\section{Limits}

\subsection{Definition}
The limit of a functor $F:J\to C$ is the universal cone over $F$. That is, an
object $\lim F \in \mathrm{ob}_C$, together with a natural transformation
$\eta: \lim F \Rightarrow F$ such that for any other objects
$X \in \mathrm{ob}_C$ with natural transformation $\alpha: X \Rightarrow F$,
there exists a unique arrow $f : X \to \lim F$ such that $\alpha = \eta \circ f$
\parencite{math3ma:limits2}
% https://q.uiver.app/?q=WzAsMyxbMCwxLCJcXGxpbSBGIl0sWzAsMiwiRiJdLFswLDAsIlgiXSxbMCwxLCJcXGV0YSIsMix7ImxldmVsIjoyfV0sWzIsMCwiZiIsMix7InN0eWxlIjp7ImJvZHkiOnsibmFtZSI6ImRvdHRlZCJ9fX1dLFsyLDEsIlxcYWxwaGEiLDAseyJjdXJ2ZSI6LTQsImxldmVsIjoyfV1d
\[\begin{tikzcd}
	X \\
	{\lim F} \\
	F
	\arrow["\eta"', Rightarrow, from=2-1, to=3-1]
	\arrow["f"', dotted, from=1-1, to=2-1]
	\arrow["\alpha", curve={height=-24pt}, Rightarrow, from=1-1, to=3-1]
\end{tikzcd}\]

\subsection{Terminal Objects}
A terminal object $a$ is the limit of the empty diagram.
\cite{adamek_herrlich_strecker:joy_cats}
If there are multiple terminal objects, they are all isomorphic.
It is an object such that for every $c\in \mathrm{ob}_C$ there exists exactly
one arrow $f: c\to a$.

e.g. In the category Set, the terminal objects are the singleton sets, whose only
incoming arrow from any category is the constant function.

\subsection{Product}
\subsubsection*{Between objects}
\todo{it actually says coproducs are the colimits of discrete diagrams} The product of $n$ objects is the limit of a diagram with $n$ disconnected
objects (discrete category). \parencite{adamek_herrlich_strecker:joy_cats} If there are multiple products, they are all isomorphic.
In a category $C$ the cartesian product of $2$ objects $a,\ b$ is another
object $a\times b$ with morphisms $\mathrm{fst}: a\times b \to a,
\ \mathrm{snd}: a\times b \to b$ such that, for any other object $c$ with the
same property there exist a unique $f:c \to a\times b$ such that the following
diagram commutes
% https://q.uiver.app/?q=WzAsNCxbMSwwLCJhXFx0aW1lcyBiIl0sWzAsMSwiYSJdLFsyLDEsImIiXSxbMSwyLCJjIl0sWzAsMSwiXFxtYXRocm17ZnN0fSIsMl0sWzAsMiwiXFxtYXRocm17c25kfSJdLFszLDAsImYiLDIseyJzdHlsZSI6eyJib2R5Ijp7Im5hbWUiOiJkYXNoZWQifX19XSxbMywxLCJnIl0sWzMsMiwiaCIsMl1d
\[\begin{tikzcd}
	& {a\times b} \\
	a && b \\
	& c
	\arrow["{\mathrm{fst}}"', from=1-2, to=2-1]
	\arrow["{\mathrm{snd}}", from=1-2, to=2-3]
	\arrow["f"', dashed, from=3-2, to=1-2]
	\arrow["g", from=3-2, to=2-1]
	\arrow["h"', from=3-2, to=2-3]
\end{tikzcd}\]

e.g.
Consider the following preorder
% https://q.uiver.app/?q=WzAsNixbMSwyLCI2Il0sWzAsMiwiMyJdLFsxLDEsIjUiXSxbMSwwLCI0Il0sWzAsMSwiMiJdLFswLDAsIjEiXSxbMCwxLCJnIl0sWzAsMiwiZiIsMl0sWzIsMywiYyIsMl0sWzEsNCwiZSJdLFs0LDUsImIiXSxbMyw1LCJhIiwyXSxbNCwyLCJkIl1d
\[\begin{tikzcd}
	1 & 4 \\
	2 & 5 \\
	3 & 6
	\arrow["g", from=3-2, to=3-1]
	\arrow["f"', from=3-2, to=2-2]
	\arrow["c"', from=2-2, to=1-2]
	\arrow["e", from=3-1, to=2-1]
	\arrow["b", from=2-1, to=1-1]
	\arrow["a"', from=1-2, to=1-1]
	\arrow["d", from=2-1, to=2-2]
\end{tikzcd}\]
The product candidates for $2\times 4$ are $2$ with $\mathrm{id}_2 : 2\to 2$
and $d\circ c:2\to 4$, $3$ with $e: 3\to 2$ and $c\circ d\circ e: 3\to 4$,
and $6$ with $e \circ g: 6\to 2$ and $c\circ f$. The product $2\times 4$
corresponds then to the object $2$. The cartesian product in a
preorder corresponds to the meet operation $\land$ or infimum.
\parencite{fong:7sketches}

\subsubsection*{Between categories}
For a product category $C\times D$ with categories $C,\ D$:
\parencite{awodey:category_theory}
\begin{itemize}
  \item Objects:\\
    $(c, d) \in \mathrm{ob}_{C\times D},\ \forall (c \in \mathrm{ob}_C, d \in \mathrm{ob}_D)$
  \item Morphisms:\\
    $(f,g) : (c,d) \to (c', d'),\ \forall (f: c \to c',\ g:d \to d')$
\end{itemize}
This operation is the cartesian product in the category of categories Cat

e.g.
\[
  % https://q.uiver.app/?q=WzAsMixbMCwwLCIxIl0sWzAsMSwiMiJdLFswLDEsImYiLDJdXQ==
\begin{tikzcd}
	1 \\
	2
	\arrow["f", from=1-1, to=2-1]
\end{tikzcd}
\times
% https://q.uiver.app/?q=WzAsMixbMCwwLCJhIl0sWzAsMSwiYiJdLFswLDEsImciXV0=
\begin{tikzcd}
	a \\
	b
	\arrow["g", from=1-1, to=2-1]
\end{tikzcd}
=
% https://q.uiver.app/?q=WzAsNCxbMCwwLCIoMSxhKSJdLFsxLDAsIigxLGIpIl0sWzAsMSwiKDIsYSkiXSxbMSwxLCIoMixiKSJdLFswLDEsIihpZF8xLGcpIl0sWzAsMiwiKGYsaWRfYSkiLDJdLFsyLDMsIihpZF8yLGcpIiwyXSxbMSwzLCIoZixpZF9iKSJdLFswLDMsIihmLGcpIiwxXV0=
\begin{tikzcd}
	{(1,a)} & {(1,b)} \\
	{(2,a)} & {(2,b)}
	\arrow["{(\mathrm{id}_1,g)}", from=1-1, to=1-2]
	\arrow["{(f,\mathrm{id}_a)}"', from=1-1, to=2-1]
	\arrow["{(\mathrm{id}_2,g)}"', from=2-1, to=2-2]
	\arrow["{(f,\mathrm{id}_b)}", from=1-2, to=2-2]
	\arrow["{(f,g)}"{description}, from=1-1, to=2-2]
\end{tikzcd}
\]