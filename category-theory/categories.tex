\section{Categories}

\subsection{Definition}
For a category $C$: \parencite{fong:7sketches}
\begin{itemize}
  \item Objects:\\
    Can be thought of as ``points'' with no internal structure.\\
    $\mathrm{ob}_C \coloneq$ collection of objects in the category $C$.
  \item Morphisms:\\
    They are directed connections between objects\\
    $\mathrm{hom}_C(a, b) \coloneqq$ collection of morphisms from $a$ to $b$,
    $\forall(a, b \in \mathrm{ob}_C$)\\
    $\mathrm{hom}_C \coloneqq$ collection of all morphisms in the category
    $C$\\
    $\mathrm{hom}_C \ni f: a \to b$ means $f \in \mathrm{hom}_C(a,c)$
  \item Identity:\\
    $\mathrm{hom}_C \ni \mathrm{id}_c: c \to c,\ \forall(c \in \mathrm{ob}_C)$
  \item Composition:\\
    $\mathrm{hom}_C \ni g \circ f : a \to c,
    \ \forall(a, b, c \in \mathrm{ob}_C, f: a \to b,\ g: b \to c)$
\end{itemize}

s.t. satisfies
\begin{itemize}
  \item Unitality:\\
    $\mathrm{id}_b \circ f = f \circ \mathrm{id}_a = f,\ \forall(a, b \in \mathrm{ob}_C, f: a \to b)$
  \item Assosiativity:\\
    $h \circ (g \circ f) = (h \circ g) \circ f,\
    \forall(a,b,c\in \mathrm{ob}_C, f:a\to b,\ g:b\to c,\ h:c\to d)$
\end{itemize}

e.g.
\[\begin{tikzcd}
  a \arrow[out=90, in=180, loop, swap, "\mathrm{id}_a", looseness=5]
    \arrow[r, "f"]
    \arrow[dr, swap, "g \circ f"]
  &
  b \arrow[out=0, in=90, loop, swap, "\mathrm{id}_b", looseness=4.5]
    \arrow[d, "g"]\\
  &
  c \arrow[out=-90, in=0, swap, "\mathrm{id}_c", looseness=5]
\end{tikzcd}\]

\noindent A category $C$ is said to be small if $\mathrm{ob}_C$ and
$\mathrm{hom}_C$ are proper sets instead of proper classes, otherwise it is said
to be large. A category is locally small if $\mathrm{hom}_C(a, b)$ is a proper
set $\forall a, b\ in ob_C$. \parencite{awodey:category_theory}

\subsection{Duality}
Given a category $C$, its opposite category $C^\mathrm{op}$ is given by keeping
all of $C$'s objects and reversing its arrows.
\parencite{maclane:working_mathematician}
\begin{itemize}
  \item Objects:\\
    $ob_{C^\mathrm{op}} = ob_C$
  \item Morphisms:\\
    $C^{\mathrm{op}} \ni f^{\mathrm{op}} : c' \to c,
      \ \forall (C \ni f : c \to c')$
\end{itemize}
As the axioms of category theory are self-dual, for any statement $\Sigma$
that holds for all categories, its dual statement $\Sigma^*$ ($\Sigma$ in
$C^\mathrm{op}$) must also hold for all categories.
\parencite{awodey:category_theory}

\subsection{Arrow properties}
An arrow is said to be: \parencite{maclane:working_mathematician}
\begin{itemize}
  \item Invertible: when there exists another arrow $g:c \to b$ such that
    $g\circ f = \mathrm{id}_b$ and $f\circ g = \mathrm{id}_c$. If such g exists, it is unique, and
    is written as $g= f^{-1}$. Two objects are isomorphic if there is an
    invertible arrow (an insomorphism) between them. If two objects $a,\ b$
    are isomorphic we write $a\cong b$.
    % https://q.uiver.app/?q=WzAsMixbMCwwLCJhIl0sWzEsMCwiYiJdLFswLDEsImYiLDAseyJvZmZzZXQiOi0xfV0sWzEsMCwiZl57LTF9IiwwLHsib2Zmc2V0IjotMX1dXQ==
    \[\begin{tikzcd}
      a & b
      \arrow["f", shift left=1, from=1-1, to=1-2]
      \arrow["{f^{-1}}", shift left=1, from=1-2, to=1-1]
      \arrow["\mathrm{id}_a"', out=90+45, in=-90-45, looseness=4.5, from=1-1, to=1-1]
      \arrow["\mathrm{id}_b", out=45, in=-45, looseness=4.5, from=1-2, to=1-2]
    \end{tikzcd}\]
  \item Monic: when it is left cancellable i.e.
    $f \circ g_1 = f \circ g_2 \implies g_1 = g_2,\ \forall (g_1, g_2:a\to b)$.
    % https://q.uiver.app/?q=WzAsMyxbMCwwLCJhIl0sWzEsMCwiYiJdLFsyLDAsImMiXSxbMCwxLCJnIiwwLHsiY3VydmUiOi0xfV0sWzAsMSwiZyciLDIseyJjdXJ2ZSI6MX1dLFsxLDIsImYiLDJdXQ==
    \[\begin{tikzcd}
      a & b & c
      \arrow["g", shift left=1, from=1-1, to=1-2]
      \arrow["{g'}"', shift right=1, from=1-1, to=1-2]
      \arrow["f"', from=1-2, to=1-3]
    \end{tikzcd}\]
    In Set the monic arrows are the injections (monomorphisms).
    Suppose sets $b$ and $c$ with a non injective function $f$ i.e. for some
    pair of elements $x \neq y$ in $b$, $f(x) = f(y)$. Take a third set $a$ with
    a pair of functions $g, g':a\to b$ that differ only in that one maps an
    element to $x$ and the other one maps the same element to $y$. Now we have
    that $f\circ g = f\circ g'$ completing the test for non-injectivity.
    \[\text{Non-Injective}(f:b\to c)
      \coloneq(\exists g, g':a\to b).(g \neq g' \land f\circ g = f\circ g')\]
    Or negating the logic.
    \[\text{Injective}(f:b\to c)
      \coloneq(\forall g, g':a\to b).(f\circ g = f\circ g' \implies g=g')\]

  \item Epic: when it is right cancellable i.e.
    $g_1 \circ f = g_2 \circ f \implies g_1 = g_2,\ \forall (g_1, g_2: c\to d)$.
    % https://q.uiver.app/?q=WzAsMyxbMCwwLCJhIl0sWzEsMCwiYiJdLFsyLDAsImMiXSxbMSwyLCJnJyIsMix7ImN1cnZlIjoxfV0sWzAsMSwiZiIsMl0sWzEsMiwiZyIsMCx7ImN1cnZlIjotMX1dXQ==
    \[\begin{tikzcd}
      a & b & c
      \arrow["{g'}"', shift right=1, from=1-2, to=1-3]
      \arrow["f"', from=1-1, to=1-2]
      \arrow["g", shift left=1, from=1-2, to=1-3]
    \end{tikzcd}\]
    In Set the epic arrows are the surjections (epimorphisms).
    Suppose sets $a$ and $b$ with a non surjective function $f$ i.e. for some
    element $y\in b$ there exists no $x\in a$ such that $f(x) = y$. Take a third
    set $c$ with a pair of functions $g, g':b\to c$ that differ only in that
    they map $y$ differently. Now we have $g\circ f = g'\circ f$ completing the
    test for non-surjectivity.
    \[\text{Non-Surjective}(f:a\to b)
      \coloneq(\exists g, g':b\to c).(g \neq g' \land g\circ f = g'\circ f)\]
    Or negating the logic.
    \[\text{Surjective}(f:a\to b)
      \coloneq(\forall g, g':b\to c).(g\circ f = g'\circ f \implies g=g')\]

  \item Idempotent: when $f\circ f = f$. It is said to split when there exists
    arrows $g$ and $h$ such that $hg = f$ and $gh = id$
\end{itemize}

\subsection{Important Categories}

\begin{itemize}
  \item Product Category:\\
    For categories $C,D$, a product category $C\times D$ consists of:
    \parencite{awodey:category_theory}
    \begin{itemize}
      \item Objects:\\
        $(c, d) \in \mathrm{ob}_{C\times D},\ \forall (c \in \mathrm{ob}_C, d \in \mathrm{ob}_D)$
      \item Morphisms:\\
        $(f,g) : (c,d) \to (c', d'),\ \forall (f: c \to c',\ g:d \to d')$
    \end{itemize}
    This operation is the product in the category of categories Cat

    e.g.
    \[
      % https://q.uiver.app/?q=WzAsMixbMCwwLCIxIl0sWzAsMSwiMiJdLFswLDEsImYiLDJdXQ==
    \begin{tikzcd}
      1 \\
      2
      \arrow["f", from=1-1, to=2-1]
    \end{tikzcd}
    \times
    % https://q.uiver.app/?q=WzAsMixbMCwwLCJhIl0sWzAsMSwiYiJdLFswLDEsImciXV0=
    \begin{tikzcd}
      a \\
      b
      \arrow["g", from=1-1, to=2-1]
    \end{tikzcd}
    =
    % https://q.uiver.app/?q=WzAsNCxbMCwwLCIoMSxhKSJdLFsxLDAsIigxLGIpIl0sWzAsMSwiKDIsYSkiXSxbMSwxLCIoMixiKSJdLFswLDEsIihpZF8xLGcpIl0sWzAsMiwiKGYsaWRfYSkiLDJdLFsyLDMsIihpZF8yLGcpIiwyXSxbMSwzLCIoZixpZF9iKSJdLFswLDMsIihmLGcpIiwxXV0=
    \begin{tikzcd}
      {(1,a)} & {(1,b)} \\
      {(2,a)} & {(2,b)}
      \arrow["{(\mathrm{id}_1,g)}", from=1-1, to=1-2]
      \arrow["{(f,\mathrm{id}_a)}"', from=1-1, to=2-1]
      \arrow["{(\mathrm{id}_2,g)}"', from=2-1, to=2-2]
      \arrow["{(f,\mathrm{id}_b)}", from=1-2, to=2-2]
      \arrow["{(f,g)}"{description}, from=1-1, to=2-2]
    \end{tikzcd}
    \]
  \item Free Category:\\
    The free category of any directed graph is that which is obtained by using
    vertices as objects, edges as arrows, and adding all necessary identity
    arrows and compositions to fulfill the category requirements. It has no
    more contraints than are necessary.

  \item Thin Category:\\
    A thin category is a category where there is an arrow from objects $a$ to
    $b$ whenever $a \leq b$. This is the opposite of a free category in the
    sense that every pair of arrows with matching source and targets are
    identical. It has every possible constraint. Every hom-set is empty or a
    singleton.

  \item Set:\\
    It is the category of all sets with functions as their morphisms.
    It is a large category and also a locally small one. \parencite{awodey:category_theory}
    \begin{itemize}
      \item Objects:\\
        Every set is an object in Set.
      \item Morphisms:\\
        Morphisms in Set are functions between sets.
    \end{itemize}

  \item Cat:\\
    It is the category of all small categories. It is therefore a large category.
    \parencite{adamek_herrlich_strecker:joy_cats}
    \begin{itemize}
      \item Objects:\\
        Every small category is an object in Cat.
      \item Morphisms:\\
        Morphisms in Cat are functors between categories.
    \end{itemize}
\end{itemize}
