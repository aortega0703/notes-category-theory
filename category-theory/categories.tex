\section{Categories}

\subsection{Definition}
\cite{7sketches_compositionality} For a category $C$:
\begin{itemize}
  \item Objects:\\
    Can be thought of as ``points'' with no internal structure.\\
    $\mathrm{ob}_C \coloneq$ collection of objects in the category $C$.
  \item Morphisms:\\
    They are directed connections between objects\\
    $\mathrm{hom}_C(a, b) \coloneqq$ collection of morphisms from $a$ to $b$,
    $\forall(a, b \in \mathrm{ob}_C$)\\
    $\mathrm{hom}_C \coloneqq$ collection of all morphisms in the category
    $C$\\
    $\mathrm{hom}_C \ni f: a \to b$ means $f \in \mathrm{hom}_C(a,c)$
  \item Identity:\\
    $\mathrm{hom}_C \ni \mathrm{id}_c: c \to c,\ \forall(c \in \mathrm{ob}_C)$
  \item Composition:\\
    $\mathrm{hom}_C \ni g \circ f : a \to c,
    \ \forall(a, b, c \in \mathrm{ob}_C, f: a \to b,\ g: b \to c)$
\end{itemize}

s.t. satisfies
\begin{itemize}
  \item Unitality:\\
    $\mathrm{id}_b \circ f = f \circ \mathrm{id}_a = f,\ \forall(a, b \in \mathrm{ob}_C, f: a \to b)$
  \item Assosiativity:\\
    $h \circ (g \circ f) = (h \circ g) \circ f,\
    \forall(a,b,c\in \mathrm{ob}_C, f:a\to b,\ g:b\to c,\ h:c\to d)$
\end{itemize}

e.g.
\[\begin{tikzcd}
  a \arrow[out=90, in=180, loop, swap, "\mathrm{id}_a", looseness=5]
    \arrow[r, "f"]
    \arrow[dr, swap, "g \circ f"]
  &
  b \arrow[out=0, in=90, loop, swap, "\mathrm{id}_b", looseness=4.5]
    \arrow[d, "g"]\\
  &
  c \arrow[out=-90, in=0, swap, "\mathrm{id}_c", looseness=5]
\end{tikzcd}\]

\noindent A category $C$ is said to be small if $\mathrm{ob}_C$ and
$\mathrm{hom}_C$ are proper sets instead of proper classes. A category locally
small if $\mathrm{hom}_C(a, b)$ is a proper set $\forall a, b\ in ob_C$.
Otherwise the category is said to be large. \cite{nlab:category}
\medskip

\subsection{Arrow properties}
\noindent\cite{categories_working_mathematician} An arrow is said to be
\begin{itemize}
  \item Invertible: when there exists another arrow $g:c \to b$ such that
    $g\circ f = id_b$ and $f\circ g = id_c$. If such g exists, it is unique, and
    is written as $g= f^{-1}$. Two objects are isomorphic if there is an
    invertible arrow (an insomorphism) between them. If two objects $a,\ b$
    are isomorphic we write $a\cong b$.
  \item Monic: when it is left cancellable i.e.
    $f \circ g_1 = f \circ g_2 \implies g_1 = g_2,\ \forall (g_1, g_2:a\to b)$.
    In Set the monic arrows are the injections (monomorphisms).
  \item Epic: when it is right cancellable i.e.
    $g_1 \circ f = g_2 \circ f \implies g_1 = g_2,\ \forall (g_1, g_2: c\to d)$.
    In Set the epic arrows are the surjections (epimorphisms).
  \item Idempotent: when $f\circ f = f$. It is said to split when there exists
    arrows $g$ and $h$ such that $hg = f$ and $gh = id$
\end{itemize}

\subsection{Important Categories}

\begin{itemize}
  \item Opposite Category:\\
    Given a category $C$, the opposite category $C^\mathrm{op}$ is given by
    keeping all of $C$'s objects and reversing its arrows.
    \cite{nlab:opposite_category}
    \begin{itemize}
      \item Objects:\\
        $ob_{C^\mathrm{op}} = ob_C$
      \item Morphisms:\\
        $C^{\mathrm{op}} \ni f^{\mathrm{op}} : c' \to c,
          \ \forall (C \ni f : c \to c')$
    \end{itemize}

  \item Set:\\
    It is the category of all sets with functions as their morphisms.
    It is a large category and also a locally small one. \cite{nlab:set}
    \begin{itemize}
      \item Objects:\\
        Every set is an object in Set.
      \item Morphisms:\\
        Morphisms in Set are functions between sets.
    \end{itemize}

  \item Cat:\\
    It is the category of all small categories. It is therefore a large category.
    \cite{nlab:cat}
    \begin{itemize}
      \item Objects:\\
        Every small category is an object in Cat.
      \item Morphisms:\\
        Morphisms in Cat are functors between categories.
    \end{itemize}
\end{itemize}
