\section{Functors}
A functor can be thought of as an structure preserving transformation between
categories, or a way to identify one category inside another.

\begin{definition}[Functor\index{Functor}]
  For categories $\C,\D$, a functor $F: \C \to \D$ consist of maps of objects
  and morphisms, such that it preserves:
  \parencite{awodey:category_theory}
  \begin{itemize}
    \item Domain and Codomain:
      \[\big(\forall (f:c\to c')\in \C_1\big)
        \big(F(f): F(c) \to F(c')\big)\]
    \item Identity:
      \[(\forall c \in \C_0)
        (F(\id_c) = \id_{F(c)})\]
    \item Composition:
      \[\big(\forall (f: a \to b)\in \C_1,\ (g: b\to c) \in \C_1\big)
        \big(F(g \circ f) = F(g) \circ F(f)\big)\]
  \end{itemize}
\end{definition}

\begin{example}
  % https://q.uiver.app/?q=WzAsNyxbMCwwLCIxIl0sWzMsMCwiRigxKSJdLFs0LDAsIkYoMikiXSxbNCwxLCJcXGJ1bGxldCJdLFsxLDAsIjIiXSxbMiwwLCIzIl0sWzUsMCwiRigzKSJdLFswLDQsImYiXSxbMSwyLCJGKGYpIl0sWzIsMywiaCIsMl0sWzEsMywiaCBcXGNpcmMgRihmKSIsMix7ImxhYmVsX3Bvc2l0aW9uIjoyMH1dLFs0LDUsImciXSxbMCw1LCJnXFxjaXJjIGYiLDAseyJjdXJ2ZSI6LTN9XSxbMiw2LCJGKGcpIl0sWzMsNiwiaSIsMl0sWzEsNiwiRihnXFxjaXJjIGYpIiwwLHsiY3VydmUiOi00fV1d
  \[\begin{tikzcd}[ampersand replacement=\&]
    1 \& 2 \& 3 \& {F(1)} \& {F(2)} \& {F(3)} \\
    \&\&\&\& \bullet
    \arrow["f", from=1-1, to=1-2]
    \arrow["{F(f)}", from=1-4, to=1-5]
    \arrow["h"', from=1-5, to=2-5]
    \arrow["{h \circ F(f)}"'{pos=0.2}, from=1-4, to=2-5]
    \arrow["g", from=1-2, to=1-3]
    \arrow["{g\circ f}", curve={height=-18pt}, from=1-1, to=1-3]
    \arrow["{F(g)}", from=1-5, to=1-6]
    \arrow["i"', from=2-5, to=1-6]
    \arrow["{F(g\circ f)}", curve={height=-24pt}, from=1-4, to=1-6]
  \end{tikzcd}\]
\end{example}

\begin{theorem}[Functors Preserve Isomorphisms]\label{thm:isomorphism_functor}

  Given two isomorphic objects, their images under any functor are also
  isomorphic.

  \begin{proof}
    Consider two objects $c,d \in \C_0$ with an isomorphism $f: c\to d$, and any
    functor $F:\C \to \D$ then:
    \[
      \begin{aligned}
        f^{-1}\circ f &= \id_c\\
        F(f^{-1}\circ f) &= F(\id_c)\\
        F(f^{-1})\circ F(f) &= \id_{F(c)}
      \end{aligned}
      \qquad
      \begin{aligned}
        f \circ f^{-1} &= \id_d\\
        F(f \circ f^{-1}) &= F(\id_d)\\
        F(f) \circ F(f^{-1}) &= \id_{F(d)}
      \end{aligned}
    \]
  \end{proof}
\end{theorem}

\subsection{Composition}
\begin{definition}[Identity Functor\index{Identity Functor}]\label{def:id_functor}

  For a category $\C$ there exists an identity functor $\id_\C: \C \to \C$ which
  consists of maps of:
  \parencite{adamek_herrlich_strecker:joy_cats}
  \begin{itemize}
    \item Objects:
      \[(\forall c \in \C_0)
        (\id_\C(c) = c)\]
    \item Morphisms:
      \[\big(\forall (f: c\to c') \in \C_1\big)
        (\id_C(f) = f)\]
  \end{itemize}
\end{definition}

\begin{theorem}[$\id_\C$ Preserves Identity\label{thm:id_functor_identity}]
  For a category $\C$, the identity functor $\id_\C: \C \to \C$ preserves
  identity.

  \begin{proof}
    For all objects $c\in \C$, $\id_\C(\id_c) = \id_c$ by definition.
  \end{proof}
\end{theorem}

\begin{theorem}[$\id_\C$ Preserves Composition]\label{thm:id_functor_composition}

  For a category $\C$, the identity functor $\id_\C: \C \to \C$ preserves
  composition.

  \begin{proof}
    For all morphisms $f,g\in \C$ with $g$ composable after $f$:
    \[
      \begin{aligned}
        \id_\C(g\circ f)
        &= g\circ f\\
        &= \id_\C(g) \circ \id_\C(f)
      \end{aligned}
    \]
  \end{proof}
\end{theorem}

\begin{theorem}[$\id_\C$ is a Functor]
  For a category $\C$, the identity functor $\id_\C$ is a functor.

  \begin{proof}
    By Theorems \ref{thm:id_functor_identity} and
    \ref{thm:id_functor_composition} the identity functor is a functor.
  \end{proof}
\end{theorem}

\begin{definition}[Functor Composition\index{Functor
  Composition}]\label{def:functor_composition}

  For functors $F:\A\to \B$ and $G:\B\to \C$, the composite functor $G\circ F$
  consists of maps of:
  \parencite{leinster:basic_category_theory}
  \begin{itemize}
    \item Objects:
      \[(\forall a \in \A_0)
        \big((G\circ F)(a) = G(F(a))\big)\]
    \item Morphisms:
      \[(\forall f\in \A_1)
        \big((G\circ F)(f) = G(F(f))\big)\]
  \end{itemize}
\end{definition}

\begin{theorem}[Functor Composition Preserves Identity\label{thm:functor_cmp_id}]
  For functors $F:\A\to \B$ and $G:\B\to \C$, the composite functor $G\circ F$
  preserves indentity.

  \begin{proof}
    For all objects $a\in \A_0$:
    \[
      \begin{aligned}
        (G\circ F)(\id_a)
        &= G(F(\id_a))\\
        &= G(\id_{F(a)})\\
        &= \id_{(G\circ F)(a)}
      \end{aligned}
    \]
  \end{proof}
\end{theorem}

\begin{theorem}[Functor Composition Preserves Composition\label{thm:functor_cmp_cmp}]
  For functors $F:\A\to \B$ and $G:\B\to \C$, the composite functor $G\circ F$
  preserves composition.

  \begin{proof}
    For all morphisms $f,g\in \A_1$ with $g$ composable after $f$:
    \[
      \begin{aligned}
        (G\circ F)(g\circ F)
        &= G(F(g\circ f))\\
        &= G(F(g)\circ F(f))\\
        &= G(F(g))\circ G(F(f))\\
        &= (G\circ F)(g)\circ(G\circ F)(f)
      \end{aligned}
    \]
  \end{proof}
\end{theorem}

\begin{theorem}[Functor Composition is a Functor]
  For functors $F, G$, the composition $G\circ F$ is a functor.

  \begin{proof}
    By Theorems \ref{thm:functor_cmp_id} and \ref{thm:functor_cmp_cmp} functor
    composition fulfills the axioms of a functor.
  \end{proof}
\end{theorem}

\begin{theorem}[Unitality of Functor Composition\label{thm:unitality_functor}]
  For a functor $F:\C\to \D$ the following holds:
  \[F \circ \id_C = \id_D \circ F = F\]

  \begin{proof}
    Let either $c\in \C_0$ or $c\in \C_1$, then:
    \[
      \begin{aligned}
        (F \circ \id_C)(c)
          &= F(\id_C(c))\\
          &= F (c)
      \end{aligned}
      \quad
      \begin{aligned}
        (\id_D \circ F)(c)
          &= \id_D (F (c))\\
          &= F (c)
      \end{aligned}
    \]
  \end{proof}
\end{theorem}

\begin{theorem}[Associativity of Functor Composition\label{thm:assoc_functor}]
  For a set-up of categories and functors:
  % https://q.uiver.app/?q=WzAsNCxbMCwwLCJcXEEiXSxbMSwwLCJcXEIiXSxbMiwwLCJcXEMiXSxbMywwLCJcXEQiXSxbMCwxLCJGIl0sWzEsMiwiRyJdLFsyLDMsIkgiXV0=&macro_url=https%3A%2F%2Fraw.githubusercontent.com%2Faortega0703%2Fnotes-category-theory%2Fmain%2Fsrc%2Fmacros.tex
  \[\begin{tikzcd}[ampersand replacement=\&]
    \A \& \B \& \C \& \D
    \arrow["F", from=1-1, to=1-2]
    \arrow["G", from=1-2, to=1-3]
    \arrow["H", from=1-3, to=1-4]
  \end{tikzcd}\]

  The following holds:
  \[(H \circ G) \circ F = H\circ(G\circ F)\]

  \begin{proof}
    Let either $c\in \C_0$ or $c\in \C_1$, then:
    \[
      \begin{aligned}
        ((H \circ G) \circ F) (c)
          &= (H\circ G) (F (c))\\
          &= H(G(F(c)))\\
          &= H((G\circ F)(c))\\
          &= (H\circ(G\circ F)) (c)
      \end{aligned}
    \]
  \end{proof}
\end{theorem}

\subsection{Functor Properties}

\begin{definition}[Faithful Functor\index{Faithful Functor}]
  A functor $F:\C\to \D$ is faithful when for every morphism $(f: c\to c') \in
  \C_1$, the following function is injective:
  \parencite{leinster:basic_category_theory}
    \[
      \begin{aligned}
        F: \C(c, c') &\to \D(F(c), F(c'))\\
        f &\mapsto F(f)
      \end{aligned}
    \]
\end{definition}

\begin{definition}[Full Functor\index{Full Functor}]
  A functor $F:\C\to \D$ is full when for every morphism $(f: c\to c') \in
  \C_1$, the following function is surjective:
  \parencite{leinster:basic_category_theory}
  \[
    \begin{aligned}
      F: \C(c, c') &\to \D(F(c), F(c'))\\
      f &\mapsto F(f)
    \end{aligned}
  \]
\end{definition}

\begin{definition}[Endofunctor\index{Endofunctor}]
  A functor $F:\C\to \C$ (one from $\C$ to itself) is an endofunctor.
  \parencite{adamek_herrlich_strecker:joy_cats}
\end{definition}

\begin{definition}[Bifunctor\index{Bifunctor}]
  A functor $F:\A\times \B \to \C$ (one with $2$ parameters) is said to be a
  bifunctor.
  \parencite{lane:working_mathematician}
\end{definition}

\begin{definition}[Presheaf\index{Presheaf}]
  A functor $\C^\op \to \Set$ is a presheaf on $\C$.
  \parencite{leinster:basic_category_theory}
\end{definition}

\begin{theorem}[Fully Faithful Isomorphism]\label{thm:full_faithful_isomorphism}

  For any full and faithful functor $F:\C\to \D$ it follows that $c \cong c'
  \iff F(c) \cong F(c')$ for objects $c, c'\in \C_0$.

  \begin{proof}
    The $(\implies)$ holds via Theorem \ref{thm:isomorphism_functor}. For the
    $(\impliedby)$ part, consider the isomorphism $Ff: F(c) \overset{\cong}{\to}
    F(c')$. As $F$ is fully faithful, morphisms $c\to c'$ are in bijection
    with morphisms $F(c)\to F(c')$, therefore there exists unique arrows $f:c\to c'$ $g:c'\to c$ such that $F(f) = Ff$ and $F(g) = Ff^{-1}$, then:
    \[
      \begin{aligned}
        &F(\id_c)\\
        =& \id_{F(c)}\\
        =& Ff^{-1} \circ Ff\\
        =& F(g) \circ F(f)\\
        =& F(g\circ f)
      \end{aligned}
      \qquad
      \begin{aligned}
        &F(\id_{c'})\\
        =& \id_{F(c')}\\
        =& Ff \circ Ff^{-1}\\
        =& F(f) \circ F(g)\\
        =& F(f\circ g)
      \end{aligned}
    \]

    As $F$ is faithful and $F(\id_c) = F(g\circ f)$, then $g\circ f = \id_c$.
    Similarly, $f\circ g = \id_{c'}$.
  \end{proof}
\end{theorem}