\section{Functors}
A functor can be thought of as an structure preserving transformation between
categories, or a way to identify one category inside another.

\begin{definition}[Functor]
  For categories $C,D$, a functor $F: C \to D$ consist of mappings of:
  \parencite{fong_spivak:7sketches}
  \begin{itemize}
    \item Object to objects:\\
      $F(c) \in D,\ (\forall c \in C)$
    \item Morphism to morphisms:\\
      $(F(f): F(c) \to F(c')) \in D,
        \ (\forall c,c' \in C,\ f : c \to c')$
  \end{itemize}
  Such that it preserves:
  \begin{itemize}
    \item Identity\\
      $F(\id_c) = \id_{F(c)},
        \ (\forall c \in C)$
    \item Composition\\
      $F(g \circ f) = F(g) \circ F(f),
        \ (\forall a,b,c \in C,\ f: a \to b,\ g: b \to c)$
  \end{itemize}
\end{definition}

\begin{example}
  % https://q.uiver.app/?q=WzAsNSxbMCwwLCIxIl0sWzAsMiwiYSJdLFsxLDIsImIiXSxbMiwyLCJjIl0sWzIsMCwiMiJdLFswLDQsImYiXSxbMSwyLCJnIl0sWzIsMywiaCJdLFswLDEsIkYiLDEseyJjb2xvdXIiOlsyNDAsNjAsNjBdLCJzdHlsZSI6eyJib2R5Ijp7Im5hbWUiOiJkb3R0ZWQifX19XSxbNCwzLCJGIiwxLHsiY29sb3VyIjpbMjQwLDYwLDYwXSwic3R5bGUiOnsiYm9keSI6eyJuYW1lIjoiZG90dGVkIn19fV0sWzEsMywiaCBcXGNpcmMgZyIsMSx7ImN1cnZlIjotM31dLFs1LDEwLCJGIiwxLHsic2hvcnRlbiI6eyJzb3VyY2UiOjEwLCJ0YXJnZXQiOjIwfSwibGV2ZWwiOjEsImNvbG91ciI6WzI0MCw2MCw2MF0sInN0eWxlIjp7ImJvZHkiOnsibmFtZSI6ImRvdHRlZCJ9fX1dXQ==
  \[\begin{tikzcd}
    1 && 2 \\
    \\
    a & b & c
    \arrow[""{name=0, anchor=center, inner sep=0}, "f", from=1-1, to=1-3]
    \arrow["g", from=3-1, to=3-2]
    \arrow["h", from=3-2, to=3-3]
    \arrow["F"{description}, draw={rgb,255:red,92;green,92;blue,214}, dotted, from=1-1, to=3-1]
    \arrow["F"{description}, draw={rgb,255:red,92;green,92;blue,214}, dotted, from=1-3, to=3-3]
    \arrow[""{name=1, anchor=center, inner sep=0}, "{h \circ g}"{description}, curve={height=-18pt}, from=3-1, to=3-3]
    \arrow["F"{description}, draw={rgb,255:red,92;green,92;blue,214}, shorten <=3pt, shorten >=6pt, dotted, from=0, to=1]
  \end{tikzcd}\]
\end{example}

\subsection{Composition}
\begin{definition}[Composition of Functors]
  For categories $A,B,C$ and functors $F:A\to B$ and $G:B\to C$ the composite
  functor $G\circ F$ consists of maps of:
  \parencite{leinster:basic_category_theory}
  \begin{itemize}
    \item Objects:\\
      $(G\circ F)(a) = G(F(a)) \in C,\ (\forall a\in A)$
    \item Morphisms:\\
      $(G\circ F)(f): (G\circ F)(a) \to (G\circ F)(a') = G(F(f)) \in C,\ (\forall
        a, a'\in A,\ f:a\to a')$
  \end{itemize}
\end{definition}

\begin{definition}[Identity Functor]
  For a category $C$ there exists an identity functor $\id_C: C \to C$ which
  trivially preserves identity and composition. It consists of maps of:
  \parencite{adamek_herrlich_strecker:joy_cats}
  \begin{itemize}
    \item Objects:\\
      $\id_C(c) = c \in C,\ (\forall c\in C)$
    \item Morphisms:\\
      $\id_C(f): c\to c' = f \in C,\ (\forall c, c'\in C,\ f:c\to c')$
  \end{itemize}
\end{definition}

\begin{theorem}[Unitality of Composition of Functors]
  For categories $\mathcal{C},\mathcal{D}$ and a functor $F:C\to D$ the following holds:
  \[F \circ \id_C = \id_D \circ F = F\]

  \begin{proof}
    Let either $c\in \ob_C$ or $c\in \hom_C$, then:
    \[
      \begin{aligned}
        (F \circ \id_C)(c)
          &= F(\id_C(c))\\
          &= F (c)
      \end{aligned}
      \quad
      \begin{aligned}
        (\id_D \circ F)(c)
          &= \id_D (F (c))\\
          &= F (c)
      \end{aligned}
    \]
  \end{proof}
\end{theorem}

\begin{theorem}[Associativity]
  For a configuration of categories and functors:
  % https://q.uiver.app/?q=WzAsNCxbMCwwLCJBIl0sWzEsMCwiQiJdLFsyLDAsIkMiXSxbMywwLCJEIl0sWzAsMSwiRiJdLFsxLDIsIkciXSxbMiwzLCJIIl1d
  \[\begin{tikzcd}
    A & B & C & D
    \arrow["F", from=1-1, to=1-2]
    \arrow["G", from=1-2, to=1-3]
    \arrow["H", from=1-3, to=1-4]
  \end{tikzcd}\]

  The following holds:
  \[(H \circ G) \circ F = H\circ(G\circ F)\]

  \begin{proof}
    Let either $c\in \ob_C$ or $c\in \hom_C$, then:
    \[
      \begin{aligned}
        ((H \circ G) \circ F) (c)
          &= (H\circ G) (F (c))\\
          &= H(G(F(c)))\\
          &= H((G\circ F)(c))\\
          &= (H\circ(G\circ F)) (c)
      \end{aligned}
    \]
  \end{proof}
\end{theorem}

\subsection{Functor Properties}

\subsubsection*{Faithful}
For categories $C, D$, a functor $F:C\to D$ is said to be faithful when for
every $x,y\in C$, the function $F_{x, y}: \hom_C(x, y) \to
\hom_D(F(x), F(y))$ is injective. \parencite{awodey:category_theory}

\subsubsection*{Full}
For categories $C, D$, a functor $F:C\to D$ is said to be full when for every
$x,y\in C$, the function $F_{x, y}: \hom_C(x, y) \to
\hom_D(F(x), F(y))$ is surjective.
\parencite{adamek_herrlich_strecker:joy_cats}

\subsubsection*{Endofunctor}
For a category $C$, a functor $F:C\to C$ (one from $C$ to itself) is said to be
an endofunctor. \parencite{adamek_herrlich_strecker:joy_cats}

\subsubsection*{Bifunctor}
For categories $A, B, C$, a functor $F:A\times B \to C$ (one with $2$
parameters) is said to be a bifunctor.
\parencite{maclane:working_mathematician}