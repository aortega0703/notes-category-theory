\section{Adjunctions}

\subsection{Via Natural Isomorphism}

\begin{definition}[Adjunction\index{Adjunction!via Natural Isomorphism}\label{def:adjunction_isomorphism}]
  For categories $C, D$ and functors $L: C\to D$ and $R: D\to C$, the pair ($L$,
  $R$) is said to be a left/right adjoint of each other when there is a natural
  isomorphism between the hom-functors:
  \parencite{leinster:basic_category_theory}
  \[
    \begin{aligned}
      F&: C^\op \times D \to \mathrm{Set}\\
      F&=\Hom_D(L(\underline{\ \ }), \underline{\ \ })
    \end{aligned}
    \qquad
    \begin{aligned}
      G&: C^\op \times D \to \mathrm{Set}\\
      G&=\Hom_C(\underline{\ \ }, R(\underline{\ \ }))
    \end{aligned}
  \]

  \begin{itemize}
    \item $(c', d)$-Components:
      \[
        \alpha_{(c', d)}
          = \Hom_D(L(c), d) \overset{\cong}{\to} \Hom_C(c, R(d))\\
      \]
    \item Naturality Condition:\\
      For every $f = (g, h): (c', d) \to (c, d')$, the following commutes:
      % https://q.uiver.app/?q=WzAsNCxbMCwwLCJcXEhvbV9EKEwoYyksIGQpIl0sWzEsMCwiXFxIb21fQyhjLCBSKGQpKSJdLFswLDEsIlxcSG9tX0QoTChjJyksZCcpIl0sWzEsMSwiXFxIb21fQyhjJyxSKGQnKSkiXSxbMCwyLCJcXEhvbV9EKEwoZyksIGgpIiwyXSxbMSwzLCJcXEhvbV9DKGcsIFIoaCkpIl0sWzAsMSwiXFxhbHBoYV97KGMsIGQpfSJdLFsyLDMsIlxcYWxwaGFfeyhjJyxkJyl9IiwyXSxbMCwxLCJcXGNvbmciLDJdLFsyLDMsIlxcY29uZyJdXQ==&macro_url=https%3A%2F%2Fraw.githubusercontent.com%2Faortega0703%2Fnotes-category-theory%2Fmain%2Fsrc%2Fmacros.tex
      \[\begin{tikzcd}[ampersand replacement=\&]
        {\Hom_D(L(c), d)} \& {\Hom_C(c, R(d))} \\
        {\Hom_D(L(c'),d')} \& {\Hom_C(c',R(d'))}
        \arrow["{\Hom_D(L(g), h)}"', from=1-1, to=2-1]
        \arrow["{\Hom_C(g, R(h))}", from=1-2, to=2-2]
        \arrow["{\alpha_{(c, d)}}", from=1-1, to=1-2]
        \arrow["{\alpha_{(c',d')}}"', from=2-1, to=2-2]
        \arrow["\cong"', from=1-1, to=1-2]
        \arrow["\cong", from=2-1, to=2-2]
      \end{tikzcd}\]
  \end{itemize}
\end{definition}

\begin{remark}
  In diagram a pair of left/right adjoints $(L, R)$ is indicated by a double
  arrow ($\Rightarrow$) going in the same direction as the left adjoint, or as a
  turnstile ($\vdash$) pointing towards the left adjoint as follows:
  % https://q.uiver.app/?q=WzAsMixbMCwwLCJEIl0sWzEsMCwiQyJdLFsxLDAsIkwiLDIseyJjdXJ2ZSI6Mn1dLFswLDEsIlIiLDIseyJjdXJ2ZSI6Mn1dLFsxLDAsIiIsMSx7InNob3J0ZW4iOnsic291cmNlIjoyMCwidGFyZ2V0IjoyMH0sImxldmVsIjoyfV1d
  \[\begin{tikzcd}
    D & C
    \arrow["L"', curve={height=18pt}, from=1-2, to=1-1]
    \arrow["R"', curve={height=18pt}, from=1-1, to=1-2]
    \arrow[shorten <=3pt, shorten >=3pt, Rightarrow, from=1-2, to=1-1]
  \end{tikzcd}
    \text{or}
  % https://q.uiver.app/?q=WzAsMixbMCwwLCJEIl0sWzEsMCwiQyJdLFsxLDAsIkwiLDIseyJjdXJ2ZSI6Mn1dLFswLDEsIlIiLDIseyJjdXJ2ZSI6Mn1dLFsyLDMsIiIsMix7ImxldmVsIjoxLCJzdHlsZSI6eyJuYW1lIjoiYWRqdW5jdGlvbiJ9fV1d
    \begin{tikzcd}
    D & C
    \arrow[""{name=0, anchor=center, inner sep=0}, "L"', curve={height=18pt}, from=1-2, to=1-1]
    \arrow[""{name=1, anchor=center, inner sep=0}, "R"', curve={height=18pt}, from=1-1, to=1-2]
    \arrow["\dashv"{anchor=center, rotate=-90}, draw=none, from=0, to=1]
  \end{tikzcd}\]
\end{remark}

\begin{definition}[Transpose\index{Transpose}\label{def:transpose}]
  For a pair of left/right adjoints $(L, R)$, consider the following maps that
  constitute the isomorphism in Definition \ref{def:adjunction_isomorphism}:
  \[
    \begin{aligned}
      \phi_{(c, d)}: \hom_D(L(c), d) &\to \hom_C(c, R(d))\\
      u &\mapsto \overline{u}\\
      \psi_{(c, d)}: \hom_C(c, R(d)) &\to \hom_D(L(c), d)\\
      v &\mapsto \overline{v}
    \end{aligned}
  \]
  Such that $\overline{\overline{u}} = u$ and $\overline{\overline{v}} = v$.
  Here $\overline{f}$ is called the transpose of $f$.
\end{definition}

\begin{theorem}[General Tranpose\label{thm:transpose}]
  For a pair of left/right adjoints $(L, R)$, the naturality condition is
  equivalent to stating that:
  \begin{align*}
    \overline{h\circ u \circ L(g)} = R(h) \circ \overline{u} \circ g
  \end{align*}

  \begin{proof}
    It follows from equating both paths in the naturality condition, that:
    \[
      \begin{aligned}
      \big(\alpha_{(c', d')} \circ \Hom_D(L(g), h)\big)(u)
        &= \big(\Hom_C(g, R(h)) \circ \alpha_{(c, d)}\big)(u)\\
      \alpha_{(c', d')}(h\circ u \circ L(g))
        &= \Hom_C(g, R(h))(\overline{u})\\
      \overline{h\circ u \circ L(g)}
        &= R(h) \circ \overline{u} \circ g
      \end{aligned}
    \]
  \end{proof}
\end{theorem}

\begin{remark}
  In Theorem \ref{thm:transpose}, the overline goes in either side of the
  equation as $\phi_{(c, d)}$ and $\psi_{(c, d)}$ are inverses.
\end{remark}

\begin{theorem}[Alternative Transposes\label{thm:transpose2}]
  Theorem \ref{thm:transpose} is equivalent to stating that for a pair of
  left/right adjoints $(L:C\to D, R:D\to C)$, the following two equations hold:
  \begin{align*}
    \overline{h\circ u} &= R(h) \circ \overline{u}\\
    \overline{v\circ g} &= \overline{v} \circ L(g)
  \end{align*}

  \begin{proof}
    The proof consists of two parts:
    \begin{description}
      \item[$(\implies)$] By setting either $g=\id_c$ or $h=\id_d$:
        \[
          \begin{aligned}
            \overline{h\circ u}
              &= \overline{h\circ u \circ L(\id_c)}\\
              &= R(h) \circ \overline{u} \circ \id_c\\
              &= R(h) \circ \overline{u}
          \end{aligned}
          \qquad
          \begin{aligned}
            \overline{v\circ g}
              &= \overline{R(\id_d)\circ v\circ g}\\
              &= \id_d \circ \overline{v} \circ L(g)\\
              &= \overline{v} \circ L(g)
          \end{aligned}
        \]
      \item[$(\impliedby)$] Consider the following:
        \[
          \begin{aligned}
            \overline{h \circ u\circ L(g)}
              &= \overline{(h \circ u)\circ L(g)}\\
              &= \overline{h\circ u} \circ g\\
              &= R(h) \circ \overline{u} \circ g
          \end{aligned}
        \]
    \end{description}
  \end{proof}
\end{theorem}

\subsection{Via Initial Morphism}

\begin{definition}[Adjunction\index{Adjunction!via Initial Morphism}\label{def:adjunction_initial}]
  For categories $C,D$ and functors $L: C\to D$ and $R: D\to C$, $L$ is said to
  be a left adjoint of $R$ when there is an initial morphism $(L(c), \eta_c:c\to
  (R\circ L)(c))$ from every $c\in C$ to $R$ such that $\eta: \id_C \to (R\circ
  L)$ is a natural transformation:
  \parencite{awodey:category_theory}
  % https://q.uiver.app/?q=WzAsNSxbMCwwLCJMKGMpIl0sWzAsMSwiZCJdLFsxLDAsIihSXFxjaXJjIEwpKGMpIl0sWzEsMSwiUihkKSJdLFsyLDAsImMiXSxbMCwxLCJcXG92ZXJsaW5le3Z9IiwyLHsic3R5bGUiOnsiYm9keSI6eyJuYW1lIjoiZGFzaGVkIn19fV0sWzIsMywiUihcXG92ZXJsaW5le3Z9KSIsMix7InN0eWxlIjp7ImJvZHkiOnsibmFtZSI6ImRhc2hlZCJ9fX1dLFs0LDMsInYiXSxbNCwyLCJcXGV0YV9jIiwyXV0=
  \[\begin{tikzcd}[ampersand replacement=\&]
    {L(c)} \& {(R\circ L)(c)} \& c \\
    d \& {R(d)}
    \arrow["{\overline{v}}"', dashed, from=1-1, to=2-1]
    \arrow["{R(\overline{v})}"', dashed, from=1-2, to=2-2]
    \arrow["v", from=1-3, to=2-2]
    \arrow["{\eta_c}"', from=1-3, to=1-2]
  \end{tikzcd}\]
\end{definition}

\begin{theorem}
  Definitions \ref{def:adjunction_isomorphism} and \ref{def:adjunction_initial}
  are equivalent.

  \begin{proof}
    The proof consists on two parts.
    \begin{description}
      \item[$(\implies)$] To obtain the existence of the counit consider:
    \end{description}
  \end{proof}
\end{theorem}

\subsection{Via Terminal Morphism}

\begin{definition}[Adjunction\index{Adjunction!via Terminal Morphism}\label{def:adjunction_terminal}]
  For categories $C,D$ and functors $L: C\to D$ and $R: D\to C$, $L$ is said to
  be a left adjoint of $R$ when there exists a terminal morphism $(R(d), \eta_d:
  (L\circ R)(d) \to d)$ from every $d\in D$ to $L$ such that $\varepsilon: (L\circ R)\to \id_D$ is a natural transformation:
  \parencite{awodey:category_theory}
  % https://q.uiver.app/?q=WzAsNSxbMCwxLCJSKGQpIl0sWzAsMCwiYyJdLFsxLDAsIkwoYykiXSxbMSwxLCIoTFxcY2lyYyBSKShkKSJdLFsyLDAsImQiXSxbMSwwLCJcXG92ZXJsaW5le3V9IiwyLHsic3R5bGUiOnsiYm9keSI6eyJuYW1lIjoiZGFzaGVkIn19fV0sWzIsMywiTChcXG92ZXJsaW5le3V9KSIsMix7InN0eWxlIjp7ImJvZHkiOnsibmFtZSI6ImRhc2hlZCJ9fX1dLFszLDQsIlxcdmFyZXBzaWxvbl9kIiwyXSxbMiw0LCJ1Il1d
  \[\begin{tikzcd}[ampersand replacement=\&]
    c \& {L(c)} \& d \\
    {R(d)} \& {(L\circ R)(d)}
    \arrow["{\overline{u}}"', dashed, from=1-1, to=2-1]
    \arrow["{L(\overline{u})}"', dashed, from=1-2, to=2-2]
    \arrow["{\varepsilon_d}"', from=2-2, to=1-3]
    \arrow["u", from=1-2, to=1-3]
  \end{tikzcd}\]
\end{definition}

\subsection{Via Triangle Identities}

\begin{definition}[Adjunction\index{Adjunction!via Triangle Identities}\label{def:adjunction_triangle}]
  For categories $C,D$ and functors $L: C\to D$ and $R: D\to C$, $L$ is said to
  be a left adjoint of $R$ when there exists a unit ($\eta: \id_C\to R\circ L$)
  and counit ($\varepsilon: L\circ R \to \id_D$):
  \parencite{leinster:basic_category_theory}

  \[
    \begin{aligned}
      \eta:&\!\!\!&\id_C\ &\Rightarrow R \circ L\\
      \varepsilon:&\!\!\!& L \circ R &\Rightarrow\ \ \id_D
    \end{aligned}
  \]
  Such that the triangular identities are satisfied (the following diagrams
  commute):

  % https://q.uiver.app/?q=WzAsMyxbMCwwLCJMKGMpIl0sWzAsMSwiKExcXGNpcmMgUlxcY2lyYyBMKShjKSJdLFsxLDEsIkwoYykiXSxbMCwxLCJMKFxcZXRhX2MpIiwyXSxbMSwyLCJcXHZhcmVwc2lsb25fe0woYyl9IiwyXSxbMCwyLCJcXGlkX3tMKGMpfSJdXQ==&macro_url=https%3A%2F%2Fraw.githubusercontent.com%2Faortega0703%2Fnotes-category-theory%2Fmain%2Fsrc%2Fmacros.tex
  \[\begin{tikzcd}[ampersand replacement=\&]
    {L(c)} \\
    {(L\circ R\circ L)(c)} \& {L(c)}
    \arrow["{L(\eta_c)}"', from=1-1, to=2-1]
    \arrow["{\varepsilon_{L(c)}}"', from=2-1, to=2-2]
    \arrow["{\id_{L(c)}}", from=1-1, to=2-2]
  \end{tikzcd}
  \quad
  % https://q.uiver.app/?q=WzAsMyxbMCwwLCJSKGQpIl0sWzEsMCwiKFJcXGNpcmMgTFxcY2lyYyBSKShkKSJdLFsxLDEsIlIoZCkiXSxbMCwxLCJcXGV0YV97UihkKX0iXSxbMSwyLCJSKFxcdmFyZXBzaWxvbl9kKSJdLFswLDIsIlxcaWRfe1IoZCl9IiwyXV0=&macro_url=https%3A%2F%2Fraw.githubusercontent.com%2Faortega0703%2Fnotes-category-theory%2Fmain%2Fsrc%2Fmacros.tex
  \begin{tikzcd}[ampersand replacement=\&]
    {R(d)} \& {(R\circ L\circ R)(d)} \\
    \& {R(d)}
    \arrow["{\eta_{R(d)}}", from=1-1, to=1-2]
    \arrow["{R(\varepsilon_d)}", from=1-2, to=2-2]
    \arrow["{\id_{R(d)}}"', from=1-1, to=2-2]
  \end{tikzcd}\]
\end{definition}

\begin{theorem}[Adjoint Functors Definitions]
  Definitions \ref{def:adjunction_isomorphism} and \ref{def:adjunction_triangle}
  of adjoint functors are equivalent.

  \begin{proof}
    The proof consists of two parts:
    \begin{description}
      \item[($\implies$)] For categories $C,D$ with left/right adjoints $L:C\to
        D$, $R:D\to C$, consider:
        \[
          \begin{aligned}
            \Hom_D(L(c), d) &\ \:\cong\ \: \Hom_C(c, R(d))\\
            \Hom_D(L(c), L(c)) &\ \:\cong\ \: \Hom_C(c, (R\circ L)(c))\\
            \big(\id_{L(c)}: L(c) \to L(c)\big) \in D
            &\implies \big(\eta_c : c \to (R\circ L)(c)\big)\in C\\
            \overline{\id_{L(c)}} &\ \:=\ \:\eta_c
          \end{aligned}
        \]

        Obtaining the existence of the unit $\eta:\id_D\Rightarrow (R\circ L)$.
        Similarly, the existence of the counit $\varepsilon:(L\circ
        R)\Rightarrow\id_C$ can be obtained by:
        \[
          \begin{aligned}
            \Hom_D(L(c), d) &\ \ \cong\ \ \Hom_C(c, R(d))\\
            \Hom_D((L\circ R)(d), d) &\ \ \cong\ \ \Hom_C(R(d), R(d))\\
            \big(\varepsilon_{d} : (L\circ R)(d) \to d) \in D
            &\impliedby \big(\id_{R(d)}: R(d) \to R(d)\big) \in C\\
            \overline{\id_{R(d)}} &= \varepsilon_d
          \end{aligned}
        \]

        To obtain the triangle identities consider:
        \[
          \begin{aligned}
            \id_{L(c)} &= \overline{\eta_c}\\
              &= \overline{R(\id_{L(c)})\circ R(\id_{L(c)})\circ \eta_c}\\
              &= \id_{L(c)} \circ \varepsilon_{L(c)}\circ L(\eta_c)\\
              &= \varepsilon_{L(c)}\circ L(\eta_c)
          \end{aligned}
          \qquad
          \begin{aligned}
            \id_{R(d)} &= \overline{\varepsilon_d}\\
              &= \overline{\varepsilon_d
                \circ L(\id_{R(d)})\circ L(\id_{R(d)})}\\
              &= R(\varepsilon_d) \circ \eta_{R(d)} \circ \id_{R(d)}\\
              &= R(\varepsilon_d) \circ \eta_{R(d)}
          \end{aligned}
        \]
      \item[($\impliedby$)] For categories $C,D$ with left/right adjoints
        $L:C\to D$, $R:D\to C$, consider two functions:
        \[
          \begin{aligned}
            \phi: \hom_D(L(c), d) &\to \hom_C(c, R(d))\\
            u &\mapsto R(u)\circ \eta_c\\
            \psi: \hom_C(c, R(d)) &\to \hom_D(L(c), d)\\
            v &\mapsto \varepsilon_d \circ L(v)
          \end{aligned}
        \]
        Given the triangle identities, and using the naturality condition of $\eta$ and $\varepsilon$, the following diagrams commute:
        % https://q.uiver.app/?q=WzAsNSxbMCwwLCJMKHgpIl0sWzAsMSwiKExcXGNpcmMgUlxcY2lyYyBMKSh4KSJdLFsxLDEsIkwoeCkiXSxbMCwyLCIoTFxcY2lyYyBSKSh5KSJdLFsxLDIsInkiXSxbMCwyLCJcXGlkX3tMKHgpfSJdLFswLDEsIkwoXFxldGFfeCkiLDJdLFsxLDIsIlxcdmFyZXBzaWxvbl97TCh4KX0iXSxbMiw0LCJ1Il0sWzEsMywiKExcXGNpcmMgUikodSkiLDJdLFszLDQsIlxcdmFyZXBzaWxvbl95IiwyXV0=&macro_url=https%3A%2F%2Fraw.githubusercontent.com%2Faortega0703%2Fnotes-category-theory%2Fmain%2Fsrc%2Fmacros.tex
        \[\begin{tikzcd}[ampersand replacement=\&]
          {L(x)} \\
          {(L\circ R\circ L)(x)} \& {L(x)} \\
          {(L\circ R)(y)} \& y
          \arrow["{\id_{L(x)}}", from=1-1, to=2-2]
          \arrow["{L(\eta_x)}"', from=1-1, to=2-1]
          \arrow["{\varepsilon_{L(x)}}", from=2-1, to=2-2]
          \arrow["u", from=2-2, to=3-2]
          \arrow["{(L\circ R)(u)}"', from=2-1, to=3-1]
          \arrow["{\varepsilon_y}"', from=3-1, to=3-2]
        \end{tikzcd}
        \qquad
        % https://q.uiver.app/?q=WzAsNSxbMCwxLCJSKHkpIl0sWzEsMSwiKFJcXGNpcmMgTFxcY2lyYyBSKSh5KSJdLFsxLDIsIlIoeSkiXSxbMSwwLCIoTFxcY2lyYyBSKSh4KSJdLFswLDAsIngiXSxbMCwyLCJcXGlkX3tSKHgpfSIsMl0sWzAsMSwiXFxldGFfe1IoeSl9Il0sWzEsMiwiUihcXHZhcmVwc2lsb25feSkiXSxbMywxLCIoTFxcY2lyYyBSKSh2KSJdLFs0LDAsInYiLDJdLFs0LDMsIlxcZXRhX3giXV0=&macro_url=https%3A%2F%2Fraw.githubusercontent.com%2Faortega0703%2Fnotes-category-theory%2Fmain%2Fsrc%2Fmacros.tex
        \begin{tikzcd}[ampersand replacement=\&]
          x \& {(L\circ R)(x)} \\
          {R(y)} \& {(R\circ L\circ R)(y)} \\
          \& {R(y)}
          \arrow["{\id_{R(x)}}"', from=2-1, to=3-2]
          \arrow["{\eta_{R(y)}}", from=2-1, to=2-2]
          \arrow["{R(\varepsilon_y)}", from=2-2, to=3-2]
          \arrow["{(L\circ R)(v)}", from=1-2, to=2-2]
          \arrow["v"', from=1-1, to=2-1]
          \arrow["{\eta_x}", from=1-1, to=1-2]
        \end{tikzcd}\]

        To show that $\phi$ and $\psi$ are an isomorphism consider their
        compositions, both of which are equal to the identity taking the
        external paths of the respective diagrams above:
        \[
          \begin{aligned}
            (\psi\circ\phi)(u) &=
              \varepsilon \circ (L\circ R(u)) \circ L(\eta_x) = u\\
            (\phi\circ\psi)(v) &=
              R(\varepsilon_y)\circ (R\circ L(v)) \circ \eta_x = v\\
          \end{aligned}
        \]
        In order to show naturality it is sufficient to show that for morphisms
        $g:c'\to c$ and $h:d\to d'$ Theorem \ref{thm:transpose2} holds:
        \[
          \begin{aligned}
            \phi(h\circ u)
              &= R(h) \circ R(u) \circ \eta_c\\
              &= R(h) \phi(u)
          \end{aligned}
          \qquad
          \begin{aligned}
            \psi(v\circ g)
              &= \varepsilon_d \circ L(v) \circ L(g)\\
              &= \psi(v) \circ L(g)
          \end{aligned}
        \]
    \end{description}
  \end{proof}
\end{theorem}