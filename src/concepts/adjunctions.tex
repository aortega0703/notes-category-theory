\section{Adjunctions}

\subsection{Via Natural Isomorphism}

\begin{definition}[Adjunction\index{Adjunction!via Natural Isomorphism}\label{def:adjunction_isomorphism}]
  For categories $C, D$ and functors $L: C\to D$ and $R: D\to C$, the pair ($L$,
  $R$) is said to be a left/right adjoint of each other when there is a natural
  isomorphism between the hom-functors:
  \parencite{leinster:basic_category_theory}
  \[
    \begin{aligned}
      F&: C^\op \times D \to \mathrm{Set}\\
      F&=\Hom_D(L(\underline{\ \ }), \underline{\ \ })
    \end{aligned}
    \qquad
    \begin{aligned}
      G&: C^\op \times D \to \mathrm{Set}\\
      G&=\Hom_C(\underline{\ \ }, R(\underline{\ \ }))
    \end{aligned}
  \]

  \begin{itemize}
    \item $(c', d)$-Components:
      \[
        \alpha_{(c', d)}
          = \Hom_D(L(c), d) \overset{\cong}{\to} \Hom_C(c, R(d))\\
      \]
    \item Naturality Condition:\\
      For every $f = (g, h): (c', d) \to (c, d')$, the following commutes:
      % https://q.uiver.app/?q=WzAsNCxbMCwwLCJcXEhvbV9EKEwoYyksIGQpIl0sWzEsMCwiXFxIb21fQyhjLCBSKGQpKSJdLFswLDEsIlxcSG9tX0QoTChjJyksZCcpIl0sWzEsMSwiXFxIb21fQyhjJyxSKGQnKSkiXSxbMCwyLCJcXEhvbV9EKEwoZyksIGgpIiwyXSxbMSwzLCJcXEhvbV9DKGcsIFIoaCkpIl0sWzAsMSwiXFxhbHBoYV97KGMsIGQpfSJdLFsyLDMsIlxcYWxwaGFfeyhjJyxkJyl9IiwyXSxbMCwxLCJcXGNvbmciLDJdLFsyLDMsIlxcY29uZyJdXQ==&macro_url=https%3A%2F%2Fraw.githubusercontent.com%2Faortega0703%2Fnotes-category-theory%2Fmain%2Fsrc%2Fmacros.tex
      \[\begin{tikzcd}[ampersand replacement=\&]
        {\Hom_D(L(c), d)} \& {\Hom_C(c, R(d))} \\
        {\Hom_D(L(c'),d')} \& {\Hom_C(c',R(d'))}
        \arrow["{\Hom_D(L(g), h)}"', from=1-1, to=2-1]
        \arrow["{\Hom_C(g, R(h))}", from=1-2, to=2-2]
        \arrow["{\alpha_{(c, d)}}", from=1-1, to=1-2]
        \arrow["{\alpha_{(c',d')}}"', from=2-1, to=2-2]
        \arrow["\cong"', from=1-1, to=1-2]
        \arrow["\cong", from=2-1, to=2-2]
      \end{tikzcd}\]
  \end{itemize}
\end{definition}

\begin{remark}
  In diagram a pair of left/right adjoints $(L, R)$ is indicated by a double
  arrow ($\Rightarrow$) going in the same direction as the left adjoint, or as a
  turnstile ($\vdash$) pointing towards the left adjoint as follows:
  % https://q.uiver.app/?q=WzAsMixbMCwwLCJEIl0sWzEsMCwiQyJdLFsxLDAsIkwiLDIseyJjdXJ2ZSI6Mn1dLFswLDEsIlIiLDIseyJjdXJ2ZSI6Mn1dLFsxLDAsIiIsMSx7InNob3J0ZW4iOnsic291cmNlIjoyMCwidGFyZ2V0IjoyMH0sImxldmVsIjoyfV1d
  \[\begin{tikzcd}
    D & C
    \arrow["L"', curve={height=18pt}, from=1-2, to=1-1]
    \arrow["R"', curve={height=18pt}, from=1-1, to=1-2]
    \arrow[shorten <=3pt, shorten >=3pt, Rightarrow, from=1-2, to=1-1]
  \end{tikzcd}
    \text{or}
  % https://q.uiver.app/?q=WzAsMixbMCwwLCJEIl0sWzEsMCwiQyJdLFsxLDAsIkwiLDIseyJjdXJ2ZSI6Mn1dLFswLDEsIlIiLDIseyJjdXJ2ZSI6Mn1dLFsyLDMsIiIsMix7ImxldmVsIjoxLCJzdHlsZSI6eyJuYW1lIjoiYWRqdW5jdGlvbiJ9fV1d
    \begin{tikzcd}
    D & C
    \arrow[""{name=0, anchor=center, inner sep=0}, "L"', curve={height=18pt}, from=1-2, to=1-1]
    \arrow[""{name=1, anchor=center, inner sep=0}, "R"', curve={height=18pt}, from=1-1, to=1-2]
    \arrow["\dashv"{anchor=center, rotate=-90}, draw=none, from=0, to=1]
  \end{tikzcd}\]
\end{remark}

\begin{definition}[Transpose\index{Transpose}\label{def:transpose}]
  For a pair of left/right adjoints $(L, R)$, consider the following maps that
  constitute the isomorphism in Definition \ref{def:adjunction_isomorphism}:
  \[
    \begin{aligned}
      \phi_{(c, d)}: \hom_D(L(c), d) &\to \hom_C(c, R(d))\\
      u &\mapsto \bar{u}\\
      \psi_{(c, d)}: \hom_C(c, R(d)) &\to \hom_D(L(c), d)\\
      v &\mapsto \bar{v}
    \end{aligned}
  \]
  As $\phi$ and $\psi$ form an isomorphism $\bar{\bar{f}} = f$. Here $\bar{f}$
  is called the transpose of $f$.
\end{definition}

\begin{theorem}[General Tranpose\label{thm:transpose}]
  For a pair of left/right adjoints $(L, R)$, the naturality condition is
  equivalent to stating that:
  \begin{align*}
    \bar{h\circ u \circ L(g)} = R(h) \circ \bar{u} \circ g
  \end{align*}

  \begin{proof}
    It follows from equating both paths in the naturality condition, that:
    \[
      \begin{aligned}
      \big(\alpha_{(c', d')} \circ \Hom_D(L(g), h)\big)(u)
        &= \big(\Hom_C(g, R(h)) \circ \alpha_{(c, d)}\big)(u)\\
      \alpha_{(c', d')}(h\circ u \circ L(g))
        &= \Hom_C(g, R(h))(\bar{u})\\
      \bar{h\circ u \circ L(g)}
        &= R(h) \circ \bar{u} \circ g
      \end{aligned}
    \]
  \end{proof}
\end{theorem}

\begin{remark}
  In Theorem \ref{thm:transpose}, the overline goes in either side of the
  equation as $\phi_{(c, d)}$ and $\psi_{(c, d)}$ are inverses.
\end{remark}

\begin{theorem}[Alternative Transposes\label{thm:transpose2}]
  Theorem \ref{thm:transpose} is equivalent to stating that for a pair of
  left/right adjoints $(L:C\to D, R:D\to C)$, the following two equations hold:
  \begin{align*}
    \bar{h\circ u} &= R(h) \circ \bar{u}\\
    \bar{v\circ g} &= \bar{v} \circ L(g)
  \end{align*}

  \begin{proof}
    The proof consists of two parts:
    \begin{description}
      \item[$(\implies)$] By setting either $g=\id_c$ or $h=\id_d$:
        \[
          \begin{aligned}
            \bar{h\circ u}
              &= \bar{h\circ u \circ L(\id_c)}\\
              &= R(h) \circ \bar{u} \circ \id_c\\
              &= R(h) \circ \bar{u}
          \end{aligned}
          \qquad
          \begin{aligned}
            \bar{v\circ g}
              &= \bar{R(\id_d)\circ v\circ g}\\
              &= \id_d \circ \bar{v} \circ L(g)\\
              &= \bar{v} \circ L(g)
          \end{aligned}
        \]
      \item[$(\impliedby)$] Consider the following:
        \[
          \begin{aligned}
            \bar{h \circ u\circ L(g)}
              &= \bar{(h \circ u)\circ L(g)}\\
              &= \bar{h\circ u} \circ g\\
              &= R(h) \circ \bar{u} \circ g
          \end{aligned}
        \]
    \end{description}
  \end{proof}
\end{theorem}

\begin{theorem}[Existence of a unit $\eta$\label{thm:unit_exists}]
  For a pair of left/right adjoints ($L: \C \to \D$, $R:\D\to \C$)
  there exists a morphism $\eta_c: c\to (R\circ L)(c)$ that corresponds to
  $\bar{\id_{L(c)}}$ for every $c\in \C_0$.

  \begin{proof}
    Consider definition \ref{def:adjunction_isomorphism} of adjunction:
    \[
      \begin{aligned}
        \D(L(c), d) &\ \:\cong\ \: \C(c, R(d))\\
        \D(L(c), L(c)) &\ \:\cong\ \: \C(c, (R\circ L)(c))\\
        \big(\id_{L(c)}: L(c) \to L(c)\big) \in \D_1
        &\implies \big(\eta_c : c \to (R\circ L)(c)\big)\in \C_1\\
        \bar{\id_{L(c)}} &\ \:=\ \:\eta_c
      \end{aligned}
    \]
  \end{proof}
\end{theorem}

\begin{theorem}[Existence of a counit $\epsilon$\label{thm:counit_exists}]
  For an adjunction ($L: \C \to \D$, $R:\D\to \C$)
  there exists a morphism $\epsilon_d: (L\circ R)(d)\to d$ that corresponds to
  $\bar{\id_{R(d)}}$ for every $d\in \D_0$.

  \begin{proof}
    Consider Definition \ref{def:adjunction_isomorphism} of adjunction:
    \[
      \begin{aligned}
        \D(L(c), d) &\ \:\cong\ \: \C(c, R(d))\\
        \D((L\circ R)(d), d) &\ \:\cong\ \: \C(R(d), R(d))\\
        \big(\epsilon:(L\circ R)(d)\to d\big)\in D_1 &\impliedby
          \big(\id_{R(d)}: R(d) \to R(d)\big)\\
        \epsilon_d &\ \:=\ \:\bar{\id_{R(d)}}
      \end{aligned}
    \]
  \end{proof}
\end{theorem}

\subsection{Via Initial Morphism}

\begin{definition}[Adjunction\index{Adjunction!via Initial Morphism}\label{def:adjunction_initial}]
  For categories $C,D$ and functors $L: C\to D$ and $R: D\to C$, $L$ is said to
  be a left adjoint of $R$ when there is an initial morphism $(L(c), \eta_c:c\to
  (R\circ L)(c))$ from every $c\in C$ to $R$:
  \parencite{awodey:category_theory}
  % https://q.uiver.app/?q=WzAsNSxbMCwwLCJMKGMpIl0sWzAsMSwiZCJdLFsxLDAsIihSXFxjaXJjIEwpKGMpIl0sWzEsMSwiUihkKSJdLFsyLDAsImMiXSxbMCwxLCJcXG92ZXJsaW5le3Z9IiwyLHsic3R5bGUiOnsiYm9keSI6eyJuYW1lIjoiZGFzaGVkIn19fV0sWzIsMywiUihcXG92ZXJsaW5le3Z9KSIsMix7InN0eWxlIjp7ImJvZHkiOnsibmFtZSI6ImRhc2hlZCJ9fX1dLFs0LDMsInYiXSxbNCwyLCJcXGV0YV9jIiwyXSxbMCwyLCIiLDAseyJjb2xvdXIiOlsyNDAsNjAsNjBdLCJzdHlsZSI6eyJib2R5Ijp7Im5hbWUiOiJkb3R0ZWQifX19XSxbMSwzLCIiLDIseyJjb2xvdXIiOlsyNDAsNjAsNjBdLCJzdHlsZSI6eyJib2R5Ijp7Im5hbWUiOiJkb3R0ZWQifX19XSxbNSw2LCIiLDIseyJzaG9ydGVuIjp7InNvdXJjZSI6MjAsInRhcmdldCI6NTB9LCJsZXZlbCI6MSwiY29sb3VyIjpbMjQwLDYwLDYwXSwic3R5bGUiOnsiYm9keSI6eyJuYW1lIjoiZG90dGVkIn19fV1d
  \[\begin{tikzcd}[ampersand replacement=\&]
    {L(c)} \& {(R\circ L)(c)} \& c \\
    d \& {R(d)}
    \arrow[""{name=0, anchor=center, inner sep=0}, "{\bar{v}}"', dashed, from=1-1, to=2-1]
    \arrow[""{name=1, anchor=center, inner sep=0}, "{R(\bar{v})}"', dashed, from=1-2, to=2-2]
    \arrow["v", from=1-3, to=2-2]
    \arrow["{\eta_c}"', from=1-3, to=1-2]
    \arrow[color={rgb,255:red,92;green,92;blue,214}, dotted, from=1-1, to=1-2]
    \arrow[color={rgb,255:red,92;green,92;blue,214}, dotted, from=2-1, to=2-2]
    \arrow[color={rgb,255:red,92;green,92;blue,214}, shorten <=8pt, shorten >=21pt, dotted, from=0, to=1]
  \end{tikzcd}\]
\end{definition}

\begin{theorem}[Naturality of $\eta$]
  In Definition \ref{def:adjunction_initial}, $\eta:\id_C \to R\circ L$ is a
  natural transformation.

  \begin{proof}
    Take two objects $c,c'\in \C_0$ with a morphism $g:c'\to c$. Then by the
    definition of $\eta_{c'}$ it follows that $\eta_c\circ g = (R\circ
    L)(c)\circ \eta_{c'}$, fulfilling the naturality condition in:

    % https://q.uiver.app/?q=WzAsNixbMiwxLCJjIl0sWzAsMSwiTChjKSJdLFsxLDEsIihSXFxjaXJjIEwpKGMpIl0sWzIsMCwiYyciXSxbMCwwLCJMKGMnKSJdLFsxLDAsIihSXFxjaXJjIEwpKGMnKSJdLFswLDIsIlxcZXRhX2MiXSxbMyw1LCJcXGV0YV97Yyd9IiwyXSxbMywwLCJnIl0sWzUsMiwiKFJcXGNpcmMgTCkoZykiLDIseyJzdHlsZSI6eyJib2R5Ijp7Im5hbWUiOiJkYXNoZWQifX19XSxbNCwxLCJMKGcpIiwyLHsic3R5bGUiOnsiYm9keSI6eyJuYW1lIjoiZGFzaGVkIn19fV0sWzEsMiwiIiwyLHsiY29sb3VyIjpbMjQwLDYwLDYwXSwic3R5bGUiOnsiYm9keSI6eyJuYW1lIjoiZG90dGVkIn19fV0sWzQsNSwiIiwyLHsiY29sb3VyIjpbMjQwLDYwLDYwXSwic3R5bGUiOnsiYm9keSI6eyJuYW1lIjoiZG90dGVkIn19fV0sWzMsMiwiXFxldGFfY1xcY2lyYyBnIiwxXV0=
    \[\begin{tikzcd}[ampersand replacement=\&]
      {L(c')} \& {(R\circ L)(c')} \& {c'} \\
      {L(c)} \& {(R\circ L)(c)} \& c
      \arrow["{\eta_c}", from=2-3, to=2-2]
      \arrow["{\eta_{c'}}"', from=1-3, to=1-2]
      \arrow["g", from=1-3, to=2-3]
      \arrow["{(R\circ L)(g)}"', dashed, from=1-2, to=2-2]
      \arrow["{L(g)}"', dashed, from=1-1, to=2-1]
      \arrow[draw={rgb,255:red,92;green,92;blue,214}, dotted, from=2-1, to=2-2]
      \arrow[draw={rgb,255:red,92;green,92;blue,214}, dotted, from=1-1, to=1-2]
      \arrow["{\eta_c\circ g}"{description}, from=1-3, to=2-2]
    \end{tikzcd}\]
  \end{proof}
\end{theorem}

\begin{theorem}
  Definitions \ref{def:adjunction_isomorphism} and \ref{def:adjunction_initial}
  are equivalent.

  \begin{proof}
    The proof consists on two parts.
    \begin{description}
      \item[$(\implies)$] By Theorem \ref{thm:unit_exists} $\eta:\id_\C\to
        (R\circ L)$ exists. As $\bar{v}$ is unique for every $v$, then for any
        $c\in \C$ the initial morphism from $c$ to $R$ can be constructed in the
        following manner:
        % https://q.uiver.app/?q=WzAsNixbMSwwLCJMKGMpIl0sWzAsMCwiTChjKSJdLFszLDAsImMiXSxbMiwwLCIoUlxcY2lyYyBMKShjKSJdLFsyLDEsIlIoZCkiXSxbMCwxLCJkIl0sWzAsMSwiXFxpZF97TChjKX0iLDJdLFsyLDMsIlxcZXRhX2MiLDJdLFsyLDQsInYiXSxbMSw1LCJcXGJhcnt2fSIsMix7InN0eWxlIjp7ImJvZHkiOnsibmFtZSI6ImRhc2hlZCJ9fX1dLFszLDQsIlIoXFxiYXJ7dn0pIiwyLHsic3R5bGUiOnsiYm9keSI6eyJuYW1lIjoiZGFzaGVkIn19fV1d&macro_url=https%3A%2F%2Fraw.githubusercontent.com%2Faortega0703%2Fnotes-category-theory%2Fmain%2Fsrc%2Fmacros.tex
        \[\begin{tikzcd}[ampersand replacement=\&]
          {L(c)} \& {L(c)} \& {(R\circ L)(c)} \& c \\
          d \&\& {R(d)}
          \arrow["{\id_{L(c)}}"', from=1-2, to=1-1]
          \arrow["{\eta_c}"', from=1-4, to=1-3]
          \arrow["v", from=1-4, to=2-3]
          \arrow["{\bar{v}}"', dashed, from=1-1, to=2-1]
          \arrow["{R(\bar{v})}"', dashed, from=1-3, to=2-3]
        \end{tikzcd}\]
      \item[$(\impliedby)$] Consider the function:
        \[
          \begin{aligned}
            \phi:&&\mathcal{D}(L(c), d) &\to \mathcal{C}(c, R(d))\\
            && u &\mapsto R(u) \circ \eta_c
          \end{aligned}
        \]
        By Definition \ref{def:adjunction_initial} every $v:c\to R(d)$ can be
        expressed as $\phi(\bar{v})$ for a unique $\bar{v}:L(c)\to d$, which
        makes $\psi$ a bijection and therefore an isomorphism. To show
        naturality of $\phi$ consider $g:c'\to c$ and $h:d\to d'$:
        % https://q.uiver.app/?q=WzAsNCxbMCwwLCJcXG1hdGhjYWx7RH0oTChjKSwgZCkiXSxbMSwwLCJcXG1hdGhjYWx7Q30oYywgUihkKSkiXSxbMCwxLCJcXG1hdGhjYWx7RH0oTChjJyksIGQnKSJdLFsxLDEsIlxcbWF0aGNhbHtDfShjJywgUihkJykpIl0sWzAsMSwiXFxwaGlfe2MsZH0iXSxbMCwyLCJcXG1hdGhjYWx7RH0oTChnKSxoKSIsMl0sWzIsMywiXFxwaGlfe2MnLGQnfSIsMl0sWzEsMywiXFxtYXRoY2Fse0N9KGcsUihoKSkiXSxbMCwxLCJcXGNvbmciLDJdLFsyLDMsIlxcY29uZyJdXQ==&macro_url=https%3A%2F%2Fraw.githubusercontent.com%2Faortega0703%2Fnotes-category-theory%2Fmain%2Fsrc%2Fmacros.tex
        \[\begin{tikzcd}[ampersand replacement=\&]
          {\mathcal{D}(L(c), d)} \& {\mathcal{C}(c, R(d))} \\
          {\mathcal{D}(L(c'), d')} \& {\mathcal{C}(c', R(d'))}
          \arrow["{\phi_{c,d}}", from=1-1, to=1-2]
          \arrow["{\mathcal{D}(L(g),h)}"', from=1-1, to=2-1]
          \arrow["{\phi_{c',d'}}"', from=2-1, to=2-2]
          \arrow["{\mathcal{C}(g,R(h))}", from=1-2, to=2-2]
          \arrow["\cong"', from=1-1, to=1-2]
          \arrow["\cong", from=2-1, to=2-2]
        \end{tikzcd}\]
        \[
          \begin{aligned}
            \big(\C(g, R(h))\circ \phi_{c,d}\big)(u)
            &= \C(g, R(h))(R(u)\circ\eta_c)\\
            &= R(h)\circ R(u)\circ\eta_c \circ g\\
            &= R(h)\circ R(u) \circ (R\circ L)(g)\circ \eta_{c'}\\
            &= R(h\circ u\circ L(g))\circ \eta_{c'}\\
            &= \phi_{c', d'}\big(h\circ u \circ L(g)\big)\\
            &= \phi_{c', d'}\big(\D(L(g), h)(u)\big)\\
            &= \big(\phi_{c', d'}\circ \D(L(g), h))(u)
          \end{aligned}
        \]
    \end{description}
  \end{proof}
\end{theorem}

\subsection{Via Terminal Morphism}

\begin{definition}[Adjunction\index{Adjunction!via Terminal Morphism}\label{def:adjunction_terminal}]
  For categories $C,D$ and functors $L: C\to D$ and $R: D\to C$, $L$ is said to
  be a left adjoint of $R$ when there exists a terminal morphism $(R(d),
  \epsilon_d: (L\circ R)(d) \to d)$ from every $d\in D$ to $L$:
  \parencite{awodey:category_theory}
  % https://q.uiver.app/?q=WzAsNSxbMCwxLCJSKGQpIl0sWzAsMCwiYyJdLFsxLDAsIkwoYykiXSxbMSwxLCIoTFxcY2lyYyBSKShkKSJdLFsyLDAsImQiXSxbMSwwLCJcXG92ZXJsaW5le3V9IiwyLHsic3R5bGUiOnsiYm9keSI6eyJuYW1lIjoiZGFzaGVkIn19fV0sWzIsMywiTChcXG92ZXJsaW5le3V9KSIsMix7InN0eWxlIjp7ImJvZHkiOnsibmFtZSI6ImRhc2hlZCJ9fX1dLFszLDQsIlxcdmFyZXBzaWxvbl9kIiwyXSxbMiw0LCJ1Il0sWzEsMiwiIiwwLHsiY29sb3VyIjpbMjQwLDYwLDYwXSwic3R5bGUiOnsiYm9keSI6eyJuYW1lIjoiZG90dGVkIn19fV0sWzAsMywiIiwyLHsiY29sb3VyIjpbMjQwLDYwLDYwXSwic3R5bGUiOnsiYm9keSI6eyJuYW1lIjoiZG90dGVkIn19fV0sWzUsNiwiIiwyLHsic2hvcnRlbiI6eyJzb3VyY2UiOjIwLCJ0YXJnZXQiOjUwfSwibGV2ZWwiOjEsImNvbG91ciI6WzI0MCw2MCw2MF0sInN0eWxlIjp7ImJvZHkiOnsibmFtZSI6ImRvdHRlZCJ9fX1dXQ==
  \[\begin{tikzcd}[ampersand replacement=\&]
    c \& {L(c)} \& d \\
    {R(d)} \& {(L\circ R)(d)}
    \arrow[""{name=0, anchor=center, inner sep=0}, "{\bar{u}}"', dashed, from=1-1, to=2-1]
    \arrow[""{name=1, anchor=center, inner sep=0}, "{L(\bar{u})}"', dashed, from=1-2, to=2-2]
    \arrow["{\epsilon_d}"', from=2-2, to=1-3]
    \arrow["u", from=1-2, to=1-3]
    \arrow[color={rgb,255:red,92;green,92;blue,214}, dotted, from=1-1, to=1-2]
    \arrow[color={rgb,255:red,92;green,92;blue,214}, dotted, from=2-1, to=2-2]
    \arrow[color={rgb,255:red,92;green,92;blue,214}, shorten <=8pt, shorten >=21pt, dotted, from=0, to=1]
  \end{tikzcd}\]
\end{definition}

\begin{theorem}[Naturality of $\epsilon$]
  In Definition \ref{def:adjunction_terminal}, $\epsilon: L\circ R \to \id_d$
  is a natural transformation.

  \begin{proof}
    Take two objects $d,d'\in \D_0$ with a morphism $h:d\to d'$. Then by the
    definition of $\varepsilon_{d}$ it follows that $\eta_c\circ h = (R\circ
    L)(c)\circ \eta_{c'}$, fulfilling the naturality condition in:
    % https://q.uiver.app/?q=WzAsNixbMiwwLCJkIl0sWzIsMSwiZCciXSxbMSwwLCIoTFxcY2lyYyBSKShkKSJdLFswLDAsIlIoZCkiXSxbMCwxLCJSKGQnKSJdLFsxLDEsIihMXFxjaXJjIFIpKGQnKSJdLFswLDEsImgiXSxbMyw0LCJSKGgpIiwyLHsic3R5bGUiOnsiYm9keSI6eyJuYW1lIjoiZGFzaGVkIn19fV0sWzIsMCwiXFxlcHNpbG9uX2QiXSxbNSwxLCJcXGVwc2lsb25fe2QnfSIsMl0sWzIsMSwiaFxcY2lyY1xcZXBzaWxvbl9kIiwxXSxbMiw1LCIoTFxcY2lyYyBSKShoKSIsMix7InN0eWxlIjp7ImJvZHkiOnsibmFtZSI6ImRhc2hlZCJ9fX1dXQ==&macro_url=https%3A%2F%2Fraw.githubusercontent.com%2Faortega0703%2Fnotes-category-theory%2Fmain%2Fsrc%2Fmacros.tex
    \[\begin{tikzcd}[ampersand replacement=\&]
      {R(d)} \& {(L\circ R)(d)} \& d \\
      {R(d')} \& {(L\circ R)(d')} \& {d'}
      \arrow["h", from=1-3, to=2-3]
      \arrow["{R(h)}"', dashed, from=1-1, to=2-1]
      \arrow["{\epsilon_d}", from=1-2, to=1-3]
      \arrow["{\epsilon_{d'}}"', from=2-2, to=2-3]
      \arrow["{h\circ\epsilon_d}"{description}, from=1-2, to=2-3]
      \arrow["{(L\circ R)(h)}"', dashed, from=1-2, to=2-2]
    \end{tikzcd}\]
  \end{proof}
\end{theorem}

\begin{theorem}[Adjunction Definitions]
  Definitions \ref{def:adjunction_isomorphism} and \ref{def:adjunction_terminal}
  are equivalent.

  \begin{proof}
    The proof consists of two parts:
    \begin{description}
      \item[$(\implies)$] By Theorem \ref{thm:counit_exists} $\epsilon:(L\circ
        R) \to \id_\D$ exists. As $\bar{u}$ is unique for every $u$, then for
        any $d\in\D$ the terminal morphism from $L$ to $d$ can be constructed in
        the following manner:
        % https://q.uiver.app/?q=WzAsNixbMiwwLCJkIl0sWzIsMSwiZCciXSxbMCwwLCJSKGQpIl0sWzAsMSwiUihkJykiXSxbMSwwLCIoTFxcY2lyYyBSKShkKSJdLFsxLDEsIihMXFxjaXJjIFIpKGQnKSJdLFs0LDAsIlxcZXBzaWxvbl9kIl0sWzUsMSwiXFxlcHNpbG9uX3tkJ30iLDJdLFswLDEsImciXSxbMiwzLCJSKGcpIiwyLHsic3R5bGUiOnsiYm9keSI6eyJuYW1lIjoiZGFzaGVkIn19fV0sWzQsMSwiZ1xcY2lyYyBcXGVwc2lsb25fZCIsMV0sWzQsNSwiKExcXGNpcmMgUikoZykiLDIseyJzdHlsZSI6eyJib2R5Ijp7Im5hbWUiOiJkYXNoZWQifX19XV0=&macro_url=https%3A%2F%2Fraw.githubusercontent.com%2Faortega0703%2Fnotes-category-theory%2Fmain%2Fsrc%2Fmacros.tex
        \[\begin{tikzcd}[ampersand replacement=\&] {R(d)} \& {(L\circ R)(d)} \&
          d \\
          {R(d')} \& {(L\circ R)(d')} \& {d'}
          \arrow["{\epsilon_d}", from=1-2, to=1-3]
          \arrow["{\epsilon_{d'}}"', from=2-2, to=2-3]
          \arrow["g", from=1-3, to=2-3]
          \arrow["{R(g)}"', dashed, from=1-1, to=2-1]
          \arrow["{g\circ \epsilon_d}"{description}, from=1-2, to=2-3]
          \arrow["{(L\circ R)(g)}"', dashed, from=1-2, to=2-2]
        \end{tikzcd}\]
      \item[$(\impliedby)$] Consider the function:
        \[
          \begin{aligned}
            \psi:&&\C(c, R(d)) &\to \D(L(c), d)\\
            &&v&\mapsto \epsilon_d\circ L(v)
          \end{aligned}
        \]
        By Definition \ref{def:adjunction_terminal} every $u:L(c)\to d$ can be
        expressed as $\psi(\bar{u})$ for a unique $\bar{u}:c\to R(d)$, which
        makes $\psi$ a bijection and therefore an isomorphism. To show
        naturality of $\psi$ consider $g:c'\to c$ and $h:d\to d'$:
        % https://q.uiver.app/?q=WzAsNCxbMSwwLCJcXG1hdGhjYWx7RH0oTChjKSwgZCkiXSxbMCwwLCJcXG1hdGhjYWx7Q30oYywgUihkKSkiXSxbMSwxLCJcXG1hdGhjYWx7RH0oTChjJyksIGQnKSJdLFswLDEsIlxcbWF0aGNhbHtDfShjJywgUihkJykpIl0sWzEsMCwiXFxwc2lfe2MsZH0iXSxbMCwyLCJcXG1hdGhjYWx7RH0oTChnKSxoKSJdLFszLDIsIlxccHNpX3tjJyxkJ30iLDJdLFsxLDMsIlxcbWF0aGNhbHtDfShnLFIoaCkpIiwyXSxbMSwwLCJcXGNvbmciLDJdLFszLDIsIlxcY29uZyJdXQ==&macro_url=https%3A%2F%2Fraw.githubusercontent.com%2Faortega0703%2Fnotes-category-theory%2Fmain%2Fsrc%2Fmacros.tex
        \[\begin{tikzcd}[ampersand replacement=\&]
          {\mathcal{C}(c, R(d))} \& {\mathcal{D}(L(c), d)} \\
          {\mathcal{C}(c', R(d'))} \& {\mathcal{D}(L(c'), d')}
          \arrow["{\psi_{c,d}}", from=1-1, to=1-2]
          \arrow["{\mathcal{D}(L(g),h)}", from=1-2, to=2-2]
          \arrow["{\psi_{c',d'}}"', from=2-1, to=2-2]
          \arrow["{\mathcal{C}(g,R(h))}"', from=1-1, to=2-1]
          \arrow["\cong"', from=1-1, to=1-2]
          \arrow["\cong", from=2-1, to=2-2]
        \end{tikzcd}\]
        \[
          \begin{aligned}
            \big(\D(L(g), h)\circ \psi_{c, d}\big)(v)
            &= \D(L(g), h) (\epsilon_d \circ L(v))\\
            &= h \circ \epsilon_d \circ L(v) \circ L(g)\\
            &= \epsilon_{d'}\circ (L\circ R)(h) \circ L(v) \circ L(g)\\
            &= \epsilon_{d'}\circ L(R(h) \circ v \circ g)\\
            &= \psi_{c', d'} (R(h) \circ v \circ g)\\
            &= \psi_{c', d'} (\C(g, R(h))(v))\\
            &= \big(\psi_{c', d'} \circ \C(g, R(h))\big)(v)
          \end{aligned}
        \]
    \end{description}
  \end{proof}
\end{theorem}

\subsection{Via Triangle Identities}

\begin{definition}[Adjunction\index{Adjunction!via Triangle Identities}\label{def:adjunction_triangle}]
  For categories $C,D$ and functors $L: C\to D$ and $R: D\to C$, $L$ is said to
  be a left adjoint of $R$ when there exists a unit ($\eta: \id_C\to R\circ L$)
  and counit ($\epsilon: L\circ R \to \id_D$):
  \parencite{leinster:basic_category_theory}

  \[
    \begin{aligned}
      \eta:&\!\!\!&\id_C\ &\Rightarrow R \circ L\\
      \epsilon:&\!\!\!& L \circ R &\Rightarrow\ \ \id_D
    \end{aligned}
  \]
  Such that the triangular identities are satisfied (the following diagrams
  commute):

  % https://q.uiver.app/?q=WzAsMyxbMCwwLCJMKGMpIl0sWzAsMSwiKExcXGNpcmMgUlxcY2lyYyBMKShjKSJdLFsxLDEsIkwoYykiXSxbMCwxLCJMKFxcZXRhX2MpIiwyXSxbMSwyLCJcXHZhcmVwc2lsb25fe0woYyl9IiwyXSxbMCwyLCJcXGlkX3tMKGMpfSJdXQ==&macro_url=https%3A%2F%2Fraw.githubusercontent.com%2Faortega0703%2Fnotes-category-theory%2Fmain%2Fsrc%2Fmacros.tex
  \[\begin{tikzcd}[ampersand replacement=\&]
    {L(c)} \\
    {(L\circ R\circ L)(c)} \& {L(c)}
    \arrow["{L(\eta_c)}"', from=1-1, to=2-1]
    \arrow["{\epsilon_{L(c)}}"', from=2-1, to=2-2]
    \arrow["{\id_{L(c)}}", from=1-1, to=2-2]
  \end{tikzcd}
  \quad
  % https://q.uiver.app/?q=WzAsMyxbMCwwLCJSKGQpIl0sWzEsMCwiKFJcXGNpcmMgTFxcY2lyYyBSKShkKSJdLFsxLDEsIlIoZCkiXSxbMCwxLCJcXGV0YV97UihkKX0iXSxbMSwyLCJSKFxcdmFyZXBzaWxvbl9kKSJdLFswLDIsIlxcaWRfe1IoZCl9IiwyXV0=&macro_url=https%3A%2F%2Fraw.githubusercontent.com%2Faortega0703%2Fnotes-category-theory%2Fmain%2Fsrc%2Fmacros.tex
  \begin{tikzcd}[ampersand replacement=\&]
    {R(d)} \& {(R\circ L\circ R)(d)} \\
    \& {R(d)}
    \arrow["{\eta_{R(d)}}", from=1-1, to=1-2]
    \arrow["{R(\epsilon_d)}", from=1-2, to=2-2]
    \arrow["{\id_{R(d)}}"', from=1-1, to=2-2]
  \end{tikzcd}\]
\end{definition}

\begin{theorem}[Adjunctions Definitions]
  Definitions \ref{def:adjunction_isomorphism} and \ref{def:adjunction_triangle}
  of adjoint functors are equivalent.

  \begin{proof}
    The proof consists of two parts:
    \begin{description}
      \item[($\implies$)] Theorems \ref{thm:unit_exists} and
        \ref{thm:counit_exists} provide the existence of $\eta: \id_C \to R\circ
        L$ and $\epsilon: L\circ R \to \id_D$. Now to obtain the triangle identities consider:
        \[
          \begin{aligned}
            \id_{L(c)} &= \bar{\eta_c}\\
              &= \bar{R(\id_{L(c)})\circ R(\id_{L(c)})\circ \eta_c}\\
              &= \id_{L(c)} \circ \epsilon_{L(c)}\circ L(\eta_c)\\
              &= \epsilon_{L(c)}\circ L(\eta_c)
          \end{aligned}
          \qquad
          \begin{aligned}
            \id_{R(d)} &= \bar{\epsilon_d}\\
              &= \bar{\epsilon_d
                \circ L(\id_{R(d)})\circ L(\id_{R(d)})}\\
              &= R(\epsilon_d) \circ \eta_{R(d)} \circ \id_{R(d)}\\
              &= R(\epsilon_d) \circ \eta_{R(d)}
          \end{aligned}
        \]
      \item[($\impliedby$)] For categories $C,D$ with left/right adjoints
        $L:C\to D$, $R:D\to C$, consider two functions:
        \[
          \begin{aligned}
            \phi: \hom_D(L(c), d) &\to \hom_C(c, R(d))\\
            u &\mapsto R(u)\circ \eta_c\\
            \psi: \hom_C(c, R(d)) &\to \hom_D(L(c), d)\\
            v &\mapsto \epsilon_d \circ L(v)
          \end{aligned}
        \]
        Given the triangle identities, and using the naturality condition of $\eta$ and $\epsilon$, the following diagrams commute:
        % https://q.uiver.app/?q=WzAsNSxbMCwwLCJMKHgpIl0sWzAsMSwiKExcXGNpcmMgUlxcY2lyYyBMKSh4KSJdLFsxLDEsIkwoeCkiXSxbMCwyLCIoTFxcY2lyYyBSKSh5KSJdLFsxLDIsInkiXSxbMCwyLCJcXGlkX3tMKHgpfSJdLFswLDEsIkwoXFxldGFfeCkiLDJdLFsxLDIsIlxcdmFyZXBzaWxvbl97TCh4KX0iXSxbMiw0LCJ1Il0sWzEsMywiKExcXGNpcmMgUikodSkiLDJdLFszLDQsIlxcdmFyZXBzaWxvbl95IiwyXV0=&macro_url=https%3A%2F%2Fraw.githubusercontent.com%2Faortega0703%2Fnotes-category-theory%2Fmain%2Fsrc%2Fmacros.tex
        \[\begin{tikzcd}[ampersand replacement=\&]
          {L(x)} \\
          {(L\circ R\circ L)(x)} \& {L(x)} \\
          {(L\circ R)(y)} \& y
          \arrow["{\id_{L(x)}}", from=1-1, to=2-2]
          \arrow["{L(\eta_x)}"', from=1-1, to=2-1]
          \arrow["{\epsilon_{L(x)}}", from=2-1, to=2-2]
          \arrow["u", from=2-2, to=3-2]
          \arrow["{(L\circ R)(u)}"', from=2-1, to=3-1]
          \arrow["{\epsilon_y}"', from=3-1, to=3-2]
        \end{tikzcd}
        \qquad
        % https://q.uiver.app/?q=WzAsNSxbMCwxLCJSKHkpIl0sWzEsMSwiKFJcXGNpcmMgTFxcY2lyYyBSKSh5KSJdLFsxLDIsIlIoeSkiXSxbMSwwLCIoTFxcY2lyYyBSKSh4KSJdLFswLDAsIngiXSxbMCwyLCJcXGlkX3tSKHgpfSIsMl0sWzAsMSwiXFxldGFfe1IoeSl9Il0sWzEsMiwiUihcXHZhcmVwc2lsb25feSkiXSxbMywxLCIoTFxcY2lyYyBSKSh2KSJdLFs0LDAsInYiLDJdLFs0LDMsIlxcZXRhX3giXV0=&macro_url=https%3A%2F%2Fraw.githubusercontent.com%2Faortega0703%2Fnotes-category-theory%2Fmain%2Fsrc%2Fmacros.tex
        \begin{tikzcd}[ampersand replacement=\&]
          x \& {(L\circ R)(x)} \\
          {R(y)} \& {(R\circ L\circ R)(y)} \\
          \& {R(y)}
          \arrow["{\id_{R(x)}}"', from=2-1, to=3-2]
          \arrow["{\eta_{R(y)}}", from=2-1, to=2-2]
          \arrow["{R(\epsilon_y)}", from=2-2, to=3-2]
          \arrow["{(L\circ R)(v)}", from=1-2, to=2-2]
          \arrow["v"', from=1-1, to=2-1]
          \arrow["{\eta_x}", from=1-1, to=1-2]
        \end{tikzcd}\]

        To show that $\phi$ and $\psi$ are an isomorphism consider their
        compositions, both of which are equal to the identity taking the
        external paths of the respective diagrams above:
        \[
          \begin{aligned}
            (\psi\circ\phi)(u) &=
              \epsilon \circ (L\circ R(u)) \circ L(\eta_x) = u\\
            (\phi\circ\psi)(v) &=
              R(\epsilon_y)\circ (R\circ L(v)) \circ \eta_x = v\\
          \end{aligned}
        \]
        In order to show naturality it is sufficient to show that for morphisms
        $g:c'\to c$ and $h:d\to d'$ Theorem \ref{thm:transpose2} holds:
        \[
          \begin{aligned}
            \phi(h\circ u)
              &= R(h) \circ R(u) \circ \eta_c\\
              &= R(h) \phi(u)
          \end{aligned}
          \qquad
          \begin{aligned}
            \psi(v\circ g)
              &= \epsilon_d \circ L(v) \circ L(g)\\
              &= \psi(v) \circ L(g)
          \end{aligned}
        \]
    \end{description}
  \end{proof}
\end{theorem}