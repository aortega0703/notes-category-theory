\section{Adjoint functors}

\begin{definition}[Adjoint Functors\index{Adjoint Functors}\label{def:adjoint_hom}]
  For categories $C, D$ and functors $L: C\to
  D$ and $R: D\to C$, the pair ($L$, $R$) is said to be a left/right adjoint of
  eachother when there is a natural isomorphism between hom-functors given by:
  \parencite{leinster:basic_category_theory}
  \[
    \begin{aligned}
      \phi_{(c', d)}: \Hom_D(L(c), d) &\to \Hom_C(c, R(d))\\
      u &\mapsto \overline{u}\\
      \psi_{(c', d)}: \Hom_C(c, R(d)) &\to \Hom_D(L(c), d)\\
      v &\mapsto \overline{v}
    \end{aligned}
  \]
  Such that $\overline{\overline{u}} = u$ and $\overline{\overline{v}} = v$.

  \begin{itemize}
    \item $(c', d)$-Components:
      \[
        \alpha_{(c', d)}
          = \Hom_D(L(c), d) \overset{\cong}{\to} \Hom_C(c, R(d))\\
      \]
    \item Naturality Condition:\\
      For every $f = (g, h): (c', d) \to (c, d')$, the following commutes:
      % https://q.uiver.app/?q=WzAsNCxbMCwwLCJcXEhvbV9EKEwoYyksIGQpIl0sWzEsMCwiXFxIb21fQyhjLCBSKGQpKSJdLFswLDEsIlxcSG9tX0QoTChjJyksZCcpIl0sWzEsMSwiXFxIb21fQyhjJyxSKGQnKSkiXSxbMCwyLCJcXEhvbV9EKEwoZyksIGgpIiwyXSxbMSwzLCJcXEhvbV9DKGcsIFIoaCkpIl0sWzAsMSwiXFxwaGlfeyhjLCBkKX0iLDAseyJvZmZzZXQiOi0xfV0sWzIsMywiXFxwaGlfeyhjJyxkJyl9IiwyLHsib2Zmc2V0IjoxfV0sWzMsMiwiXFxwc2lfeyhjJyxkJyl9IiwyLHsib2Zmc2V0IjoxfV0sWzEsMCwiXFxwc2lfeyhjLGQpfSIsMCx7Im9mZnNldCI6LTF9XV0=&macro_url=https%3A%2F%2Fraw.githubusercontent.com%2Faortega0703%2Fnotes-category-theory%2Fmain%2Fsrc%2Fmacros.tex
      \[\begin{tikzcd}[ampersand replacement=\&]
        {\Hom_D(L(c), d)} \& {\Hom_C(c, R(d))} \\
        {\Hom_D(L(c'),d')} \& {\Hom_C(c',R(d'))}
        \arrow["{\Hom_D(L(g), h)}"', from=1-1, to=2-1]
        \arrow["{\Hom_C(g, R(h))}", from=1-2, to=2-2]
        \arrow["{\phi_{(c, d)}}", shift left=1, from=1-1, to=1-2]
        \arrow["{\phi_{(c',d')}}"', shift right=1, from=2-1, to=2-2]
        \arrow["{\psi_{(c',d')}}"', shift right=1, from=2-2, to=2-1]
        \arrow["{\psi_{(c,d)}}", shift left=1, from=1-2, to=1-1]
      \end{tikzcd}\]

      Equating both $\phi$ paths:
      \[
        \begin{aligned}
          \big(\phi_{(c', d')} \circ \Hom_D(L(g), h)\big)(u)
            &= \big(\Hom_C(g, R(h)) \circ \phi_{(c, d)}\big)(u)\\
          \phi_{(c', d')}(h\circ u \circ L(g))
            &= \Hom_C(g, R(h))(\overline{u})\\
          \overline{h\circ u \circ L(g)}
            &= R(h) \circ \overline{u} \circ g
        \end{aligned}
      \]
      Equating both $\psi$ paths:
      \[
        \begin{aligned}
          \big(\psi_{(c', d')} \circ \Hom_C(g, R(h))\big)(v)
            &= \big(\Hom_D(L(g), h) \circ \psi_{(c, d)}\big)(v)\\
          \psi_{(c', d')}(R(h)\circ v \circ g)
            &= \Hom_D(L(g), h)(\overline{v})\\
          \overline{R(h)\circ v \circ g}
            &= h \circ \overline{v} \circ L(g)
        \end{aligned}
      \]
  \end{itemize}
\end{definition}

\begin{remark}
  In diagram a pair of left/right adjoints $(L, R)$ is indicated by a double
  arrow ($\Rightarrow$) going in the same direction as the left adjoint, or as a
  turnstile ($\vdash$) pointing towards the left adjoint as follows:
  % https://q.uiver.app/?q=WzAsMixbMCwwLCJEIl0sWzEsMCwiQyJdLFsxLDAsIkwiLDIseyJjdXJ2ZSI6Mn1dLFswLDEsIlIiLDIseyJjdXJ2ZSI6Mn1dLFsxLDAsIiIsMSx7InNob3J0ZW4iOnsic291cmNlIjoyMCwidGFyZ2V0IjoyMH0sImxldmVsIjoyfV1d
  \[\begin{tikzcd}
    D & C
    \arrow["L"', curve={height=18pt}, from=1-2, to=1-1]
    \arrow["R"', curve={height=18pt}, from=1-1, to=1-2]
    \arrow[shorten <=3pt, shorten >=3pt, Rightarrow, from=1-2, to=1-1]
  \end{tikzcd}
    \text{or}
  % https://q.uiver.app/?q=WzAsMixbMCwwLCJEIl0sWzEsMCwiQyJdLFsxLDAsIkwiLDIseyJjdXJ2ZSI6Mn1dLFswLDEsIlIiLDIseyJjdXJ2ZSI6Mn1dLFsyLDMsIiIsMix7ImxldmVsIjoxLCJzdHlsZSI6eyJuYW1lIjoiYWRqdW5jdGlvbiJ9fV1d
    \begin{tikzcd}
    D & C
    \arrow[""{name=0, anchor=center, inner sep=0}, "L"', curve={height=18pt}, from=1-2, to=1-1]
    \arrow[""{name=1, anchor=center, inner sep=0}, "R"', curve={height=18pt}, from=1-1, to=1-2]
    \arrow["\dashv"{anchor=center, rotate=-90}, draw=none, from=0, to=1]
  \end{tikzcd}\]
\end{remark}

\begin{remark}
  $\overline{f}$ is called the transpose of $f$.
\end{remark}

\begin{remark}
  Note the similarity of $\Hom_D(L(c'), d)\cong \Hom_C(c', R(d))$ to
  $f^{-1}(y)=x \iff y=f(x)$.
\end{remark}

\begin{definition}[Adjoint Functors\index{Adjoint Functors}]\label{def:adjoint_unit_counit}
  For categories $C,D$ and functors $L: C\to D$ and $R: D\to C$, $L$ is said to
  be a left adjoint of $R$ when there exists a unit ($\eta$) and counit
  ($\varepsilon$): \parencite{leinster:basic_category_theory}

  \[
    \begin{aligned}
      \eta&:\\
      \varepsilon&:
    \end{aligned}
    \ \begin{aligned}
      \id_C\ \ &\Rightarrow R \circ L\\
      L \circ R &\Rightarrow\ \ \id_D
    \end{aligned}
  \]
  Such that the triangular identities are satisfied (the following diagrams
  commute)

  % https://q.uiver.app/?q=WzAsMyxbMCwwLCJMKGMpIl0sWzAsMSwiKExcXGNpcmMgUlxcY2lyYyBMKShjKSJdLFsxLDEsIkwoYykiXSxbMCwxLCJMKFxcZXRhX2MpIiwyXSxbMSwyLCJcXHZhcmVwc2lsb25fe0woYyl9IiwyXSxbMCwyLCJcXGlkX3tMKGMpfSJdXQ==&macro_url=https%3A%2F%2Fraw.githubusercontent.com%2Faortega0703%2Fnotes-category-theory%2Fmain%2Fsrc%2Fmacros.tex
  \[\begin{tikzcd}[ampersand replacement=\&]
    {L(c)} \\
    {(L\circ R\circ L)(c)} \& {L(c)}
    \arrow["{L(\eta_c)}"', from=1-1, to=2-1]
    \arrow["{\varepsilon_{L(c)}}"', from=2-1, to=2-2]
    \arrow["{\id_{L(c)}}", from=1-1, to=2-2]
  \end{tikzcd}
  \quad
  % https://q.uiver.app/?q=WzAsMyxbMCwwLCJSKGQpIl0sWzEsMCwiKFJcXGNpcmMgTFxcY2lyYyBSKShkKSJdLFsxLDEsIlIoZCkiXSxbMCwxLCJcXGV0YV97UihkKX0iXSxbMSwyLCJSKFxcdmFyZXBzaWxvbl9kKSJdLFswLDIsIlxcaWRfe1IoZCl9IiwyXV0=&macro_url=https%3A%2F%2Fraw.githubusercontent.com%2Faortega0703%2Fnotes-category-theory%2Fmain%2Fsrc%2Fmacros.tex
  \begin{tikzcd}[ampersand replacement=\&]
    {R(d)} \& {(R\circ L\circ R)(d)} \\
    \& {R(d)}
    \arrow["{\eta_{R(d)}}", from=1-1, to=1-2]
    \arrow["{R(\varepsilon_d)}", from=1-2, to=2-2]
    \arrow["{\id_{R(d)}}"', from=1-1, to=2-2]
  \end{tikzcd}\]
\end{definition}

\begin{theorem}[Adjoint Functors Definitions]
  Definitions \ref{def:adjoint_hom} and \ref{def:adjoint_unit_counit} of adjoint
  functors are equivalent.

  \begin{proof}
    The proof consists of two parts:
    \begin{description}
      \item[($\implies$)] For categories $C,D$ with left/right adjoints $L:C\to
        D$, $R:D\to C$, consider:
        \[
          \begin{aligned}
            \Hom_D(L(c), d) &\ \:\cong\ \: \Hom_C(c, R(d))\\
            \Hom_D(L(c), L(c)) &\ \:\cong\ \: \Hom_C(c, (R\circ L)(c))\\
            \big(\id_{L(c)}: L(c) \to L(c)\big) \in D
            &\implies \big(\eta_c : c \to (R\circ L)(c)\big)\in C\\
            \overline{\id_{L(c)}} &\ \:=\ \:\eta_c
          \end{aligned}
        \]

        Obtaining the existence of the unit $\eta:\id_D\Rightarrow (R\circ L)$.
        Similarly, the existence of the counit $\varepsilon:(L\circ
        R)\Rightarrow\id_C$ can be obtained by:
        \[
          \begin{aligned}
            \Hom_D(L(c), d) &\ \ \cong\ \ \Hom_C(c, R(d))\\
            \Hom_D((L\circ R)(d), d) &\ \ \cong\ \ \Hom_C(R(d), R(d))\\
            \big(\varepsilon_{d} : (L\circ R)(d) \to d) \in D
            &\impliedby \big(\id_{R(d)}: R(d) \to R(d)\big) \in C\\
            \overline{\id_{R(d)}} &= \varepsilon_d
          \end{aligned}
        \]

        To obtain the triangle identities consider:
        \[
          \begin{aligned}
            \id_{L(c)} &= \overline{\eta_c}\\
              &= \overline{R(\id_{L(c)})\circ R(\id_{L(c)})\circ \eta_c}\\
              &= \id_{L(c)} \circ \varepsilon_{L(c)}\circ L(\eta_c)\\
              &= \varepsilon_{L(c)}\circ L(\eta_c)
          \end{aligned}
          \qquad
          \begin{aligned}
            \id_{R(d)} &= \overline{\varepsilon_d}\\
              &= \overline{\varepsilon_d
                \circ L(\id_{R(d)})\circ L(\id_{R(d)})}\\
              &= R(\varepsilon_d) \circ \eta_{R(d)} \circ \id_{R(d)}\\
              &= R(\varepsilon_d) \circ \eta_{R(d)}
          \end{aligned}
        \]
      \item[($\impliedby$)] For categories $C,D$ with left/right adjoints
        $L:C\to D$, $R:D\to C$, consider two functions:
        \[
          \begin{aligned}
            \phi: \hom_D(L(x), y) &\to \hom_C(x, R(y))\\
            u &\mapsto R(u)\circ \eta_x\\
            \psi: \hom_C(x, R(y)) &\to \hom_D(L(x), y)\\
            v &\mapsto \varepsilon_y \circ L(v)
          \end{aligned}
        \]
        Given the triangle identities, and using the naturality condition of $\eta$ and $\varepsilon$, the following diagrams commute:
        % https://q.uiver.app/?q=WzAsNSxbMCwwLCJMKHgpIl0sWzAsMSwiKExcXGNpcmMgUlxcY2lyYyBMKSh4KSJdLFsxLDEsIkwoeCkiXSxbMCwyLCIoTFxcY2lyYyBSKSh5KSJdLFsxLDIsInkiXSxbMCwyLCJcXGlkX3tMKHgpfSJdLFswLDEsIkwoXFxldGFfeCkiLDJdLFsxLDIsIlxcdmFyZXBzaWxvbl97TCh4KX0iXSxbMiw0LCJ1Il0sWzEsMywiKExcXGNpcmMgUikodSkiLDJdLFszLDQsIlxcdmFyZXBzaWxvbl95IiwyXV0=&macro_url=https%3A%2F%2Fraw.githubusercontent.com%2Faortega0703%2Fnotes-category-theory%2Fmain%2Fsrc%2Fmacros.tex
        \[\begin{tikzcd}[ampersand replacement=\&]
          {L(x)} \\
          {(L\circ R\circ L)(x)} \& {L(x)} \\
          {(L\circ R)(y)} \& y
          \arrow["{\id_{L(x)}}", from=1-1, to=2-2]
          \arrow["{L(\eta_x)}"', from=1-1, to=2-1]
          \arrow["{\varepsilon_{L(x)}}", from=2-1, to=2-2]
          \arrow["u", from=2-2, to=3-2]
          \arrow["{(L\circ R)(u)}"', from=2-1, to=3-1]
          \arrow["{\varepsilon_y}"', from=3-1, to=3-2]
        \end{tikzcd}
        \qquad
        % https://q.uiver.app/?q=WzAsNSxbMCwxLCJSKHkpIl0sWzEsMSwiKFJcXGNpcmMgTFxcY2lyYyBSKSh5KSJdLFsxLDIsIlIoeSkiXSxbMSwwLCIoTFxcY2lyYyBSKSh4KSJdLFswLDAsIngiXSxbMCwyLCJcXGlkX3tSKHgpfSIsMl0sWzAsMSwiXFxldGFfe1IoeSl9Il0sWzEsMiwiUihcXHZhcmVwc2lsb25feSkiXSxbMywxLCIoTFxcY2lyYyBSKSh2KSJdLFs0LDAsInYiLDJdLFs0LDMsIlxcZXRhX3giXV0=&macro_url=https%3A%2F%2Fraw.githubusercontent.com%2Faortega0703%2Fnotes-category-theory%2Fmain%2Fsrc%2Fmacros.tex
        \begin{tikzcd}[ampersand replacement=\&]
          x \& {(L\circ R)(x)} \\
          {R(y)} \& {(R\circ L\circ R)(y)} \\
          \& {R(y)}
          \arrow["{\id_{R(x)}}"', from=2-1, to=3-2]
          \arrow["{\eta_{R(y)}}", from=2-1, to=2-2]
          \arrow["{R(\varepsilon_y)}", from=2-2, to=3-2]
          \arrow["{(L\circ R)(v)}", from=1-2, to=2-2]
          \arrow["v"', from=1-1, to=2-1]
          \arrow["{\eta_x}", from=1-1, to=1-2]
        \end{tikzcd}\]

        To show that $\phi$ and $\psi$ are an isomorphism consider their
        compositions, both of which are the identity by use of the respective
        diagram above:
        \[
          \begin{aligned}
            (\psi\circ\phi)(u) &=
              \varepsilon \circ (L\circ R(u)) \circ L(\eta_x) = u\\
            (\phi\circ\psi)(v) &=
              R(\varepsilon_y)\circ (R\circ L(v)) \circ \eta_x = v\\
          \end{aligned}
        \]
        \todo{Show naturality}
    \end{description}
  \end{proof}
\end{theorem}