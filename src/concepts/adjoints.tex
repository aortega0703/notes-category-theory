\section{Adjoint functors}

\begin{definition}[Adjoint Functors\index{Adjoint Functors}]\label{def:adjoint_hom}
  For categories $C, D$ and functors $L: C\to D$ and $R: D\to C$, the pair ($L$,
  $R$) is said to be a left/right adjoint of eachother when there is a natural
  isomorphism between functors: \parencite{fong_spivak:7sketches}
  \[
    \begin{aligned}
      F&: C^\op \times D \to \mathrm{Set}\\
      F&=\Hom_D(L(\underline{\ \ }), \underline{\ \ })
    \end{aligned}
    \qquad
    \begin{aligned}
      G&: C^\op \times D \to \mathrm{Set}\\
      G&=\Hom_C(\underline{\ \ }, R(\underline{\ \ }))
    \end{aligned}
  \]

  % https://q.uiver.app/?q=WzAsNCxbMSwwLCJkIl0sWzAsMCwiTChkKSJdLFswLDEsImMiXSxbMSwxLCJSKGMpIl0sWzAsM10sWzEsMl0sWzAsMSwiIiwxLHsiY29sb3VyIjpbMjQwLDYwLDYwXSwic3R5bGUiOnsiYm9keSI6eyJuYW1lIjoiZG90dGVkIn19fV0sWzIsMywiIiwxLHsiY29sb3VyIjpbMCw2MCw2MF0sInN0eWxlIjp7ImJvZHkiOnsibmFtZSI6ImRvdHRlZCJ9fX1dXQ==
  \[\begin{tikzcd}
    {L(d)} & d \\
    c & {R(c)}
    \arrow[from=1-2, to=2-2]
    \arrow[from=1-1, to=2-1]
    \arrow[draw={rgb,255:red,92;green,92;blue,214}, dotted, from=1-2, to=1-1]
    \arrow[draw={rgb,255:red,214;green,92;blue,92}, dotted, from=2-1, to=2-2]
  \end{tikzcd}\]

  \begin{itemize}
    \item $(c', d)$-Components:\\
      $\alpha_{(c', d)} \coloneqq \Hom_D(L(c'), d)
      \overset{\cong}{\to} \Hom_C(c', R(d))$\\
      Note the similarity to $f^{-1}(y)=x \iff y=f(x)$
    \item Naturality Condition:\\
      $\forall (f = (g^{op}, h): (c', d) \to (c, d')),$
      the following commutes
      % https://q.uiver.app/?q=WzAsNCxbMCwwLCJIb21fRChMKGMnKSwgZCkiXSxbMCwxLCJIb21fQyhjJywgUihkKSkiXSxbMiwwLCJIb21fRChMKGMpLGQnKSJdLFsyLDEsIkhvbV9DKGMsUihkJykpIl0sWzAsMSwiXFxhbHBoYV97YycsZH0iLDJdLFsyLDMsIlxcYWxwaGFfe2MsZCd9IiwyXSxbMCwyLCJIb21fRChMKGhee29wfSksIGcpIl0sWzEsMywiSG9tX0MoaF57b3B9LCBSKGcpKSIsMl1d
      \[\begin{tikzcd}
        {\Hom_D(L(c'), d)} && {\Hom_D(L(c),d')} \\
        {\Hom_C(c', R(d))} && {\Hom_C(c,R(d'))}
        \arrow["{\alpha_{c',d}}"', from=1-1, to=2-1]
        \arrow["{\alpha_{c,d'}}"', from=1-3, to=2-3]
        \arrow["{\Hom_D(L(h^{op}), g)}", from=1-1, to=1-3]
        \arrow["{\Hom_C(h^{op}, R(g))}"', from=2-1, to=2-3]
      \end{tikzcd}\]
  \end{itemize}
\end{definition}

\begin{remark}
  In diagram a pair of left/right adjoints $(L, R)$ is indicated by a double
  arrow ($\Rightarrow$) going in the same direction as the left adjoint, or as a
  turnstile ($\vdash$) pointing towards the left adjoint as follows:
  % https://q.uiver.app/?q=WzAsMixbMCwwLCJEIl0sWzEsMCwiQyJdLFsxLDAsIkwiLDIseyJjdXJ2ZSI6Mn1dLFswLDEsIlIiLDIseyJjdXJ2ZSI6Mn1dLFsxLDAsIiIsMSx7InNob3J0ZW4iOnsic291cmNlIjoyMCwidGFyZ2V0IjoyMH0sImxldmVsIjoyfV1d
  \[\begin{tikzcd}
    D & C
    \arrow["L"', curve={height=18pt}, from=1-2, to=1-1]
    \arrow["R"', curve={height=18pt}, from=1-1, to=1-2]
    \arrow[shorten <=3pt, shorten >=3pt, Rightarrow, from=1-2, to=1-1]
  \end{tikzcd}
    \text{or}
  % https://q.uiver.app/?q=WzAsMixbMCwwLCJEIl0sWzEsMCwiQyJdLFsxLDAsIkwiLDIseyJjdXJ2ZSI6Mn1dLFswLDEsIlIiLDIseyJjdXJ2ZSI6Mn1dLFsyLDMsIiIsMix7ImxldmVsIjoxLCJzdHlsZSI6eyJuYW1lIjoiYWRqdW5jdGlvbiJ9fV1d
    \begin{tikzcd}
    D & C
    \arrow[""{name=0, anchor=center, inner sep=0}, "L"', curve={height=18pt}, from=1-2, to=1-1]
    \arrow[""{name=1, anchor=center, inner sep=0}, "R"', curve={height=18pt}, from=1-1, to=1-2]
    \arrow["\dashv"{anchor=center, rotate=-90}, draw=none, from=0, to=1]
  \end{tikzcd}\]
\end{remark}

\begin{definition}[Adjoint Functors\index{Adjoint Functors}]\label{def:adjoint_unit_counit}
  For categories $C,D$ and functors $L: C\to D$ and $R: D\to C$, $L$ is said to
  be a left adjoint of $R$ when there exists a unit ($\eta$) and counit
  ($\varepsilon$): \parencite{leinster:basic_category_theory}

  \[
    \begin{aligned}
      \eta&:\\
      \varepsilon&:
    \end{aligned}
    \ \begin{aligned}
      \id_C\ \ &\Rightarrow R \circ L\\
      L \circ R &\Rightarrow\ \ \id_D
    \end{aligned}
  \]
  Such that the triangular identities are satisfied (the following diagrams
  commute)

  % https://q.uiver.app/?q=WzAsMyxbMCwwLCJMIl0sWzAsMSwiTFxcY2lyYyBSXFxjaXJjIEwiXSxbMSwxLCJMIl0sWzAsMSwiXFxtYXRocm17aWR9X0xcXGNpcmNcXGV0YSIsMl0sWzEsMiwiXFx2YXJlcHNpbG9uXFxjaXJjIFxcbWF0aHJte2lkfV9MIiwyXSxbMCwyLCJcXG1hdGhybXtpZH1fTCJdXQ==
  \[\begin{tikzcd} L \\
    {L\circ R\circ L} & L
    \arrow["{\id_L\circ\eta}"', from=1-1, to=2-1]
    \arrow["{\varepsilon\circ \id_L}"', from=2-1, to=2-2]
    \arrow["{\id_L}", from=1-1, to=2-2]
  \end{tikzcd}
  % https://q.uiver.app/?q=WzAsMyxbMCwwLCJSIl0sWzEsMCwiUlxcY2lyYyBMXFxjaXJjIFIiXSxbMSwxLCJSIl0sWzAsMSwiXFxldGFcXGNpcmNcXG1hdGhybXtpZH1fUiJdLFsxLDIsIlxcbWF0aHJte2lkfV9SIFxcY2lyYyBcXHZhcmVwc2lsb24iXSxbMCwyLCJcXG1hdGhybXtpZH1fUiIsMl1d
  \begin{tikzcd}
    R & {R\circ L\circ R} \\
    & R
    \arrow["{\eta\circ\id_R}", from=1-1, to=1-2]
    \arrow["{\id_R \circ \varepsilon}", from=1-2, to=2-2]
    \arrow["{\id_R}"', from=1-1, to=2-2]
  \end{tikzcd}\]
\end{definition}

\begin{theorem}[Adjoint Functors Definitions]
  Definitions \ref{def:adjoint_hom} and \ref{def:adjoint_unit_counit} of adjoint
  functors are equivalent.

  \begin{proof}
    The proof consists of two parts:
    \begin{description}
      \item[($\implies$)] For categories $C,D$ with left/right adjoints $L:C\to
        D$, $R:D\to C$, consider:
        \[
          \begin{aligned}
            \Hom_D(L(x), y) &\ \:\cong\ \: \Hom_C(x, R(y))\\
            \Hom_D(L(x), L(x)) &\ \:\cong\ \: \Hom_C(x, (R\circ L)(x))\\
            \big(\id_{L(x)}: L(x) \to L(x)\big) \in D
            &\implies \big(\eta_x : x \to (R\circ L)(x)\big)\in C
          \end{aligned}
        \]
        Obtaining the existence of the unit $\eta:\id_D\Rightarrow (R\circ L)$.
        Similarly, the existence of the counit $\varepsilon:(L\circ
        R)\Rightarrow\id_C$ can be obtained by:
        \[
          \begin{aligned}
            \Hom_D(L(x), y) &\ \ \cong\ \ \Hom_C(x, R(y))\\
            \Hom_D((L\circ R)(y), y) &\ \ \cong\ \ \Hom_C(R(y), R(y))\\
            \big(\varepsilon_{y} : (L\circ R)(y) \to y) \in D
            &\impliedby \big(\id_{R(y)}: R(y) \to R(y)\big) \in C
          \end{aligned}
        \]
        \todo{Triangle identities}
      \item[($\impliedby$)] For categories $C,D$ with left/right adjoints
        $L:C\to D$, $R:D\to C$, consider two functions:
        \[
          \begin{aligned}
            f: \hom_D(L(x), y) &\to \hom_C(x, R(y))\\
            u &\mapsto R(u)\circ \eta_x\\
            f^{-1} : \hom_C(x, R(y)) &\to \hom_D(L(x), y)\\
            v &\mapsto \varepsilon_y \circ L(v)
          \end{aligned}
        \]
        \todo{Show isomorphism}
    \end{description}
  \end{proof}
\end{theorem}
\subsection{Notable examples}
\begin{itemize}
  \item Coproduct $\dashv$ Diagonal $\dashv$ Product $\dashv$ Exponential
    \parencite{lane_moerdijk:sheaves_geometry_logic}
    % https://q.uiver.app/?q=WzAsMixbMCwwLCJDIl0sWzAsMiwiQ1xcdGltZXMgQyJdLFsxLDAsIisiLDEseyJjdXJ2ZSI6LTV9XSxbMCwxLCJcXERlbHRhIiwxLHsiY3VydmUiOjJ9XSxbMSwwLCJcXHRpbWVzIiwxLHsiY3VydmUiOjJ9XSxbMCwxLCJcXHJpZ2h0YXJyb3ciLDEseyJjdXJ2ZSI6LTV9XSxbMyw0LCIiLDEseyJsZXZlbCI6MSwic3R5bGUiOnsibmFtZSI6ImFkanVuY3Rpb24ifX1dLFs0LDUsIiIsMSx7ImxldmVsIjoxLCJzdHlsZSI6eyJuYW1lIjoiYWRqdW5jdGlvbiJ9fV0sWzIsMywiIiwxLHsibGV2ZWwiOjEsInN0eWxlIjp7Im5hbWUiOiJhZGp1bmN0aW9uIn19XV0=
    \[\begin{tikzcd}
      C \\
      \\
      {C\times C}
      \arrow[""{name=0, anchor=center, inner sep=0}, "{+}"{description}, curve={height=-30pt}, from=3-1, to=1-1]
      \arrow[""{name=1, anchor=center, inner sep=0}, "\Delta"{description}, curve={height=12pt}, from=1-1, to=3-1]
      \arrow[""{name=2, anchor=center, inner sep=0}, "\times"{description}, curve={height=12pt}, from=3-1, to=1-1]
      \arrow[""{name=3, anchor=center, inner sep=0}, "\rightarrow"{description}, curve={height=-30pt}, from=1-1, to=3-1]
      \arrow["\dashv"{anchor=center}, draw=none, from=1, to=2]
      \arrow["\dashv"{anchor=center, rotate=4}, draw=none, from=2, to=3]
      \arrow["\dashv"{anchor=center, rotate=-4}, draw=none, from=0, to=1]
    \end{tikzcd}\]
\end{itemize}