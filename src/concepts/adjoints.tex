\section{Adjoint functors}
In diagram a pair of left/right adjoints $(L, R)$ is written as:
% https://q.uiver.app/?q=WzAsMixbMCwwLCJEIl0sWzEsMCwiQyJdLFsxLDAsIkwiLDIseyJjdXJ2ZSI6Mn1dLFswLDEsIlIiLDIseyJjdXJ2ZSI6Mn1dLFsxLDAsIiIsMSx7InNob3J0ZW4iOnsic291cmNlIjoyMCwidGFyZ2V0IjoyMH0sImxldmVsIjoyfV1d
\[\begin{tikzcd}
	D & C
	\arrow["L"', curve={height=18pt}, from=1-2, to=1-1]
	\arrow["R"', curve={height=18pt}, from=1-1, to=1-2]
	\arrow[shorten <=3pt, shorten >=3pt, Rightarrow, from=1-2, to=1-1]
\end{tikzcd}
  \text{or}
% https://q.uiver.app/?q=WzAsMixbMCwwLCJEIl0sWzEsMCwiQyJdLFsxLDAsIkwiLDIseyJjdXJ2ZSI6Mn1dLFswLDEsIlIiLDIseyJjdXJ2ZSI6Mn1dLFsyLDMsIiIsMix7ImxldmVsIjoxLCJzdHlsZSI6eyJuYW1lIjoiYWRqdW5jdGlvbiJ9fV1d
  \begin{tikzcd}
	D & C
	\arrow[""{name=0, anchor=center, inner sep=0}, "L"', curve={height=18pt}, from=1-2, to=1-1]
	\arrow[""{name=1, anchor=center, inner sep=0}, "R"', curve={height=18pt}, from=1-1, to=1-2]
	\arrow["\dashv"{anchor=center, rotate=-90}, draw=none, from=0, to=1]
\end{tikzcd}\]
Where the double arrow points in the same direction as the left adjoint or the
turnstile ($\vdash$) points towards the left adjoint.

\subsection{Definition}
\subsubsection*{As a natural isomorphism of hom-sets}
For categories $C, D$ and functors $L: C\to D$ and $R: D\to C$, $L$ is said to
be a left adjoint of $R$ when there is a natural isomorphism between functors:
\parencite{fong_spivak:7sketches}
\begin{equation*}
  \begin{aligned}
    F&: C^\mathrm{op} \times D \to \mathrm{Set}\\
    F&=\mathrm{Hom}_D(L(\underline{\ \ }), \underline{\ \ })
  \end{aligned}
  \qquad
  \begin{aligned}
    G&: C^\mathrm{op} \times D \to \mathrm{Set}\\
    G&=\mathrm{Hom}_C(\underline{\ \ }, R(\underline{\ \ }))
  \end{aligned}
\end{equation*}

% https://q.uiver.app/?q=WzAsNixbMSwxLCJkIl0sWzAsMSwiTChkKSJdLFswLDIsImMiXSxbMSwyLCJSKGMpIl0sWzAsMCwiQyJdLFsxLDAsIkQiXSxbMCwzXSxbMSwyXSxbMCwxLCIiLDEseyJjb2xvdXIiOlsyNDAsNjAsNjBdLCJzdHlsZSI6eyJib2R5Ijp7Im5hbWUiOiJkb3R0ZWQifX19XSxbMiwzLCIiLDEseyJjb2xvdXIiOlswLDYwLDYwXSwic3R5bGUiOnsiYm9keSI6eyJuYW1lIjoiZG90dGVkIn19fV0sWzUsNCwiTCIsMSx7ImN1cnZlIjozLCJjb2xvdXIiOlsyNDAsNjAsNjBdfV0sWzQsNSwiUiAiLDEseyJjdXJ2ZSI6MywiY29sb3VyIjpbMCw2MCw2MF19XSxbMTAsMTEsIiIsMSx7ImxldmVsIjoxLCJzdHlsZSI6eyJuYW1lIjoiYWRqdW5jdGlvbiJ9fV1d
\[\begin{tikzcd}
	C & D \\
	{L(d)} & d \\
	c & {R(c)}
	\arrow[from=2-2, to=3-2]
	\arrow[from=2-1, to=3-1]
	\arrow[color={rgb,255:red,92;green,92;blue,214}, dotted, from=2-2, to=2-1]
	\arrow[color={rgb,255:red,214;green,92;blue,92}, dotted, from=3-1, to=3-2]
	\arrow[""{name=0, anchor=center, inner sep=0}, "L"{description}, draw={rgb,255:red,92;green,92;blue,214}, curve={height=18pt}, from=1-2, to=1-1]
	\arrow[""{name=1, anchor=center, inner sep=0}, "{R }"{description}, draw={rgb,255:red,214;green,92;blue,92}, curve={height=18pt}, from=1-1, to=1-2]
	\arrow["\dashv"{anchor=center, rotate=-90}, draw=none, from=0, to=1]
\end{tikzcd}\]

\begin{itemize}
  \item $\alpha_{(c', d)} \coloneqq \mathrm{Hom}_D(L(c'), d)
    \overset{\cong}{\to} \mathrm{Hom}_C(c', R(d))$\\
    Note the similarity to $f^{-1}(y)=x \iff y=f(x)$
  \item $\forall (f = (g^{op}, h): (c', d) \to (c, d')),$
    the following commutes
    % https://q.uiver.app/?q=WzAsNCxbMCwwLCJIb21fRChMKGMnKSwgZCkiXSxbMCwxLCJIb21fQyhjJywgUihkKSkiXSxbMiwwLCJIb21fRChMKGMpLGQnKSJdLFsyLDEsIkhvbV9DKGMsUihkJykpIl0sWzAsMSwiXFxhbHBoYV97YycsZH0iLDJdLFsyLDMsIlxcYWxwaGFfe2MsZCd9IiwyXSxbMCwyLCJIb21fRChMKGhee29wfSksIGcpIl0sWzEsMywiSG9tX0MoaF57b3B9LCBSKGcpKSIsMl1d
    \[\begin{tikzcd}
      {\mathrm{Hom}_D(L(c'), d)} && {\mathrm{Hom}_D(L(c),d')} \\
      {\mathrm{Hom}_C(c', R(d))} && {\mathrm{Hom}_C(c,R(d'))}
      \arrow["{\alpha_{c',d}}"', from=1-1, to=2-1]
      \arrow["{\alpha_{c,d'}}"', from=1-3, to=2-3]
      \arrow["{\mathrm{Hom}_D(L(h^{op}), g)}", from=1-1, to=1-3]
      \arrow["{\mathrm{Hom}_C(h^{op}, R(g))}"', from=2-1, to=2-3]
    \end{tikzcd}\]
\end{itemize}

\subsubsection*{As the existence of a unit and counit}
For categories $C,D$ and functors $L: C\to D$ and $R: D\to C$, $L$ is said to
be a left adjoint of $R$ when there exists a unit ($\eta$) and counit
($\varepsilon$)

\begin{equation*}
  \begin{aligned}
    \eta&:\\
    \varepsilon&:
  \end{aligned}
  \ \begin{aligned}
    \mathrm{id}_C\ \ &\Rightarrow R \circ L\\
    L \circ R &\Rightarrow\ \ \mathrm{id}_D
  \end{aligned}
\end{equation*}
Such that the triangular identities are satisfied (the following diagrams commute)
% https://q.uiver.app/?q=WzAsMyxbMCwwLCJMIl0sWzAsMSwiTFxcY2lyYyBSXFxjaXJjIEwiXSxbMSwxLCJMIl0sWzAsMSwiXFxtYXRocm17aWR9X0xcXGNpcmNcXGV0YSIsMl0sWzEsMiwiXFx2YXJlcHNpbG9uXFxjaXJjIFxcbWF0aHJte2lkfV9MIiwyXSxbMCwyLCJcXG1hdGhybXtpZH1fTCJdXQ==
\[\begin{tikzcd}
	L \\
	{L\circ R\circ L} & L
	\arrow["{\mathrm{id}_L\circ\eta}"', from=1-1, to=2-1]
	\arrow["{\varepsilon\circ \mathrm{id}_L}"', from=2-1, to=2-2]
	\arrow["{\mathrm{id}_L}", from=1-1, to=2-2]
\end{tikzcd}
% https://q.uiver.app/?q=WzAsMyxbMCwwLCJSIl0sWzEsMCwiUlxcY2lyYyBMXFxjaXJjIFIiXSxbMSwxLCJSIl0sWzAsMSwiXFxldGFcXGNpcmNcXG1hdGhybXtpZH1fUiJdLFsxLDIsIlxcbWF0aHJte2lkfV9SIFxcY2lyYyBcXHZhcmVwc2lsb24iXSxbMCwyLCJcXG1hdGhybXtpZH1fUiIsMl1d
\begin{tikzcd}
	R & {R\circ L\circ R} \\
	& R
	\arrow["{\eta\circ\mathrm{id}_R}", from=1-1, to=1-2]
	\arrow["{\mathrm{id}_R \circ \varepsilon}", from=1-2, to=2-2]
	\arrow["{\mathrm{id}_R}"', from=1-1, to=2-2]
\end{tikzcd}\]

\subsection{Notable examples}
\begin{itemize}
  \item Coproduct $\dashv$ Diagonal $\dashv$ Product $\dashv$ Exponential
    \parencite{maclane_moerdijk:sheaves_geometry_logic}
    % https://q.uiver.app/?q=WzAsMixbMCwwLCJDIl0sWzAsMiwiQ1xcdGltZXMgQyJdLFsxLDAsIisiLDEseyJjdXJ2ZSI6LTV9XSxbMCwxLCJcXERlbHRhIiwxLHsiY3VydmUiOjJ9XSxbMSwwLCJcXHRpbWVzIiwxLHsiY3VydmUiOjJ9XSxbMCwxLCJcXHJpZ2h0YXJyb3ciLDEseyJjdXJ2ZSI6LTV9XSxbMyw0LCIiLDEseyJsZXZlbCI6MSwic3R5bGUiOnsibmFtZSI6ImFkanVuY3Rpb24ifX1dLFs0LDUsIiIsMSx7ImxldmVsIjoxLCJzdHlsZSI6eyJuYW1lIjoiYWRqdW5jdGlvbiJ9fV0sWzIsMywiIiwxLHsibGV2ZWwiOjEsInN0eWxlIjp7Im5hbWUiOiJhZGp1bmN0aW9uIn19XV0=
    \[\begin{tikzcd}
      C \\
      \\
      {C\times C}
      \arrow[""{name=0, anchor=center, inner sep=0}, "{+}"{description}, curve={height=-30pt}, from=3-1, to=1-1]
      \arrow[""{name=1, anchor=center, inner sep=0}, "\Delta"{description}, curve={height=12pt}, from=1-1, to=3-1]
      \arrow[""{name=2, anchor=center, inner sep=0}, "\times"{description}, curve={height=12pt}, from=3-1, to=1-1]
      \arrow[""{name=3, anchor=center, inner sep=0}, "\rightarrow"{description}, curve={height=-30pt}, from=1-1, to=3-1]
      \arrow["\dashv"{anchor=center}, draw=none, from=1, to=2]
      \arrow["\dashv"{anchor=center, rotate=4}, draw=none, from=2, to=3]
      \arrow["\dashv"{anchor=center, rotate=-4}, draw=none, from=0, to=1]
    \end{tikzcd}\]
\end{itemize}