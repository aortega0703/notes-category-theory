\section{Limits and Colimits}

\begin{definition}[Limit\index{Limit}\label{def:limit}]
	The limit of a functor $F:C\to D$ is a cone with vertex $d\in D$ and morphisms
	$\big(\alpha_c: d \to F(c)\big)_{c\in C}$ such that for any other cone with
	vertex $d'\in D$ and morphisms $\big(\beta_c: d' \to F(c)\big)_{c\in C}$ there
	exists a unique map $f:d'\to d$ such that the following diagram commutes for
	all $c\in C$: \parencite{leinster:basic_category_theory}

	% https://q.uiver.app/?q=WzAsMyxbMCwxLCJkIl0sWzAsMCwiZCciXSxbMSwxLCJGKGMpIl0sWzAsMiwiXFxhbHBoYV9jIiwyXSxbMSwyLCJcXGJldGFfYyJdLFsxLDAsImYiLDIseyJzdHlsZSI6eyJib2R5Ijp7Im5hbWUiOiJkYXNoZWQifX19XV0=
	\[\begin{tikzcd}
		{d'} \\
		d & {F(c)}
		\arrow["{\alpha_c}"', from=2-1, to=2-2]
		\arrow["{\beta_c}", from=1-1, to=2-2]
		\arrow["f"', dashed, from=1-1, to=2-1]
	\end{tikzcd}\]
\end{definition}

\begin{remark}
	Note that Definition \ref{def:limit} is equivalent to saying that a limit of a
	functor $F:C\to D$ is a terminal object in the category cone($F$).
\end{remark}

\begin{remark}
	Note that Definition \ref{def:limit} can also be thought of as a terminal
	morphism from the diagonal functor $\Delta$ to the functor $F$ (thought of as
	an object in $D^C$) as shown in the following diagram:
	% https://q.uiver.app/?q=WzAsNSxbMCwwLCJjIl0sWzAsMSwiYyciXSxbMSwwLCJcXERlbHRhX2MiXSxbMSwxLCJcXERlbHRhX3tjJ30iXSxbMiwwLCJGIl0sWzEsMCwiaCIsMCx7InN0eWxlIjp7ImJvZHkiOnsibmFtZSI6ImRhc2hlZCJ9fX1dLFszLDIsIlxcRGVsdGEoaCkiLDAseyJzdHlsZSI6eyJib2R5Ijp7Im5hbWUiOiJkYXNoZWQifX19XSxbMyw0LCJcXGJldGEiLDJdLFsyLDQsIlxcYWxwaGEiXSxbMCwyLCIiLDAseyJjb2xvdXIiOlsyNDAsNjAsNjBdLCJzdHlsZSI6eyJib2R5Ijp7Im5hbWUiOiJkb3R0ZWQifX19XSxbMSwzLCIiLDAseyJjb2xvdXIiOlsyNDAsNjAsNjBdLCJzdHlsZSI6eyJib2R5Ijp7Im5hbWUiOiJkb3R0ZWQifX19XSxbNSw2LCIiLDAseyJzaG9ydGVuIjp7InNvdXJjZSI6MjAsInRhcmdldCI6NjB9LCJsZXZlbCI6MSwiY29sb3VyIjpbMjQwLDYwLDYwXSwic3R5bGUiOnsiYm9keSI6eyJuYW1lIjoiZG90dGVkIn19fV1d&macro_url=https%3A%2F%2Fraw.githubusercontent.com%2Faortega0703%2Fnotes-category-theory%2Fmain%2Fsrc%2Fmacros.tex
	\[\begin{tikzcd}[ampersand replacement=\&]
		c \& {\Delta_c} \& F \\
		{c'} \& {\Delta_{c'}}
		\arrow[""{name=0, anchor=center, inner sep=0}, "h", dashed, from=2-1, to=1-1]
		\arrow[""{name=1, anchor=center, inner sep=0}, "{\Delta(h)}", dashed, from=2-2, to=1-2]
		\arrow["\beta"', from=2-2, to=1-3]
		\arrow["\alpha", from=1-2, to=1-3]
		\arrow[color={rgb,255:red,92;green,92;blue,214}, dotted, from=1-1, to=1-2]
		\arrow[color={rgb,255:red,92;green,92;blue,214}, dotted, from=2-1, to=2-2]
		\arrow[color={rgb,255:red,92;green,92;blue,214}, shorten <=6pt, shorten >=19pt, dotted, from=0, to=1]
	\end{tikzcd}\]
\end{remark}

\begin{theorem}[Uniqueness of Limits\label{thm:limit_iso}]
	A limit can be seen as a terminal morphism, which are unique up to Isomorphism
	(Theorem \ref{thm:terminal_morphism_iso}), therefore limits are also unique
	up to isomorphism.
\end{theorem}

\begin{definition}[Colimit\index{Colimit}\label{def:colimit}]
	The colimit of a functor $F:C\to D$ is a cocone with vertex $d\in D$ and
	morphisms $\big(\alpha_c: F(c)\to d\big)_{c\in C}$ such that for any other
	cone with vertex $d'\in D$ and morphisms $\big(\beta_c: F(c)\to d'\big)_{c\in
	C}$ there exists a unique map $f:d\to d'$ such that the following diagram
	commutes for all $c\in C$: \parencite{leinster:basic_category_theory}

	% https://q.uiver.app/?q=WzAsMyxbMCwxLCJkIl0sWzAsMCwiZCciXSxbMSwxLCJGKGMpIl0sWzIsMCwiXFxhbHBoYV9jIl0sWzIsMSwiXFxiZXRhX2MiLDJdLFswLDEsImYiLDAseyJzdHlsZSI6eyJib2R5Ijp7Im5hbWUiOiJkYXNoZWQifX19XV0=
	\[\begin{tikzcd}[ampersand replacement=\&]
		{d'} \\
		d \& {F(c)}
		\arrow["{\alpha_c}", from=2-2, to=2-1]
		\arrow["{\beta_c}"', from=2-2, to=1-1]
		\arrow["f", dashed, from=2-1, to=1-1]
	\end{tikzcd}\]
\end{definition}

\begin{remark}
	Note that Definition \ref{def:colimit} is equivalent to saying that a colimit
	of a functor $F:C\to D$ is an initial object in the category cocone($F$).
\end{remark}

\begin{remark}
	Note that Definition \ref{def:colimit} can also be thought of as an initial
	morphism from the functor $F$ (thought of as an object in $D^C$) to the diagonal functor $\Delta$ as shown in the following diagram:
	% https://q.uiver.app/?q=WzAsNSxbMCwwLCJjIl0sWzAsMSwiYyciXSxbMSwwLCJcXERlbHRhX2MiXSxbMSwxLCJcXERlbHRhX3tjJ30iXSxbMiwwLCJGIl0sWzAsMSwiaCIsMix7InN0eWxlIjp7ImJvZHkiOnsibmFtZSI6ImRhc2hlZCJ9fX1dLFsyLDMsIlxcRGVsdGEoaCkiLDIseyJzdHlsZSI6eyJib2R5Ijp7Im5hbWUiOiJkYXNoZWQifX19XSxbNCwzLCJcXGJldGEiXSxbNCwyLCJcXGFscGhhIiwyXSxbMCwyLCIiLDAseyJjb2xvdXIiOlsyNDAsNjAsNjBdLCJzdHlsZSI6eyJib2R5Ijp7Im5hbWUiOiJkb3R0ZWQifX19XSxbMSwzLCIiLDAseyJjb2xvdXIiOlsyNDAsNjAsNjBdLCJzdHlsZSI6eyJib2R5Ijp7Im5hbWUiOiJkb3R0ZWQifX19XSxbNSw2LCIiLDAseyJzaG9ydGVuIjp7InNvdXJjZSI6MjAsInRhcmdldCI6NjB9LCJsZXZlbCI6MSwiY29sb3VyIjpbMjQwLDYwLDYwXSwic3R5bGUiOnsiYm9keSI6eyJuYW1lIjoiZG90dGVkIn19fV1d&macro_url=https%3A%2F%2Fraw.githubusercontent.com%2Faortega0703%2Fnotes-category-theory%2Fmain%2Fsrc%2Fmacros.tex
	\[\begin{tikzcd}[ampersand replacement=\&]
		c \& {\Delta_c} \& F \\
		{c'} \& {\Delta_{c'}}
		\arrow[""{name=0, anchor=center, inner sep=0}, "h"', dashed, from=1-1, to=2-1]
		\arrow[""{name=1, anchor=center, inner sep=0}, "{\Delta(h)}"', dashed, from=1-2, to=2-2]
		\arrow["\beta", from=1-3, to=2-2]
		\arrow["\alpha"', from=1-3, to=1-2]
		\arrow[draw={rgb,255:red,92;green,92;blue,214}, dotted, from=1-1, to=1-2]
		\arrow[draw={rgb,255:red,92;green,92;blue,214}, dotted, from=2-1, to=2-2]
		\arrow[draw={rgb,255:red,92;green,92;blue,214}, shorten <=6pt, shorten >=19pt, dotted, from=0, to=1]
	\end{tikzcd}\]
\end{remark}

\begin{theorem}[Uniqueness of Colimits\label{thm:colimit_iso}]
	A colimit can be seen as an initial morphism, which are unique up to
	Isomorphism (Theorem \ref{thm:initial_morphism_iso}), therefore colimits are
	also unique up to isomorphism.
\end{theorem}