\section{Categories}

\subsection{Definition}
A category $C$ is composed of: \parencite{fong:7sketches}
\begin{itemize}
  \item Objects:\\
    They can be thought of as ``points'' with no additional structure.\\
    $\mathrm{ob}_C$ is defined as the collection of objects in the category $C$.
    For simplicity's sake, $c\in C$ is meant to mean $c\in \mathrm{ob}_C$.
  \item Morphisms:\\
    Commonly referred to as arrows, they are directed connections between
    objects.\\
    $\mathrm{hom}_C(a, b)$ is called an hom-set and is defined as the collection
    of morphisms from $a$ to $b$ for all $a,b\in C$. For simplicity's sake, $f:
    a \to b$ is meant to mean $f \in \mathrm{hom}_C(a,c)$ when talking of a
    morphism $f$ in a category $C$\\
    $\mathrm{hom}_C = \bigcup\limits_{a,b\in C} \mathrm{hom}_C(a,b)$
\end{itemize}

Such that it satisfies:
\begin{itemize}
  \item Identity:\\
    $(\mathrm{id}_c: c \to c) \in C,\ (\forall c \in C)$
  \item Composition:\\
    $(g \circ f : a \to c) \in C,
      \ (\forall a, b, c \in C,\ f: a \to b,\ g:b \to c)$
  \item Unitality:\\
    $\mathrm{id}_b \circ f = f \circ \mathrm{id}_a = f,
      \ (\forall a, b \in C,\ f: a \to b)$
  \item Associativity:\\
    $h \circ (g \circ f) = (h \circ g) \circ f,
      \ (\forall a,b,c\in C,\ f:a\to b,\ g:b\to c,\ h:c\to d)$
\end{itemize}

e.g.
\[
  \begin{tikzcd}
    a
    \arrow[out=90, in=180, loop, swap, "\mathrm{id}_a", looseness=5]
    \arrow[r, "f"]
    \arrow[dr, swap, "g \circ f"]
    & b
    \arrow[out=0, in=90, loop, swap, "\mathrm{id}_b", looseness=4.5]
    \arrow[d,"g"]\\
    & c
    \arrow[out=-90, in=0, swap, "\mathrm{id}_c", looseness=5]
  \end{tikzcd}
\]

\subsection{Duality}
Given a category $C$, its opposite category $C^\mathrm{op}$ is given by keeping
all objects in $C$ and reversing its arrows. That is:
\parencite{maclane:working_mathematician}
\begin{itemize}
  \item Objects:\\
    $c\in C^\mathrm{op},\ (\forall c \in C$)
  \item Morphisms:\\
    $(f^\mathrm{op} : c' \to c)\in C^\mathrm{op},
      \ (\forall c, c' \in C,\ f : c \to c')$
\end{itemize}
As the axioms of category theory are self-dual, any statement $\Sigma$ that
holds for all categories implies that its dual statement ($\Sigma$ in
$C^\mathrm{op}$) must also hold for all categories.
\parencite{awodey:category_theory}

\subsection{Category Properties}

\subsubsection*{Small/Large Category}
A category $C$ is said to be small when $\mathrm{ob}_C$ and $\mathrm{hom}_C$ are
sets instead of proper classes. It is said to be large otherwise.
\parencite{awodey:category_theory}

\subsubsection*{Locally Small Category}
A category $C$ is said to be locally small when $\mathrm{hom}_C(a, b)$ is a
proper set for all $a,b\in C$. \parencite{awodey:category_theory}

\subsubsection*{Free Category}
A category is said to be free when it has no more
constraints\footnote{Equalities between paths} than the ones enforced from the
category axioms. \parencite{adamek_herrlich_strecker:joy_cats} A free category
can be constructed from an arbitrary graph by adding morphisms as necessary to
fulfill such axioms.

\subsubsection*{Thin Category}
A category $C$ is said to be thin when $\mathrm{hom}_C(a, b)$ is either empty or
a singleton for every $a,b\in C$. \parencite{adamek_herrlich_strecker:joy_cats}
Every thin category can be seen as a preorder. This is the opposite of a free
category in the sense that it has every possible constraint.

\subsubsection*{Discrete Category}
A category $C$ is said to be discrete when $\mathrm{hom}_C$ is conformed
entirely out of identity morphisms. \parencite{awodey:category_theory} Discrete
categories are going to be denoted by $\underline{n}$, with $n$ being the number
of objects in it.

\subsection{Morphism Properties}

\subsubsection*{Isomorphism}
In a category $C$ with objects $a, b$, a morphism $f: a\to b$ is said to be
invertible or an isomorphism when there exists another morphism $g:b \to a$ such
that $g\circ f = \mathrm{id}_b$ and $f\circ g = \mathrm{id}_c$.
\parencite{maclane:working_mathematician}
If such $g$ exists, it is unique and often written as $g= f^{-1}$. Two objects
are said to be isomorphic if there is an isomorphism between them. If two
objects $a,b$ are isomorphic we write $a\cong b$.
% https://q.uiver.app/?q=WzAsNCxbMCwwLCJhIl0sWzEsMCwiYiJdLFswLDEsImEiXSxbMSwxLCJiIl0sWzAsMSwiZiIsMCx7Im9mZnNldCI6LTF9XSxbMywyLCJmXnstMX0iXSxbMiwwLCJcXG1hdGhybXtpZH1fYSJdLFsxLDMsIlxcbWF0aHJte2lkfV9iIl1d
\[\begin{tikzcd}
	a & b \\
	a & b
	\arrow["f", shift left=1, from=1-1, to=1-2]
	\arrow["{f^{-1}}", from=2-2, to=2-1]
	\arrow["{\mathrm{id}_a}", from=2-1, to=1-1]
	\arrow["{\mathrm{id}_b}", from=1-2, to=2-2]
\end{tikzcd}\]
e.g. In Set the isomorphisms are precisely the bijections.

\subsubsection*{Monomorphism}
In a category with objects $a, b, c$, a morphism is said to be monic or a
monomorphism when it is left cancellable i.e. $f \circ g_1 = f \circ g_2
\implies g_1 = g_2,\ (\forall g_1, g_2:a\to b)$.
\parencite{maclane:working_mathematician}\\
$f:b\rightarrowtail c$ will be used to denote monomorphisms.
% https://q.uiver.app/?q=WzAsMyxbMCwwLCJhIl0sWzEsMCwiYiJdLFsyLDAsImMiXSxbMCwxLCJnIiwwLHsiY3VydmUiOi0yfV0sWzAsMSwiZyciLDIseyJjdXJ2ZSI6Mn1dLFsxLDIsImYiLDAseyJzdHlsZSI6eyJ0YWlsIjp7Im5hbWUiOiJtb25vIn19fV1d
\[\begin{tikzcd}
	a & b & c
	\arrow["g", curve={height=-12pt}, from=1-1, to=1-2]
	\arrow["{g'}"', curve={height=12pt}, from=1-1, to=1-2]
	\arrow["f", tail, from=1-2, to=1-3]
\end{tikzcd}\]
e.g. In Set the monomorphisms are precisely the injections.
Suppose sets $b$ and $c$ with a non-injective function $f$ i.e. for some
pair of elements $x \neq y$ in $b$, $f(x) = f(y)$. Take a third set $a$ with
a pair of functions $g, g':a\to b$ that differ only in that one maps an
element to $x$ and the other one maps the same element to $y$. Now we have
that $f\circ g = f\circ g'$ completing the test for non-injectivity.
\[\text{Non-Injective}(f:b\to c) \coloneq(\exists g, g':a\to b).(g \neq g'
  \land f\circ g = f\circ g')\] Or negating the logic.
\[\text{Injective}(f:b\to c) \coloneq(\forall g, g':a\to b).(f\circ g =
  f\circ g' \implies g=g')\]

\subsubsection*{Epimorphism}
In a category with objects $a, b, c$, a morphism is said to be epic or an
epimorphism when it is right cancellable i.e. $g_1 \circ f = g_2 \circ f
\implies g_1 = g_2,\ (\forall g_1, g_2: c\to d)$.
\parencite{maclane:working_mathematician}\\
$f:a\twoheadrightarrow b$ will be used to denote epimorphisms.
% https://q.uiver.app/?q=WzAsMyxbMCwwLCJhIl0sWzEsMCwiYiJdLFsyLDAsImMiXSxbMSwyLCJnJyIsMix7ImN1cnZlIjoyfV0sWzAsMSwiZiIsMCx7InN0eWxlIjp7ImhlYWQiOnsibmFtZSI6ImVwaSJ9fX1dLFsxLDIsImciLDAseyJjdXJ2ZSI6LTJ9XV0=
\[\begin{tikzcd}
	a & b & c
	\arrow["{g'}"', curve={height=12pt}, from=1-2, to=1-3]
	\arrow["f", two heads, from=1-1, to=1-2]
	\arrow["g", curve={height=-12pt}, from=1-2, to=1-3]
\end{tikzcd}\]
e.g. In Set the epimorphisms are precisely the surjections.
Suppose sets $a$ and $b$ with a non-surjective function $f$ i.e. for some
element $y\in b$ there exists no $x\in a$ such that $f(x) = y$. Take a third
set $c$ with a pair of functions $g, g':b\to c$ that differ only in that
they map $y$ differently. Now we have $g\circ f = g'\circ f$ completing the
test for non-surjectivity.
\[\text{Non-Surjective}(f:a\to b) \coloneq(\exists g, g':b\to c).(g \neq g'
  \land g\circ f = g'\circ f)\] Or negating the logic.
\[\text{Surjective}(f:a\to b) \coloneq(\forall g, g':b\to c).(g\circ f =
  g'\circ f \implies g=g')\]

\subsection{Important Categories}

\begin{itemize}
  \item Set:\\
    It is the category of all sets with functions as their morphisms.
    \parencite{awodey:category_theory}
    \begin{itemize}
      \item Objects:\\
            Every set is an object in Set.
      \item Morphisms:\\
            Morphisms in Set are functions between sets.
    \end{itemize}

  \item Cat:\\
    It is the category of all small categories.
    \parencite{adamek_herrlich_strecker:joy_cats}
    \begin{itemize}
      \item Objects:\\
            Every small category is an object in Cat.
      \item Morphisms:\\
            Morphisms in Cat are functors between categories.
    \end{itemize}
\end{itemize}