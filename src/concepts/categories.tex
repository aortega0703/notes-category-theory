\section{Categories}
A category can be thought of as a way to formalize the presence of certain
structure, only caring about the underlying relationships.

\begin{definition}[Category]
  A category $C$ is composed of: \parencite{fong_spivak:7sketches}
  \begin{itemize}
    \item Objects:\\
      $\ob_C$: the collection of objects.
    \item Morphisms:\footnote{Also referred to as arrows.}\\
      $\hom_C(c, c'),\ (\forall c, c' \in \ob_C)$: the collection of all
      morphisms from $c$ to $c'$ where $c$ is called the domain and $c'$ the
      codomain.\\
      $\hom_C = \bigcup\limits_{a,b\in C} \hom_C(a,b)$
  \end{itemize}

  Such that the following are satisfied:
  \begin{itemize}
    \item Composition:\\
      $(g \circ f : a \to c) \in C,
        \ (\forall a, b, c \in C,\ f: a \to b,\ g:b \to c)$
    \item Associativity:\\
      $(h \circ g) \circ f = h \circ (g \circ f), \ (\forall a,b,c\in C,\ f:a\to
        b,\ g:b\to c,\ h:c\to d)$
    \item Identity:\\
      $(\id_c: c \to c) \in C,\ (\forall c \in C)$
    \item Unitality:\\
      $f \circ \id_a = \id_b \circ f = f,
        \ (\forall a, b \in C,\ f: a \to b)$
  \end{itemize}
\end{definition}

\begin{remark}
  For simplicity's sake, the following conventions are usually followed:
  \begin{itemize}
    \item $c\in C$ to mean $c\in\ob_C$.
    \item $(f: c\to c')\in C$ to mean $f\in \hom_C(c, c')$.
    \item Identities are not drawn.
    \item Omission of parenthesis when associativity holds.
  \end{itemize}
\end{remark}

\begin{example}
  % https://q.uiver.app/?q=WzAsNCxbMCwwLCJhIl0sWzEsMCwiYiJdLFsyLDAsImMiXSxbMywwLCJkIl0sWzAsMSwiZiJdLFswLDIsImdcXGNpcmMgZiIsMCx7ImN1cnZlIjotNH1dLFsyLDMsImgiXSxbMSwyLCJnIl0sWzEsMywiaFxcY2lyYyBnIiwwLHsiY3VydmUiOi00fV0sWzAsMywiaFxcY2lyYyBnXFxjaXJjIGYiLDIseyJjdXJ2ZSI6NH1dXQ==
  \[\begin{tikzcd}
    a & b & c & d
    \arrow["f", from=1-1, to=1-2]
    \arrow["{g\circ f}", curve={height=-24pt}, from=1-1, to=1-3]
    \arrow["h", from=1-3, to=1-4]
    \arrow["g", from=1-2, to=1-3]
    \arrow["{h\circ g}", curve={height=-24pt}, from=1-2, to=1-4]
    \arrow["{h\circ g\circ f}"', curve={height=24pt}, from=1-1, to=1-4]
  \end{tikzcd}\]
\end{example}

\begin{definition}[Opposite category]
  For a category $C$, its opposite category $C^\op$ is given by keeping all
  objects in $C$ and reversing its arrows. i.e.
  \parencite{awodey:category_theory}
  \begin{itemize}
    \item Objects:\\
      $c\in C^\op,\ (\forall c \in C$)
    \item Morphisms:\\
      $(f^\op : c' \to c)\in C^\op,
        \ (\forall c, c' \in C,\ f : c \to c')$
  \end{itemize}
\end{definition}

\begin{remark}
  As the axioms of category theory are self-dual, any statement $\Sigma$ that
  holds for all categories implies that its dual statement ($\Sigma$ in
  $C^\op$) must also hold for all categories.
  \parencite{awodey:category_theory}
\end{remark}

\subsection{Category Properties}

\subsubsection*{Small/Large Category}
A category $C$ is said to be small when $\ob_C$ and $\hom_C$ are
sets instead of proper classes. It is said to be large otherwise.
\parencite{awodey:category_theory}

\subsubsection*{Locally Small Category}
A category $C$ is said to be locally small when $\hom_C(a, b)$ is a
proper set for all $a,b\in C$. \parencite{awodey:category_theory}

\subsubsection*{Free Category}
A category is said to be free when it has no more constraints
\footnote{Equalities between paths} than the ones enforced from the category
axioms. \parencite{adamek_herrlich_strecker:joy_cats} A free category can be
constructed from an arbitrary graph by adding morphisms as necessary to fulfill
such axioms.

\subsubsection*{Thin Category}
A category $C$ is said to be thin when $\hom_C(a, b)$ is either empty or
a singleton for every $a,b\in C$. \parencite{adamek_herrlich_strecker:joy_cats}
Every thin category can be seen as a preorder. This is the opposite of a free
category in the sense that it has every possible constraint.

\subsubsection*{Discrete Category}
A category $C$ is said to be discrete when $\hom_C$ is composed entirely
out of identity morphisms. \parencite{awodey:category_theory} Discrete
categories are going to be denoted by $\underline{n}$, with $n$ being the number
of objects in it.

\subsection{Morphism Properties}

\subsubsection*{Isomorphism}
In a category $C$ with objects $a, b$, a morphism $f: a\to b$ is said to be
invertible or an isomorphism when there exists another morphism $g:b \to a$ such
that $g\circ f = \id_b$ and $f\circ g = \id_c$.
\parencite{maclane:working_mathematician}
If such $g$ exists, it is unique and often written as $g= f^{-1}$. Two objects
are said to be isomorphic if there is an isomorphism between them. If two
objects $a,b$ are isomorphic we write $a\cong b$.

% https://q.uiver.app/?q=WzAsNCxbMCwwLCJhIl0sWzEsMCwiYiJdLFswLDEsImEiXSxbMSwxLCJiIl0sWzAsMSwiZiIsMCx7Im9mZnNldCI6LTF9XSxbMywyLCJmXnstMX0iXSxbMiwwLCJcXG1hdGhybXtpZH1fYSJdLFsxLDMsIlxcbWF0aHJte2lkfV9iIl1d
\[\begin{tikzcd}
	a & b \\
	a & b
	\arrow["f", shift left=1, from=1-1, to=1-2]
	\arrow["{f^{-1}}", from=2-2, to=2-1]
	\arrow["{\id_a}", from=2-1, to=1-1]
	\arrow["{\id_b}", from=1-2, to=2-2]
\end{tikzcd}\]

\begin{definition}[Monomorphisms]
  A morphism $f:c\to c'$ is said to be monic or a monomorphism when it is left
  cancellable \parencite{maclane:working_mathematician} i.e. when for any
  configuration of objects and morphisms:
  % https://q.uiver.app/?q=WzAsMyxbMSwwLCJjIl0sWzIsMCwiYyciXSxbMCwwLCJ4Il0sWzAsMSwiZiIsMCx7InN0eWxlIjp7InRhaWwiOnsibmFtZSI6Im1vbm8ifX19XSxbMiwwLCJnIiwwLHsiY3VydmUiOi0zfV0sWzIsMCwiaCIsMix7ImN1cnZlIjozfV1d
  \[\begin{tikzcd}
    x & c & {c'}
    \arrow["f", tail, from=1-2, to=1-3]
    \arrow["g", curve={height=-18pt}, from=1-1, to=1-2]
    \arrow["h"', curve={height=18pt}, from=1-1, to=1-2]
  \end{tikzcd}\]

  The following holds:
  \[f \circ g = f \circ h \implies g = h\]
\end{definition}

\begin{remark}
  $f:b\mono c$ will be used to denote monomorphisms.
\end{remark}

\begin{theorem}[Isomorphism implies Monomorphisms]
  If $f:c\to c'$ is an isomorphism, then it is a monomorphism.

  \begin{proof}
    Consider the configuration of objects and morphisms:
    % https://q.uiver.app/?q=WzAsMyxbMSwwLCJjIl0sWzIsMCwiYyciXSxbMCwwLCJ4Il0sWzAsMSwiZiIsMCx7ImN1cnZlIjotM31dLFsyLDAsImciLDAseyJjdXJ2ZSI6LTN9XSxbMiwwLCJoIiwyLHsiY3VydmUiOjN9XSxbMSwwLCJmXnstMX0iLDAseyJjdXJ2ZSI6LTN9XV0=
    \[\begin{tikzcd}
      x & c & {c'}
      \arrow["f", curve={height=-18pt}, from=1-2, to=1-3]
      \arrow["g", curve={height=-18pt}, from=1-1, to=1-2]
      \arrow["h"', curve={height=18pt}, from=1-1, to=1-2]
      \arrow["{f^{-1}}", curve={height=-18pt}, from=1-3, to=1-2]
    \end{tikzcd}\]
    Suppose that $f\circ g = f\circ h$, then:
    \[
      \begin{aligned}
        f\circ g &= f\circ h\\
        f^{-1}\circ f\circ g &= f^{-1}\circ f\circ h\\
        g &= h
      \end{aligned}
    \]
  \end{proof}
\end{theorem}

\begin{definition}[Epimorphism]
  A morphism $f:c\to c'$ is said to be epic or an epimorphism when it is right
  cancellable \parencite{maclane:working_mathematician} i.e. when for any
  configuration of objects and morphisms:
  % https://q.uiver.app/?q=WzAsMyxbMCwwLCJjIl0sWzEsMCwiYyciXSxbMiwwLCJ4Il0sWzEsMiwiaCIsMix7ImN1cnZlIjozfV0sWzAsMSwiZiIsMCx7InN0eWxlIjp7ImhlYWQiOnsibmFtZSI6ImVwaSJ9fX1dLFsxLDIsImciLDAseyJjdXJ2ZSI6LTN9XV0=
  \[\begin{tikzcd}
    c & {c'} & x
    \arrow["h"', curve={height=18pt}, from=1-2, to=1-3]
    \arrow["f", two heads, from=1-1, to=1-2]
    \arrow["g", curve={height=-18pt}, from=1-2, to=1-3]
  \end{tikzcd}\]

  The following holds:
  \[g \circ f = h \circ f \implies g = h\]
\end{definition}

\begin{remark}
  $f:a\epi b$ $\epi$will be used to denote epimorphisms.
\end{remark}

\begin{theorem}[Isomorphism implies Epimorphism]
  If $f:c\to c'$ is an isomorphism, then it is an epimorphism.

  \begin{proof}
    Consider the configuration of objects and morphisms:
    % https://q.uiver.app/?q=WzAsMyxbMCwwLCJjIl0sWzEsMCwiYyciXSxbMiwwLCJ4Il0sWzAsMSwiZiIsMCx7ImN1cnZlIjotM31dLFsxLDAsImZeey0xfSIsMCx7ImN1cnZlIjotM31dLFsxLDIsImciLDAseyJjdXJ2ZSI6LTN9XSxbMSwyLCJoIiwyLHsiY3VydmUiOjN9XV0=
    \[\begin{tikzcd}
      c & {c'} & x
      \arrow["f", curve={height=-18pt}, from=1-1, to=1-2]
      \arrow["{f^{-1}}", curve={height=-18pt}, from=1-2, to=1-1]
      \arrow["g", curve={height=-18pt}, from=1-2, to=1-3]
      \arrow["h"', curve={height=18pt}, from=1-2, to=1-3]
    \end{tikzcd}\]

    Suppose that $g \circ f = h \circ f$, then:
    \[
      \begin{aligned}
        g \circ f &= h \circ f\\
        g \circ f \circ f^{-1} &= h \circ f \circ f^{-1}\\
        g &= h
      \end{aligned}
    \]
  \end{proof}
\end{theorem}
e.g. In Set the monomorphisms are precisely the injections.
Suppose sets $b$ and $c$ with a non-injective function $f$ i.e. for some
pair of elements $x \neq y$ in $b$, $f(x) = f(y)$. Take a third set $a$ with
a pair of functions $g, g':a\to b$ that differ only in that one maps an
element to $x$ and the other one maps the same element to $y$. Now we have
that $f\circ g = f\circ g'$ completing the test for non-injectivity.
\[\text{Non-Injective}(f:b\to c) \coloneq(\exists g, g':a\to b).(g \neq g'
  \land f\circ g = f\circ g')\] Or negating the logic.
\[\text{Injective}(f:b\to c) \coloneq(\forall g, g':a\to b).(f\circ g =
  f\circ g' \implies g=g')\]

e.g. In Set the epimorphisms are precisely the surjections.
Suppose sets $a$ and $b$ with a non-surjective function $f$ i.e. for some
element $y\in b$ there exists no $x\in a$ such that $f(x) = y$. Take a third
set $c$ with a pair of functions $g, g':b\to c$ that differ only in that
they map $y$ differently. Now we have $g\circ f = g'\circ f$ completing the
test for non-surjectivity.
\[\text{Non-Surjective}(f:a\to b) \coloneq(\exists g, g':b\to c).(g \neq g'
  \land g\circ f = g'\circ f)\] Or negating the logic.
\[\text{Surjective}(f:a\to b) \coloneq(\forall g, g':b\to c).(g\circ f =
  g'\circ f \implies g=g')\]