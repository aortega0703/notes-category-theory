\section{Categories}
A category can be thought of as the abstraction of a certain structure.

\begin{definition}[Category\index{Category}]
  A category $\C$ is composed of~\parencite{fong_spivak:7sketches}:
  \begin{itemize}
    \item Collections of Objects ($\C_0$) and Morphisms ($\C_1$).
    \item Functions Domain ($\mathrm{Dom}:\C_1\to \C_0$) and Codomain
      ($\mathrm{Cod}: \C_1 \to \C_0$).
      \[
        \begin{gathered}
          (f : c\to c') \in \C_1
          \coloneqq f\in \C_1
          \land \mathrm{Dom}(f) = c \land \mathrm{Cod}(f) = c'\\
          \C(c, c') \coloneqq \{f\ |\ f : c \to c'\}\subseteq \C_1
        \end{gathered}
      \]
  \end{itemize}

  Such that the following are satisfied:
  \begin{itemize}
    \item Identity:
      \[\big(\forall c \in \C_0\big)
        \big((\id_c : c\to c)\in \C_1\big)\]
      % https://q.uiver.app/?q=WzAsMixbMCwwLCJjIl0sWzEsMCwiYyJdLFswLDEsIlxcaWRfYyJdXQ==&macro_url=https%3A%2F%2Fraw.githubusercontent.com%2Faortega0703%2Fnotes-category-theory%2Fmain%2Fsrc%2Fmacros.tex
      \[\begin{tikzcd}[ampersand replacement=\&]
        c \& c
        \arrow["{\id_c}", from=1-1, to=1-2]
      \end{tikzcd}\]
    \item Composition:
      \[\big(\forall (f :a \to b),\ (g : b\to c) \in \C_1\big)
        \big((g\circ f :a\to c) \in \C_1\big)\]
      % https://q.uiver.app/?q=WzAsMyxbMCwwLCJhIl0sWzEsMCwiYiJdLFsxLDEsImMiXSxbMCwxLCJmIl0sWzEsMiwiZyJdLFswLDIsImdcXGNpcmMgZiIsMix7ImxhYmVsX3Bvc2l0aW9uIjo0MH1dXQ==&macro_url=https%3A%2F%2Fraw.githubusercontent.com%2Faortega0703%2Fnotes-category-theory%2Fmain%2Fsrc%2Fmacros.tex
      \[\begin{tikzcd}[ampersand replacement=\&]
        a \& b \\
        \& c
        \arrow["f", from=1-1, to=1-2]
        \arrow["g", from=1-2, to=2-2]
        \arrow["{g\circ f}"'{pos=0.4}, from=1-1, to=2-2]
      \end{tikzcd}\]
    \item Unitality:
      \[\big(\forall (f:a\to b)\in \C_1\big)
        \big(f \circ \id_a = \id_b \circ f = f\big)\]
      % https://q.uiver.app/?q=WzAsNCxbMSwwLCJhIl0sWzEsMSwiYiJdLFsyLDEsImIiXSxbMCwwLCJhIl0sWzAsMSwiZiIsMl0sWzEsMiwiXFxpZF9iIiwyXSxbMywwLCJcXGlkX2EiXSxbMywxLCJmIiwyXSxbMCwyLCJmIl1d&macro_url=https%3A%2F%2Fraw.githubusercontent.com%2Faortega0703%2Fnotes-category-theory%2Fmain%2Fsrc%2Fmacros.tex
      \[\begin{tikzcd}[ampersand replacement=\&]
        a \& a \\
        \& b \& b
        \arrow["f"', from=1-2, to=2-2]
        \arrow["{\id_b}"', from=2-2, to=2-3]
        \arrow["{\id_a}", from=1-1, to=1-2]
        \arrow["f"', from=1-1, to=2-2]
        \arrow["f", from=1-2, to=2-3]
      \end{tikzcd}\]
    \item Associativity:
      \[\big(\forall (f : a\to b),\ (g: b\to c),\ (h: c\to d)\in\C_1\big)
        \big((h \circ g) \circ f = h \circ (g \circ f)\big)\]
      % https://q.uiver.app/?q=WzAsNCxbMCwwLCJhIl0sWzEsMCwiYiJdLFsxLDEsImMiXSxbMiwwLCJkIl0sWzAsMSwiZiJdLFsxLDIsImciLDJdLFsyLDMsImgiLDJdLFswLDIsImdcXGNpcmMgZiIsMix7ImxhYmVsX3Bvc2l0aW9uIjozMH1dLFsxLDMsImhcXGNpcmMgZyJdLFswLDMsImhcXGNpcmMgZyBcXGNpcmMgZiIsMCx7ImN1cnZlIjotNH1dXQ==&macro_url=https%3A%2F%2Fraw.githubusercontent.com%2Faortega0703%2Fnotes-category-theory%2Fmain%2Fsrc%2Fmacros.tex
      \[\begin{tikzcd}[ampersand replacement=\&]
        a \& b \& d \\
        \& c
        \arrow["f", from=1-1, to=1-2]
        \arrow["g"', from=1-2, to=2-2]
        \arrow["h"', from=2-2, to=1-3]
        \arrow["{g\circ f}"'{pos=0.3}, from=1-1, to=2-2]
        \arrow["{h\circ g}", from=1-2, to=1-3]
        \arrow["{h\circ g \circ f}", curve={height=-24pt}, from=1-1, to=1-3]
      \end{tikzcd}\]
  \end{itemize}
\end{definition}

\begin{definition}[Opposite category\index{Opposite category}]
  For a category $\C$, its opposite category $\C^\op$ is composed
  of~\parencite{awodey:category_theory}:

  \begin{itemize}
    \item Objects:
      \[(\forall c \in \C_0)
        (c\in \C^\op_0)\]
    \item Morphisms:
      \[\big(\forall (f: c\to c') \in \C_1\big)
        \big((f^\op : c'\to c)\in \C^\op_1\big)\]
  \end{itemize}
\end{definition}

\begin{remark}
  As the axioms of category theory are self-dual, any statement $\Sigma$ that
  holds for all categories implies that its dual statement ($\Sigma$ in
  $\C^\op$) must also hold for all
  categories~\parencite{awodey:category_theory}.
\end{remark}

\subsection{Category Properties}

\begin{definition}[Small/Large Category\index{Small Category}\index{Large Category}]
  A category $\C$ is small when $\C_0$ and $\C_1$ are sets instead of proper
  classes. It is large otherwise~\parencite{awodey:category_theory}.
\end{definition}

\begin{definition}[Locally Small Category\index{Locally Small Category}]
  A category $\C$ is locally small when $\C(c, c')$ is a set for all $c,c'\in
  \C_0$~\parencite{awodey:category_theory}.
\end{definition}

\begin{definition}[Free Category\index{Free Category}]
  A category is free when it has no more constraints\footnote{Equalities
  between paths} than the ones enforced from the category axioms~\parencite{adamek_herrlich_strecker:joy_cats}.
\end{definition}

\begin{definition}[Thin Category\index{Thin Category}]
  A category $\C$ is thin when $\C(a, b)$ is either empty or a singleton for
  every $a,b\in C$~\parencite{adamek_herrlich_strecker:joy_cats}. It has every
  possible constraint.
\end{definition}

\begin{definition}[Discrete Category\index{Discrete Category}]
  A category $\C$ is discrete when $\C_1$ is composed entirely out of identity
  morphisms~\parencite{awodey:category_theory}. Discrete categories are going to
  be denoted by $\underline{n}$, with $n=|\C_0|$.
\end{definition}

\subsection{Morphism Properties}

\begin{definition}[Endomorphism\index{Endomorphism}]
  A morphism whose domain equals its codomain is an
  endomorphism~\parencite{riehl:category_theory_in_context}.
\end{definition}

\begin{definition}[Left Inverse\index{Left Inverse}]
  A morphism $f: c\to c'$ has a left inverse $g: c'\to c$ if $g\circ f =
  \id_c$~\parencite[p.~19]{lane:working_mathematician} i.e. the following
  diagram commutes:
  % https://q.uiver.app/#q=WzAsMyxbMCwwLCJDIl0sWzEsMCwiQyciXSxbMSwxLCJDIl0sWzAsMSwiZiJdLFsxLDIsImciXSxbMCwyLCJcXGlkX0MiLDJdXQ==&macro_url=https%3A%2F%2Fgist.githubusercontent.com%2Faortega0703%2Fa1fd97cb097b8142e63a6fbf0cdb0f76%2Fraw%2Fb46a955b6b1f06908105b870088e59a10049fc60%2Fmacros.tex
  \[\begin{tikzcd}[ampersand replacement=\&]
    C \& {C'} \\
    \& C
    \arrow["f", from=1-1, to=1-2]
    \arrow["g", from=1-2, to=2-2]
    \arrow["{\id_C}"', from=1-1, to=2-2]
  \end{tikzcd}\]
\end{definition}

\remark{Left inverse is the dual of right inverse.}

\begin{definition}[Right Inverse\index{Right Inverse}]
  A morphism $f: c\to c'$ has a right inverse $g: c'\to c$ if $f\circ g =
  \id_{c'}$~\parencite[p.~19]{lane:working_mathematician} i.e. the following
  diagram commutes:
  % https://q.uiver.app/#q=WzAsMyxbMCwxLCJDIl0sWzEsMSwiQyciXSxbMCwwLCJDJyJdLFswLDEsImYiLDJdLFsyLDAsImciLDJdLFsyLDEsIlxcaWRfe0MnfSJdXQ==&macro_url=https%3A%2F%2Fgist.githubusercontent.com%2Faortega0703%2Fa1fd97cb097b8142e63a6fbf0cdb0f76%2Fraw%2Fb46a955b6b1f06908105b870088e59a10049fc60%2Fmacros.tex
  \[\begin{tikzcd}[ampersand replacement=\&]
    {C'} \\
    C \& {C'}
    \arrow["f"', from=2-1, to=2-2]
    \arrow["g"', from=1-1, to=2-1]
    \arrow["{\id_{C'}}", from=1-1, to=2-2]
  \end{tikzcd}\]
\end{definition}

\remark{Right inverse is the dual of left inverse.}

\begin{definition}[Isomorphism\index{Isomorphism}]
  A morphism $f: C\to C'$ is invertible or an isomorphism when it has a left and
  right inverce, and both are
  equal~\parencite[p.~19]{lane:working_mathematician} i.e. the following diagram
  commutes:
  % https://q.uiver.app/#q=WzAsNCxbMSwxLCJDIl0sWzIsMSwiQyciXSxbMSwwLCJDJyJdLFswLDAsIkMiXSxbMCwxLCJmIiwyXSxbMiwwLCJnIiwyXSxbMiwxLCJcXGlkX3tDJ30iXSxbMywyLCJmIl0sWzMsMCwiXFxpZF9DIiwyXV0=&macro_url=https%3A%2F%2Fgist.githubusercontent.com%2Faortega0703%2Fa1fd97cb097b8142e63a6fbf0cdb0f76%2Fraw%2Fb46a955b6b1f06908105b870088e59a10049fc60%2Fmacros.tex
  \[\begin{tikzcd}[ampersand replacement=\&]
    C \& {C'} \\
    \& C \& {C'}
    \arrow["f"', from=2-2, to=2-3]
    \arrow["g"', from=1-2, to=2-2]
    \arrow["{\id_{C'}}", from=1-2, to=2-3]
    \arrow["f", from=1-1, to=1-2]
    \arrow["{\id_C}"', from=1-1, to=2-2]
  \end{tikzcd}\]
\end{definition}

\begin{remark}
  Two objects are said to be isomorphic (written $a\cong b$) if there exists is
  an isomorphism between them.
\end{remark}

\begin{remark}
  Isomorphism is its own dual.
\end{remark}

\begin{theorem}[Uniqueness of Inverse]
  If $f: c\to c'$ is an isomorphism, then it has a unique inverse $f^{-1}:c'\to
  c$:
  \begin{proof}
    Consider two inverses of $f$, $f^{-1}_1$ and $f^{-1}_2$, then:
    \[
      \begin{aligned}
        \id_c &= \id_c\\
        f\circ f^{-1}_1 &= f\circ f^{-1}_2\\
        f^{-1}_1 \circ f \circ f^{-1}_1 &= f^{-1}_1 \circ f \circ f^{-1}_2\\
        f^{-1}_1 &= f^{-1}_2
      \end{aligned}
    \]
  \end{proof}
\end{theorem}

\begin{definition}[Automorphism\index{Automorphism}]
  An endomorphism which is also an isomorphism is an
  automorphism~\parencite{riehl:category_theory_in_context}.
\end{definition}

\begin{definition}[Monomorphisms\index{Monomorphisms}]
  A morphism $f:c\to c'$ is monic or a monomorphism when it is left
  cancellable~\parencite{lane:working_mathematician} i.e. when for any
  set-up of objects and morphisms:
  % https://q.uiver.app/?q=WzAsMyxbMSwwLCJjIl0sWzIsMCwiYyciXSxbMCwwLCJ4Il0sWzAsMSwiZiJdLFsyLDAsImciLDAseyJvZmZzZXQiOi0xfV0sWzIsMCwiaCIsMix7Im9mZnNldCI6MX1dXQ==
  \[\begin{tikzcd}[ampersand replacement=\&]
    x \& c \& {c'}
    \arrow["f", from=1-2, to=1-3]
    \arrow["g", shift left=1, from=1-1, to=1-2]
    \arrow["h"', shift right=1, from=1-1, to=1-2]
  \end{tikzcd}\]

  The following holds:
  \[f \circ g = f \circ h \implies g = h\]
\end{definition}

\begin{remark}
  $f:b\mono c$ will be used to denote monomorphisms.
\end{remark}

\begin{remark}
  $f:b\sub c$ will be used instead when $f$ represents inclusions.
\end{remark}

\begin{remark}
  Monomorphisms are the dual of epimorphisms.
\end{remark}

\begin{theorem}\label{thm:right_inverse_implies_mono}

  If $f:C\to C'$ has a left inverse $f':C'\to C$, then it is a monomorphism.

  \begin{proof}
    Consider the following set-up of objects and morphisms:
    % https://q.uiver.app/#q=WzAsMyxbMSwwLCJDIl0sWzIsMCwiQyciXSxbMCwwLCJYIl0sWzAsMSwiZiIsMCx7Im9mZnNldCI6LTF9XSxbMiwwLCJnIiwwLHsib2Zmc2V0IjotMX1dLFsyLDAsImgiLDIseyJvZmZzZXQiOjF9XSxbMSwwLCJmJyIsMCx7Im9mZnNldCI6LTF9XV0=
    \[
      \begin{tikzcd}[ampersand replacement=\&]
        X \& C \& {C'}
        \arrow["f", shift left=1, from=1-2, to=1-3]
        \arrow["g", shift left=1, from=1-1, to=1-2]
        \arrow["h"', shift right=1, from=1-1, to=1-2]
        \arrow["{f'}", shift left=1, from=1-3, to=1-2]
      \end{tikzcd}
    \]
    If $f\circ g = f\circ h$ then:
    \[
      \begin{aligned}
        f\circ g &= f\circ h\\
        f'\circ f \circ g &= f'\circ f\circ h\\
        g &= h
      \end{aligned}
    \]
  \end{proof}
  \vspace{-\baselineskip}
\end{theorem}

\begin{definition}[Equivalent Monomorphisms]
  Two monomorphisms $f:A\mono C$ and $g:B\mono C$ are equivalent if there exists
  an isomorphism $h: A \overset{\cong}{\to} B$ such
  that $g\circ h=f$~\parencite{lane_moerdijk:sheaves_geometry_logic}.
\end{definition}

\begin{definition}[Epimorphism\index{Epimorphism}]
  A morphism $f:c\to c'$ is epic or an epimorphism when it is right
  cancellable~\parencite{lane:working_mathematician} i.e. when for any set-up of
  objects and morphisms:
  % https://q.uiver.app/?q=WzAsMyxbMCwwLCJjIl0sWzEsMCwiYyciXSxbMiwwLCJ4Il0sWzEsMiwiaCIsMix7Im9mZnNldCI6MX1dLFswLDEsImYiXSxbMSwyLCJnIiwwLHsib2Zmc2V0IjotMX1dXQ==
  \[\begin{tikzcd}[ampersand replacement=\&]
    c \& {c'} \& x
    \arrow["h"', shift right=1, from=1-2, to=1-3]
    \arrow["f", from=1-1, to=1-2]
    \arrow["g", shift left=1, from=1-2, to=1-3]
  \end{tikzcd}\]

  The following holds:
  \[g \circ f = h \circ f \implies g = h\]
\end{definition}

\begin{remark}
  $f:a\epi b$ will be used to denote epimorphisms.
\end{remark}

\begin{remark}
  Epimorphism are the dual of monomorphisms.
\end{remark}

\begin{theorem}
  If $f:C\to C'$ has a right inverse $f':C'\to C$, then it is a epimorphism.

  \begin{proof}
    By duality with Theorem \ref{thm:right_inverse_implies_mono}.
  \end{proof}
\end{theorem}

\begin{definition}[Subobject]
  A subobject of an object $C$ is an equivalence class of monomorphisms into
  $C$. The collection of subobjects of $C$ induces a partial order defined by
  $[f]\leq[g]$ iff there exist some $h: A\to B$ such that $f = g\circ
  h$~\parencite[p.~11]{lane_moerdijk:sheaves_geometry_logic}.
\end{definition}