\section{Categories}
A category can be thought of as a way to formalize the presence of certain
structure, only caring about the underlying relationships.

\begin{definition}[Category\index{Category}]
  A category $C$ is composed of: \parencite{fong_spivak:7sketches}
  \begin{itemize}
    \item Objects:\\
      $\ob_C$: the collection of objects.
    \item Morphisms:\footnote{Also referred to as arrows.}\\
      $\hom_C(c, c'),\ (\forall c, c' \in \ob_C)$: the collection of all
      morphisms from $c$ to $c'$ where $c$ is called the domain and $c'$ the
      codomain.\\
      $\hom_C \coloneqq \bigcup\limits_{a,b\in C} \hom_C(a,b)$
  \end{itemize}

  Such that the following are satisfied:
  \begin{itemize}
    \item Identity:\\
      $(\id_c: c \to c) \in C,\ (\forall c \in C)$
    \item Composition:\\
      $(g \circ f : a \to c) \in C,
        \ (\forall a, b, c \in C,\ f: a \to b,\ g:b \to c)$
    \item Unitality:\\
      $f \circ \id_a = \id_b \circ f = f,
        \ (\forall a, b \in C,\ f: a \to b)$
    \item Associativity:\\
      $(h \circ g) \circ f = h \circ (g \circ f), \ (\forall a,b,c\in C,\ f:a\to
        b,\ g:b\to c,\ h:c\to d)$
  \end{itemize}
\end{definition}

\begin{remark}
  For simplicity's sake, the following conventions are usually followed:
  \begin{itemize}
    \item $c\in C$ to mean $c\in\ob_C$.
    \item $(f: c\to c')\in C$ to mean $f\in \hom_C(c, c')$.
    \item Identities are not drawn.
    \item Omission of parenthesis when associativity holds.
  \end{itemize}
\end{remark}

\begin{example}
  % https://q.uiver.app/?q=WzAsNCxbMCwwLCJhIl0sWzEsMCwiYiJdLFsyLDAsImMiXSxbMywwLCJkIl0sWzAsMSwiZiIsMl0sWzAsMiwiZ1xcY2lyYyBmIiwwLHsiY3VydmUiOi00fV0sWzIsMywiaCIsMl0sWzEsMiwiZyIsMl0sWzEsMywiaFxcY2lyYyBnIiwwLHsiY3VydmUiOi00fV0sWzAsMywiaFxcY2lyYyBnXFxjaXJjIGYiLDIseyJjdXJ2ZSI6NH1dXQ==
  \[\begin{tikzcd}[ampersand replacement=\&]
    a \& b \& c \& d
    \arrow["f"', from=1-1, to=1-2]
    \arrow["{g\circ f}", curve={height=-24pt}, from=1-1, to=1-3]
    \arrow["h"', from=1-3, to=1-4]
    \arrow["g"', from=1-2, to=1-3]
    \arrow["{h\circ g}", curve={height=-24pt}, from=1-2, to=1-4]
    \arrow["{h\circ g\circ f}"', curve={height=24pt}, from=1-1, to=1-4]
  \end{tikzcd}\]
\end{example}

\begin{definition}[Opposite category\index{Opposite category}]
  For a category $C$, its opposite category $C^\op$ is given by keeping all
  objects in $C$ and reversing its arrows. i.e. it is composed of:
  \parencite{awodey:category_theory}
  \begin{itemize}
    \item Objects:\\
      $c\in C^\op,\ (\forall c \in C$)
    \item Morphisms:\\
      $(f^\op : c' \to c)\in C^\op,
        \ (\forall c, c' \in C,\ f : c \to c')$
  \end{itemize}
\end{definition}

\begin{remark}
  As the axioms of category theory are self-dual, any statement $\Sigma$ that
  holds for all categories implies that its dual statement ($\Sigma$ in
  $C^\op$) must also hold for all categories.
  \parencite{awodey:category_theory}
\end{remark}

\subsection{Category Properties}

\begin{definition}[Small/Large Category\index{Small/Large Category}]
  A category $C$ is said to be small when $\ob_C$ and $\hom_C$ are sets instead
  of proper classes. It is said to be large otherwise.
  \parencite{awodey:category_theory}
\end{definition}

\begin{definition}[Locally Small Category\index{Locally Small Category}]
  A category $C$ is said to be locally small when $\hom_C(c, c')$ is a set for
  all $c,c'\in C$. \parencite{awodey:category_theory}
\end{definition}

\begin{definition}[Free Category\index{Free Category}]
  A category is said to be free when it has no more constraints
  \footnote{Equalities between paths} than the ones enforced from the category
  axioms. \parencite{adamek_herrlich_strecker:joy_cats} A free category can be
  constructed from an arbitrary graph by adding morphisms as necessary to
  fulfill such axioms.
\end{definition}

\begin{definition}[Thin Category\index{Thin Category}]
  A category $C$ is said to be thin when $\hom_C(a, b)$ is either empty or
  a singleton for every $a,b\in C$.
  \parencite{adamek_herrlich_strecker:joy_cats} Every thin category can be seen
  as a preorder. This is the opposite of a free category in the sense that it
  has every possible constraint.
\end{definition}

\begin{definition}[Discrete Category\index{Discrete Category}]
  A category $C$ is said to be discrete when $\hom_C$ is composed entirely
  out of identity morphisms. \parencite{awodey:category_theory} Discrete
  categories are going to be denoted by $\underline{n}$, with $n=|\ob_C|$.
\end{definition}

\subsection{Morphism Properties}

\begin{definition}[Isomorphism\index{Isomorphism}]
  A morphism $f: c\to c'$ is said to be invertible or an isomorphism when there
  exists an inverse morphism $f^{-1}:c'\to c$ such that $f^{-1}\circ f =
  \id_{c}$ and $f\circ f^{-1} = \id_{c'}$.
  \parencite{lane:working_mathematician} i.e. the following diagram
  commutes:\footnote{Parallel morphisms i.e. those with the same domain and
  codomain, are equal.}
  % https://q.uiver.app/?q=WzAsNCxbMCwwLCJjIl0sWzEsMCwiYyciXSxbMCwxLCJjIl0sWzEsMSwiYyciXSxbMCwxLCJmIiwwLHsib2Zmc2V0IjotMX1dLFszLDIsImZeey0xfSJdLFsyLDAsIlxcbWF0aHJte2lkfV9jIl0sWzEsMywiXFxtYXRocm17aWR9X3tjJ30iXV0=
  \[\begin{tikzcd}
    c & {c'} \\
    c & {c'}
    \arrow["f", shift left=1, from=1-1, to=1-2]
    \arrow["{f^{-1}}", from=2-2, to=2-1]
    \arrow["{\mathrm{id}_c}", from=2-1, to=1-1]
    \arrow["{\mathrm{id}_{c'}}", from=1-2, to=2-2]
  \end{tikzcd}\]
\end{definition}

\begin{remark}
  Two objects are said to be isomorphic (written $a\cong b$) if there exists is
  an isomorphism between them.
\end{remark}

\begin{theorem}[Uniqueness of Inverse]
  If $f: c\to c'$ is an isomorphism, then it has a unique inverse $f^{-1}:c'\to
  c$:

  \begin{proof}
    Consider two inverses of $f$, $f^{-1}_1$ and $f^{-1}_2$, then:
    \[
      \begin{aligned}
        \id_c &= \id_c\\
        f\circ f^{-1}_1 &= f\circ f^{-1}_2\\
        f^{-1}_1 \circ f \circ f^{-1}_1 &= f^{-1}_1 \circ f \circ f^{-1}_2\\
        f^{-1}_1 &= f^{-1}_2
      \end{aligned}
    \]
  \end{proof}
\end{theorem}

\begin{definition}[Monomorphisms\index{Monomorphisms}]
  A morphism $f:c\to c'$ is said to be monic or a monomorphism when it is left
  cancellable \parencite{lane:working_mathematician} i.e. when for any
  set-up of objects and morphisms:
  % https://q.uiver.app/?q=WzAsMyxbMSwwLCJjIl0sWzIsMCwiYyciXSxbMCwwLCJ4Il0sWzAsMSwiZiIsMCx7InN0eWxlIjp7InRhaWwiOnsibmFtZSI6Im1vbm8ifX19XSxbMiwwLCJnIiwwLHsiY3VydmUiOi0zfV0sWzIsMCwiaCIsMix7ImN1cnZlIjozfV1d
  \[\begin{tikzcd}
    x & c & {c'}
    \arrow["f", tail, from=1-2, to=1-3]
    \arrow["g", curve={height=-18pt}, from=1-1, to=1-2]
    \arrow["h"', curve={height=18pt}, from=1-1, to=1-2]
  \end{tikzcd}\]

  The following holds:
  \[f \circ g = f \circ h \implies g = h\]
\end{definition}

\begin{remark}
  $f:b\mono c$ will be used to denote monomorphisms.
\end{remark}

\begin{theorem}[Isomorphism implies Monomorphisms]\label{thm:iso_then_mono}
  If $f:c\to c'$ is an isomorphism, then it is a monomorphism.

  \begin{proof}
    Consider the set-up of objects and morphisms:
    % https://q.uiver.app/?q=WzAsMyxbMSwwLCJjIl0sWzIsMCwiYyciXSxbMCwwLCJ4Il0sWzAsMSwiZiIsMCx7ImN1cnZlIjotM31dLFsyLDAsImciLDAseyJjdXJ2ZSI6LTN9XSxbMiwwLCJoIiwyLHsiY3VydmUiOjN9XSxbMSwwLCJmXnstMX0iLDAseyJjdXJ2ZSI6LTN9XV0=
    \[\begin{tikzcd}
      x & c & {c'}
      \arrow["f", curve={height=-18pt}, from=1-2, to=1-3]
      \arrow["g", curve={height=-18pt}, from=1-1, to=1-2]
      \arrow["h"', curve={height=18pt}, from=1-1, to=1-2]
      \arrow["{f^{-1}}", curve={height=-18pt}, from=1-3, to=1-2]
    \end{tikzcd}\]
    Suppose that $f\circ g = f\circ h$, then:
    \[
      \begin{aligned}
        f\circ g &= f\circ h\\
        f^{-1}\circ f\circ g &= f^{-1}\circ f\circ h\\
        g &= h
      \end{aligned}
    \]
  \end{proof}
\end{theorem}

\begin{definition}[Epimorphism\index{Epimorphism}]
  A morphism $f:c\to c'$ is said to be epic or an epimorphism when it is right
  cancellable \parencite{lane:working_mathematician} i.e. when for any
  set-up of objects and morphisms:
  % https://q.uiver.app/?q=WzAsMyxbMCwwLCJjIl0sWzEsMCwiYyciXSxbMiwwLCJ4Il0sWzEsMiwiaCIsMix7ImN1cnZlIjozfV0sWzAsMSwiZiIsMCx7InN0eWxlIjp7ImhlYWQiOnsibmFtZSI6ImVwaSJ9fX1dLFsxLDIsImciLDAseyJjdXJ2ZSI6LTN9XV0=
  \[\begin{tikzcd}[ampersand replacement=\&]
    c \& {c'} \& x
    \arrow["h"', curve={height=18pt}, from=1-2, to=1-3]
    \arrow["f", two heads, from=1-1, to=1-2]
    \arrow["g", curve={height=-18pt}, from=1-2, to=1-3]
  \end{tikzcd}\]

  The following holds:
  \[g \circ f = h \circ f \implies g = h\]
\end{definition}

\begin{remark}
  $f:a\epi b$ will be used to denote epimorphisms.
\end{remark}

\begin{remark}
  Monomorphism and epimorphism are dual concepts
\end{remark}

\begin{theorem}[Isomorphism implies Epimorphism]\label{thm:iso_then_epi}
  If $f:c\to c'$ is an isomorphism, then it is an epimorphism.

  \begin{proof}
    Consider the set-up of objects and morphisms:
    % https://q.uiver.app/?q=WzAsMyxbMCwwLCJjIl0sWzEsMCwiYyciXSxbMiwwLCJ4Il0sWzAsMSwiZiIsMCx7ImN1cnZlIjotM31dLFsxLDAsImZeey0xfSIsMCx7ImN1cnZlIjotM31dLFsxLDIsImciLDAseyJjdXJ2ZSI6LTN9XSxbMSwyLCJoIiwyLHsiY3VydmUiOjN9XV0=
    \[\begin{tikzcd}
      c & {c'} & x
      \arrow["f", curve={height=-18pt}, from=1-1, to=1-2]
      \arrow["{f^{-1}}", curve={height=-18pt}, from=1-2, to=1-1]
      \arrow["g", curve={height=-18pt}, from=1-2, to=1-3]
      \arrow["h"', curve={height=18pt}, from=1-2, to=1-3]
    \end{tikzcd}\]

    Suppose that $g \circ f = h \circ f$, then:
    \[
      \begin{aligned}
        g \circ f &= h \circ f\\
        g \circ f \circ f^{-1} &= h \circ f \circ f^{-1}\\
        g &= h
      \end{aligned}
    \]
  \end{proof}
\end{theorem}