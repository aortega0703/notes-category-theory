\section{Topos}

\begin{definition}[Topos]\index{Topos}
  An elementary topos is a category such that
  \parencite[p.~84]{goldblatt:topoi}:

  \begin{itemize}
    \item It Cartesian Closed.
    \item It has a subobject classifier.
  \end{itemize}
\end{definition}

\begin{proposition}\label{prop:epicmono_factorization}\index{Epic-Mono Factorization}
  In a topos, every arrow $f:A\to B$ can be factorized as monic after an epic as
  folows \parencite[p.~110]{goldblatt:topoi}:
  % https://q.uiver.app/#q=WzAsMyxbMCwwLCJBIl0sWzIsMCwiQiJdLFsxLDEsIlxcbWF0aHJte0ltfVxcIGYiXSxbMCwxLCJmIl0sWzAsMiwiZl4qIiwyLHsic3R5bGUiOnsiaGVhZCI6eyJuYW1lIjoiZXBpIn19fV0sWzIsMSwiIiwyLHsic3R5bGUiOnsidGFpbCI6eyJuYW1lIjoiaG9vayIsInNpZGUiOiJ0b3AifX19XV0=
  \[\begin{tikzcd}[ampersand replacement=\&]
    A \&\& B \\
    \& {f(A)}
    \arrow["f", from=1-1, to=1-3]
    \arrow["{f^*}"', two heads, from=1-1, to=2-2]
    \arrow[hook, from=2-2, to=1-3]
  \end{tikzcd}\]
\end{proposition}

\subsection{Algebra of Subobjects}

\subsubsection{Negation}

\begin{definition}[False]\index{False}
  The arrow $\bot: 1 \mono \Omega$ that represents false is the characteristic
  function of $!: 0 \to 1$ or $0 \subseteq 1$
  \parencite[p.~117]{goldblatt:topoi}:
  % https://q.uiver.app/#q=WzAsNCxbMCwwLCIwIl0sWzEsMCwiMSJdLFswLDEsIjEiXSxbMSwxLCJcXE9tZWdhIl0sWzEsMywiXFxtYXRocm17XFx0b3B9IiwwLHsic3R5bGUiOnsidGFpbCI6eyJuYW1lIjoibW9ubyJ9fX1dLFswLDEsIiEiXSxbMCwyLCIhIiwyLHsic3R5bGUiOnsidGFpbCI6eyJuYW1lIjoibW9ubyJ9fX1dLFsyLDMsIlxcYm90IiwyXSxbMCwzLCIiLDEseyJzdHlsZSI6eyJuYW1lIjoiY29ybmVyIn19XV0=
  \[\begin{tikzcd}[ampersand replacement=\&]
    0 \& 1 \\
    1 \& \Omega
    \arrow["{\mathrm{\top}}", tail, from=1-2, to=2-2]
    \arrow["{!}", from=1-1, to=1-2]
    \arrow["{!}"', tail, from=1-1, to=2-1]
    \arrow["\bot"', from=2-1, to=2-2]
    \arrow["\lrcorner"{anchor=center, pos=0.125}, draw=none, from=1-1, to=2-2]
  \end{tikzcd}\]
\end{definition}

\begin{remark}
  The characteristic function of $!:0\to 1$ ($0 \subseteq 1$) being regarded as
  false implies that the initial object $0$ should behave as empty-like.
\end{remark}

\begin{definition}[Negation]\index{Negation}
  The unary function $\bot: \Omega\to Omega$ that represents negation is given
  by the characteristic function of $\bot: 1 \to \Omega$ or $0 \subseteq \Omega$
  \parencite[p.~137]{goldblatt:topoi}:
  % https://q.uiver.app/#q=WzAsNCxbMCwwLCIxIl0sWzEsMCwiMSJdLFswLDEsIlxcT21lZ2EiXSxbMSwxLCJcXE9tZWdhIl0sWzEsMywiXFx0b3AiXSxbMCwyLCJcXGJvdCIsMl0sWzIsMywiXFxuZWciLDJdLFswLDEsIiEiXSxbMCwzLCIiLDEseyJzdHlsZSI6eyJuYW1lIjoiY29ybmVyIn19XV0=
  \[\begin{tikzcd}[ampersand replacement=\&]
    1 \& 1 \\
    \Omega \& \Omega
    \arrow["\top", from=1-2, to=2-2]
    \arrow["\bot"', from=1-1, to=2-1]
    \arrow["\neg"', from=2-1, to=2-2]
    \arrow["{!}", from=1-1, to=1-2]
    \arrow["\lrcorner"{anchor=center, pos=0.125}, draw=none, from=1-1, to=2-2]
  \end{tikzcd}\]
\end{definition}

\begin{definition}[Complement]\index{Complement}
  For a subobject $a: A\mono B$, the characteristic function for its complement
  $\neg a:B-A\mono B$ is given by \parencite[p.~146]{goldblatt:topoi}:
  % https://q.uiver.app/#q=WzAsNSxbMCwwLCJCLUEiXSxbMiwwLCIxIl0sWzAsMSwiQiJdLFsyLDEsIlxcT21lZ2EiXSxbMSwxLCJcXE9tZWdhIl0sWzEsMywiXFx0b3AiXSxbMCwyLCJcXGJhcnthfSIsMl0sWzAsMSwiISJdLFsyLDQsIlxcY2hpX2EiLDJdLFs0LDMsIlxcbmVnIiwyXSxbMCwzLCIiLDEseyJzdHlsZSI6eyJuYW1lIjoiY29ybmVyIn19XV0=
  \[\begin{tikzcd}[ampersand replacement=\&]
    {B-A} \&\& 1 \\
    B \& \Omega \& \Omega
    \arrow["\top", from=1-3, to=2-3]
    \arrow["{\bar{a}}"', from=1-1, to=2-1]
    \arrow["{!}", from=1-1, to=1-3]
    \arrow["{\chi_a}"', from=2-1, to=2-2]
    \arrow["\neg"', from=2-2, to=2-3]
    \arrow["\lrcorner"{anchor=center, pos=0.125}, draw=none, from=1-1, to=2-3]
  \end{tikzcd}\]
\end{definition}

\subsubsection{Conjunction}

\begin{definition}[Conjunction]\index{Conjunction}
  Consider the arrow $f = \<\top,\top\>: 1\to \Omega$ constructed as follows:
  % https://q.uiver.app/#q=WzAsNCxbMiwxLCJcXE9tZWdhXFx0aW1lc1xcT21lZ2EiXSxbMiwwLCJcXE9tZWdhIl0sWzIsMiwiXFxPbWVnYSJdLFswLDEsIjEiXSxbMCwxLCJcXHBpXzAiLDJdLFswLDIsIlxccGlfMSJdLFszLDEsIlxcdG9wIiwwLHsic3R5bGUiOnsidGFpbCI6eyJuYW1lIjoibW9ubyJ9fX1dLFszLDIsIlxcdG9wIiwyLHsic3R5bGUiOnsidGFpbCI6eyJuYW1lIjoibW9ubyJ9fX1dLFszLDAsImYiLDEseyJzdHlsZSI6eyJ0YWlsIjp7Im5hbWUiOiJtb25vIn0sImJvZHkiOnsibmFtZSI6ImRhc2hlZCJ9fX1dXQ==
  \[\begin{tikzcd}[ampersand replacement=\&]
    \&\& \Omega \\
    1 \&\& \Omega\times\Omega \\
    \&\& \Omega
    \arrow["{\pi_0}"', from=2-3, to=1-3]
    \arrow["{\pi_1}", from=2-3, to=3-3]
    \arrow["\top", tail, from=2-1, to=1-3]
    \arrow["\top"', tail, from=2-1, to=3-3]
    \arrow["f"{description}, dashed, tail, from=2-1, to=2-3]
  \end{tikzcd}\]

  The binary function $\land$ that represents conjunction is given by the
  characteristic function of the subobject $f$
  \parencite[p.~137]{goldblatt:topoi}, which selects when both operands are
  $\top$:
  % https://q.uiver.app/#q=WzAsNCxbMCwwLCIxIl0sWzEsMCwiMSJdLFswLDEsIlxcT21lZ2FcXHRpbWVzIFxcT21lZ2EiXSxbMSwxLCJcXE9tZWdhIl0sWzEsMywiXFx0b3AiXSxbMCwyLCJmIiwyXSxbMiwzLCJcXGxhbmQiLDJdLFswLDEsIiEiXSxbMCwzLCIiLDEseyJzdHlsZSI6eyJuYW1lIjoiY29ybmVyIn19XV0=
  \[\begin{tikzcd}[ampersand replacement=\&]
    1 \& 1 \\
    {\Omega\times \Omega} \& \Omega
    \arrow["\top", from=1-2, to=2-2]
    \arrow["f"', from=1-1, to=2-1]
    \arrow["\land"', from=2-1, to=2-2]
    \arrow["{!}", from=1-1, to=1-2]
    \arrow["\lrcorner"{anchor=center, pos=0.125}, draw=none, from=1-1, to=2-2]
  \end{tikzcd}\]
\end{definition}

\begin{definition}[Intersection]\index{Intersection}
  For subobjects $a: A\mono C$ and $b: B\mono C$, the characteristic function of
  their intersection $a\cap b:A\cap B \mono C$ is given by
  \parencite[p.~147]{goldblatt:topoi}:
  % https://q.uiver.app/#q=WzAsNSxbMCwwLCJBXFxjYXAgQiJdLFsyLDAsIjEiXSxbMCwxLCJDIl0sWzIsMSwiXFxPbWVnYSJdLFsxLDEsIlxcT21lZ2FcXHRpbWVzXFxPbWVnYSJdLFsxLDMsIlxcdG9wIl0sWzAsMiwiYVxcY2FwIGIiLDJdLFswLDEsIiEiXSxbMiw0LCJcXGxhbmdsZVxcY2hpX2EsXFxjaGlfYlxccmFuZ2xlIiwyXSxbNCwzLCJcXGxhbmQiLDJdLFswLDMsIiIsMSx7InN0eWxlIjp7Im5hbWUiOiJjb3JuZXIifX1dXQ==
  \[\begin{tikzcd}[ampersand replacement=\&]
    {A\cap B} \&\& 1 \\
    C \& \Omega\times\Omega \& \Omega
    \arrow["\top", from=1-3, to=2-3]
    \arrow["{a\cap b}"', from=1-1, to=2-1]
    \arrow["{!}", from=1-1, to=1-3]
    \arrow["{\langle\chi_a,\chi_b\rangle}"', from=2-1, to=2-2]
    \arrow["\land"', from=2-2, to=2-3]
    \arrow["\lrcorner"{anchor=center, pos=0.125}, draw=none, from=1-1, to=2-3]
  \end{tikzcd}\]
\end{definition}

\subsubsection{Disjunction}

\begin{definition}[Disjunction]\index{Disjunction}
  Consider the arrow $f = [\<\top\circ\ !,\id_\Omega\>, \<\id_\Omega,\top\circ\
  !\>]: \Omega + \Omega \to \Omega \times \Omega$ constructed as follows:
  % https://q.uiver.app/#q=WzAsNCxbMCwwLCJcXE9tZWdhIl0sWzIsMSwiXFxPbWVnYVxcdGltZXNcXE9tZWdhIl0sWzAsMiwiXFxPbWVnYSJdLFswLDEsIlxcT21lZ2ErXFxPbWVnYSJdLFswLDEsIlxcbGFuZ2xlXFx0b3BcXGNpcmNcXCAhLFxcaWRfXFxPbWVnYVxccmFuZ2xlIl0sWzIsMSwiXFxsYW5nbGVcXGlkX1xcT21lZ2EsXFx0b3BcXGNpcmNcXCAhXFxyYW5nbGUiLDJdLFswLDMsImlfMCIsMl0sWzIsMywiaV8xIl0sWzMsMSwiZiIsMSx7InN0eWxlIjp7ImJvZHkiOnsibmFtZSI6ImRhc2hlZCJ9fX1dXQ==&macro_url=https%3A%2F%2Fgist.githubusercontent.com%2Faortega0703%2Fa1fd97cb097b8142e63a6fbf0cdb0f76%2Fraw%2Fb46a955b6b1f06908105b870088e59a10049fc60%2Fmacros.tex
  \[\begin{tikzcd}[ampersand replacement=\&]
    \Omega \\
    {\Omega+\Omega} \&\& \Omega\times\Omega \\
    \Omega
    \arrow["{\langle\top\circ\ !,\id_\Omega\rangle}", from=1-1, to=2-3]
    \arrow["{\langle\id_\Omega,\top\circ\ !\rangle}"', from=3-1, to=2-3]
    \arrow["{i_0}"', from=1-1, to=2-1]
    \arrow["{i_1}", from=3-1, to=2-1]
    \arrow["f"{description}, dashed, from=2-1, to=2-3]
  \end{tikzcd}\]

  Factorizing $f$ using \ref{prop:epicmono_factorization} leads to a subobject
  $i: f(\Omega+\Omega)\mono \Omega\times \Omega$ as follows:
  % https://q.uiver.app/#q=WzAsMyxbMCwwLCJcXE9tZWdhK1xcT21lZ2EiXSxbMiwwLCJcXE9tZWdhXFx0aW1lc1xcT21lZ2EiXSxbMSwxLCJcXG1hdGhybXtJbX1cXCBmIl0sWzAsMiwiZl4qIiwyLHsic3R5bGUiOnsiaGVhZCI6eyJuYW1lIjoiZXBpIn19fV0sWzAsMSwiZiJdLFsyLDEsImkiLDIseyJzdHlsZSI6eyJ0YWlsIjp7Im5hbWUiOiJtb25vIn19fV1d
  \[\begin{tikzcd}[ampersand replacement=\&]
    {\Omega+\Omega} \&\& \Omega\times\Omega \\
    \& {f(\Omega+\Omega)}
    \arrow["{f^*}"', two heads, from=1-1, to=2-2]
    \arrow["f", from=1-1, to=1-3]
    \arrow["i"', tail, from=2-2, to=1-3]
  \end{tikzcd}\]

  The binary function $\lor$ that represents disjuction is given by the
  characteristic function of $i$ \parencite[p.~138]{goldblatt:topoi}, which
  selects when either operand is $\top$:
  % https://q.uiver.app/#q=WzAsNCxbMCwwLCJmKFxcT21lZ2ErXFxPbWVnYSkiXSxbMSwwLCIxIl0sWzEsMSwiXFxPbWVnYSJdLFswLDEsIlxcT21lZ2FcXHRpbWVzXFxPbWVnYSJdLFswLDEsIiEiXSxbMSwyLCJcXHRvcCJdLFszLDIsIlxcbG9yIiwyXSxbMCwyLCIiLDEseyJzdHlsZSI6eyJuYW1lIjoiY29ybmVyIn19XSxbMCwzLCJpIiwyXV0=
  \[\begin{tikzcd}[ampersand replacement=\&]
    {f(\Omega+\Omega)} \& 1 \\
    \Omega\times\Omega \& \Omega
    \arrow["{!}", from=1-1, to=1-2]
    \arrow["\top", from=1-2, to=2-2]
    \arrow["\lor"', from=2-1, to=2-2]
    \arrow["\lrcorner"{anchor=center, pos=0.125}, draw=none, from=1-1, to=2-2]
    \arrow["i"', from=1-1, to=2-1]
  \end{tikzcd}\]
\end{definition}

\begin{definition}[Union]\index{Union}
  For subobjects $a:A\mono C$ and $b:B\mono C$, the characteristic function of
  their union $a\cup b:A\cup B \mono C$ is given by
  \parencite[p.~147]{goldblatt:topoi}:
  % https://q.uiver.app/#q=WzAsNSxbMCwwLCJBXFxjdXAgQiJdLFsyLDAsIjEiXSxbMiwxLCJcXE9tZWdhIl0sWzEsMSwiXFxPbWVnYVxcdGltZXNcXE9tZWdhIl0sWzAsMSwiQyJdLFswLDEsIiEiXSxbMSwyLCJcXHRvcCJdLFszLDIsIlxcbG9yIiwyXSxbMCw0LCJhXFxjdXAgYiIsMl0sWzQsMywiXFxsYW5nbGVcXGNoaV9hLCBcXGNoaV9iXFxyYW5nbGUiLDJdLFswLDIsIiIsMSx7InN0eWxlIjp7Im5hbWUiOiJjb3JuZXIifX1dXQ==
  \[\begin{tikzcd}[ampersand replacement=\&]
    {A\cup B} \&\& 1 \\
    C \& \Omega\times\Omega \& \Omega
    \arrow["{!}", from=1-1, to=1-3]
    \arrow["\top", from=1-3, to=2-3]
    \arrow["\lor"', from=2-2, to=2-3]
    \arrow["{a\cup b}"', from=1-1, to=2-1]
    \arrow["{\langle\chi_a, \chi_b\rangle}"', from=2-1, to=2-2]
    \arrow["\lrcorner"{anchor=center, pos=0.125}, draw=none, from=1-1, to=2-3]
  \end{tikzcd}\]
\end{definition}

\subsubsection{Implication}

\begin{definition}[Implication]\index{Implication}
  In any lattice $x \sqsubseteq y \iff x \sqcap y = x$. Therefore consider the
  subobject $i: \leq\mono \Omega\times\Omega$ that equalizes $\land$ and
  $\pi_0$:
  % https://q.uiver.app/#q=WzAsMyxbMiwwLCJcXE9tZWdhIl0sWzEsMCwiXFxPbWVnYVxcdGltZXNcXE9tZWdhIl0sWzAsMCwiXFxsZXEiXSxbMSwwLCJcXGxhbmQiLDAseyJvZmZzZXQiOi0xfV0sWzEsMCwiXFxwaV8wIiwyLHsib2Zmc2V0IjoxfV0sWzIsMSwiaSIsMCx7InN0eWxlIjp7InRhaWwiOnsibmFtZSI6Im1vbm8ifX19XV0=
  \[\begin{tikzcd}[ampersand replacement=\&]
    \leq \& \Omega\times\Omega \& \Omega
    \arrow["\land", shift left=1, from=1-2, to=1-3]
    \arrow["{\pi_0}"', shift right=1, from=1-2, to=1-3]
    \arrow["i", tail, from=1-1, to=1-2]
  \end{tikzcd}\]

  The binary function $\implies$ that represents implication is given by the
  characteristic function of $i$ \parencite[p.~137]{goldblatt:topoi}:
  % https://q.uiver.app/#q=WzAsNCxbMCwxLCJcXE9tZWdhXFx0aW1lc1xcT21lZ2EiXSxbMCwwLCJcXGxlcSJdLFsxLDEsIlxcT21lZ2EiXSxbMSwwLCIxIl0sWzEsMCwiaSIsMix7InN0eWxlIjp7InRhaWwiOnsibmFtZSI6Im1vbm8ifX19XSxbMSwzLCIhIl0sWzMsMiwiXFx0b3AiXSxbMCwyLCJcXGltcGxpZXMiLDJdLFsxLDIsIiIsMSx7InN0eWxlIjp7Im5hbWUiOiJjb3JuZXIifX1dXQ==
  \[\begin{tikzcd}[ampersand replacement=\&]
    \leq \& 1 \\
    \Omega\times\Omega \& \Omega
    \arrow["i"', tail, from=1-1, to=2-1]
    \arrow["{!}", from=1-1, to=1-2]
    \arrow["\top", from=1-2, to=2-2]
    \arrow["\implies"', from=2-1, to=2-2]
    \arrow["\lrcorner"{anchor=center, pos=0.125}, draw=none, from=1-1, to=2-2]
  \end{tikzcd}\]
\end{definition}

\begin{definition}[Universal Quantificator]\index{Topos!Universal
Quantificator}
  Consider the maximal subobject of $\id: A\to A$, the name of its
  characteristic function is then given by:
  % https://q.uiver.app/#q=WzAsNCxbMCwwLCJBIl0sWzAsMSwiQSJdLFsxLDAsIjEiXSxbMSwxLCJcXE9tZWdhIl0sWzAsMSwiXFxpZF9BIiwyLHsic3R5bGUiOnsidGFpbCI6eyJuYW1lIjoibW9ubyJ9fX1dLFswLDIsIiEiXSxbMiwzLCJcXHRvcCIsMCx7InN0eWxlIjp7InRhaWwiOnsibmFtZSI6Im1vbm8ifX19XSxbMSwzLCJcXHRvcF9BIiwyXSxbMCwzLCIiLDEseyJzdHlsZSI6eyJuYW1lIjoiY29ybmVyIn19XV0=&macro_url=https%3A%2F%2Fgist.githubusercontent.com%2Faortega0703%2Fa1fd97cb097b8142e63a6fbf0cdb0f76%2Fraw%2Fb08e21d744ff9df9903ede35f3944ad2f54020e7%2Fmacros.tex
  \[\begin{tikzcd}[ampersand replacement=\&]
    A \& 1 \\
    A \& \Omega
    \arrow["{\id_A}"', tail, from=1-1, to=2-1]
    \arrow["{!}", from=1-1, to=1-2]
    \arrow["\top", tail, from=1-2, to=2-2]
    \arrow["{\top_A}"', from=2-1, to=2-2]
    \arrow["\lrcorner"{anchor=center, pos=0.125}, draw=none, from=1-1, to=2-2]
  \end{tikzcd}\]

  Then, consider the element $\name{\top_A}: 1\to \Omega^A$ which selects the
  maximal subobject from the power object $\Omega^A$. The characteristic
  function of this element which will be called $\forall_A:\Omega^A\to\Omega$
  \parencite[p.~121]{mclarty:elementary}:
  % https://q.uiver.app/#q=WzAsNCxbMCwwLCIxIl0sWzAsMSwiXFxPbWVnYV5BIl0sWzEsMCwiMSJdLFsxLDEsIlxcT21lZ2EiXSxbMCwxLCJcXG5hbWV7XFx0b3BfQX0iLDIseyJzdHlsZSI6eyJ0YWlsIjp7Im5hbWUiOiJtb25vIn19fV0sWzAsMiwiISJdLFsyLDMsIlxcdG9wIiwwLHsic3R5bGUiOnsidGFpbCI6eyJuYW1lIjoibW9ubyJ9fX1dLFsxLDMsIlxcZm9yYWxsX0EiLDJdLFswLDMsIiIsMSx7InN0eWxlIjp7Im5hbWUiOiJjb3JuZXIifX1dXQ==&macro_url=https%3A%2F%2Fgist.githubusercontent.com%2Faortega0703%2Fa1fd97cb097b8142e63a6fbf0cdb0f76%2Fraw%2Fb08e21d744ff9df9903ede35f3944ad2f54020e7%2Fmacros.tex
  \[\begin{tikzcd}[ampersand replacement=\&]
    1 \& 1 \\
    {\Omega^A} \& \Omega
    \arrow["{\name{\top_A}}"', tail, from=1-1, to=2-1]
    \arrow["{!}", from=1-1, to=1-2]
    \arrow["\top", tail, from=1-2, to=2-2]
    \arrow["{\forall_A}"', from=2-1, to=2-2]
    \arrow["\lrcorner"{anchor=center, pos=0.125}, draw=none, from=1-1, to=2-2]
  \end{tikzcd}\]
\end{definition}

\begin{definition}[Universal Quantification]
  Let the subobject $r: R\mono B\times A$ be a relation. The arrow
  $\overline{\chi_r}: B \to \Omega^A$ sends every element in $B$ to the arrow
  classifying the subobject of all whose are related to it. Then the composite
  $\forall_A \circ \overline{\chi_r}$ classifies the largest subobject of $B$
  whose elements are related to all of $A$
  \parencite[p.~121]{mclarty:elementary}
  % https://q.uiver.app/#q=WzAsNSxbMiwxLCJcXE9tZWdhIl0sWzAsMSwiQiJdLFsxLDEsIlxcT21lZ2FeQSJdLFsyLDAsIjEiXSxbMCwwLCJcXGZvcmFsbF9hLlIiXSxbMSwyLCJcXG92ZXJsaW5le1xcY2hpX3J9IiwyXSxbMiwwLCJcXGZvcmFsbF9BIiwyXSxbMywwLCJcXHRvcCIsMCx7InN0eWxlIjp7InRhaWwiOnsibmFtZSI6Im1vbm8ifX19XSxbNCwxLCJcXGZvcmFsbCBhLnIiLDIseyJzdHlsZSI6eyJ0YWlsIjp7Im5hbWUiOiJtb25vIn19fV0sWzQsMywiISJdLFs0LDAsIiIsMSx7InN0eWxlIjp7Im5hbWUiOiJjb3JuZXIifX1dXQ==&macro_url=https%3A%2F%2Fgist.githubusercontent.com%2Faortega0703%2Fa1fd97cb097b8142e63a6fbf0cdb0f76%2Fraw%2Fb08e21d744ff9df9903ede35f3944ad2f54020e7%2Fmacros.tex
  \[\begin{tikzcd}[ampersand replacement=\&]
    {\forall_a.R} \&\& 1 \\
    B \& {\Omega^A} \& \Omega
    \arrow["{\overline{\chi_r}}"', from=2-1, to=2-2]
    \arrow["{\forall_A}"', from=2-2, to=2-3]
    \arrow["\top", tail, from=1-3, to=2-3]
    \arrow["{\forall a.r}"', tail, from=1-1, to=2-1]
    \arrow["{!}", from=1-1, to=1-3]
    \arrow["\lrcorner"{anchor=center, pos=0.125}, draw=none, from=1-1, to=2-3]
  \end{tikzcd}\]
\end{definition}

\begin{definition}[Existential Quantificator]\index{Topos!Existential
  Quantificator} Consider the inclusion subobject
  $\in_a:\in_A\mono\Omega^A\times A$. The mono part of the epic-mono
  factorization of $f=\pi_0\circ \in_a:\in_A\to \Omega^A$ selects all subobjects
  which have a element in them. The characteristic function of such a subobject
  is the existential quantificator $\exists_A:\Omega^A \to \Omega$
  \parencite[p.~245]{goldblatt:topoi}:
  % https://q.uiver.app/#q=WzAsNixbMCwxLCJcXE9tZWdhXkFcXHRpbWVzIEEiXSxbMSwxLCJcXE9tZWdhXkEiXSxbMCwwLCJcXGluX0EiXSxbMSwwLCJmKFxcaW5fQSkiXSxbMiwwLCIxIl0sWzIsMSwiXFxPbWVnYSJdLFswLDEsIlxccGlfMCIsMl0sWzIsMCwiXFxpbl9hIiwyLHsic3R5bGUiOnsidGFpbCI6eyJuYW1lIjoibW9ubyJ9fX1dLFsyLDMsIiIsMCx7InN0eWxlIjp7ImhlYWQiOnsibmFtZSI6ImVwaSJ9fX1dLFszLDEsIiIsMCx7InN0eWxlIjp7InRhaWwiOnsibmFtZSI6Im1vbm8ifX19XSxbNCw1LCJcXHRvcCIsMCx7InN0eWxlIjp7InRhaWwiOnsibmFtZSI6Im1vbm8ifX19XSxbMyw0LCIhIl0sWzEsNSwiXFxleGlzdHNfQSIsMl0sWzMsNSwiIiwwLHsic3R5bGUiOnsibmFtZSI6ImNvcm5lciJ9fV1d
  \[\begin{tikzcd}[ampersand replacement=\&]
    {\in_A} \& {f(\in_A)} \& 1 \\
    {\Omega^A\times A} \& {\Omega^A} \& \Omega
    \arrow["{\pi_0}"', from=2-1, to=2-2]
    \arrow["{\in_a}"', tail, from=1-1, to=2-1]
    \arrow[two heads, from=1-1, to=1-2]
    \arrow[tail, from=1-2, to=2-2]
    \arrow["\top", tail, from=1-3, to=2-3]
    \arrow["{!}", from=1-2, to=1-3]
    \arrow["{\exists_A}"', from=2-2, to=2-3]
    \arrow["\lrcorner"{anchor=center, pos=0.125}, draw=none, from=1-2, to=2-3]
  \end{tikzcd}\]
\end{definition}

\subsection{Classification}

\begin{definition}[Bivalent Topos]\index{Topos!Bivalent}
  A topos is bivalent if $\top$ and $\bot$ are the only elements of $\Omega$
  \parencite[p.~118]{goldblatt:topoi}.
\end{definition}

\begin{definition}[Well-Pointed Topos]\index{Topos!Well-Pointed}
  A topos is well pointed if it satisfies extensionality
  \parencite[p.~116]{goldblatt:topoi}, i.e. for any pair of distinct arrows
  $f,g:A\to B$ there exists an element $x:1\to A$ such that $f\circ x \neq
  g\circ x$.
\end{definition}

\begin{proposition}
  If a topos is well pointed, then it is bivalent
  \parencite[p.~118]{goldblatt:topoi}
\end{proposition}

\begin{definition}[Classical Topos]\index{Topos!Classical}\index{Topos!Boolean}
  A topos is Classical (or boolean) if $[\top, \bot]:1+1 \to \Omega$ is an
  isomorphism \parencite[p.~118, p.~156]{goldblatt:topoi}
\end{definition}

\begin{definition}[Disjunctive Topos]
  A topos is called disjunctive if for any truth values $x,y:1\to \Omega$,
  \[\lor \circ \<x, y\> = \top \text{ iff } x=\top \text{ or } y=\top\]
  \parencite[p.~171]{goldblatt:topoi}.
\end{definition}

\begin{proposition}
  Every Bivalent topos is disjunctive.
\end{proposition}

\begin{definition}[Axiom of Choice]
  Every epic $f:A\epi B$ has a section $s:B\to A$, i.e. $f\circ s= \id_B$
  \parencite[p.~290]{goldblatt:topoi}.
\end{definition}

\begin{proposition}
  If the axiom of choice holds for a topos, then it is boolean
  \parencite[p.~163]{mclarty:elementary}.
\end{proposition}

