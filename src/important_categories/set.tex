\subsection{Set}
\begin{definition}[Set]
  Set is the category of all sets, composed by:
  \parencite{awodey:category_theory}
  \begin{itemize}
    \item Objects:\\
      $\ob_{\mathrm{Set}}$: The collection of all sets.
    \item Morphisms:\\
      $\hom_{\mathrm{Set}}(c, c')$: The set of all functions between sets $c$ and
      $c'$.
  \end{itemize}
\end{definition}

\begin{definition}[Injective Function]
  For sets $A,B$, a function $f:A\to B$ is said to be injective if the following
  holds:
  \[f(a)=f(b) \implies a=b,\ (\forall a\in A,\ b\in B)\]
\end{definition}

\begin{theorem}[Monomorphisms in Set]\label{thm:mono_iff_inj}
  In Set a morphism is monic if and only if it is injective.

  \begin{proof}
    The proof consists of two parts:
    \begin{description}
      \item[$(\implies)$] Consider a non-injective function $f:A\to B$, i.e.
        there exists a pair of different elements $x,y\in A$ such that
        $f(x)=f(y)$. Let:
        \[
          \begin{aligned}
            \bar{x} : \{\bullet\} &\to A\\
            \bullet &\mapsto x
          \end{aligned}
          \quad
          \begin{aligned}
            \bar{y} : \{\bullet\} &\to A\\
            \bullet &\mapsto y
          \end{aligned}
        \]
        \[
          \begin{aligned}
            f(x) &= f(y)\\
            (f\circ\bar{x})(\bullet) &= (f\circ\bar{y})(\bullet)\\
            (f\circ\bar{x}) &= (f\circ\bar{y})
          \end{aligned}
        \]
        Furthermore, $f$ cannot bet monic as it would imply $\bar{x} = \bar{y}$.
        Then by contrapositive, if $f$ is monic then it is injective.
      \item Consider an injective function $f:A\to B$. Let $X$ be another set
        with a pair of distinct functions $g, h:X\to A$.
        \[
          \begin{aligned}
            (\exists x\in X)(g(x) &\neq h(x))\\
            (\exists x\in X)((f\circ g)(x) &\neq (f\circ h)(x))
            ,\quad\text{by injectivity of $f$}\\
            f\circ g &\neq f\circ h
          \end{aligned}
        \]
        Then by contrapositive, $f\circ g = f\circ h \implies g = h$, implying
        that $f$ is monic.
    \end{description}
  \end{proof}
\end{theorem}

\begin{definition}[Surjective Function]
  For sets $A,B$, a function $f:A\to B$ is said to be surjective if the
  following holds:
  \[(\forall b\in B)(\exists a\in A).(f(a)=b)\]
\end{definition}

\begin{theorem}[Epimorphisms in Set]\label{thm:epi_iff_sur}
  In Set a morphism is epic if and only if it is surjective.

  \begin{proof}
    The proof consists of two parts:
    \begin{description}
      \item[$(\implies)$] Consider a non-surjective function $f:A\to B$, i.e.
        for some $x\in B$ there exists no $a\in A$ such that $f(a) = x$. Let:
        \[
          \begin{aligned}[t]
            g:B &\to B \cup \{B\}\\
            b &\mapsto b
          \end{aligned}
          \quad
          \begin{aligned}[t]
            h:B &\to B \cup \{B\}\\
            b &\mapsto
            \begin{cases}
              \begin{aligned}
                B &\text{ if } b = x\\
                b &\text{ otherwise}
              \end{aligned}
            \end{cases}
          \end{aligned}
        \]
        Note that $g(b) = h(b)$ for $b\neq x$, particularly, as $f(a)\neq b$ for
        any $a$ then $(g\circ f)(a) = (h\circ f)(a)$ i.e. $g\circ f = h\circ f$.
        Furthermore, $f$ cannot be epic since that would imply $g=f$. Then by
        contrapositive, if $f$ is epic then it is a surjection.
      \item[$(\impliedby)$] Consider a surjective function $f:A\to B$. Let $X$
        be a set with two distinct functions $g, h: B \to X$.
        \[
          \begin{aligned}
            (\exists b\in B)(g(b) &\neq h(b))\\
            (\exists a\in A)((g\circ f)(a) &\neq (h\circ f)(a))
            ,\quad\text{(by surjectivity of $f$)}\\
            g\circ f &\neq h\circ f
          \end{aligned}
        \]
        Then, by contrapositive $g\circ f = h\circ f \implies g=h$, implying
        that $f$ is epic.
    \end{description}
  \end{proof}
\end{theorem}

\begin{definition}[Bijective Function]
  A function is said to be bijective if it is both injective and surjective
\end{definition}

\begin{theorem}[Isomorphisms in Set]
  In the category Set an arrow is an isomorphism if and only if it is
  bijective.

  \begin{proof}
    The proof consists of two parts:

    \begin{description}
      \item[$(\implies)$] By Qusing Theorems \ref{thm:iso_then_mono},
        \ref{thm:iso_then_epi}, \ref{thm:mono_iff_inj}, and
        \ref{thm:epi_iff_sur} we get that any isomorphism is also a bijective
        function.
      \item[$(\impliedby)$] Consider a bijection $f:A\to B$. Then define
        $f^{-1}$ as follows. Since $f$ is injective, every $f(a)\in B$ has a
        unique pre-image $a\in A$, then let $(f^{-1}\circ f)(a) = a$. This
        implies that the range of $f^{-1}$ is precisely the domain of $f$ and
        its domain, the range of $f$. Since $f$ is surjective, its range is
        equal to its codomain, making $f:B\to A$. Let $a\in A$ and $b=f(a)$,
        which by definition implies $f^{-1}(b)=a$, then:
        \[
          \begin{aligned}
            (f^{-1}\circ f)(a)
            = f^{-1}(b)\\
            &= a\\
            f^{-1}\circ f &= \id_A
          \end{aligned}
        \]
        Similarly, let $b\in B$ and $a=f^{-1}(b)$, which by definition implies
        $f(a)=b$, then:
        \[
          \begin{aligned}
            (f\circ f^{-1})(b)
            &= f(a)\\
            &= b\\
            (f\circ f^{-1}) &= \id_B
          \end{aligned}
        \]
    \end{description}
  \end{proof}
\end{theorem}