\subsection{Set}
\begin{definition}[Set]
  Set is the category of all sets, composed by:
  \parencite{awodey:category_theory}
  \begin{itemize}
    \item Objects:\\
      $\ob_{\mathrm{Set}}$: The collection of all sets.
    \item Morphisms:\\
      $\hom_{\mathrm{Set}}(c, c')$: The set of all functions between sets $c$ and
      $c'$.
  \end{itemize}
\end{definition}

\begin{definition}[Injective Function]
  For sets $A,B$, a function $f:A\to B$ is said to be injective if the following
  holds:
  \[f(a)=f(b) \implies a=b,\ (\forall a\in A,\ b\in B)\]
\end{definition}

\begin{definition}[Surjective Function]
  For sets $A,B$, a function $f:A\to B$ is said to be surjective if the
  following holds:
  \[(\forall b\in B)(\exists a\in A).(f(a)=b)\]
\end{definition}

\begin{definition}[Bijective Function]
  A function is said to be bijective if it is both injective and surjective
\end{definition}

\begin{theorem}[Monomorphisms in Set]
  In Set a morphism is mono if and only if it is injective.

  \begin{proof}
    The proof consists of two parts:
    \begin{description}
      \item[$(\implies)$] Consider a non-injective function $f:A\to B$, i.e.
        there exists a pair of different elements $a,a'\in A$ such that
        $f(a)=f(a')$. Additionally, $f$ cannot be a monomorphism as it would
        imply that $a=a'$, then:
        \[\text{Non-injective} \implies \text{Non-monic}\]
        Or by contrapositive:
        \[\text{Monic} \implies \text{Injective}\]
      \item[$(\impliedby)$] Consider a non-injective function $f:A\to B$, i.e.
        there exists a pair of different elements $a,a'\in A$ such that
        $f(a)=f(a')$. Let $X$ be another set with a pair of functions $g,
        h:X\to A$ that differ only in that for some $x\in X$, $g(x)=a'$ and
        $h(x)=a$ so that $f\circ g = f\circ h$, then:
        \[\text{Non-Injective}(f:B\to C):\ (\exists X)(\exists g,h:X\to
        B).(g \neq h \land f\circ g = f\circ h)\] Then by negating the logic:
        \[\text{Injective}(f:B\to C):\ (\forall X)(\forall g, h:X\to B).(f\circ
        g = f\circ h \implies g=h)\]
    \end{description}
  \end{proof}
\end{theorem}

\begin{theorem}[Epimorphisms in Set]
  In Set a morphism is epi if and only if it is surjective.

  \begin{proof}
    The proof consists of two parts:
    \begin{description}
      \item[$(\implies)$] TODO
      \item[$(\impliedby)$] Consider a non-surjective function $f:A\to B$, i.e.
        for some $b\in B$ there exists no $a\in A$ such that $f(a) = b$. Let $X$
        be another set with a pair of functions $g, h:B\to X$ that differ only
        in how they map such $b$ so that $g\circ f = h\circ f$, then:
        \[\text{Non-Surjective}(f:a\to b):\ (\exists g, g':b\to c).(g \neq
        g' \land g\circ f = g'\circ f)\]
        Or negating the logic:
        \[\text{Surjective}(f:a\to b):\ (\forall g, g':b\to c).(g\circ f =
          g'\circ f \implies g=g')\]
    \end{description}
  \end{proof}
\end{theorem}

\begin{theorem}[Isomorphisms in Set]
  In the category Set an arrow is an isomorphism if and only if it is
  bijective.

  \begin{proof}
    The proof consists of two parts:
    \begin{description}
      \item[$(\implies)$] For sets $A,B$ consider a isomorphism $f:A\to B$ with
        its respective inverse being $f^{-1}: A\to B$. Consider some $b\in B$
        and let $a = f^{-1}(b)$, then:
        \[
          \begin{aligned}
            a &= f^{-1}(b)\\
            f(a) &= (f\circ f^{-1})(b)\\
            &= b
          \end{aligned}
        \]
      \item[$(\impliedby)$]
    \end{description}
  \end{proof}
\end{theorem}