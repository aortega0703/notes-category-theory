\section{Colimits}

\begin{definition}[Colimit\index{Colimit}\label{def:colimit}]
	The colimit of a functor $F:\C\to \D$ is a cocone with vertex $d\in \D_0$ and
	morphisms $\big(\alpha_c: F(c)\to d\big)_{c\in \C_0}$ such that for any other
	cone with vertex $d'\in \D_0$ and morphisms $\big(\beta_c: F(c)\to
	d'\big)_{c\in \C_0}$ there exists a unique map $f:d\to d'$ such that the
	following diagram commutes for all $c\in
	\C_0$~\parencite[p.~126]{leinster:basic_category_theory}:
	% https://q.uiver.app/?q=WzAsMyxbMCwwLCJkIl0sWzAsMSwiZCciXSxbMSwwLCJGKGMpIl0sWzIsMCwiXFxhbHBoYV9jIiwyXSxbMiwxLCJcXGJldGFfYyJdLFswLDEsImYiLDIseyJzdHlsZSI6eyJib2R5Ijp7Im5hbWUiOiJkYXNoZWQifX19XV0=
	\[\begin{tikzcd}[ampersand replacement=\&]
		d \& {F(c)} \\
		{d'}
		\arrow["{\alpha_c}"', from=1-2, to=1-1]
		\arrow["{\beta_c}", from=1-2, to=2-1]
		\arrow["f"', dashed, from=1-1, to=2-1]
	\end{tikzcd}\]
\end{definition}

\begin{definition}[Colimit as an Initial Object\index{Colimit!as an Initial Object}]
	Equivalently to Definition \ref{def:limit}, the colimit of a functor $F:\I\to
	\C$ can be expressed as an initial object in the category
	$\mathrm{cocone}(F)$.
\end{definition}

\begin{definition}[Colimit as an Initial Morphism\index{Colimit!as an Initial Morphism}]
	Equivalently to Definition \ref{def:limit}, the colimit of a functor $F:\I\to
	\C$ can be expressed as an initial morphism from $F$ (thought of as an object in $\C^\I$) to the diagonal functor $\Delta:\C\to\C^\I$:
	% https://q.uiver.app/?q=WzAsNSxbMCwwLCJjIl0sWzAsMSwiYyciXSxbMSwwLCJcXERlbHRhX2MiXSxbMSwxLCJcXERlbHRhX3tjJ30iXSxbMiwwLCJGIl0sWzAsMSwiaCIsMix7InN0eWxlIjp7ImJvZHkiOnsibmFtZSI6ImRhc2hlZCJ9fX1dLFsyLDMsIlxcRGVsdGEoaCkiLDIseyJzdHlsZSI6eyJib2R5Ijp7Im5hbWUiOiJkYXNoZWQifX19XSxbNCwzLCJcXGJldGEiXSxbNCwyLCJcXGFscGhhIiwyXV0=&macro_url=https%3A%2F%2Fraw.githubusercontent.com%2Faortega0703%2Fnotes-category-theory%2Fmain%2Fsrc%2Fmacros.tex
	\[\begin{tikzcd}[ampersand replacement=\&]
		c \& {\Delta_c} \& F \\
		{c'} \& {\Delta_{c'}}
		\arrow["h"', dashed, from=1-1, to=2-1]
		\arrow["{\Delta(h)}"', dashed, from=1-2, to=2-2]
		\arrow["\beta", from=1-3, to=2-2]
		\arrow["\alpha"', from=1-3, to=1-2]
	\end{tikzcd}\]
\end{definition}

\begin{theorem}[Uniqueness of Colimits\label{thm:colimit_iso}]
	If there exists multiple colimits for the same functor, then they are all
	isomorphic.

	\begin{proof}
		A colimit can be seen as an initial morphism, which are unique up to
		Isomorphism (Theorem \ref{thm:initial_morphism_iso}), therefore colimits are
		also unique up to isomorphism.
	\end{proof}
\end{theorem}

\subimport{coproduct/}{coproduct.tex}

\subsection{Coequalizer}

\begin{definition}[Coequalizer]\index{Coequalizer}

  A coequalizer is the colimit of a
  diagram~\parencite[p.~128]{leinster:basic_category_theory}:
  % https://q.uiver.app/?q=WzAsMixbMCwwLCJ4Il0sWzEsMCwieSJdLFswLDEsImYiLDAseyJvZmZzZXQiOi0xfV0sWzAsMSwiZyIsMix7Im9mZnNldCI6MX1dXQ==
  \[\begin{tikzcd}[ampersand replacement=\&]
    x \& y
    \arrow["f", shift left=1, from=1-1, to=1-2]
    \arrow["g"', shift right=1, from=1-1, to=1-2]
  \end{tikzcd}\]
\end{definition}

\subsection{Pushouts}

\begin{definition}[Pushout]\index{Pushout}\label{def:pushout}

  A pushout is the colimit of a
  diagram~\parencite[p.~130]{leinster:basic_category_theory}:
  % https://q.uiver.app/?q=WzAsMyxbMiwwLCJ6Il0sWzAsMCwieSJdLFsxLDAsIngiXSxbMiwwLCJ0Il0sWzIsMSwicyIsMl1d
  \[\begin{tikzcd}[ampersand replacement=\&]
    y \& x \& z
    \arrow["t", from=1-2, to=1-3]
    \arrow["s"', from=1-2, to=1-1]
  \end{tikzcd}\]
\end{definition}