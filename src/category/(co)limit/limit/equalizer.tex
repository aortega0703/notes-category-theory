\subsection{Equalizer}

\begin{definition}[Equalizer]\index{Equalizer}
  An equalizer is the limit of a
  diagram~\parencite[p.~112]{leinster:basic_category_theory}:
  % https://q.uiver.app/#q=WzAsMixbMCwwLCJYIl0sWzEsMCwiWSJdLFswLDEsImYiLDAseyJvZmZzZXQiOi0xfV0sWzAsMSwiZyIsMix7Im9mZnNldCI6MX1dXQ==
  \[\begin{tikzcd}[ampersand replacement=\&]
    X \& Y
    \arrow["f", shift left=1, from=1-1, to=1-2]
    \arrow["g"', shift right=1, from=1-1, to=1-2]
  \end{tikzcd}\]
\end{definition}

\begin{remark}
  An equalizer of objects $A, B$ and morphisms $f, g: A\to B$ is composed by an
  object $E$ together with a morphism $i:E\to A$.
\end{remark}

\begin{theorem}
  Every equalizer morphism is a monomorphism.

  \begin{proof}
    Consider objects $A, B$ and morphisms $f, g: A\to B$ with their respective
    equalizer $\<E, i\>$. Now consider an object $X$ with morphisms $x_0, x_1: X
    \to E$ such that $i\circ x_0 = i\circ x_1$, then:
    \[
      \begin{aligned}
        f \circ i \circ x_0
        &= f \circ i \circ x_1\\
        &= g \circ i \circ x_1
      \end{aligned}
    \]

    Finally by the universal property of equalizers $i\circ x_0$ (and $i\circ
    x_1$) must be uniquely factorized as $i\circ k$ for some $k$. As this
    factorization is unique, it must be the case that $x_0 = x_1 = k$.
  \end{proof}
\end{theorem}