\subsection{Product}

\begin{definition}[Product]\index{Product}\label{def:product_limit}

	For a discrete category $\underline{n}$, the functor $F:\underline{n}\to \C$
	is a family of objects $(F(n))_{n\in \underline{n}_0}$ in $\C$. The product
	$\prod\limits_{n\in \underline{n}_0} F(n)$ is the limit of such an
	$F$~\parencite[p.~108]{leinster:basic_category_theory}.
\end{definition}

\begin{remark}
	In the case $\underline{n}=\underline{2}$, the product of objects $A$ and $B$
	is comformed by an object $A\times B$ with morphisms $\pi_0:A\times B\to A$,
	$\pi_1: A\times B \to B$.
\end{remark}

\begin{remark}
	In the case $\underline{n}=\underline{0}$ the product is denoted as $1$
	and corresponds to a terminal object.
\end{remark}

\begin{theorem}\label{thm:ternary_product}
	For objects $A,B,C$ it holds that $A\times(B\times C)\cong A\times B\times C$.

	\begin{proof}
		% https://q.uiver.app/% https://q.uiver.app/#q=WzAsNixbMCwxLCJYXzAiXSxbMSwxLCJYXzEiXSxbMiwxLCJYXzIiXSxbMSwyLCJCXFx0aW1lcyBDIl0sWzEsMywiQVxcdGltZXMoQlxcdGltZXMgQykiXSxbMSwwLCJTIl0sWzMsMSwicl8wIl0sWzMsMiwicl8xIiwyXSxbNCwwLCJcXHBpXzAiXSxbNCwzLCJcXHBpXzEiLDJdLFs1LDAsIlxccmhvXzAiLDJdLFs1LDEsIlxccmhvXzEiLDJdLFs1LDIsIlxccmhvXzIiXV0=
		\[\begin{tikzcd}[ampersand replacement=\&]
			\& S \\
			{X_0} \& {X_1} \& {X_2} \\
			\& {B\times C} \\
			\& {A\times(B\times C)}
			\arrow["{r_0}", from=3-2, to=2-2]
			\arrow["{r_1}"', from=3-2, to=2-3]
			\arrow["{\pi_0}", from=4-2, to=2-1]
			\arrow["{\pi_1}"', from=4-2, to=3-2]
			\arrow["{\rho_0}"', from=1-2, to=2-1]
			\arrow["{\rho_1}"', from=1-2, to=2-2]
			\arrow["{\rho_2}", from=1-2, to=2-3]
		\end{tikzcd}\]

		Consider an object $S$ with morphisms $x_i: S\to X_i$. There exists
		morphisms from $S$ to $X_1$ and $X_2$, therefore by the universal property
		of the product there exists a unique morphism $u: S\to X_1 \times X_2$ such
		that:
		\[
			r_0 \circ u = \rho_1
			\qquad
			r_1 \circ u = \rho_2
		\]

		Now, as there exists morphisms from $S$ to $X_0$ and $X_1\times X_2$, by the
		universal property of the product there exists a unique morphism $v: S \to
		X_0 \times(X_1 \times X_2)$ such that:
		\[
			\pi_0 \circ v = \rho_0
			\qquad
			\pi_1 \circ v = u
		\]

		Then, $\rho_i$ is factorized via $v$ by:
		\[
			\rho_0 = \pi_0 \circ v
			\qquad
			\rho_1 = r_0 \circ \pi_1 \circ v
			\qquad
			\rho_2 = r_1 \circ \pi_1 \circ v
		\]

		Finally by the uniqueness of $u$ and $v$ this factorization is unique, and
		as $X_0 \times(X_1 \times X_2)$ has the same universal property as $X_0
		\times X_1 \times X_2$ both are isomorphic.
	\end{proof}
\end{theorem}

\begin{proposition}
	For objects $A,B,C$ it holds that $(A\times B)\times C\cong A\times B\times
	C$, making products weakly associative. The proof is similar to that of Theorem \ref{thm:ternary_product}.
\end{proposition}

\begin{proposition}
	A category has all finite limits if it has a terminal object and all binary
	products. The proof relies on using theorem \ref{thm:ternary_product}
	inductively.
\end{proposition}

\begin{theorem}
	Product projections are jointly monic, i.e. for a collection of $I$-indexed
	objects $(A_i)_{i\in I}$, their product $A$ with projections $\pi_i: A \to
	A_i$, and an object $B$ with morphisms $f, g: B \to A$, it follows that:
	\[
		f = g \iff \big(\forall i\in I\big)\big(\pi_i \circ f = \pi_i \circ g\big)
	\]

	\begin{proof}
		% https://q.uiver.app/#q=WzAsMyxbMSwwLCJBIl0sWzEsMSwiQV9pIl0sWzAsMCwiQiJdLFsyLDAsImciLDIseyJvZmZzZXQiOjF9XSxbMiwwLCJmIiwwLHsib2Zmc2V0IjotMX1dLFswLDEsIlxccGlfaSJdLFsyLDEsIlxccGlfaVxcY2lyYyBmIiwxXSxbMiwxLCJcXHBpX2lcXGNpcmMgZyIsMSx7ImN1cnZlIjozfV1d&macro_url=https%3A%2F%2Fgist.githubusercontent.com%2Faortega0703%2Fa1fd97cb097b8142e63a6fbf0cdb0f76%2Fraw%2Fb46a955b6b1f06908105b870088e59a10049fc60%2Fmacros.tex
		\[\begin{tikzcd}[ampersand replacement=\&]
			B \& A \\
			\& {A_i}
			\arrow["g"', shift right=1, from=1-1, to=1-2]
			\arrow["f", shift left=1, from=1-1, to=1-2]
			\arrow["{\pi_i}", from=1-2, to=2-2]
			\arrow["{\pi_i\circ f}"{description}, from=1-1, to=2-2]
			\arrow["{\pi_i\circ g}"{description}, curve={height=18pt}, from=1-1, to=2-2]
		\end{tikzcd}\]

		Now, by making $\pi_i \circ f = \pi_i \circ g$ we obtain the following:
		% https://q.uiver.app/#q=WzAsMyxbMSwwLCJBIl0sWzEsMSwiQV9pIl0sWzAsMCwiQiJdLFsyLDAsImciLDIseyJvZmZzZXQiOjF9XSxbMiwwLCJmIiwwLHsib2Zmc2V0IjotMX1dLFswLDEsIlxccGlfaSJdLFsyLDEsIlxccGlfaVxcY2lyYyBmPVxccGlfaVxcY2lyYyBnIiwyXV0=&macro_url=https%3A%2F%2Fgist.githubusercontent.com%2Faortega0703%2Fa1fd97cb097b8142e63a6fbf0cdb0f76%2Fraw%2Fb46a955b6b1f06908105b870088e59a10049fc60%2Fmacros.tex
		\[\begin{tikzcd}[ampersand replacement=\&]
			B \& A \\
			\& {A_i}
			\arrow["g"', shift right=1, from=1-1, to=1-2]
			\arrow["f", shift left=1, from=1-1, to=1-2]
			\arrow["{\pi_i}", from=1-2, to=2-2]
			\arrow["{\pi_i\circ f=\pi_i\circ g}"', from=1-1, to=2-2]
		\end{tikzcd}\]

		Clearly both $f$ and $g$ make the triangles commute. Then by the universal
		property of products, there exists a unique arrow that makes them commute,
		therefore $f=g$.
	\end{proof}
\end{theorem}

\newpage

\subsubsection{Product Category}
\begin{definition}[Product Category]\index{Product Category}

  For categories $\C,\D$, a product category $\C\times \D$ consists
  of~\parencite[p.~16]{awodey:category_theory}:

  \begin{itemize}
    \item Objects:
      \[\big(\forall C \in \C_0,\ D \in \D_0\big)
        \big(\<C, D\>\in (\C\times \D)_0)\]
    \item Morphisms:
      \[
        \begin{gathered}
          \big(\forall (f: C \to C')\in \C_1,\ (g:D \to D')\in \D_1\big)\\
          \big((\<f,g\> : \<C,D\> \to \<C', D'\>)\in (\C\times \D)_1\big)
        \end{gathered}
      \]
  \end{itemize}
\end{definition}

\begin{remark}
  The product category coincides with the product in Cat.
\end{remark}

\begin{example}
  \[
    % https://q.uiver.app/#q=WzAsMixbMCwwLCJBIl0sWzAsMSwiQiJdLFswLDEsImYiLDJdXQ==
    \begin{tikzcd}[ampersand replacement=\&]
      A \\
      B
      \arrow["f"', from=1-1, to=2-1]
    \end{tikzcd}
    \times
    % https://q.uiver.app/#q=WzAsMixbMCwwLCJDIl0sWzAsMSwiRCJdLFswLDEsImciXV0=
    \begin{tikzcd}[ampersand replacement=\&]
      C \\
      D
      \arrow["g", from=1-1, to=2-1]
    \end{tikzcd}
    =
    % https://q.uiver.app/#q=WzAsNCxbMCwwLCJcXGxhbmdsZSBBLCBDXFxyYW5nbGUiXSxbMSwwLCJcXGxhbmdsZSBBLERcXHJhbmdsZSJdLFswLDEsIlxcbGFuZ2xlIEIsQ1xccmFuZ2xlIl0sWzEsMSwiXFxsYW5nbGUgQixEXFxyYW5nbGUiXSxbMCwxLCJcXGxhbmdsZSBcXGlkX0EsZ1xccmFuZ2xlIl0sWzAsMiwiXFxsYW5nbGUgZixcXGlkX0NcXHJhbmdsZSIsMl0sWzIsMywiXFxsYW5nbGUgXFxpZF9CLGdcXHJhbmdsZSIsMl0sWzEsMywiXFxsYW5nbGUgZixcXGlkX0RcXHJhbmdsZSJdLFswLDMsIlxcbGFuZ2xlIGYsZ1xccmFuZ2xlIiwxXV0=&macro_url=https%3A%2F%2Fraw.githubusercontent.com%2Faortega0703%2Fnotes-category-theory%2Fmain%2Fsrc%2Fmacros.tex
    \begin{tikzcd}[ampersand replacement=\&]
      {\langle A, C\rangle} \& {\langle A,D\rangle} \\
      {\langle B,C\rangle} \& {\langle B,D\rangle}
      \arrow["{\langle \id_A,g\rangle}", from=1-1, to=1-2]
      \arrow["{\langle f,\id_C\rangle}"', from=1-1, to=2-1]
      \arrow["{\langle \id_B,g\rangle}"', from=2-1, to=2-2]
      \arrow["{\langle f,\id_D\rangle}", from=1-2, to=2-2]
      \arrow["{\langle f,g\rangle}"{description}, from=1-1, to=2-2]
    \end{tikzcd}
  \]
\end{example}

\begin{theorem}[Naturality on Two Variables\label{thm:binaturality}]
  For bifunctors $F, G: \A \times \B \to \C$ a natural transformation is natural
  in both variables simultaneously if and only if it is natural on each
  separately.

  \begin{proof}
    The proof consists of two parts:

    \begin{description}
      \item[$(\implies)$] Consider the functions ${\< \id_a, g \>: \<a, b\>
        \to \<a, b'\>}$, ${\< f, \id_b \>: \<a, b\> \to \<a', b\>}$:
        % https://q.uiver.app/#q=WzAsNCxbMCwwLCJGKEEsQikiXSxbMSwwLCJGKEEsIEInKSJdLFswLDEsIkcoQSwgQikiXSxbMSwxLCJHKEEsQicpIl0sWzAsMSwiRihcXGlkX0EsIGcpIl0sWzAsMiwiXFxhbHBoYV97QSxCfSIsMl0sWzIsMywiRyhcXGlkX0EsIGcpIiwyXSxbMSwzLCJcXGFscGhhX3tBLCBCJ30iXV0=&macro_url=https%3A%2F%2Fraw.githubusercontent.com%2Faortega0703%2Fnotes-category-theory%2Fmain%2Fsrc%2Fmacros.tex
        \[\begin{tikzcd}[ampersand replacement=\&]
          {F(A,B)} \& {F(A, B')} \\
          {G(A, B)} \& {G(A,B')}
          \arrow["{F(\id_A, g)}", from=1-1, to=1-2]
          \arrow["{\alpha_{A,B}}"', from=1-1, to=2-1]
          \arrow["{G(\id_A, g)}"', from=2-1, to=2-2]
          \arrow["{\alpha_{A, B'}}", from=1-2, to=2-2]
        \end{tikzcd}
        \qquad
        % https://q.uiver.app/#q=WzAsNCxbMCwwLCJGKEEsIEIpIl0sWzAsMSwiRyhBLEIpIl0sWzEsMSwiRyhBJyxCKSJdLFsxLDAsIkYoQScsIEIpIl0sWzAsMSwiXFxhbHBoYV97QSwgQn0iLDJdLFsxLDIsIkcoZiwgXFxpZF97Qn0pIiwyXSxbMywyLCJcXGFscGhhX3tBJywgQn0iXSxbMCwzLCJGKGYsIFxcaWRfe0J9KSJdXQ==&macro_url=https%3A%2F%2Fraw.githubusercontent.com%2Faortega0703%2Fnotes-category-theory%2Fmain%2Fsrc%2Fmacros.tex
        \begin{tikzcd}[ampersand replacement=\&]
          {F(A, B)} \& {F(A', B)} \\
          {G(A,B)} \& {G(A',B)}
          \arrow["{\alpha_{A, B}}"', from=1-1, to=2-1]
          \arrow["{G(f, \id_{B})}"', from=2-1, to=2-2]
          \arrow["{\alpha_{A', B}}", from=1-2, to=2-2]
          \arrow["{F(f, \id_{B})}", from=1-1, to=1-2]
        \end{tikzcd}\]
      \item[$(\impliedby)$] By taking naturality on each variable and
        considering functions $f:a \to a',\ g:b\to b'$ it is possible to form
        the following commutative diagram:
        % https://q.uiver.app/#q=WzAsNixbMCwwLCJGKEEsIEIpIl0sWzEsMCwiRihBLCBCJykiXSxbMiwwLCJGKEEnLCBCJykiXSxbMCwxLCJHKEEsIEIpIl0sWzEsMSwiRyhBLEInKSJdLFsyLDEsIkcoQScsQicpIl0sWzAsMSwiRihcXGlkX0EsIGcpIl0sWzEsMiwiRihmLCBcXGlkX3tCJ30pIl0sWzAsMywiXFxhbHBoYV97QSwgQn0iLDJdLFszLDQsIkcoXFxpZF9BLCBnKSIsMl0sWzEsNCwiXFxhbHBoYV97QSwgQid9IiwxXSxbNCw1LCJHKGYsIFxcaWRfe0InfSkiLDJdLFsyLDUsIlxcYWxwaGFfe0EnLCBCJ30iXV0=&macro_url=https%3A%2F%2Fraw.githubusercontent.com%2Faortega0703%2Fnotes-category-theory%2Fmain%2Fsrc%2Fmacros.tex
        \[\begin{tikzcd}[ampersand replacement=\&]
          {F(A, B)} \& {F(A, B')} \& {F(A', B')} \\
          {G(A, B)} \& {G(A,B')} \& {G(A',B')}
          \arrow["{F(\id_A, g)}", from=1-1, to=1-2]
          \arrow["{F(f, \id_{B'})}", from=1-2, to=1-3]
          \arrow["{\alpha_{A, B}}"', from=1-1, to=2-1]
          \arrow["{G(\id_A, g)}"', from=2-1, to=2-2]
          \arrow["{\alpha_{A, B'}}"{description}, from=1-2, to=2-2]
          \arrow["{G(f, \id_{B'})}"', from=2-2, to=2-3]
          \arrow["{\alpha_{A', B'}}", from=1-3, to=2-3]
        \end{tikzcd}\]

        \[
          \begin{aligned}
            \alpha_{A', B'} \circ F(f, g)
            &= \alpha_{A', B'} \circ F(\< f, \id_{B'}\> \circ \<\id_A, g\>)\\
            &= \alpha_{A', B'} \circ F(f, \id_{B'}) \circ F(\id_A, g)\\
            &= G(f, \id_{B'}) \circ \alpha_{A, b'} \circ F(\id_A, g)\\
            &= G(f, \id_{B'}) \circ G(\id_A, g) \circ \alpha_{A, b}\\
            &= G(\< f, \id_{B'}\> \circ \< \id_A,g \>) \circ \alpha_{A, b}\\
            &= G(f, g) \circ \alpha_{A, b}
          \end{aligned}
        \]
    \end{description}
  \end{proof}
  \vspace{-1.5\baselineskip}
\end{theorem}

\section{Functors}
A functor can be thought of as an structure preserving transformation between
categories, or a way to identify one category inside another.

\begin{definition}[Functor]\index{Functor}
  For categories $\C,\D$, a functor $F: \C \to \D$ consist of maps of objects
  and morphisms, such that it
  preserves~\parencite[p.~8]{awodey:category_theory}:
  \begin{itemize}
    \item Domain and Codomain:
      \[\big(\forall (f:c\to c')\in \C_1\big)
        \big(F(f): F(c) \to F(c')\big)\]
    \item Identity:
      \[(\forall c \in \C_0)
        (F(\id_c) = \id_{F(c)})\]
    \item Composition:
      \[\big(\forall (f: a \to b)\in \C_1,\ (g: b\to c) \in \C_1\big)
        \big(F(g \circ f) = F(g) \circ F(f)\big)\]
  \end{itemize}
\end{definition}

\begin{example}
  % https://q.uiver.app/?q=WzAsNyxbMCwwLCIxIl0sWzMsMCwiRigxKSJdLFs0LDAsIkYoMikiXSxbNCwxLCJcXGJ1bGxldCJdLFsxLDAsIjIiXSxbMiwwLCIzIl0sWzUsMCwiRigzKSJdLFswLDQsImYiXSxbMSwyLCJGKGYpIl0sWzIsMywiaCIsMl0sWzEsMywiaCBcXGNpcmMgRihmKSIsMix7ImxhYmVsX3Bvc2l0aW9uIjoyMH1dLFs0LDUsImciXSxbMCw1LCJnXFxjaXJjIGYiLDAseyJjdXJ2ZSI6LTN9XSxbMiw2LCJGKGcpIl0sWzMsNiwiaSIsMl0sWzEsNiwiRihnXFxjaXJjIGYpIiwwLHsiY3VydmUiOi00fV1d
  \[\begin{tikzcd}[ampersand replacement=\&]
    1 \& 2 \& 3 \& {F(1)} \& {F(2)} \& {F(3)} \\
    \&\&\&\& \bullet
    \arrow["f", from=1-1, to=1-2]
    \arrow["{F(f)}", from=1-4, to=1-5]
    \arrow["h"', from=1-5, to=2-5]
    \arrow["{h \circ F(f)}"'{pos=0.2}, from=1-4, to=2-5]
    \arrow["g", from=1-2, to=1-3]
    \arrow["{g\circ f}", curve={height=-18pt}, from=1-1, to=1-3]
    \arrow["{F(g)}", from=1-5, to=1-6]
    \arrow["i"', from=2-5, to=1-6]
    \arrow["{F(g\circ f)}", curve={height=-24pt}, from=1-4, to=1-6]
  \end{tikzcd}\]
\end{example}

\begin{theorem}[Functors Preserve Isomorphisms]\label{thm:isomorphism_functor}

  Given two isomorphic objects, their images under any functor are also
  isomorphic.

  \begin{proof}
    Consider two objects $c,d \in \C_0$ with an isomorphism $f: c\to d$, and any
    functor $F:\C \to \D$ then:
    \[
      \begin{aligned}
        f^{-1}\circ f &= \id_c\\
        F(f^{-1}\circ f) &= F(\id_c)\\
        F(f^{-1})\circ F(f) &= \id_{F(c)}
      \end{aligned}
      \qquad
      \begin{aligned}
        f \circ f^{-1} &= \id_d\\
        F(f \circ f^{-1}) &= F(\id_d)\\
        F(f) \circ F(f^{-1}) &= \id_{F(d)}
      \end{aligned}
    \]
  \end{proof}
\end{theorem}

\subsection{Special Functors}

\subsubsection{Constant Functor}
\begin{definition}[Constant Functor]\index{Constant Functor}
  For categories $\C,\D$ and an object $D\in \D$, the constant functor
  $\Delta_D: \C\to \D$ consists of maps
  of~\parencite[p.~142]{leinster:basic_category_theory}:
  \begin{itemize}
    \item Objects:
      \[\big(\forall C\in \C_0\big)
        \big(\Delta_D(C) = D\big)\]
    \item Morphisms:
      \[\big(\forall f\in \C_1\big)
        \big(\Delta_D(F) = \id_D\big)\]
  \end{itemize}
\end{definition}

\subsubsection{Diagonal Functor}
\begin{definition}[Diagonal Functor]\index{Diagonal
  Functor}\label{def:diagonal_functor_binary}
  For a category $\C$, the diagonal
  functor $\Delta: \C \to \C\times \C$ is defined as maps of:
  \begin{itemize}
    \item Objects:
      \[\big(\forall c\in \C_0\big)
        \big(\Delta(C) = \<C,C\>\big)\]
    \item Morphisms:
      \[\big(\forall f\in \C_1\big)
        \big(\Delta(f) = \<f,f\>\big)\]
  \end{itemize}
\end{definition}

\begin{definition}[Diagonal Functor]\index{Diagonal
  Functor}\label{def:diagonal_functor}
  More generally, for categories $\C, \D$,
  the diagonal functor $\Delta: \C \to \D^\C$ consists of maps
  of~\parencite[p.~142]{leinster:basic_category_theory}:
  \begin{itemize}
    \item Objects:
      \[\big(\forall C\in \C_0\big)
        \big(\Delta(C) = \Delta_C : \C \to \D\big)\]
      Where $\Delta_C$ is the constant functor.
    \item Morphisms:
      \[\big(\forall C\in C_0,\ f\in \C_1\big)
        \big((\Delta(f))_C= f \big)\]
      The naturality condition of $\Delta(f)$ is given by:
      % https://q.uiver.app/#q=WzAsNCxbMCwwLCJcXERlbHRhX0MoQykiXSxbMSwwLCJcXERlbHRhX0MoRCkiXSxbMCwxLCJcXERlbHRhX3tEfShDKSJdLFsxLDEsIlxcRGVsdGFfe0R9KEQpIl0sWzAsMSwiXFxEZWx0YV9DKGYpIl0sWzAsMiwiXFxEZWx0YShmKV9DIiwyXSxbMiwzLCJcXERlbHRhX3tEfShmKSIsMl0sWzEsMywiXFxEZWx0YShmKV97RH0iXV0=
      \[\begin{tikzcd}[ampersand replacement=\&]
        {\Delta_C(C)} \& {\Delta_C(D)} \\
        {\Delta_{D}(C)} \& {\Delta_{D}(D)}
        \arrow["{\Delta_C(f)}", from=1-1, to=1-2]
        \arrow["{\Delta(f)_C}"', from=1-1, to=2-1]
        \arrow["{\Delta_{D}(f)}"', from=2-1, to=2-2]
        \arrow["{\Delta(f)_{D}}", from=1-2, to=2-2]
      \end{tikzcd}
      =
      % https://q.uiver.app/#q=WzAsNCxbMCwwLCJDIl0sWzEsMCwiQyJdLFswLDEsIkQiXSxbMSwxLCJEIl0sWzAsMSwiXFxpZF9DIl0sWzAsMiwiZiIsMl0sWzIsMywiXFxpZF9EIiwyXSxbMSwzLCJmIl1d&macro_url=https%3A%2F%2Fraw.githubusercontent.com%2Faortega0703%2Fnotes-category-theory%2Fmain%2Fsrc%2Fmacros.tex
      \begin{tikzcd}[ampersand replacement=\&]
        C \& C \\
        D \& D
        \arrow["{\id_C}", from=1-1, to=1-2]
        \arrow["f"', from=1-1, to=2-1]
        \arrow["{\id_D}"', from=2-1, to=2-2]
        \arrow["f", from=1-2, to=2-2]
      \end{tikzcd}\]
  \end{itemize}
\end{definition}

\subsection{Classification of Functors}

\begin{definition}[Faithful Functor\index{Faithful Functor}]
  A functor $F:\C\to \D$ is faithful when for every morphism $(f: c\to c') \in
  \C_1$, the following function is
  injective~\parencite[p.~25]{leinster:basic_category_theory}:
    \[
      \begin{aligned}
        F: \C(c, c') &\to \D(F(c), F(c'))\\
        f &\mapsto F(f)
      \end{aligned}
    \]
\end{definition}

\begin{definition}[Full Functor\index{Full Functor}]
  A functor $F:\C\to \D$ is full when for every morphism $(f: c\to c') \in
  \C_1$, the following function is
  surjective~\parencite[p.~25]{leinster:basic_category_theory}:
  \[
    \begin{aligned}
      F: \C(c, c') &\to \D(F(c), F(c'))\\
      f &\mapsto F(f)
    \end{aligned}
  \]
\end{definition}

\begin{definition}[Endofunctor\index{Endofunctor}]
  A functor whose domain equals its codomain is an
  endofunctor~\parencite[p.~30]{adamek_herrlich_strecker:joy_cats}.
\end{definition}

\begin{definition}[Bifunctor\index{Bifunctor}]
  A functor $F:\A\times \B \to \C$ (one with $2$ parameters) is said to be a
  bifunctor~\parencite[p.~37]{lane:working_mathematician}.
\end{definition}

\begin{definition}[Presheaf\index{Presheaf}]
  A functor $\C^\op \to \Set$ is a presheaf on
  $\C$~\parencite[p.~24]{leinster:basic_category_theory}.
\end{definition}

\begin{theorem}[Fully Faithful Isomorphism]\label{thm:full_faithful_isomorphism}

  For any full and faithful functor $F:\C\to \D$ it follows that $c \cong c'
  \iff F(c) \cong F(c')$ for objects $c, c'\in \C_0$.

  \begin{proof}
    The $(\implies)$ holds via Theorem \ref{thm:isomorphism_functor}. For the
    $(\impliedby)$ part, consider the isomorphism $Ff: F(c) \overset{\cong}{\to}
    F(c')$. As $F$ is fully faithful, morphisms $c\to c'$ are in bijection
    with morphisms $F(c)\to F(c')$, therefore there exists unique arrows $f:c\to c'$ $g:c'\to c$ such that $F(f) = Ff$ and $F(g) = Ff^{-1}$, then:
    \[
      \begin{aligned}
        &F(\id_c)\\
        =& \id_{F(c)}\\
        =& Ff^{-1} \circ Ff\\
        =& F(g) \circ F(f)\\
        =& F(g\circ f)
      \end{aligned}
      \qquad
      \begin{aligned}
        &F(\id_{c'})\\
        =& \id_{F(c')}\\
        =& Ff \circ Ff^{-1}\\
        =& F(f) \circ F(g)\\
        =& F(f\circ g)
      \end{aligned}
    \]

    As $F$ is faithful and $F(\id_c) = F(g\circ f)$, then $g\circ f = \id_c$.
    Similarly, $f\circ g = \id_{c'}$.
  \end{proof}
\end{theorem}

\subsection{Identity}

\begin{definition}[Identity Functor]\index{Identity
  Functor}\label{def:id_functor}
  For a category $\C$ there exists an identity functor $\id_\C: \C \to \C$ which
  consists of maps of~\parencite[p.~27]{adamek_herrlich_strecker:joy_cats}:
  \begin{itemize}
    \item Objects:
      \[(\forall c \in \C_0)
        (\id_\C(c) = c)\]
    \item Morphisms:
      \[\big(\forall (f: c\to c') \in \C_1\big)
        (\id_C(f) = f)\]
  \end{itemize}
\end{definition}

\begin{theorem}[$\id_\C$ Preserves Identity\label{thm:id_functor_identity}]
  For a category $\C$, the identity functor $\id_\C: \C \to \C$ preserves
  identity.

  \begin{proof}
    For all objects $c\in \C$, $\id_\C(\id_c) = \id_c$ by definition.
  \end{proof}
\end{theorem}

\begin{theorem}[$\id_\C$ Preserves Composition]\label{thm:id_functor_composition}

  For a category $\C$, the identity functor $\id_\C: \C \to \C$ preserves
  composition.

  \begin{proof}
    For all morphisms $f,g\in \C$ with $g$ composable after $f$:
    \[
      \begin{aligned}
        \id_\C(g\circ f)
        &= g\circ f\\
        &= \id_\C(g) \circ \id_\C(f)
      \end{aligned}
    \]
  \end{proof}
\end{theorem}

\begin{theorem}[$\id_\C$ is a Functor]
  For a category $\C$, the identity functor $\id_\C$ is a functor.

  \begin{proof}
    By Theorems \ref{thm:id_functor_identity} and
    \ref{thm:id_functor_composition} the identity functor is a functor.
  \end{proof}
\end{theorem}

\subsection{Composition}

\begin{definition}[Functor Composition\index{Functor
  Composition}]\label{def:functor_composition}

  For functors $F:\A\to \B$ and $G:\B\to \C$, the composite functor $G\circ F$
  consists of maps of~\parencite[p.~17]{leinster:basic_category_theory}:
  \begin{itemize}
    \item Objects:
      \[(\forall A \in \A_0)
        \big((G\circ F)(A) = G(F(A))\big)\]
    \item Morphisms:
      \[(\forall f\in \A_1)
        \big((G\circ F)(f) = G(F(f))\big)\]
  \end{itemize}
\end{definition}

\begin{theorem}[Functor Composition Preserves Identity\label{thm:functor_cmp_id}]
  For functors $F:\A\to \B$ and $G:\B\to \C$, the composite functor $G\circ F$
  preserves indentity.

  \begin{proof}
    For all objects $A\in \A_0$:
    \[
      \begin{aligned}
        (G\circ F)(\id_A)
        &= G(F(\id_A))\\
        &= G(\id_{F(A)})\\
        &= \id_{(G\circ F)(A)}
      \end{aligned}
    \]
  \end{proof}
  \vspace{-\baselineskip}
\end{theorem}

\begin{theorem}[Functor Composition Preserves Composition\label{thm:functor_cmp_cmp}]
  For functors $F:\A\to \B$ and $G:\B\to \C$, the composite functor $G\circ F$
  preserves composition.

  \begin{proof}
    For all morphisms $f,g\in \A_1$ with $g$ composable after $f$:
    \[
      \begin{aligned}
        (G\circ F)(g\circ f)
        &= G(F(g\circ f))\\
        &= G(F(g)\circ F(f))\\
        &= G(F(g))\circ G(F(f))\\
        &= (G\circ F)(g)\circ(G\circ F)(f)
      \end{aligned}
    \]
  \end{proof}
  \vspace{-\baselineskip}
\end{theorem}

\begin{theorem}[Functor Composition is a Functor]
  For functors $F: \A \to \B,\ G: \B\to \C$, the composition $G\circ F$ is a functor.

  \begin{proof}
    By Theorems \ref{thm:functor_cmp_id} and \ref{thm:functor_cmp_cmp} functor
    composition fulfills the axioms of a functor.
  \end{proof}
\end{theorem}

\begin{theorem}[Unitality of Functor Composition\label{thm:unitality_functor}]
  For a functor $F:\C\to \D$ the following holds:
  \[F \circ \id_\C = \id_\D \circ F = F\]

  \begin{proof}
    Let either $C\in \C_0$ or $C\in \C_1$, then:
    \[
      \begin{aligned}
        (F \circ \id_\C)(C)
          &= F(\id_\C(C))\\
          &= F (C)
      \end{aligned}
      \qquad
      \begin{aligned}
        (\id_\D \circ F)(C)
          &= \id_\D (F (C))\\
          &= F (C)
      \end{aligned}
    \]
  \end{proof}
  \vspace{-\baselineskip}
\end{theorem}

\begin{theorem}[Associativity of Functor Composition\label{thm:assoc_functor}]
  For a set-up of categories and functors:
  % https://q.uiver.app/?q=WzAsNCxbMCwwLCJcXEEiXSxbMSwwLCJcXEIiXSxbMiwwLCJcXEMiXSxbMywwLCJcXEQiXSxbMCwxLCJGIl0sWzEsMiwiRyJdLFsyLDMsIkgiXV0=&macro_url=https%3A%2F%2Fraw.githubusercontent.com%2Faortega0703%2Fnotes-category-theory%2Fmain%2Fsrc%2Fmacros.tex
  \[\begin{tikzcd}[ampersand replacement=\&]
    \A \& \B \& \C \& \D
    \arrow["F", from=1-1, to=1-2]
    \arrow["G", from=1-2, to=1-3]
    \arrow["H", from=1-3, to=1-4]
  \end{tikzcd}\]

  The following holds:
  \[(H \circ G) \circ F = H\circ(G\circ F)\]

  \begin{proof}
    Let either $C\in \C_0$ or $C\in \C_1$, then:
    \[
      \begin{aligned}
        ((H \circ G) \circ F) (C)
          &= (H\circ G) (F (C))\\
          &= H(G(F(C)))\\
          &= H((G\circ F)(C))\\
          &= (H\circ(G\circ F)) (C)
      \end{aligned}
    \]
  \end{proof}
  \vspace{-\baselineskip}
\end{theorem}

\begin{theorem}
  There is a category with (small) categories as objects and functors as
  morphisms. It is called Cat.

  \begin{proof}
    By Definitions \ref{def:id_functor}, \ref{def:functor_composition} and Theorems \ref{thm:unitality_functor}, \ref{thm:assoc_functor}.
  \end{proof}
\end{theorem}

\begin{remark}
  The functor category coincides with the exponential in Cat.
\end{remark}