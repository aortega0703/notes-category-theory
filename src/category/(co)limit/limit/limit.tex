\section{Limits}

\begin{definition}[Limit\index{Limit}\label{def:limit}]
	The limit of a functor $F:\C\to \D$ is a cone with vertex $d\in \D$ and
	morphisms $\big(\alpha_c: d \to F(c)\big)_{c\in \C_0}$ such that for any other
	cone with vertex $d'\in D$ and morphisms $\big(\beta_c: d' \to F(c)\big)_{c\in
	\C_0}$ there exists a unique map $f:d'\to d$ such that the following diagram
	commutes for all $c\in
	\C_0$~\parencite[p.~118]{leinster:basic_category_theory}:
	% https://q.uiver.app/?q=WzAsMyxbMCwxLCJkIl0sWzAsMCwiZCciXSxbMSwxLCJGKGMpIl0sWzAsMiwiXFxhbHBoYV9jIiwyXSxbMSwyLCJcXGJldGFfYyJdLFsxLDAsImYiLDIseyJzdHlsZSI6eyJib2R5Ijp7Im5hbWUiOiJkYXNoZWQifX19XV0=
	\[\begin{tikzcd}
		{d'} \\
		d & {F(c)}
		\arrow["{\alpha_c}"', from=2-1, to=2-2]
		\arrow["{\beta_c}", from=1-1, to=2-2]
		\arrow["f"', dashed, from=1-1, to=2-1]
	\end{tikzcd}\]
\end{definition}

\begin{definition}[Limit as a Terminal Object\index{Limit!as a Terminal Object}]
	Equivalently to Definition \ref{def:limit}, the limit of a functor $F:\I\to
	\C$ can be expressed as a terminal object in the category $\mathrm{cone}(F)$.
\end{definition}

\begin{definition}[Limit as a Terminal Morphism\index{Limit!as a Terminal Morphism}]
	Equivalently to Definition \ref{def:limit}, the limit of a functor $F:\I\to
	\C$ can be expressed as a terminal morphism from the diagonal functor $\Delta:
	\C \to \C^\I$ to $F$ (thought of as an object in $\C^\I$).
	% https://q.uiver.app/?q=WzAsNSxbMCwxLCJjIl0sWzAsMCwiYyciXSxbMSwxLCJcXERlbHRhX2MiXSxbMSwwLCJcXERlbHRhX3tjJ30iXSxbMiwxLCJGIl0sWzEsMCwiaCIsMix7InN0eWxlIjp7ImJvZHkiOnsibmFtZSI6ImRhc2hlZCJ9fX1dLFszLDIsIlxcRGVsdGEoaCkiLDIseyJzdHlsZSI6eyJib2R5Ijp7Im5hbWUiOiJkYXNoZWQifX19XSxbMyw0LCJcXGJldGEiXSxbMiw0LCJcXGFscGhhIiwyXV0=&macro_url=https%3A%2F%2Fraw.githubusercontent.com%2Faortega0703%2Fnotes-category-theory%2Fmain%2Fsrc%2Fmacros.tex
	\[\begin{tikzcd}[ampersand replacement=\&]
		{c'} \& {\Delta_{c'}} \\
		c \& {\Delta_c} \& F
		\arrow["h"', dashed, from=1-1, to=2-1]
		\arrow["{\Delta(h)}"', dashed, from=1-2, to=2-2]
		\arrow["\beta", from=1-2, to=2-3]
		\arrow["\alpha"', from=2-2, to=2-3]
	\end{tikzcd}\]
\end{definition}

\begin{theorem}[Uniqueness of Limits\label{thm:limit_iso}]
	If there exists multiple limits for the same functor, then they are all
	isomorphic.

	\begin{proof}
		A limit can be seen as a terminal morphism, which are unique up to
		Isomorphism (Theorem \ref{thm:terminal_morphism_iso}), therefore limits are
		also unique up to isomorphism.
	\end{proof}
\end{theorem}

\subimport{product/}{product.tex}

\subsection{Equalizer}

\begin{definition}[Equalizer]\index{Equalizer}
  An equalizer is the limit of a
  diagram~\parencite[p.~112]{leinster:basic_category_theory}:
  % https://q.uiver.app/#q=WzAsMixbMCwwLCJYIl0sWzEsMCwiWSJdLFswLDEsImYiLDAseyJvZmZzZXQiOi0xfV0sWzAsMSwiZyIsMix7Im9mZnNldCI6MX1dXQ==
  \[\begin{tikzcd}[ampersand replacement=\&]
    X \& Y
    \arrow["f", shift left=1, from=1-1, to=1-2]
    \arrow["g"', shift right=1, from=1-1, to=1-2]
  \end{tikzcd}\]
\end{definition}

\begin{remark}
  An equalizer candidate of objects $A, B$ and morphisms $f, g: A\to B$ is
  composed by an object $X$ together with a morphism $x:X\to A$ such that
  $f\circ x = g\circ x$. An equalizer is a candidate $\<E, e\>$ such that for
  any candidate $\<X, x\>$ there exists a unique arrow $h: X\to E$ that makes
  $e\circ h= x$ hold.
  % https://q.uiver.app/#q=WzAsNCxbMSwxLCJBIl0sWzIsMSwiQiJdLFswLDEsIkUiXSxbMCwwLCJYIl0sWzAsMSwiZiIsMCx7Im9mZnNldCI6LTF9XSxbMCwxLCJnIiwyLHsib2Zmc2V0IjoxfV0sWzIsMCwiZSJdLFszLDIsImgiLDIseyJzdHlsZSI6eyJib2R5Ijp7Im5hbWUiOiJkYXNoZWQifX19XSxbMywwLCJ4Il1d&macro_url=https%3A%2F%2Fgist.githubusercontent.com%2Faortega0703%2Fa1fd97cb097b8142e63a6fbf0cdb0f76%2Fraw%2Fb46a955b6b1f06908105b870088e59a10049fc60%2Fmacros.tex
  \[\begin{tikzcd}[ampersand replacement=\&]
    X \\
    E \& A \& B
    \arrow["f", shift left=1, from=2-2, to=2-3]
    \arrow["g"', shift right=1, from=2-2, to=2-3]
    \arrow["e", from=2-1, to=2-2]
    \arrow["h"', dashed, from=1-1, to=2-1]
    \arrow["x", from=1-1, to=2-2]
  \end{tikzcd}\]
\end{remark}

\begin{theorem}\label{thm:equalizer_mono}
  Every equalizer morphism is a monomorphism.

  \begin{proof}
    Consider objects $A, B$ and morphisms $f, g: A\to B$ with their respective
    equalizer $\<E, i\>$. Now consider an object $X$ with morphisms $x_0, x_1: X
    \to E$ such that $i\circ x_0 = i\circ x_1$, then:
    \[
      \begin{aligned}
        f \circ i \circ x_0
        &= f \circ i \circ x_1\\
        &= g \circ i \circ x_1
      \end{aligned}
    \]

    Finally by the universal property of equalizers $i\circ x_0$ (and $i\circ
    x_1$) must be uniquely factorized as $i\circ k$ for some $k$. As this
    factorization is unique, it must be the case that $x_0 = x_1 = k$.
  \end{proof}
\end{theorem}

\begin{theorem}
  Every epic equalizer is an isomorphism.

  \begin{proof}
    % https://q.uiver.app/#q=WzAsNCxbMCwxLCJFIl0sWzEsMSwiQSJdLFsyLDEsIkIiXSxbMCwwLCJBIl0sWzEsMiwiZiIsMCx7Im9mZnNldCI6LTF9XSxbMSwyLCJnIiwyLHsib2Zmc2V0IjoxfV0sWzMsMSwiXFxpZF9BIl0sWzMsMCwiayIsMix7InN0eWxlIjp7ImJvZHkiOnsibmFtZSI6ImRhc2hlZCJ9fX1dLFswLDEsImkiLDJdXQ==&macro_url=https%3A%2F%2Fgist.githubusercontent.com%2Faortega0703%2Fa1fd97cb097b8142e63a6fbf0cdb0f76%2Fraw%2Fb46a955b6b1f06908105b870088e59a10049fc60%2Fmacros.tex
    \[\begin{tikzcd}[ampersand replacement=\&]
      A \\
      E \& A \& B
      \arrow["f", shift left=1, from=2-2, to=2-3]
      \arrow["g"', shift right=1, from=2-2, to=2-3]
      \arrow["{\id_A}", from=1-1, to=2-2]
      \arrow["k"', dashed, from=1-1, to=2-1]
      \arrow["i"', from=2-1, to=2-2]
    \end{tikzcd}\]

    Consider an epic equalizer $i:E\to A$, then as $f\circ i = g\circ i$ it
    follows that $f= g$. Therefore $f\circ \id_A = g \circ \id_A$ which by the
    universal property of equalizers implies the existance of a unique morphism
    $k:A\to E$ such that $i\circ k = \id_A$. Then by Theorem
    \ref{thm:equalizer_mono}:
    \[
      \begin{aligned}
        i\circ k &= \id_A\\
        i\circ k \circ i &= i\\
        k\circ i &= \id_E
      \end{aligned}
    \]
  \end{proof}
\end{theorem}

\begin{theorem}
  Any category with all products and all equalizers is complete.

  \begin{proof}
    % https://q.uiver.app/#q=WzAsOCxbMiwxLCJcXHByb2RcXGxpbWl0c197SVxcaW4gXFxJXzB9IEYoSSkiXSxbMiwyLCJGKFxcbWF0aHJte2RvbX1cXCB1KSJdLFszLDEsIlxccHJvZFxcbGltaXRzX3t1XFxpbiBcXElfMX1GKFxcbWF0aHJte2NvZH1cXCB1KSJdLFszLDIsIkYoXFxtYXRocm17Y29kfVxcIHUpIl0sWzAsMCwiRSJdLFsxLDEsIkYoSSkiXSxbMCwyLCJYIl0sWzMsMCwiRihcXG1hdGhybXtjb2R9XFwgdSkiXSxbMCwxLCJcXHBpX3tcXG1hdGhybXtkb219XFwgdX0iLDJdLFswLDIsImYiLDAseyJvZmZzZXQiOi0xLCJzdHlsZSI6eyJib2R5Ijp7Im5hbWUiOiJkYXNoZWQifX19XSxbMCwyLCJnIiwyLHsib2Zmc2V0IjoxLCJzdHlsZSI6eyJib2R5Ijp7Im5hbWUiOiJkYXNoZWQifX19XSxbNCwwLCJlIiwwLHsic3R5bGUiOnsiYm9keSI6eyJuYW1lIjoiZGFzaGVkIn19fV0sWzAsNSwiXFxwaV9JIiwyXSxbNCw1LCJcXG11X0kiLDIseyJzdHlsZSI6eyJib2R5Ijp7Im5hbWUiOiJkYXNoZWQifX19XSxbNiw1LCJ4X0kiXSxbNiw0LCJoJyIsMCx7InN0eWxlIjp7ImJvZHkiOnsibmFtZSI6ImRhc2hlZCJ9fX1dLFs2LDAsImgiLDIseyJzdHlsZSI6eyJib2R5Ijp7Im5hbWUiOiJkYXNoZWQifX19XSxbMSwzLCJGKHUpIiwyXSxbMCw3LCJcXHBpX3tcXG1hdGhybXtjb2R9XFwgdX0iXSxbMiw3LCJcXHBpX3t1fSIsMl0sWzIsMywiXFxwaV97dX0iLDJdXQ==&macro_url=https%3A%2F%2Fgist.githubusercontent.com%2Faortega0703%2Fa1fd97cb097b8142e63a6fbf0cdb0f76%2Fraw%2Fb46a955b6b1f06908105b870088e59a10049fc60%2Fmacros.tex
    \[\begin{tikzcd}[ampersand replacement=\&]
      E \&\&\& {F(\mathrm{cod}\ u)} \\
      \& {F(I)} \& {\prod\limits_{I\in \I_0} F(I)} \& {\prod\limits_{u\in \I_1}F(\mathrm{cod}\ u)} \\
      X \&\& {F(\mathrm{dom}\ u)} \& {F(\mathrm{cod}\ u)}
      \arrow["{\pi_{\mathrm{dom}\ u}}"', from=2-3, to=3-3]
      \arrow["f", shift left=1, dashed, from=2-3, to=2-4]
      \arrow["g"', shift right=1, dashed, from=2-3, to=2-4]
      \arrow["e", dashed, from=1-1, to=2-3]
      \arrow["{\pi_I}"', from=2-3, to=2-2]
      \arrow["{\mu_I}"', dashed, from=1-1, to=2-2]
      \arrow["{x_I}", from=3-1, to=2-2]
      \arrow["{h'}", dashed, from=3-1, to=1-1]
      \arrow["h"', dashed, from=3-1, to=2-3]
      \arrow["{F(u)}"', from=3-3, to=3-4]
      \arrow["{\pi_{\mathrm{cod}\ u}}", from=2-3, to=1-4]
      \arrow["{\pi_{u}}"', from=2-4, to=1-4]
      \arrow["{\pi_{u}}"', from=2-4, to=3-4]
    \end{tikzcd}\]

    Consider a category $\C$ with all products and all equalizers, and a functor
    $F:\I\to \C$. It is possible to construct the products $A=\prod\limits_{I\in
    \I_0} F(I)$ and $B=\prod\limits_{u\in \I_1} F(\mathrm{cod}\ u)$. As $A$ has
    morphisms into $F(\mathrm{cod}\ u)$ for all $u$ (namely $\pi_{\mathrm{cod}\
    u}$ and $F(u) \circ \pi_{\mathrm{dom}\ u}$), then by the universal property
    of $B$ there exists unique morphisms $f, g: A \to B$ such that:
    \[
      \begin{aligned}
        \pi_u \circ f &= {\mathrm{cod}\ u}\\
        \pi_u \circ g &= F(u) \circ \pi_{\mathrm{dom}\ u}
      \end{aligned}
    \]

    Then, it is possible to construct the equalizer $\<E, e\>$ of $f,g$ which
    has morphisms into $F(I)$ for all $I$ (namely $\mu_I=\pi_I \circ i$). Then
    take $(u:J\to K)\in \I_1$:
    \[
      \begin{aligned}
        F(u) \circ \mu_J
        &= F(u) \circ \pi_J \circ e\\
        &= \pi_u \circ g \circ e\\
        &= \pi_u \circ f \circ e\\
        &= \pi_K \circ i\\
        &= \mu_K
      \end{aligned}
    \]

    This makes $\<E, \mu_*\>$ a cone over $F$. Consider a cone $\<X, x_*\>$, it
    has morphisms into $F(I)$ for all $I$, therefore by the universal property
    of $A$ there exists a unique morphism $h: X\to A$ such that $x_I = \pi_I
    \circ h$. Then, as $\<X, x_*\>$ is a cone, for any $u: j \to k\in \I_1$:
    \[
      \begin{aligned}
        F(u) \circ x_J &= x_K\\
        F(u) \circ \pi_J \circ h &= \pi_K \circ h\\
        F(u)\circ\pi_{\mathrm{dom}\ u}\circ h &= \pi_{\mathrm{cod}\ u} \circ h\\
        \pi_u \circ g \circ h &= \pi_u \circ f \circ h
      \end{aligned}
    \]

    By Theorem \ref{thm:proj_jointly_monic} $g\circ h = f \circ h$. As $h$
    equalizes $f$ and $g$ there exists a unique arrow $h': X\to E$ that makes
    $e\circ h' = h$, then:
    \[
      \begin{aligned}
        x_I &= \pi_I \circ h\\
        &= \pi_I \circ e \circ h'\\
        &= \mu_I \circ h'
      \end{aligned}
    \]

    As $x_I$ factors uniquely through $\mu_I$, $\<E, \mu_*\>$ corresponds to the
    limit of $F$.
  \end{proof}
\end{theorem}

\subsection{Pullbacks}

\begin{definition}[Pullback]\index{Pullback}

  A pullback is the limit of a
  diagram~\parencite[p.~114]{leinster:basic_category_theory}:
  % https://q.uiver.app/#q=WzAsMyxbMiwwLCJCIl0sWzAsMCwiQSJdLFsxLDAsIkMiXSxbMCwyLCJnIiwyXSxbMSwyLCJmIl1d
  \[\begin{tikzcd}[ampersand replacement=\&]
    A \& C \& B
    \arrow["g"', from=1-3, to=1-2]
    \arrow["f", from=1-1, to=1-2]
  \end{tikzcd}\]
\end{definition}

\begin{remark}
  A pullback of objects and morphism $A\overset{f}{\to} C \overset{g}{\from} B$
  is composed by the object $A \times_C B$ and morphisms $\pi_0:A \times_C B$,
  $\pi_1: A \times_C B\to B$ which form a commutative square:
  % https://q.uiver.app/#q=WzAsNCxbMSwwLCJCIl0sWzAsMSwiQSJdLFsxLDEsIkMiXSxbMCwwLCJBXFx0aW1lc19DQiJdLFswLDIsImciXSxbMSwyLCJmIiwyXSxbMywxLCJcXHBpXzAiLDJdLFszLDAsIlxccGlfMSJdLFszLDIsIiIsMSx7InN0eWxlIjp7Im5hbWUiOiJjb3JuZXIifX1dXQ==
  \[\begin{tikzcd}[ampersand replacement=\&]
    {A\times_CB} \& B \\
    A \& C
    \arrow["g", from=1-2, to=2-2]
    \arrow["f"', from=2-1, to=2-2]
    \arrow["{\pi_0}"', from=1-1, to=2-1]
    \arrow["{\pi_1}", from=1-1, to=1-2]
    \arrow["\lrcorner"{anchor=center, pos=0.125}, draw=none, from=1-1, to=2-2]
  \end{tikzcd}\]
\end{remark}

\begin{theorem}
  Pullback squares preserve monomorphisms i.e. if $g$ is a monomorphism in the
  following pullback square then so is $\pi_1$:
  % https://q.uiver.app/#q=WzAsNCxbMSwwLCJCIl0sWzAsMSwiQSJdLFsxLDEsIkMiXSxbMCwwLCJBXFx0aW1lc19DQiJdLFswLDIsImciLDAseyJzdHlsZSI6eyJ0YWlsIjp7Im5hbWUiOiJtb25vIn19fV0sWzEsMiwiZiIsMl0sWzMsMSwiXFxwaV8wIiwyXSxbMywwLCJcXHBpXzEiXSxbMywyLCIiLDEseyJzdHlsZSI6eyJuYW1lIjoiY29ybmVyIn19XV0=
  \[\begin{tikzcd}[ampersand replacement=\&]
    {A\times_CB} \& B \\
    A \& C
    \arrow["g", tail, from=1-2, to=2-2]
    \arrow["f"', from=2-1, to=2-2]
    \arrow["{\pi_0}"', from=1-1, to=2-1]
    \arrow["{\pi_1}", from=1-1, to=1-2]
    \arrow["\lrcorner"{anchor=center, pos=0.125}, draw=none, from=1-1, to=2-2]
  \end{tikzcd}\]

  \begin{proof}
    Consider an object $X$ with morphisms $p, q: X\to A\times_C B$ such that
    $\pi_0 \circ p = \pi_0 \circ q$, then:
    \[
      \begin{aligned}
        \pi_0 \circ p &= \pi_0 \circ q\\
        f \circ \pi_0 \circ p &= f \circ \pi_0 \circ q\\
        g \circ \pi_1 \circ p &= g \circ \pi_1 \circ q\\
        \pi_1 \circ p &= \pi_1 \circ q
      \end{aligned}
    \]

    $X$ has morphisms into $A$ and $B$, namely $\pi_0 \circ p$, $\pi_0 \circ q$
    and $\pi_1 \circ p$, $\pi_1 \circ q$. Then, there must be a unique $h: X\to
    A\times_C B$ such that they are equal to $\pi_0 \circ h$ and $\pi_1 \circ
    h$, implying that $p=q=h$.
  \end{proof}
\end{theorem}

\begin{theorem}
  The following set-up of pullback squares implies that the outter rectangle
  is another pullback:
  % https://q.uiver.app/#q=WzAsNixbMiwxLCJEIl0sWzEsMSwiQiJdLFsyLDAsIkMiXSxbMCwxLCJBIl0sWzEsMCwiQlxcdGltZXNfRCBDIl0sWzAsMCwiQVxcdGltZXNfQihCXFx0aW1lc19EIEMpIl0sWzQsMSwiXFxwaV8wIiwyXSxbNCwyLCJcXHBpXzEiXSxbMSwwLCJmIiwyXSxbMiwwLCJnIl0sWzMsMSwiaCIsMl0sWzUsMywiXFxyaG9fMCIsMl0sWzUsNCwiXFxyaG9fMSJdLFs0LDAsIiIsMix7InN0eWxlIjp7Im5hbWUiOiJjb3JuZXIifX1dLFs1LDEsIiIsMix7InN0eWxlIjp7Im5hbWUiOiJjb3JuZXIifX1dXQ==
  \[\begin{tikzcd}[ampersand replacement=\&]
    {A\times_B(B\times_D C)} \& {B\times_D C} \& C \\
    A \& B \& D
    \arrow["{\pi_0}"', from=1-2, to=2-2]
    \arrow["{\pi_1}", from=1-2, to=1-3]
    \arrow["f"', from=2-2, to=2-3]
    \arrow["g", from=1-3, to=2-3]
    \arrow["h"', from=2-1, to=2-2]
    \arrow["{\rho_0}"', from=1-1, to=2-1]
    \arrow["{\rho_1}", from=1-1, to=1-2]
    \arrow["\lrcorner"{anchor=center, pos=0.125}, draw=none, from=1-2, to=2-3]
    \arrow["\lrcorner"{anchor=center, pos=0.125}, draw=none, from=1-1, to=2-2]
  \end{tikzcd}\]

  Put another way, $A\times_B (B\times_D C) \cong A\times_D C$.
  \begin{proof}
    Consider an object $X$ with morphisms $p: X\to A$ and $q:X\to C$. As there
    are morphism from $X$ to $B$ and $C$ namely $h\circ p$ and $q$, there must
    be a unique morphism $\delta: X\to A\times_C B$ such that:
    \[
      \begin{gathered}
        h\circ p = \pi_0 \circ \delta
      \end{gathered}
      \qquad
      \begin{gathered}
        q = \pi_1 \circ \delta
      \end{gathered}
    \]

    Now, as there are morphisms from $X$ to $A$ and $B\times_D C$ namely $p$ and
    $\delta$, there must exist a unique morphism $\varepsilon: X\to
    A\times_B(B\times_D C)$ such that:
    \[
      \begin{gathered}
        p = \rho_0 \circ \varepsilon
      \end{gathered}
      \qquad
      \begin{gathered}
        \delta = \rho_1 \circ \varepsilon
      \end{gathered}
    \]

    Therefore, for any object $X$ with morphisms $p:X\to A$ and $q:X\to C$ there
    is a unique morphism $\epsilon: X\to A\times_B(B\times_D C)$ that factors
    them as $p=\rho_0 \circ \varepsilon$ and $q=\pi_1\circ \rho_1\circ
    \varepsilon$ making $A\times_B(B\times_D C)$ the pullback of $A$ and $C$.
    Finally by Theorem \ref{thm:limit_iso} $A\times_B(B\times_D C) \cong
    A\times_D C$.
  \end{proof}
\end{theorem}