\section{Limits}

\begin{definition}[Limit\index{Limit}\label{def:limit}]
	The limit of a functor $F:\C\to \D$ is a cone with vertex $d\in \D$ and
	morphisms $\big(\alpha_c: d \to F(c)\big)_{c\in \C_0}$ such that for any other
	cone with vertex $d'\in D$ and morphisms $\big(\beta_c: d' \to F(c)\big)_{c\in
	\C_0}$ there exists a unique map $f:d'\to d$ such that the following diagram
	commutes for all $c\in
	\C_0$~\parencite[p.~118]{leinster:basic_category_theory}:
	% https://q.uiver.app/?q=WzAsMyxbMCwxLCJkIl0sWzAsMCwiZCciXSxbMSwxLCJGKGMpIl0sWzAsMiwiXFxhbHBoYV9jIiwyXSxbMSwyLCJcXGJldGFfYyJdLFsxLDAsImYiLDIseyJzdHlsZSI6eyJib2R5Ijp7Im5hbWUiOiJkYXNoZWQifX19XV0=
	\[\begin{tikzcd}
		{d'} \\
		d & {F(c)}
		\arrow["{\alpha_c}"', from=2-1, to=2-2]
		\arrow["{\beta_c}", from=1-1, to=2-2]
		\arrow["f"', dashed, from=1-1, to=2-1]
	\end{tikzcd}\]
\end{definition}

\begin{definition}[Limit as a Terminal Object\index{Limit!as a Terminal Object}]
	Equivalently to Definition \ref{def:limit}, the limit of a functor $F:\I\to
	\C$ can be expressed as a terminal object in the category $\mathrm{cone}(F)$.
\end{definition}

\begin{definition}[Limit as a Terminal Morphism\index{Limit!as a Terminal Morphism}]
	Equivalently to Definition \ref{def:limit}, the limit of a functor $F:\I\to
	\C$ can be expressed as a terminal morphism from the diagonal functor $\Delta:
	\C \to \C^\I$ to $F$ (thought of as an object in $\C^\I$).
	% https://q.uiver.app/?q=WzAsNSxbMCwxLCJjIl0sWzAsMCwiYyciXSxbMSwxLCJcXERlbHRhX2MiXSxbMSwwLCJcXERlbHRhX3tjJ30iXSxbMiwxLCJGIl0sWzEsMCwiaCIsMix7InN0eWxlIjp7ImJvZHkiOnsibmFtZSI6ImRhc2hlZCJ9fX1dLFszLDIsIlxcRGVsdGEoaCkiLDIseyJzdHlsZSI6eyJib2R5Ijp7Im5hbWUiOiJkYXNoZWQifX19XSxbMyw0LCJcXGJldGEiXSxbMiw0LCJcXGFscGhhIiwyXV0=&macro_url=https%3A%2F%2Fraw.githubusercontent.com%2Faortega0703%2Fnotes-category-theory%2Fmain%2Fsrc%2Fmacros.tex
	\[\begin{tikzcd}[ampersand replacement=\&]
		{c'} \& {\Delta_{c'}} \\
		c \& {\Delta_c} \& F
		\arrow["h"', dashed, from=1-1, to=2-1]
		\arrow["{\Delta(h)}"', dashed, from=1-2, to=2-2]
		\arrow["\beta", from=1-2, to=2-3]
		\arrow["\alpha"', from=2-2, to=2-3]
	\end{tikzcd}\]
\end{definition}

\begin{theorem}[Uniqueness of Limits\label{thm:limit_iso}]
	If there exists multiple limits for the same functor, then they are all
	isomorphic.

	\begin{proof}
		A limit can be seen as a terminal morphism, which are unique up to
		Isomorphism (Theorem \ref{thm:terminal_morphism_iso}), therefore limits are
		also unique up to isomorphism.
	\end{proof}
\end{theorem}

\subimport{product/}{product.tex}

\subsection{Equalizer}

\begin{definition}[Equalizer]\index{Equalizer}
  An equalizer is the limit of a
  diagram~\parencite[p.~112]{leinster:basic_category_theory}:
  % https://q.uiver.app/#q=WzAsMixbMCwwLCJYIl0sWzEsMCwiWSJdLFswLDEsImYiLDAseyJvZmZzZXQiOi0xfV0sWzAsMSwiZyIsMix7Im9mZnNldCI6MX1dXQ==
  \[\begin{tikzcd}[ampersand replacement=\&]
    X \& Y
    \arrow["f", shift left=1, from=1-1, to=1-2]
    \arrow["g"', shift right=1, from=1-1, to=1-2]
  \end{tikzcd}\]
\end{definition}

\begin{remark}
  An equalizer of objects $A, B$ and morphisms $f, g: A\to B$ is composed by an
  object $E$ together with a morphism $i:E\to A$.
\end{remark}

\begin{theorem}
  Every equalizer morphism is a monomorphism.

  \begin{proof}
    Consider objects $A, B$ and morphisms $f, g: A\to B$ with their respective
    equalizer $\<E, i\>$. Now consider an object $X$ with morphisms $x_0, x_1: X
    \to E$ such that $i\circ x_0 = i\circ x_1$, then:
    \[
      \begin{aligned}
        f \circ i \circ x_0
        &= f \circ i \circ x_1\\
        &= g \circ i \circ x_1
      \end{aligned}
    \]

    Finally by the universal property of equalizers $i\circ x_0$ (and $i\circ
    x_1$) must be uniquely factorized as $i\circ k$ for some $k$. As this
    factorization is unique, it must be the case that $x_0 = x_1 = k$.
  \end{proof}
\end{theorem}

\subsection{Pullbacks}

\begin{definition}[Pullback]\index{Pullback}

  A pullback is a limit from a diagram~\parencite[p.~114]{leinster:basic_category_theory}:
  % https://q.uiver.app/#q=WzAsMyxbMiwwLCJZIl0sWzAsMCwiWCJdLFsxLDAsIloiXSxbMCwyLCJ0IiwyXSxbMSwyLCJzIl1d
  \[\begin{tikzcd}[ampersand replacement=\&]
    X \& Z \& Y
    \arrow["t"', from=1-3, to=1-2]
    \arrow["s", from=1-1, to=1-2]
  \end{tikzcd}\]
\end{definition}