\section{Via Triangle Identities}

\begin{definition}[Adjunction]\index{Adjunction!via Triangle
  Identities}\label{def:adjunction_triangle}

  The pair of functors $\<L: \C\to \D$, $R: \D\to \C\>$ is left/right adjoint of
  each other when there exists a unit ($\eta$) and counit
  ($\epsilon$)~\parencite[p.~53]{leinster:basic_category_theory}:
  \[
    \begin{gathered}
      \eta: \id_\C \Rightarrow R \circ L
    \end{gathered}
    \qquad
    \begin{gathered}
      \epsilon: L \circ R \Rightarrow \id_\D
    \end{gathered}
  \]
  Such that the triangular identities are satisfied (the following diagrams
  commute):

  % https://q.uiver.app/#q=WzAsMyxbMCwwLCJSKEQpIl0sWzEsMCwiKFJcXGNpcmMgTFxcY2lyYyBSKShEKSJdLFswLDEsIlIoRCkiXSxbMCwxLCJcXGV0YV97UihEKX0iXSxbMSwyLCJSKFxcdmFyZXBzaWxvbl9EKSJdLFswLDIsIlxcaWRfe1IoRCl9IiwyXV0=&macro_url=https%3A%2F%2Fraw.githubusercontent.com%2Faortega0703%2Fnotes-category-theory%2Fmain%2Fsrc%2Fmacros.tex
  \[\begin{tikzcd}[ampersand replacement=\&]
    {R(D)} \& {(R\circ L\circ R)(D)} \\
    {R(D)}
    \arrow["{\eta_{R(D)}}", from=1-1, to=1-2]
    \arrow["{R(\varepsilon_D)}", from=1-2, to=2-1]
    \arrow["{\id_{R(D)}}"', from=1-1, to=2-1]
  \end{tikzcd}
  % https://q.uiver.app/#q=WzAsMyxbMSwwLCJMKEMpIl0sWzAsMSwiKExcXGNpcmMgUlxcY2lyYyBMKShDKSJdLFsxLDEsIkwoQykiXSxbMCwxLCJMKFxcZXRhX0MpIiwyXSxbMSwyLCJcXHZhcmVwc2lsb25fe0woQyl9IiwyXSxbMCwyLCJcXGlkX3tMKEMpfSJdXQ==&macro_url=https%3A%2F%2Fraw.githubusercontent.com%2Faortega0703%2Fnotes-category-theory%2Fmain%2Fsrc%2Fmacros.tex
  \begin{tikzcd}[ampersand replacement=\&]
    \& {L(C)} \\
    {(L\circ R\circ L)(C)} \& {L(C)}
    \arrow["{L(\eta_C)}"', from=1-2, to=2-1]
    \arrow["{\varepsilon_{L(C)}}"', from=2-1, to=2-2]
    \arrow["{\id_{L(C)}}", from=1-2, to=2-2]
  \end{tikzcd}\]
\end{definition}

\begin{theorem}[Adjunctions Definitions]
  Definitions \ref{def:adjunction_isomorphism} and \ref{def:adjunction_triangle}
  of adjoint functors are equivalent.

  \begin{proof}
    The proof consists of two parts:
    \begin{description}
      \item[($\implies$)] Theorems \ref{thm:unit_exists} and
        \ref{thm:counit_exists} provide the existence of $\eta: \id_\C \to
        R\circ L$ and $\epsilon: L\circ R \to \id_\D$. Now to obtain the
        triangle identities consider:
        \[
          \begin{aligned}
              &\phantom{=\ }\id_{L(C)}\\
              &= \overline{\eta_C}\\
              &= \overline{R(\id_{L(C)})\circ R(\id_{L(C)})\circ \eta_C}\\
              &= \id_{L(C)} \circ \epsilon_{L(C)}\circ L(\eta_C)\\
              &= \epsilon_{L(C)}\circ L(\eta_C)
          \end{aligned}
          \qquad
          \begin{aligned}
              &\phantom{=\ }\id_{R(D)}\\
              &= \overline{\epsilon_D}\\
              &= \overline{\epsilon_D
                \circ L(\id_{R(D)})\circ L(\id_{R(D)})}\\
              &= R(\epsilon_D) \circ \eta_{R(D)} \circ \id_{R(D)}\\
              &= R(\epsilon_D) \circ \eta_{R(D)}
          \end{aligned}
        \]
      \item[($\impliedby$)] For a pair of left/right adjoints $L:\C\to \D$,
        $R:\D\to \C$, consider two functions:
        \[
          \begin{aligned}
            \phi: \D(L(C), D) &\to \C(C, R(D))\\
            u &\mapsto R(u)\circ \eta_C
          \end{aligned}
          \qquad
          \begin{aligned}
            \psi: \C(C, R(D)) &\to \D(L(C), D)\\
            v &\mapsto \epsilon_D \circ L(v)
          \end{aligned}
        \]
        By using the triangle identities, and the naturality condition of $\eta$
        and $\epsilon$:

        % https://q.uiver.app/#q=WzAsNSxbMCwxLCJSKEQpIl0sWzEsMSwiKFJcXGNpcmMgTFxcY2lyYyBSKShEKSJdLFswLDIsIlIoRCkiXSxbMSwwLCIoTFxcY2lyYyBSKShDKSJdLFswLDAsIkMiXSxbMCwyLCJcXGlkX3tSKEQpfSIsMl0sWzAsMSwiXFxldGFfe1IoRCl9Il0sWzEsMiwiUihcXHZhcmVwc2lsb25fRCkiXSxbMywxLCIoTFxcY2lyYyBSKSh2KSJdLFs0LDAsInYiLDJdLFs0LDMsIlxcZXRhX0MiXV0=&macro_url=https%3A%2F%2Fraw.githubusercontent.com%2Faortega0703%2Fnotes-category-theory%2Fmain%2Fsrc%2Fmacros.tex
        \[\begin{tikzcd}[ampersand replacement=\&]
          C \& {(L\circ R)(C)} \\
          {R(D)} \& {(R\circ L\circ R)(D)} \\
          {R(D)}
          \arrow["{\id_{R(D)}}"', from=2-1, to=3-1]
          \arrow["{\eta_{R(D)}}", from=2-1, to=2-2]
          \arrow["{R(\varepsilon_D)}", from=2-2, to=3-1]
          \arrow["{(L\circ R)(v)}", from=1-2, to=2-2]
          \arrow["v"', from=1-1, to=2-1]
          \arrow["{\eta_C}", from=1-1, to=1-2]
        \end{tikzcd}
        \quad
        \begin{aligned}
          (\phi\circ\psi)(v)
            &= R(\epsilon_D)\circ (R\circ L)(v) \circ \eta_C\\
            &= R(\epsilon_D) \circ \eta_{R(D)} \circ v\\
            &= \id_{R(D)} \circ v\\
            &= v
        \end{aligned}
        \]
        % https://q.uiver.app/#q=WzAsNSxbMSwwLCJMKEMpIl0sWzAsMSwiKExcXGNpcmMgUlxcY2lyYyBMKShDKSJdLFsxLDEsIkwoQykiXSxbMCwyLCIoTFxcY2lyYyBSKShkKSJdLFsxLDIsImQiXSxbMCwyLCJcXGlkX3tMKEMpfSJdLFswLDEsIkwoXFxldGFfQykiLDJdLFsxLDIsIlxcdmFyZXBzaWxvbl97TChDKX0iXSxbMiw0LCJ1Il0sWzEsMywiKExcXGNpcmMgUikodSkiLDJdLFszLDQsIlxcdmFyZXBzaWxvbl9EIiwyXV0=&macro_url=https%3A%2F%2Fraw.githubusercontent.com%2Faortega0703%2Fnotes-category-theory%2Fmain%2Fsrc%2Fmacros.tex
        \[\begin{tikzcd}[ampersand replacement=\&]
          \& {L(C)} \\
          {(L\circ R\circ L)(C)} \& {L(C)} \\
          {(L\circ R)(d)} \& d
          \arrow["{\id_{L(C)}}", from=1-2, to=2-2]
          \arrow["{L(\eta_C)}"', from=1-2, to=2-1]
          \arrow["{\varepsilon_{L(C)}}", from=2-1, to=2-2]
          \arrow["u", from=2-2, to=3-2]
          \arrow["{(L\circ R)(u)}"', from=2-1, to=3-1]
          \arrow["{\varepsilon_D}"', from=3-1, to=3-2]
        \end{tikzcd}
        \quad
        \begin{aligned}
          (\psi\circ \phi)(u)
          &= \epsilon_D \circ (L\circ R)(u) \circ L(\eta_C)\\
          &= u \circ \epsilon_{L(C)} \circ L(\eta_C)\\
          &= u \circ \id_{L(C)}\\
          &= u
        \end{aligned}
        \] In order to prove naturality it is sufficient to prove that for
        morphisms $g:C'\to C$ and $h:D\to D'$ Theorem \ref{thm:transpose2}
        holds:
        \[
          \begin{aligned}
            \phi(h\circ u)
              &= R(h\circ u) \circ \eta_C\\
              &= R(h) \circ R(u) \circ \eta_C\\
              &= R(h) \circ \phi(u)
          \end{aligned}
          \qquad
          \begin{aligned}
            \psi(v\circ g)
              &= \epsilon_D \circ L(v\circ g)\\
              &= \epsilon_D \circ L(v) \circ L(g)\\
              &= \psi(v) \circ L(g)
          \end{aligned}
        \]
    \end{description}
  \end{proof}
\end{theorem}