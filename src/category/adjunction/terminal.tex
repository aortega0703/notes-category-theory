\section{Via Terminal Morphism}

\begin{definition}[Adjunction]\index{Adjunction!via Terminal
  Morphism}\label{def:adjunction_terminal}

  The pair of functors $\<L: \C\to \D$, $R: \D\to \C\>$, is left/right adjoint
  of each other when there exists a terminal morphism $(R(D), \epsilon_D:
  (L\circ R)(D) \to D)$ from every $D\in \D_0$ to
  $L$~\parencite[p.~214]{awodey:category_theory}:
  % https://q.uiver.app/#q=WzAsNSxbMCwxLCJSKEQpIl0sWzAsMCwiQyJdLFsxLDAsIkwoQykiXSxbMSwxLCIoTFxcY2lyYyBSKShEKSJdLFsyLDEsIkQiXSxbMSwwLCJcXG92ZXJsaW5le3V9IiwyLHsic3R5bGUiOnsiYm9keSI6eyJuYW1lIjoiZGFzaGVkIn19fV0sWzIsMywiTChcXG92ZXJsaW5le3V9KSIsMix7InN0eWxlIjp7ImJvZHkiOnsibmFtZSI6ImRhc2hlZCJ9fX1dLFszLDQsIlxcdmFyZXBzaWxvbl9EIiwyXSxbMiw0LCJ1Il1d
  \[\begin{tikzcd}[ampersand replacement=\&]
    C \& {L(C)} \\
    {R(D)} \& {(L\circ R)(D)} \& D
    \arrow["{\overline{u}}"', dashed, from=1-1, to=2-1]
    \arrow["{L(\overline{u})}"', dashed, from=1-2, to=2-2]
    \arrow["{\varepsilon_D}"', from=2-2, to=2-3]
    \arrow["u", from=1-2, to=2-3]
  \end{tikzcd}\]
\end{definition}

\begin{theorem}[Naturality of $\epsilon$]
  In Definition \ref{def:adjunction_terminal}, $\epsilon: L\circ R \to \id_D$
  is a natural transformation.

  \begin{proof}
    Take two objects $D,D'\in \D_0$ with a morphism $h:D\to D'$. Then by the
    definition of $\varepsilon_{D}$ it follows that $\eta_C\circ h = (R\circ
    L)(C)\circ \eta_{C'}$, fulfilling the naturality condition in:
    % https://q.uiver.app/#q=WzAsNixbMiwwLCJEIl0sWzIsMSwiRCciXSxbMSwwLCIoTFxcY2lyYyBSKShEKSJdLFswLDAsIlIoRCkiXSxbMCwxLCJSKEQnKSJdLFsxLDEsIihMXFxjaXJjIFIpKEQnKSJdLFswLDEsImgiXSxbMyw0LCJSKGgpIiwyLHsic3R5bGUiOnsiYm9keSI6eyJuYW1lIjoiZGFzaGVkIn19fV0sWzIsMCwiXFxlcHNpbG9uX0QiXSxbNSwxLCJcXGVwc2lsb25fe0QnfSIsMl0sWzIsMSwiaFxcY2lyY1xcZXBzaWxvbl9EIiwxXSxbMiw1LCIoTFxcY2lyYyBSKShoKSIsMix7InN0eWxlIjp7ImJvZHkiOnsibmFtZSI6ImRhc2hlZCJ9fX1dXQ==&macro_url=https%3A%2F%2Fraw.githubusercontent.com%2Faortega0703%2Fnotes-category-theory%2Fmain%2Fsrc%2Fmacros.tex
    \[\begin{tikzcd}[ampersand replacement=\&]
      {R(D)} \& {(L\circ R)(D)} \& D \\
      {R(D')} \& {(L\circ R)(D')} \& {D'}
      \arrow["h", from=1-3, to=2-3]
      \arrow["{R(h)}"', dashed, from=1-1, to=2-1]
      \arrow["{\epsilon_D}", from=1-2, to=1-3]
      \arrow["{\epsilon_{D'}}"', from=2-2, to=2-3]
      \arrow["{h\circ\epsilon_D}"{description}, from=1-2, to=2-3]
      \arrow["{(L\circ R)(h)}"', dashed, from=1-2, to=2-2]
    \end{tikzcd}\]
  \end{proof}
  \vspace{-\baselineskip}
\end{theorem}

\begin{theorem}[Adjunction Definitions]
  Definitions \ref{def:adjunction_isomorphism} and \ref{def:adjunction_terminal}
  are equivalent.

  \begin{proof}
    The proof consists of two parts:
    \begin{description}
      \item[$(\implies)$] By Theorem \ref{thm:counit_exists} $\epsilon:(L\circ
        R) \to \id_\D$ exists. As $\overline{u}$ is unique for every $u$, then
        for any $D\in\D_0$ the terminal morphism from $L$ to $D$ can be
        constructed in the following manner:
        % https://q.uiver.app/#q=WzAsNixbMiwwLCJEIl0sWzIsMSwiRCciXSxbMCwwLCJSKEQpIl0sWzAsMSwiUihEJykiXSxbMSwwLCIoTFxcY2lyYyBSKShEKSJdLFsxLDEsIihMXFxjaXJjIFIpKEQnKSJdLFs0LDAsIlxcZXBzaWxvbl9EIl0sWzUsMSwiXFxlcHNpbG9uX3tEJ30iLDJdLFswLDEsImciXSxbMiwzLCJSKGcpIiwyLHsic3R5bGUiOnsiYm9keSI6eyJuYW1lIjoiZGFzaGVkIn19fV0sWzQsMSwiZ1xcY2lyYyBcXGVwc2lsb25fRCIsMV0sWzQsNSwiKExcXGNpcmMgUikoZykiLDIseyJzdHlsZSI6eyJib2R5Ijp7Im5hbWUiOiJkYXNoZWQifX19XV0=&macro_url=https%3A%2F%2Fraw.githubusercontent.com%2Faortega0703%2Fnotes-category-theory%2Fmain%2Fsrc%2Fmacros.tex
        \[\begin{tikzcd}[ampersand replacement=\&]
          {R(D)} \& {(L\circ R)(D)} \& D \\
          {R(D')} \& {(L\circ R)(D')} \& {D'}
          \arrow["{\epsilon_D}", from=1-2, to=1-3]
          \arrow["{\epsilon_{D'}}"', from=2-2, to=2-3]
          \arrow["g", from=1-3, to=2-3]
          \arrow["{R(g)}"', dashed, from=1-1, to=2-1]
          \arrow["{g\circ \epsilon_D}"{description}, from=1-2, to=2-3]
          \arrow["{(L\circ R)(g)}"', dashed, from=1-2, to=2-2]
        \end{tikzcd}\]
      \item[$(\impliedby)$] Consider the function:
        \[
          \begin{aligned}
            \psi:&&\C(C, R(D)) &\to \D(L(C), D)\\
            &&v&\mapsto \epsilon_D\circ L(v)
          \end{aligned}
        \]
        By Definition \ref{def:adjunction_terminal} every $u:L(C)\to D$ can be
        expressed as $\psi(\overline{u})$ for a unique $\overline{u}:C\to R(D)$,
        which makes $\psi$ a bijection and therefore an isomorphism. To prove
        naturality of $\psi$ consider $g:C'\to C$ and $h:D\to D'$:
        % https://q.uiver.app/#q=WzAsNCxbMSwwLCJcXG1hdGhjYWx7RH0oTChDKSwgRCkiXSxbMCwwLCJcXG1hdGhjYWx7Q30oQywgUihEKSkiXSxbMSwxLCJcXG1hdGhjYWx7RH0oTChDJyksIEQnKSJdLFswLDEsIlxcbWF0aGNhbHtDfShDJywgUihEJykpIl0sWzEsMCwiXFxwc2lfe1xcbGFuZ2xlIEMsRFxccmFuZ2xlfSJdLFswLDIsIlxcbWF0aGNhbHtEfShMKGcpLGgpIl0sWzMsMiwiXFxwc2lfe1xcbGFuZ2xlIEMnLEQnXFxyYW5nbGV9IiwyXSxbMSwzLCJcXG1hdGhjYWx7Q30oZyxSKGgpKSIsMl0sWzMsMiwiXFxjb25nIl0sWzEsMCwiXFxjb25nIiwyXV0=&macro_url=https%3A%2F%2Fraw.githubusercontent.com%2Faortega0703%2Fnotes-category-theory%2Fmain%2Fsrc%2Fmacros.tex
        \[\begin{tikzcd}[ampersand replacement=\&]
          {\mathcal{C}(C, R(D))} \& {\mathcal{D}(L(C), D)} \\
          {\mathcal{C}(C', R(D'))} \& {\mathcal{D}(L(C'), D')}
          \arrow["{\psi_{\langle C,D\rangle}}", from=1-1, to=1-2]
          \arrow["{\mathcal{D}(L(g),h)}", from=1-2, to=2-2]
          \arrow["{\psi_{\langle C',D'\rangle}}"', from=2-1, to=2-2]
          \arrow["{\mathcal{C}(g,R(h))}"', from=1-1, to=2-1]
          \arrow["\cong", from=2-1, to=2-2]
          \arrow["\cong"', from=1-1, to=1-2]
        \end{tikzcd}\]
        \[
          \begin{aligned}
            \big(\D(L(g), h)\circ \psi_{C, D}\big)(v)
            &= \D(L(g), h) (\epsilon_D \circ L(v))\\
            &= h \circ \epsilon_d \circ L(v) \circ L(g)\\
            &= \epsilon_{D'}\circ (L\circ R)(h) \circ L(v) \circ L(g)\\
            &= \epsilon_{D'}\circ L(R(h) \circ v \circ g)\\
            &= \psi_{C', D'} (R(h) \circ v \circ g)\\
            &= \psi_{C', D'} (\C(g, R(h))(v))\\
            &= \big(\psi_{C', D'} \circ \C(g, R(h))\big)(v)
          \end{aligned}
        \]
    \end{description}
    \vspace{-\baselineskip}
  \end{proof}
\end{theorem}