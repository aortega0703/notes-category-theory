\section{Via Natural Isomorphism}

\begin{definition}[Adjunction\index{Adjunction!via Natural Isomorphism}\label{def:adjunction_isomorphism}]
  The pair of functors $\<L: \C\to \D$, $R: \D\to \C\>$ is said to be left/right
  adjoint of each other when there is a natural isomorphism between the
  functors~\parencite[p.~41]{leinster:basic_category_theory}:
  \[
    \begin{aligned}
      \D(L(\hole), \hole)
        &: \C^\op \times \D \to \Set\\
      \C(\hole, R(\hole))
        &: \C^\op \times \D \to \Set\\
    \end{aligned}
  \]

  \begin{itemize}
    \item $(c', d)$-Components:
      \[
        \alpha_{c', d}
          = \D(L(c), d) \overset{\cong}{\to} \C(c, R(d))\\
      \]
    \item Naturality Condition:\\
      For every $f = \<g, h\>: \<c', d\> \to \<c, d'\>$, the following commutes:
      % https://q.uiver.app/?q=WzAsNCxbMCwwLCJcXEQoTChjKSwgZCkiXSxbMSwwLCJcXEMoYywgUihkKSkiXSxbMCwxLCJcXEQoTChjJyksZCcpIl0sWzEsMSwiXFxDKGMnLFIoZCcpKSJdLFswLDIsIlxcRChMKGcpLCBoKSIsMl0sWzEsMywiXFxDKGcsIFIoaCkpIl0sWzAsMSwiXFxhbHBoYV97YywgZH0iXSxbMiwzLCJcXGFscGhhX3tjJyxkJ30iLDJdLFswLDEsIlxcY29uZyIsMl0sWzIsMywiXFxjb25nIl1d&macro_url=https%3A%2F%2Fraw.githubusercontent.com%2Faortega0703%2Fnotes-category-theory%2Fmain%2Fsrc%2Fmacros.tex
      \[\begin{tikzcd}[ampersand replacement=\&]
        {\D(L(c), d)} \& {\C(c, R(d))} \\
        {\D(L(c'),d')} \& {\C(c',R(d'))}
        \arrow["{\D(L(g), h)}"', from=1-1, to=2-1]
        \arrow["{\C(g, R(h))}", from=1-2, to=2-2]
        \arrow["{\alpha_{c, d}}", from=1-1, to=1-2]
        \arrow["{\alpha_{c',d'}}"', from=2-1, to=2-2]
        \arrow["\cong"', from=1-1, to=1-2]
        \arrow["\cong", from=2-1, to=2-2]
      \end{tikzcd}\]
  \end{itemize}
\end{definition}

\begin{remark}
  In diagram a pair of left/right adjoints $\<L, R\>$ is indicated by a double
  arrow ($\Rightarrow$) going in the same direction as the left adjoint, or as a
  turnstile ($\vdash$) pointing towards the left adjoint as follows:
  % https://q.uiver.app/?q=WzAsMixbMSwwLCJcXEQiXSxbMCwwLCJcXEMiXSxbMSwwLCJMIiwwLHsiY3VydmUiOi0zfV0sWzAsMSwiUiIsMCx7ImN1cnZlIjotM31dLFsxLDAsIiIsMSx7InNob3J0ZW4iOnsic291cmNlIjoyMCwidGFyZ2V0IjoyMH0sImxldmVsIjoyfV1d&macro_url=https%3A%2F%2Fraw.githubusercontent.com%2Faortega0703%2Fnotes-category-theory%2Fmain%2Fsrc%2Fmacros.tex
  \[\begin{tikzcd}[ampersand replacement=\&]
    \C \& \D
    \arrow["L", curve={height=-18pt}, from=1-1, to=1-2]
    \arrow["R", curve={height=-18pt}, from=1-2, to=1-1]
    \arrow[shorten <=3pt, shorten >=3pt, Rightarrow, from=1-1, to=1-2]
  \end{tikzcd}
    \text{or}
  % https://q.uiver.app/?q=WzAsMixbMSwwLCJcXEQiXSxbMCwwLCJcXEMiXSxbMSwwLCJMIiwwLHsiY3VydmUiOi0zfV0sWzAsMSwiUiIsMCx7ImN1cnZlIjotM31dLFsyLDMsIiIsMCx7ImxldmVsIjoxLCJzdHlsZSI6eyJuYW1lIjoiYWRqdW5jdGlvbiJ9fV1d&macro_url=https%3A%2F%2Fraw.githubusercontent.com%2Faortega0703%2Fnotes-category-theory%2Fmain%2Fsrc%2Fmacros.tex
  \begin{tikzcd}[ampersand replacement=\&]
    \C \& \D
    \arrow[""{name=0, anchor=center, inner sep=0}, "L", curve={height=-18pt}, from=1-1, to=1-2]
    \arrow[""{name=1, anchor=center, inner sep=0}, "R", curve={height=-18pt}, from=1-2, to=1-1]
    \arrow["\dashv"{anchor=center, rotate=-90}, draw=none, from=0, to=1]
  \end{tikzcd}\]
\end{remark}

\begin{definition}[Transpose\index{Transpose}\label{def:transpose}]
  For an adjunction $(L:\C\to \D, R:\D\to \C)$, the transpose $\overline{f}$ of a
  morphism $f$ is given by the following isomorphisms of Definition
  \ref{def:adjunction_isomorphism}:
  \[
    \begin{aligned}
      \phi_{(c, d)}: \D(L(c), d) &\to \C(c, R(d))\\
      u &\mapsto \overline{u}\\
      \psi_{(c, d)}: \C(c, R(d)) &\to \D(L(c), d)\\
      v &\mapsto \overline{v}
    \end{aligned}
  \]
\end{definition}

\begin{theorem}[General Tranpose\label{thm:transpose}]
  For an adjunction $\<L: \C\to \D, R:\D\to \C\>$, the naturality condition is
  equivalent to stating that for any $u\in \D(L(\hole), \hole)$, $g \in \C_1$,
  $h\in \D_1$:
  \begin{align*}
    \overline{h\circ u \circ L(g)} = R(h) \circ \overline{u} \circ g
  \end{align*}

  \begin{proof}
    It follows from equating both paths in the naturality condition, that:
    \[
      \begin{aligned}
      \big(\alpha_{c', d'} \circ \D(L(g), h)\big)(u)
        &= \big(\C(g, R(h)) \circ \alpha_{c, d}\big)(u)\\
      \alpha_{c', d'}(h\circ u \circ L(g))
        &= \C(g, R(h))(\overline{u})\\
      \overline{h\circ u \circ L(g)}
        &= R(h) \circ \overline{u} \circ g
      \end{aligned}
    \]
  \end{proof}
\end{theorem}

\begin{remark}
  As $\phi_{(c, d)}$ and $\psi_{(c, d)}$ are inverses, in Theorem
  \ref{thm:transpose} the overline goes in either side of the equation.
\end{remark}

\begin{theorem}[Alternative Transposes\label{thm:transpose2}]
  Theorem \ref{thm:transpose} is equivalent to stating that the following two
  equations hold for any $u\in \D(L(\hole), \hole)$, $v\in \C(\hole, R(\hole))$,
  $g \in \C_1$, $h\in \D_1$:
  \begin{align*}
    \overline{h\circ u} &= R(h) \circ \overline{u}\\
    \overline{v\circ g} &= \overline{v} \circ L(g)
  \end{align*}

  \begin{proof}
    The proof consists of two parts:
    \begin{description}
      \item[$(\implies)$] By setting either $g=\id_c$ or $h=\id_d$:
        \[
          \begin{aligned}
            \overline{h\circ u}
              &= \overline{h\circ u \circ L(\id_c)}\\
              &= R(h) \circ \overline{u} \circ \id_c\\
              &= R(h) \circ \overline{u}
          \end{aligned}
          \qquad
          \begin{aligned}
            \overline{v\circ g}
              &= \overline{R(\id_d)\circ v\circ g}\\
              &= \id_d \circ \overline{v} \circ L(g)\\
              &= \overline{v} \circ L(g)
          \end{aligned}
        \]
      \item[$(\impliedby)$] Consider the following:
        \[
          \begin{aligned}
            \overline{h \circ u\circ L(g)}
              &= \overline{(h \circ u)\circ L(g)}\\
              &= \overline{h\circ u} \circ g\\
              &= R(h) \circ \overline{u} \circ g
          \end{aligned}
        \]
    \end{description}
  \end{proof}
\end{theorem}

\begin{theorem}[Existence of a unit $\eta$\label{thm:unit_exists}]
  For a pair of left/right adjoints $\<L: \C \to \D$, $R:\D\to \C\>$
  there exists a morphism $\eta_c: c\to (R\circ L)(c)$ that corresponds to
  $\overline{\id_{L(c)}}$ for every $c\in \C_0$.

  \begin{proof}
    Consider definition \ref{def:adjunction_isomorphism} of adjunction:
    \[
      \begin{aligned}
        \D(L(c), d) &\ \:\cong\ \: \C(c, R(d))\\
        \D(L(c), L(c)) &\ \:\cong\ \: \C(c, (R\circ L)(c))\\
        \big(\id_{L(c)}: L(c) \to L(c)\big) \in \D_1
        &\implies \big(\eta_c : c \to (R\circ L)(c)\big)\in \C_1\\
        \overline{\id_{L(c)}} &\ \:=\ \:\eta_c
      \end{aligned}
    \]
  \end{proof}
\end{theorem}

\begin{theorem}[Existence of a counit $\epsilon$\label{thm:counit_exists}]
  For an adjunction $\<L: \C \to \D$, $R:\D\to \C\>$
  there exists a morphism $\epsilon_d: (L\circ R)(d)\to d$ that corresponds to
  $\overline{\id_{R(d)}}$ for every $d\in \D_0$.

  \begin{proof}
    Consider Definition \ref{def:adjunction_isomorphism} of adjunction:
    \[
      \begin{aligned}
        \D(L(c), d) &\ \:\cong\ \: \C(c, R(d))\\
        \D((L\circ R)(d), d) &\ \:\cong\ \: \C(R(d), R(d))\\
        \big(\epsilon:(L\circ R)(d)\to d\big)\in D_1 &\impliedby
          \big(\id_{R(d)}: R(d) \to R(d)\big)\\
        \epsilon_d &\ \:=\ \:\overline{\id_{R(d)}}
      \end{aligned}
    \]
  \end{proof}
\end{theorem}