\section{Via Initial Morphism}

\begin{definition}[Adjunction\index{Adjunction!via Initial Morphism}\label{def:adjunction_initial}]
  The pair of functors $\<L: \C\to \D$, $R: \D\to \C\>$ is left/right adjoint of
  each other when there is an initial morphism $(L(c), \eta_c:c\to (R\circ
  L)(c))$ from every $c\in \C_0$ to
  $R$~\parencite[p.~208]{awodey:category_theory}:
  % https://q.uiver.app/?q=WzAsNSxbMCwwLCJMKGMpIl0sWzAsMSwiZCJdLFsxLDAsIihSXFxjaXJjIEwpKGMpIl0sWzEsMSwiUihkKSJdLFsyLDAsImMiXSxbMCwxLCJcXG92ZXJsaW5le3Z9IiwyLHsic3R5bGUiOnsiYm9keSI6eyJuYW1lIjoiZGFzaGVkIn19fV0sWzIsMywiUihcXG92ZXJsaW5le3Z9KSIsMix7InN0eWxlIjp7ImJvZHkiOnsibmFtZSI6ImRhc2hlZCJ9fX1dLFs0LDMsInYiXSxbNCwyLCJcXGV0YV9jIiwyXV0=
  \[\begin{tikzcd}[ampersand replacement=\&]
    {L(c)} \& {(R\circ L)(c)} \& c \\
    d \& {R(d)}
    \arrow["{\overline{v}}"', dashed, from=1-1, to=2-1]
    \arrow["{R(\overline{v})}"', dashed, from=1-2, to=2-2]
    \arrow["v", from=1-3, to=2-2]
    \arrow["{\eta_c}"', from=1-3, to=1-2]
  \end{tikzcd}\]
\end{definition}

\begin{theorem}[Naturality of $\eta$]
  In Definition \ref{def:adjunction_initial}, $\eta:\id_C \to R\circ L$ is a
  natural transformation.

  \begin{proof}
    Take two objects $c,c'\in \C_0$ with a morphism $g:c'\to c$. Then by the
    definition of $\eta_{c'}$ it follows that $\eta_c\circ g = (R\circ
    L)(c)\circ \eta_{c'}$, fulfilling the naturality condition in:

    % https://q.uiver.app/?q=WzAsNixbMiwxLCJjIl0sWzAsMSwiTChjKSJdLFsxLDEsIihSXFxjaXJjIEwpKGMpIl0sWzIsMCwiYyciXSxbMCwwLCJMKGMnKSJdLFsxLDAsIihSXFxjaXJjIEwpKGMnKSJdLFswLDIsIlxcZXRhX2MiXSxbMyw1LCJcXGV0YV97Yyd9IiwyXSxbMywwLCJnIl0sWzUsMiwiKFJcXGNpcmMgTCkoZykiLDIseyJzdHlsZSI6eyJib2R5Ijp7Im5hbWUiOiJkYXNoZWQifX19XSxbNCwxLCJMKGcpIiwyLHsic3R5bGUiOnsiYm9keSI6eyJuYW1lIjoiZGFzaGVkIn19fV0sWzMsMiwiXFxldGFfY1xcY2lyYyBnIiwxXV0=
    \[\begin{tikzcd}[ampersand replacement=\&]
      {L(c')} \& {(R\circ L)(c')} \& {c'} \\
      {L(c)} \& {(R\circ L)(c)} \& c
      \arrow["{\eta_c}", from=2-3, to=2-2]
      \arrow["{\eta_{c'}}"', from=1-3, to=1-2]
      \arrow["g", from=1-3, to=2-3]
      \arrow["{(R\circ L)(g)}"', dashed, from=1-2, to=2-2]
      \arrow["{L(g)}"', dashed, from=1-1, to=2-1]
      \arrow["{\eta_c\circ g}"{description}, from=1-3, to=2-2]
    \end{tikzcd}\]
  \end{proof}
\end{theorem}

\begin{theorem}[Adjunction Definition Equivalence]
  Definitions \ref{def:adjunction_isomorphism} and \ref{def:adjunction_initial}
  are equivalent.

  \begin{proof}
    The proof consists on two parts.
    \begin{description}
      \item[$(\implies)$] By Theorem \ref{thm:unit_exists} $\eta:\id_\C\to
        (R\circ L)$ exists. As $\overline{v}$ is unique for every $v$, then for any
        $c\in \C_0$ the initial morphism from $c$ to $R$ can be constructed in
        the following manner:
        % https://q.uiver.app/?q=WzAsNixbMSwwLCJMKGMpIl0sWzAsMCwiTChjKSJdLFszLDAsImMiXSxbMiwwLCIoUlxcY2lyYyBMKShjKSJdLFsyLDEsIlIoZCkiXSxbMCwxLCJkIl0sWzAsMSwiXFxpZF97TChjKX0iLDJdLFsyLDMsIlxcZXRhX2MiLDJdLFsyLDQsInYiXSxbMSw1LCJcXGJhcnt2fSIsMix7InN0eWxlIjp7ImJvZHkiOnsibmFtZSI6ImRhc2hlZCJ9fX1dLFszLDQsIlIoXFxiYXJ7dn0pIiwyLHsic3R5bGUiOnsiYm9keSI6eyJuYW1lIjoiZGFzaGVkIn19fV1d&macro_url=https%3A%2F%2Fraw.githubusercontent.com%2Faortega0703%2Fnotes-category-theory%2Fmain%2Fsrc%2Fmacros.tex
        \[\begin{tikzcd}[ampersand replacement=\&]
          {L(c)} \& {L(c)} \& {(R\circ L)(c)} \& c \\
          d \&\& {R(d)}
          \arrow["{\id_{L(c)}}"', from=1-2, to=1-1]
          \arrow["{\eta_c}"', from=1-4, to=1-3]
          \arrow["v", from=1-4, to=2-3]
          \arrow["{\overline{v}}"', dashed, from=1-1, to=2-1]
          \arrow["{R(\overline{v})}"', dashed, from=1-3, to=2-3]
        \end{tikzcd}\]
      \item[$(\impliedby)$] Consider the function:
        \[
          \begin{aligned}
            \phi:&&\mathcal{D}(L(c), d) &\to \mathcal{C}(c, R(d))\\
            && u &\mapsto R(u) \circ \eta_c
          \end{aligned}
        \]
        By Definition \ref{def:adjunction_initial} every $v:c\to R(d)$ can be
        expressed as $\phi(\overline{v})$ for a unique $\overline{v}:L(c)\to d$, which
        makes $\psi$ a bijection and therefore an isomorphism. To prove
        naturality of $\phi$ consider $g:c'\to c$ and $h:d\to d'$:
        % https://q.uiver.app/?q=WzAsNCxbMCwwLCJcXG1hdGhjYWx7RH0oTChjKSwgZCkiXSxbMSwwLCJcXG1hdGhjYWx7Q30oYywgUihkKSkiXSxbMCwxLCJcXG1hdGhjYWx7RH0oTChjJyksIGQnKSJdLFsxLDEsIlxcbWF0aGNhbHtDfShjJywgUihkJykpIl0sWzAsMSwiXFxwaGlfe2MsZH0iXSxbMCwyLCJcXG1hdGhjYWx7RH0oTChnKSxoKSIsMl0sWzIsMywiXFxwaGlfe2MnLGQnfSIsMl0sWzEsMywiXFxtYXRoY2Fse0N9KGcsUihoKSkiXSxbMCwxLCJcXGNvbmciLDJdLFsyLDMsIlxcY29uZyJdXQ==&macro_url=https%3A%2F%2Fraw.githubusercontent.com%2Faortega0703%2Fnotes-category-theory%2Fmain%2Fsrc%2Fmacros.tex
        \[\begin{tikzcd}[ampersand replacement=\&]
          {\mathcal{D}(L(c), d)} \& {\mathcal{C}(c, R(d))} \\
          {\mathcal{D}(L(c'), d')} \& {\mathcal{C}(c', R(d'))}
          \arrow["{\phi_{c,d}}", from=1-1, to=1-2]
          \arrow["{\mathcal{D}(L(g),h)}"', from=1-1, to=2-1]
          \arrow["{\phi_{c',d'}}"', from=2-1, to=2-2]
          \arrow["{\mathcal{C}(g,R(h))}", from=1-2, to=2-2]
          \arrow["\cong"', from=1-1, to=1-2]
          \arrow["\cong", from=2-1, to=2-2]
        \end{tikzcd}\]
        \[
          \begin{aligned}
            \big(\C(g, R(h))\circ \phi_{c,d}\big)(u)
            &= \C(g, R(h))(R(u)\circ\eta_c)\\
            &= R(h)\circ R(u)\circ\eta_c \circ g\\
            &= R(h)\circ R(u) \circ (R\circ L)(g)\circ \eta_{c'}\\
            &= R(h\circ u\circ L(g))\circ \eta_{c'}\\
            &= \phi_{c', d'}\big(h\circ u \circ L(g)\big)\\
            &= \phi_{c', d'}\big(\D(L(g), h)(u)\big)\\
            &= \big(\phi_{c', d'}\circ \D(L(g), h))(u)
          \end{aligned}
        \]
    \end{description}
  \end{proof}
\end{theorem}