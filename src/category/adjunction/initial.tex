\section{Via Initial Morphism}

\begin{definition}[Adjunction]\index{Adjunction!via Initial
  Morphism}\label{def:adjunction_initial}

  The pair of functors $\<L: \C\to \D$, $R: \D\to \C\>$ is left/right adjoint of
  each other when there exists an initial morphism $(L(C), \eta_C:C\to (R\circ
  L)(C))$ from every $C\in \C_0$ to
  $R$~\parencite[p.~208]{awodey:category_theory}:
  % https://q.uiver.app/?q=WzAsNSxbMCwwLCJMKGMpIl0sWzAsMSwiZCJdLFsxLDAsIihSXFxjaXJjIEwpKGMpIl0sWzEsMSwiUihkKSJdLFsyLDAsImMiXSxbMCwxLCJcXG92ZXJsaW5le3Z9IiwyLHsic3R5bGUiOnsiYm9keSI6eyJuYW1lIjoiZGFzaGVkIn19fV0sWzIsMywiUihcXG92ZXJsaW5le3Z9KSIsMix7InN0eWxlIjp7ImJvZHkiOnsibmFtZSI6ImRhc2hlZCJ9fX1dLFs0LDMsInYiXSxbNCwyLCJcXGV0YV9jIiwyXV0=
  \[\begin{tikzcd}[ampersand replacement=\&]
    {L(c)} \& {(R\circ L)(c)} \& c \\
    d \& {R(d)}
    \arrow["{\overline{v}}"', dashed, from=1-1, to=2-1]
    \arrow["{R(\overline{v})}"', dashed, from=1-2, to=2-2]
    \arrow["v", from=1-3, to=2-2]
    \arrow["{\eta_c}"', from=1-3, to=1-2]
  \end{tikzcd}\]
\end{definition}

\begin{theorem}[Naturality of $\eta$]
  In Definition \ref{def:adjunction_initial}, $\eta:\id_\C \to R\circ L$ is a
  natural transformation.

  \begin{proof}
    Take two objects $C,C'\in \C_0$ with a morphism $g:C'\to C$. Then by the
    definition of $\eta_{C'}$ it follows that $\eta_C\circ g = (R\circ
    L)(C)\circ \eta_{C'}$, fulfilling the naturality condition in:

    % https://q.uiver.app/#q=WzAsNixbMiwxLCJDIl0sWzAsMSwiTChDKSJdLFsxLDEsIihSXFxjaXJjIEwpKEMpIl0sWzIsMCwiQyciXSxbMCwwLCJMKEMnKSJdLFsxLDAsIihSXFxjaXJjIEwpKEMnKSJdLFswLDIsIlxcZXRhX0MiXSxbMyw1LCJcXGV0YV97Qyd9IiwyXSxbMywwLCJnIl0sWzUsMiwiKFJcXGNpcmMgTCkoZykiLDIseyJzdHlsZSI6eyJib2R5Ijp7Im5hbWUiOiJkYXNoZWQifX19XSxbNCwxLCJMKGcpIiwyLHsic3R5bGUiOnsiYm9keSI6eyJuYW1lIjoiZGFzaGVkIn19fV0sWzMsMiwiXFxldGFfQ1xcY2lyYyBnIiwxXV0=
    \[\begin{tikzcd}[ampersand replacement=\&]
      {L(C')} \& {(R\circ L)(C')} \& {C'} \\
      {L(C)} \& {(R\circ L)(C)} \& C
      \arrow["{\eta_C}", from=2-3, to=2-2]
      \arrow["{\eta_{C'}}"', from=1-3, to=1-2]
      \arrow["g", from=1-3, to=2-3]
      \arrow["{(R\circ L)(g)}"', dashed, from=1-2, to=2-2]
      \arrow["{L(g)}"', dashed, from=1-1, to=2-1]
      \arrow["{\eta_C\circ g}"{description}, from=1-3, to=2-2]
    \end{tikzcd}\]
  \end{proof}
  \vspace{-\baselineskip}
\end{theorem}

\begin{theorem}[Adjunction Definition Equivalence]
  Definitions \ref{def:adjunction_isomorphism} and \ref{def:adjunction_initial}
  are equivalent.

  \begin{proof}
    The proof consists on two parts.
    \begin{description}
      \item[$(\implies)$] By Theorem \ref{thm:unit_exists} $\eta:\id_\C\to
        (R\circ L)$ exists. As $\overline{v}$ is unique for every $v$, then for
        any $C\in \C_0$ the initial morphism from $C$ to $R$ can be constructed
        in the following manner:
        % https://q.uiver.app/#q=WzAsNixbMSwwLCJMKEMpIl0sWzAsMCwiTChDKSJdLFszLDAsIkMiXSxbMiwwLCIoUlxcY2lyYyBMKShDKSJdLFsyLDEsIlIoRCkiXSxbMCwxLCJEIl0sWzAsMSwiXFxpZF97TChDKX0iLDJdLFsyLDMsIlxcZXRhX2MiLDJdLFsyLDQsInYiXSxbMSw1LCJcXGJhcnt2fSIsMix7InN0eWxlIjp7ImJvZHkiOnsibmFtZSI6ImRhc2hlZCJ9fX1dLFszLDQsIlIoXFxiYXJ7dn0pIiwyLHsic3R5bGUiOnsiYm9keSI6eyJuYW1lIjoiZGFzaGVkIn19fV1d&macro_url=https%3A%2F%2Fraw.githubusercontent.com%2Faortega0703%2Fnotes-category-theory%2Fmain%2Fsrc%2Fmacros.tex
        \[\begin{tikzcd}[ampersand replacement=\&]
          {L(C)} \& {L(C)} \& {(R\circ L)(C)} \& C \\
          D \&\& {R(D)}
          \arrow["{\id_{L(C)}}"', from=1-2, to=1-1]
          \arrow["{\eta_c}"', from=1-4, to=1-3]
          \arrow["v", from=1-4, to=2-3]
          \arrow["{\bar{v}}"', dashed, from=1-1, to=2-1]
          \arrow["{R(\bar{v})}"', dashed, from=1-3, to=2-3]
        \end{tikzcd}\]
      \item[$(\impliedby)$] Consider the function:
        \[
          \begin{aligned}
            \phi:&&\D(L(C), D) &\to \C(C, R(D))\\
            && u &\mapsto R(u) \circ \eta_C
          \end{aligned}
        \]
        By Definition \ref{def:adjunction_initial} every $v:C\to R(D)$ can be
        expressed as $\phi(\overline{v})$ for a unique $\overline{v}:L(C)\to D$,
        which makes $\psi$ a bijection and therefore an isomorphism. To prove
        naturality of $\phi$ consider $g:C'\to C$ and $h:D\to D'$:
        % https://q.uiver.app/#q=WzAsNCxbMCwwLCJcXG1hdGhjYWx7RH0oTChDKSwgRCkiXSxbMSwwLCJcXG1hdGhjYWx7Q30oQywgUihEKSkiXSxbMCwxLCJcXG1hdGhjYWx7RH0oTChDJyksIEQnKSJdLFsxLDEsIlxcbWF0aGNhbHtDfShDJywgUihEJykpIl0sWzAsMSwiXFxwaGlfe1xcbGFuZ2xlIEMsRFxccmFuZ2xlIH0iXSxbMCwyLCJcXG1hdGhjYWx7RH0oTChnKSxoKSIsMl0sWzIsMywiXFxwaGlfe1xcbGFuZ2xlIEMnLEQnXFxyYW5nbGV9IiwyXSxbMSwzLCJcXG1hdGhjYWx7Q30oZyxSKGgpKSJdLFsyLDMsIlxcY29uZyJdLFswLDEsIlxcY29uZyIsMl1d&macro_url=https%3A%2F%2Fraw.githubusercontent.com%2Faortega0703%2Fnotes-category-theory%2Fmain%2Fsrc%2Fmacros.tex
        \[\begin{tikzcd}[ampersand replacement=\&]
          {\mathcal{D}(L(C), D)} \& {\mathcal{C}(C, R(D))} \\
          {\mathcal{D}(L(C'), D')} \& {\mathcal{C}(C', R(D'))}
          \arrow["{\phi_{\langle C,D\rangle }}", from=1-1, to=1-2]
          \arrow["{\mathcal{D}(L(g),h)}"', from=1-1, to=2-1]
          \arrow["{\phi_{\langle C',D'\rangle}}"', from=2-1, to=2-2]
          \arrow["{\mathcal{C}(g,R(h))}", from=1-2, to=2-2]
          \arrow["\cong", from=2-1, to=2-2]
          \arrow["\cong"', from=1-1, to=1-2]
        \end{tikzcd}\]
        \[
          \begin{aligned}
            \big(\C(g, R(h))\circ \phi_{\<C,D\>}\big)(u)
            &= \C(g, R(h))(R(u)\circ\eta_C)\\
            &= R(h)\circ R(u)\circ\eta_C \circ g\\
            &= R(h)\circ R(u) \circ (R\circ L)(g)\circ \eta_{C'}\\
            &= R(h\circ u\circ L(g))\circ \eta_{C'}\\
            &= \phi_{\<C', D'\>}\big(h\circ u \circ L(g)\big)\\
            &= \phi_{\<C', D'\>}\big(\D(L(g), h)(u)\big)\\
            &= \big(\phi_{\<C', D'\>}\circ \D(L(g), h))(u)
          \end{aligned}
        \]
    \end{description}
  \end{proof}
\end{theorem}