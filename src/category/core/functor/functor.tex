\section{Functors}
A functor can be thought of as an structure preserving transformation between
categories, or a way to identify one category inside another.

\begin{definition}[Functor\index{Functor}]
  For categories $\C,\D$, a functor $F: \C \to \D$ consist of maps of objects
  and morphisms, such that it
  preserves~\parencite[p.~8]{awodey:category_theory}:
  \begin{itemize}
    \item Domain and Codomain:
      \[\big(\forall (f:c\to c')\in \C_1\big)
        \big(F(f): F(c) \to F(c')\big)\]
    \item Identity:
      \[(\forall c \in \C_0)
        (F(\id_c) = \id_{F(c)})\]
    \item Composition:
      \[\big(\forall (f: a \to b)\in \C_1,\ (g: b\to c) \in \C_1\big)
        \big(F(g \circ f) = F(g) \circ F(f)\big)\]
  \end{itemize}
\end{definition}

\begin{example}
  % https://q.uiver.app/?q=WzAsNyxbMCwwLCIxIl0sWzMsMCwiRigxKSJdLFs0LDAsIkYoMikiXSxbNCwxLCJcXGJ1bGxldCJdLFsxLDAsIjIiXSxbMiwwLCIzIl0sWzUsMCwiRigzKSJdLFswLDQsImYiXSxbMSwyLCJGKGYpIl0sWzIsMywiaCIsMl0sWzEsMywiaCBcXGNpcmMgRihmKSIsMix7ImxhYmVsX3Bvc2l0aW9uIjoyMH1dLFs0LDUsImciXSxbMCw1LCJnXFxjaXJjIGYiLDAseyJjdXJ2ZSI6LTN9XSxbMiw2LCJGKGcpIl0sWzMsNiwiaSIsMl0sWzEsNiwiRihnXFxjaXJjIGYpIiwwLHsiY3VydmUiOi00fV1d
  \[\begin{tikzcd}[ampersand replacement=\&]
    1 \& 2 \& 3 \& {F(1)} \& {F(2)} \& {F(3)} \\
    \&\&\&\& \bullet
    \arrow["f", from=1-1, to=1-2]
    \arrow["{F(f)}", from=1-4, to=1-5]
    \arrow["h"', from=1-5, to=2-5]
    \arrow["{h \circ F(f)}"'{pos=0.2}, from=1-4, to=2-5]
    \arrow["g", from=1-2, to=1-3]
    \arrow["{g\circ f}", curve={height=-18pt}, from=1-1, to=1-3]
    \arrow["{F(g)}", from=1-5, to=1-6]
    \arrow["i"', from=2-5, to=1-6]
    \arrow["{F(g\circ f)}", curve={height=-24pt}, from=1-4, to=1-6]
  \end{tikzcd}\]
\end{example}

\begin{theorem}[Functors Preserve Isomorphisms]\label{thm:isomorphism_functor}

  Given two isomorphic objects, their images under any functor are also
  isomorphic.

  \begin{proof}
    Consider two objects $c,d \in \C_0$ with an isomorphism $f: c\to d$, and any
    functor $F:\C \to \D$ then:
    \[
      \begin{aligned}
        f^{-1}\circ f &= \id_c\\
        F(f^{-1}\circ f) &= F(\id_c)\\
        F(f^{-1})\circ F(f) &= \id_{F(c)}
      \end{aligned}
      \qquad
      \begin{aligned}
        f \circ f^{-1} &= \id_d\\
        F(f \circ f^{-1}) &= F(\id_d)\\
        F(f) \circ F(f^{-1}) &= \id_{F(d)}
      \end{aligned}
    \]
  \end{proof}
\end{theorem}

\subsection{Special Functors}

\subsubsection{Constant Functor}
\begin{definition}[Constant Functor]\index{Constant Functor}
  For categories $\C,\D$ and an object $D\in \D$, the constant functor
  $\Delta_D: \C\to \D$ consists of maps
  of~\parencite[p.~142]{leinster:basic_category_theory}:
  \begin{itemize}
    \item Objects:
      \[\big(\forall C\in \C_0\big)
        \big(\Delta_D(C) = D\big)\]
    \item Morphisms:
      \[\big(\forall f\in \C_1\big)
        \big(\Delta_D(F) = \id_D\big)\]
  \end{itemize}
\end{definition}

\subsubsection{Diagonal Functor}
\begin{definition}[Diagonal Functor]\index{Diagonal
  Functor}\label{def:diagonal_functor_binary}
  For a category $\C$, the diagonal
  functor $\Delta: \C \to \C\times \C$ is defined as maps of:
  \begin{itemize}
    \item Objects:
      \[\big(\forall c\in \C_0\big)
        \big(\Delta(C) = \<C,C\>\big)\]
    \item Morphisms:
      \[\big(\forall f\in \C_1\big)
        \big(\Delta(f) = \<f,f\>\big)\]
  \end{itemize}
\end{definition}

\begin{definition}[Diagonal Functor]\index{Diagonal
  Functor}\label{def:diagonal_functor}
  More generally, for categories $\C, \D$,
  the diagonal functor $\Delta: \C \to \D^\C$ consists of maps
  of~\parencite[p.~142]{leinster:basic_category_theory}:
  \begin{itemize}
    \item Objects:
      \[\big(\forall C\in \C_0\big)
        \big(\Delta(C) = \Delta_C : \C \to \D\big)\]
      Where $\Delta_C$ is the constant functor.
    \item Morphisms:
      \[\big(\forall C\in C_0,\ f\in \C_1\big)
        \big((\Delta(f))_C= f \big)\]
      The naturality condition of $\Delta(f)$ is given by:
      % https://q.uiver.app/#q=WzAsNCxbMCwwLCJcXERlbHRhX0MoQykiXSxbMSwwLCJcXERlbHRhX0MoRCkiXSxbMCwxLCJcXERlbHRhX3tEfShDKSJdLFsxLDEsIlxcRGVsdGFfe0R9KEQpIl0sWzAsMSwiXFxEZWx0YV9DKGYpIl0sWzAsMiwiXFxEZWx0YShmKV9DIiwyXSxbMiwzLCJcXERlbHRhX3tEfShmKSIsMl0sWzEsMywiXFxEZWx0YShmKV97RH0iXV0=
      \[\begin{tikzcd}[ampersand replacement=\&]
        {\Delta_C(C)} \& {\Delta_C(D)} \\
        {\Delta_{D}(C)} \& {\Delta_{D}(D)}
        \arrow["{\Delta_C(f)}", from=1-1, to=1-2]
        \arrow["{\Delta(f)_C}"', from=1-1, to=2-1]
        \arrow["{\Delta_{D}(f)}"', from=2-1, to=2-2]
        \arrow["{\Delta(f)_{D}}", from=1-2, to=2-2]
      \end{tikzcd}
      =
      % https://q.uiver.app/#q=WzAsNCxbMCwwLCJDIl0sWzEsMCwiQyJdLFswLDEsIkQiXSxbMSwxLCJEIl0sWzAsMSwiXFxpZF9DIl0sWzAsMiwiZiIsMl0sWzIsMywiXFxpZF9EIiwyXSxbMSwzLCJmIl1d&macro_url=https%3A%2F%2Fraw.githubusercontent.com%2Faortega0703%2Fnotes-category-theory%2Fmain%2Fsrc%2Fmacros.tex
      \begin{tikzcd}[ampersand replacement=\&]
        C \& C \\
        D \& D
        \arrow["{\id_C}", from=1-1, to=1-2]
        \arrow["f"', from=1-1, to=2-1]
        \arrow["{\id_D}"', from=2-1, to=2-2]
        \arrow["f", from=1-2, to=2-2]
      \end{tikzcd}\]
  \end{itemize}
\end{definition}

\subsection{Classification of Functors}

\begin{definition}[Faithful Functor\index{Faithful Functor}]
  A functor $F:\C\to \D$ is faithful when for every morphism $(f: c\to c') \in
  \C_1$, the following function is
  injective~\parencite[p.~25]{leinster:basic_category_theory}:
    \[
      \begin{aligned}
        F: \C(c, c') &\to \D(F(c), F(c'))\\
        f &\mapsto F(f)
      \end{aligned}
    \]
\end{definition}

\begin{definition}[Full Functor\index{Full Functor}]
  A functor $F:\C\to \D$ is full when for every morphism $(f: c\to c') \in
  \C_1$, the following function is
  surjective~\parencite[p.~25]{leinster:basic_category_theory}:
  \[
    \begin{aligned}
      F: \C(c, c') &\to \D(F(c), F(c'))\\
      f &\mapsto F(f)
    \end{aligned}
  \]
\end{definition}

\begin{definition}[Endofunctor\index{Endofunctor}]
  A functor whose domain equals its codomain is an
  endofunctor~\parencite[p.~30]{adamek_herrlich_strecker:joy_cats}.
\end{definition}

\begin{definition}[Bifunctor\index{Bifunctor}]
  A functor $F:\A\times \B \to \C$ (one with $2$ parameters) is said to be a
  bifunctor~\parencite[p.~37]{lane:working_mathematician}.
\end{definition}

\begin{definition}[Presheaf\index{Presheaf}]
  A functor $\C^\op \to \Set$ is a presheaf on
  $\C$~\parencite[p.~24]{leinster:basic_category_theory}.
\end{definition}

\begin{theorem}[Fully Faithful Isomorphism]\label{thm:full_faithful_isomorphism}

  For any full and faithful functor $F:\C\to \D$ it follows that $c \cong c'
  \iff F(c) \cong F(c')$ for objects $c, c'\in \C_0$.

  \begin{proof}
    The $(\implies)$ holds via Theorem \ref{thm:isomorphism_functor}. For the
    $(\impliedby)$ part, consider the isomorphism $Ff: F(c) \overset{\cong}{\to}
    F(c')$. As $F$ is fully faithful, morphisms $c\to c'$ are in bijection
    with morphisms $F(c)\to F(c')$, therefore there exists unique arrows $f:c\to c'$ $g:c'\to c$ such that $F(f) = Ff$ and $F(g) = Ff^{-1}$, then:
    \[
      \begin{aligned}
        &F(\id_c)\\
        =& \id_{F(c)}\\
        =& Ff^{-1} \circ Ff\\
        =& F(g) \circ F(f)\\
        =& F(g\circ f)
      \end{aligned}
      \qquad
      \begin{aligned}
        &F(\id_{c'})\\
        =& \id_{F(c')}\\
        =& Ff \circ Ff^{-1}\\
        =& F(f) \circ F(g)\\
        =& F(f\circ g)
      \end{aligned}
    \]

    As $F$ is faithful and $F(\id_c) = F(g\circ f)$, then $g\circ f = \id_c$.
    Similarly, $f\circ g = \id_{c'}$.
  \end{proof}
\end{theorem}

\subsection{Identity}

\begin{definition}[Identity Natural Transformation\index{Identity Natural Transformation}]
  For a functor $F:\C\to \D$, there exists a natural transformation $\id_F:
  F\Rightarrow F$ which consists
  of~\parencite[p.~30]{leinster:basic_category_theory}:

  \begin{itemize}
    \item $c$-Components:
      \[(\forall c\in \C_0)
        \big((\id_F)_c = \id_{F(c)}\big)\]
    \item Naturality Condition:
      % https://q.uiver.app/?q=WzAsNCxbMCwwLCJGKGMpIl0sWzEsMCwiRihjJykiXSxbMCwxLCJGKGMpIl0sWzEsMSwiRihjJykiXSxbMCwxLCJGKGYpIl0sWzAsMiwiXFxtYXRocm17aWR9X3tGKGMpfSIsMl0sWzIsMywiRihmKSIsMl0sWzEsMywiXFxtYXRocm17aWR9X3tGKGMnKX0iXV0=
      \[\begin{tikzcd}
        {F(c)} & {F(c')} \\
        {F(c)} & {F(c')}
        \arrow["{F(f)}", from=1-1, to=1-2]
        \arrow["{\mathrm{id}_{F(c)}}"', from=1-1, to=2-1]
        \arrow["{F(f)}"', from=2-1, to=2-2]
        \arrow["{\mathrm{id}_{F(c')}}", from=1-2, to=2-2]
      \end{tikzcd}\]
  \end{itemize}
\end{definition}

\begin{theorem}[Naturality of Identity Natural Transformation]
  The naturality condition of the identity natural transformation holds.

  \begin{proof}
    By unitality it follows that $F(f) \circ \id_{F(c)} = \id_{F(c')} \circ
    F(f)$.
  \end{proof}
\end{theorem}

\subsection{Composition}

\begin{definition}[Functor Composition\index{Functor
  Composition}]\label{def:functor_composition}

  For functors $F:\A\to \B$ and $G:\B\to \C$, the composite functor $G\circ F$
  consists of maps of~\parencite[p.~17]{leinster:basic_category_theory}:
  \begin{itemize}
    \item Objects:
      \[(\forall a \in \A_0)
        \big((G\circ F)(a) = G(F(a))\big)\]
    \item Morphisms:
      \[(\forall f\in \A_1)
        \big((G\circ F)(f) = G(F(f))\big)\]
  \end{itemize}
\end{definition}

\begin{theorem}[Functor Composition Preserves Identity\label{thm:functor_cmp_id}]
  For functors $F:\A\to \B$ and $G:\B\to \C$, the composite functor $G\circ F$
  preserves indentity.

  \begin{proof}
    For all objects $a\in \A_0$:
    \[
      \begin{aligned}
        (G\circ F)(\id_a)
        &= G(F(\id_a))\\
        &= G(\id_{F(a)})\\
        &= \id_{(G\circ F)(a)}
      \end{aligned}
    \]
  \end{proof}
\end{theorem}

\begin{theorem}[Functor Composition Preserves Composition\label{thm:functor_cmp_cmp}]
  For functors $F:\A\to \B$ and $G:\B\to \C$, the composite functor $G\circ F$
  preserves composition.

  \begin{proof}
    For all morphisms $f,g\in \A_1$ with $g$ composable after $f$:
    \[
      \begin{aligned}
        (G\circ F)(g\circ F)
        &= G(F(g\circ f))\\
        &= G(F(g)\circ F(f))\\
        &= G(F(g))\circ G(F(f))\\
        &= (G\circ F)(g)\circ(G\circ F)(f)
      \end{aligned}
    \]
  \end{proof}
\end{theorem}

\begin{theorem}[Functor Composition is a Functor]
  For functors $F, G$, the composition $G\circ F$ is a functor.

  \begin{proof}
    By Theorems \ref{thm:functor_cmp_id} and \ref{thm:functor_cmp_cmp} functor
    composition fulfills the axioms of a functor.
  \end{proof}
\end{theorem}

\begin{theorem}[Unitality of Functor Composition\label{thm:unitality_functor}]
  For a functor $F:\C\to \D$ the following holds:
  \[F \circ \id_C = \id_D \circ F = F\]

  \begin{proof}
    Let either $c\in \C_0$ or $c\in \C_1$, then:
    \[
      \begin{aligned}
        (F \circ \id_C)(c)
          &= F(\id_C(c))\\
          &= F (c)
      \end{aligned}
      \quad
      \begin{aligned}
        (\id_D \circ F)(c)
          &= \id_D (F (c))\\
          &= F (c)
      \end{aligned}
    \]
  \end{proof}
\end{theorem}

\begin{theorem}[Associativity of Functor Composition\label{thm:assoc_functor}]
  For a set-up of categories and functors:
  % https://q.uiver.app/?q=WzAsNCxbMCwwLCJcXEEiXSxbMSwwLCJcXEIiXSxbMiwwLCJcXEMiXSxbMywwLCJcXEQiXSxbMCwxLCJGIl0sWzEsMiwiRyJdLFsyLDMsIkgiXV0=&macro_url=https%3A%2F%2Fraw.githubusercontent.com%2Faortega0703%2Fnotes-category-theory%2Fmain%2Fsrc%2Fmacros.tex
  \[\begin{tikzcd}[ampersand replacement=\&]
    \A \& \B \& \C \& \D
    \arrow["F", from=1-1, to=1-2]
    \arrow["G", from=1-2, to=1-3]
    \arrow["H", from=1-3, to=1-4]
  \end{tikzcd}\]

  The following holds:
  \[(H \circ G) \circ F = H\circ(G\circ F)\]

  \begin{proof}
    Let either $c\in \C_0$ or $c\in \C_1$, then:
    \[
      \begin{aligned}
        ((H \circ G) \circ F) (c)
          &= (H\circ G) (F (c))\\
          &= H(G(F(c)))\\
          &= H((G\circ F)(c))\\
          &= (H\circ(G\circ F)) (c)
      \end{aligned}
    \]
  \end{proof}
\end{theorem}