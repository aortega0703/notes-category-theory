\subsection{Classification of Functors}

\begin{definition}[Faithful Functor]\index{Faithful Functor}
  A functor $F:\C\to \D$ is faithful when for every morphism $(f: C\to D) \in
  \C_1$, the following function is
  injective~\parencite[p.~25]{leinster:basic_category_theory}:
    \[
      \begin{aligned}
        F: \C(C, D) &\to \D(F(C), F(D))\\
        f &\mapsto F(f)
      \end{aligned}
    \]
\end{definition}

\begin{definition}[Full Functor]\index{Full Functor}
  A functor $F:\C\to \D$ is full when for every morphism $(f: C\to D) \in
  \C_1$, the following function is
  surjective~\parencite[p.~25]{leinster:basic_category_theory}:
  \[
    \begin{aligned}
      F: \C(C, D) &\to \D(F(C), F(D))\\
      f &\mapsto F(f)
    \end{aligned}
  \]
\end{definition}

\begin{definition}[Endofunctor]\index{Endofunctor}
  A functor whose domain equals its codomain is an
  endofunctor~\parencite[p.~30]{adamek_herrlich_strecker:joy_cats}.
\end{definition}

\begin{definition}[Bifunctor]\index{Bifunctor}
  A functor $F:\A\times \B \to \C$ from a product category is said to be a
  bifunctor~\parencite[p.~37]{lane:working_mathematician}.
\end{definition}

\begin{definition}[Presheaf]\index{Presheaf}
  A functor $\C^\op \to \Set$ from an opposite category is a presheaf on
  $\C$~\parencite[p.~24]{leinster:basic_category_theory}.
\end{definition}

\begin{theorem}\label{thm:full_faithful_isomorphism}
  For any full and faithful functor $F:\C\to \D$ it follows that $C \cong D
  \iff F(C) \cong F(D)$ for objects $C, D\in \C_0$.

  \begin{proof}
    The $(\implies)$ holds via Theorem \ref{thm:isomorphism_functor}. For the
    $(\impliedby)$ part, consider the isomorphism $f: F(C) \overset{\cong}{\to}
    F(D)$. As $F$ is full, there exists morphisms $g:C\to D$, $h:D\to C$
    such that $F(g) = f$ and $F(h) = f^{-1}$, then:
    \[
      \begin{aligned}
        F(\id_C)
        &= \id_{F(C)}\\
        &= f^{-1} \circ f\\
        &= F(h) \circ F(g)\\
        &= F(h\circ g)
      \end{aligned}
      \qquad
      \begin{aligned}
        F(\id_{D})
        &= \id_{F(D)}\\
        &= f \circ f^{-1}\\
        &= F(g) \circ F(h)\\
        &= F(g\circ h)
      \end{aligned}
    \]

    As $F$ is faithful and $F(\id_C) = F(h\circ g)$, they must share the same
    preimage $h\circ g = \id_C$. Similarly, $g\circ h = \id_{D}$.
  \end{proof}
\end{theorem}