\subsection{Composition}

\begin{definition}[Functor Composition\index{Functor
  Composition}]\label{def:functor_composition}

  For functors $F:\A\to \B$ and $G:\B\to \C$, the composite functor $G\circ F$
  consists of maps of~\parencite[p.~17]{leinster:basic_category_theory}:
  \begin{itemize}
    \item Objects:
      \[(\forall a \in \A_0)
        \big((G\circ F)(a) = G(F(a))\big)\]
    \item Morphisms:
      \[(\forall f\in \A_1)
        \big((G\circ F)(f) = G(F(f))\big)\]
  \end{itemize}
\end{definition}

\begin{theorem}[Functor Composition Preserves Identity\label{thm:functor_cmp_id}]
  For functors $F:\A\to \B$ and $G:\B\to \C$, the composite functor $G\circ F$
  preserves indentity.

  \begin{proof}
    For all objects $a\in \A_0$:
    \[
      \begin{aligned}
        (G\circ F)(\id_a)
        &= G(F(\id_a))\\
        &= G(\id_{F(a)})\\
        &= \id_{(G\circ F)(a)}
      \end{aligned}
    \]
  \end{proof}
\end{theorem}

\begin{theorem}[Functor Composition Preserves Composition\label{thm:functor_cmp_cmp}]
  For functors $F:\A\to \B$ and $G:\B\to \C$, the composite functor $G\circ F$
  preserves composition.

  \begin{proof}
    For all morphisms $f,g\in \A_1$ with $g$ composable after $f$:
    \[
      \begin{aligned}
        (G\circ F)(g\circ F)
        &= G(F(g\circ f))\\
        &= G(F(g)\circ F(f))\\
        &= G(F(g))\circ G(F(f))\\
        &= (G\circ F)(g)\circ(G\circ F)(f)
      \end{aligned}
    \]
  \end{proof}
\end{theorem}

\begin{theorem}[Functor Composition is a Functor]
  For functors $F, G$, the composition $G\circ F$ is a functor.

  \begin{proof}
    By Theorems \ref{thm:functor_cmp_id} and \ref{thm:functor_cmp_cmp} functor
    composition fulfills the axioms of a functor.
  \end{proof}
\end{theorem}

\begin{theorem}[Unitality of Functor Composition\label{thm:unitality_functor}]
  For a functor $F:\C\to \D$ the following holds:
  \[F \circ \id_C = \id_D \circ F = F\]

  \begin{proof}
    Let either $c\in \C_0$ or $c\in \C_1$, then:
    \[
      \begin{aligned}
        (F \circ \id_C)(c)
          &= F(\id_C(c))\\
          &= F (c)
      \end{aligned}
      \quad
      \begin{aligned}
        (\id_D \circ F)(c)
          &= \id_D (F (c))\\
          &= F (c)
      \end{aligned}
    \]
  \end{proof}
\end{theorem}

\begin{theorem}[Associativity of Functor Composition\label{thm:assoc_functor}]
  For a set-up of categories and functors:
  % https://q.uiver.app/?q=WzAsNCxbMCwwLCJcXEEiXSxbMSwwLCJcXEIiXSxbMiwwLCJcXEMiXSxbMywwLCJcXEQiXSxbMCwxLCJGIl0sWzEsMiwiRyJdLFsyLDMsIkgiXV0=&macro_url=https%3A%2F%2Fraw.githubusercontent.com%2Faortega0703%2Fnotes-category-theory%2Fmain%2Fsrc%2Fmacros.tex
  \[\begin{tikzcd}[ampersand replacement=\&]
    \A \& \B \& \C \& \D
    \arrow["F", from=1-1, to=1-2]
    \arrow["G", from=1-2, to=1-3]
    \arrow["H", from=1-3, to=1-4]
  \end{tikzcd}\]

  The following holds:
  \[(H \circ G) \circ F = H\circ(G\circ F)\]

  \begin{proof}
    Let either $c\in \C_0$ or $c\in \C_1$, then:
    \[
      \begin{aligned}
        ((H \circ G) \circ F) (c)
          &= (H\circ G) (F (c))\\
          &= H(G(F(c)))\\
          &= H((G\circ F)(c))\\
          &= (H\circ(G\circ F)) (c)
      \end{aligned}
    \]
  \end{proof}
\end{theorem}