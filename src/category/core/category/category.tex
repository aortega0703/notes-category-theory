\section{Categories}
A category can be thought of as the abstraction of a certain structure.

\begin{definition}[Category]\index{Category}

  A category $\C$ is composed of~\parencite[p.~4]{awodey:category_theory}:
  \begin{itemize}
    \item Collections of Objects ($\C_0$) and Morphisms ($\C_1$).
    \item Functions Domain ($\mathrm{Dom}:\C_1\to \C_0$) and Codomain
      ($\mathrm{Cod}: \C_1 \to \C_0$).
      \[
        \begin{gathered}
          (f : C\to D)
          \coloneqq (\mathrm{Dom}(f) = C) \land (\mathrm{Cod}(f) = D)\\
          \C(C, D) \coloneqq \{f\ |\ f : C \to D\}\subseteq \C_1
        \end{gathered}
      \]
    \item Identity morphisms:
      \[\big(\forall C \in \C_0\big)
        \big((\id_C : C\to C)\in \C_1\big)\]
    \item Composition of Morphisms:
      \[\big(\forall (f:A \to B),\ (g : B\to C) \in \C_1\big)
        \big((g\circ f:A\to C) \in \C_1\big)\]
  \end{itemize}

  Such that the following are satisfied:
  \begin{itemize}
    \item Unitality:
      \[\big(\forall (f:C\to D)\in \C_1\big)
        \big(f \circ \id_C = \id_D \circ f = f\big)\]
    \item Associativity:
      \[
        \begin{gathered}
          \big(\forall (f : A\to B),\ (g: B\to C),\ (h: C\to D)\in\C_1\big)\\
          \big((h \circ g) \circ f = h \circ (g \circ f)\big)
        \end{gathered}
      \]
  \end{itemize}
\end{definition}

\begin{remark}
  For the sake of simplicity, identity and composite arrows are usually not
  drawn.
\end{remark}

\begin{example}
  % https://q.uiver.app/#q=WzAsNCxbMCwwLCJBIl0sWzIsMSwiQiJdLFsyLDIsIkMiXSxbNCwwLCJEIl0sWzAsMSwiZiIsMV0sWzEsMiwiZyIsMV0sWzIsMywiaCIsMl0sWzAsMiwiZ1xcY2lyYyBmIiwyXSxbMSwzLCJoXFxjaXJjIGciLDFdLFswLDMsImhcXGNpcmMgZ1xcY2lyYyBmIl1d
  \[\begin{tikzcd}[ampersand replacement=\&,column sep=scriptsize]
    A \&\&\&\& D \\
    \&\& B \\
    \&\& C
    \arrow["f"{description}, from=1-1, to=2-3]
    \arrow["g"{description}, from=2-3, to=3-3]
    \arrow["h"', from=3-3, to=1-5]
    \arrow["{g\circ f}"', from=1-1, to=3-3]
    \arrow["{h\circ g}"{description}, from=2-3, to=1-5]
    \arrow["{h\circ g\circ f}", from=1-1, to=1-5]
  \end{tikzcd}\]
  \vspace{-1.5\baselineskip}
\end{example}

\begin{definition}[Opposite category]\index{Opposite category}

  For a category $\C$, its opposite category $\C^\op$ is composed
  of~\parencite[p.~15]{awodey:category_theory}:

  \begin{itemize}
    \item Objects:
      \[(\forall C \in \C_0)
        (C\in \C^\op_0)\]
    \item Morphisms:
      \[\big(\forall (f: C\to D) \in \C_1\big)
        \big((f^\op : D\to C)\in \C^\op_1\big)\]
  \end{itemize}
\end{definition}

\begin{remark}
  As the axioms of category theory are self-dual, any statement $\Sigma$ that
  holds for all categories implies that its dual statement ($\Sigma$ in
  $\C^\op$) must also hold for all
  categories~\parencite[p.~16]{leinster:basic_category_theory}.
\end{remark}

\subsection{Classification of Categories}

\begin{definition}[Small/Large Category]\index{Small Category}\index{Large Category}

  A category $\C$ is small when $\C_0$ and $\C_1$ are sets instead of proper
  classes. It is large otherwise~\parencite[p.~24]{awodey:category_theory}.
\end{definition}

\begin{definition}[Locally Small Category]\index{Locally Small Category}

  A category $\C$ is locally small when $\C(C, D)$ is a set for all $C,D\in
  \C_0$~\parencite[p.~25]{awodey:category_theory}.
\end{definition}

\begin{definition}[Discrete Category]\index{Discrete Category}

  A category $\C$ is discrete when $\C_1$ is composed entirely out of identity
  morphisms~\parencite[p.~11]{awodey:category_theory}. Discrete categories are
  going to be denoted by $\underline{n}$, with $n=|\C_0|$.
\end{definition}

\subsection{Classification Morphisms}

\begin{definition}[Endomorphism]\index{Endomorphism}
  A morphism whose domain equals its codomain is an
  endomorphism~\parencite[p.~7]{riehl:category_theory_in_context}.
\end{definition}

\begin{definition}[Idempotent]\index{Idempotent}
  An endomorphism $f:C\to C$ is idempotent if $f\circ f =
  f$~\parencite[p.~20]{lane:working_mathematician}.
\end{definition}

\begin{definition}[Split]\index{Split}
  An endomorphism $f:C\to C$ is split if there exists arrows $g:C\to D$ and
  $h:D\to C$ such that $f=h\circ g$ and $g\circ h=
  \id_{C}$~\parencite[p.~20]{lane:working_mathematician}.
\end{definition}

\begin{theorem}[Splits are Idempotent]
  Any split endomorphism is idempotent.

  \begin{proof}
    Consider an split endomorphism $f=h\circ g:C\to C$, then:
    \[
      \begin{aligned}
        f\circ f &= h\circ g \circ h\circ g\\
        &= h\circ g\\
        &= f
      \end{aligned}
    \]
  \end{proof}
  \vspace{-\baselineskip}
\end{theorem}

\begin{definition}[Left Inverse]\index{Left Inverse}\index{Retraction!see Left
  Inverse}
  A morphism $f: C\to D$ has a left inverse or a retraction $g: D\to C$ if
  $g\circ f = \id_C$~\parencite[p.~19]{lane:working_mathematician}.
\end{definition}

\begin{definition}[Right Inverse]\index{Right Inverse}\index{Section!see Right
  Inverse}
  A morphism $f: C\to D$ has a right inverse or a section $g: D\to C$ if
  $f\circ g = \id_{D}$~\parencite[p.~19]{lane:working_mathematician}.
\end{definition}

\begin{definition}[Isomorphism]\index{Isomorphism}
  A morphism is invertible or an isomorphism when it has a left and right
  inverse, and both are equal~\parencite[p.~19]{lane:working_mathematician}.
\end{definition}

\begin{remark}
  Two objects $A, B$ are said to be isomorphic (written $A\cong B$) if there
  exists is an isomorphism between them.
\end{remark}

\begin{theorem}[Uniqueness of Inverse]
  If $f: C\to D$ is an isomorphism, then it has a unique inverse $f^{-1}:D\to
  C$:
  \begin{proof}
    Consider two inverses of $f$, $f^{-1}_1$ and $f^{-1}_2$, then:
    \[
      \begin{aligned}
        \id_C &= \id_C\\
        f\circ f^{-1}_1 &= f\circ f^{-1}_2\\
        f^{-1}_1 \circ f \circ f^{-1}_1 &= f^{-1}_1 \circ f \circ f^{-1}_2\\
        f^{-1}_1 &= f^{-1}_2
      \end{aligned}
    \]
  \end{proof}
\end{theorem}

\begin{definition}[Automorphism]\index{Automorphism}
  An endomorphism which is also an isomorphism is an
  automorphism~\parencite[p.~7]{riehl:category_theory_in_context}.
\end{definition}

\begin{definition}[Monomorphisms]\index{Monomorphisms}
  A morphism $f:C\to D$ is monic or a monomorphism when it is left
  cancellable~\parencite[p.~19]{lane:working_mathematician} i.e. when for any
  set-up of objects and morphisms:
  % https://q.uiver.app/#q=WzAsMyxbMSwwLCJDIl0sWzIsMCwiRCJdLFswLDAsIlgiXSxbMCwxLCJmIl0sWzIsMCwiZyIsMCx7Im9mZnNldCI6LTF9XSxbMiwwLCJoIiwyLHsib2Zmc2V0IjoxfV1d
  \[\begin{tikzcd}[ampersand replacement=\&]
    X \& C \& D
    \arrow["f", from=1-2, to=1-3]
    \arrow["g", shift left=1, from=1-1, to=1-2]
    \arrow["h"', shift right=1, from=1-1, to=1-2]
  \end{tikzcd}\]

  The following holds:
  \[f \circ g = f \circ h \implies g = h\]
\end{definition}

\begin{remark}
  $f:C\mono D$ will be used to denote monomorphisms.
\end{remark}

\begin{remark}
  $f:C\sub D$ will be used instead when $f$ represents inclusions.
\end{remark}

\begin{theorem}[Left Inverse Imply Monomorphisms]\label{thm:left_inverse_implies_mono}

  If $f:C\to D$ has a left inverse $f':D\to C$, then it is a monomorphism.

  \begin{proof}
    Consider the following set-up of objects and morphisms:
    % https://q.uiver.app/#q=WzAsMyxbMSwwLCJDIl0sWzIsMCwiRCJdLFswLDAsIlgiXSxbMCwxLCJmIiwwLHsib2Zmc2V0IjotMX1dLFsyLDAsImciLDAseyJvZmZzZXQiOi0xfV0sWzIsMCwiaCIsMix7Im9mZnNldCI6MX1dLFsxLDAsImYnIiwwLHsib2Zmc2V0IjotMX1dXQ==
    \[\begin{tikzcd}[ampersand replacement=\&]
      X \& C \& D
      \arrow["f", shift left=1, from=1-2, to=1-3]
      \arrow["g", shift left=1, from=1-1, to=1-2]
      \arrow["h"', shift right=1, from=1-1, to=1-2]
      \arrow["{f'}", shift left=1, from=1-3, to=1-2]
    \end{tikzcd}\]
    If $f'$ is left inverse of $f$ and $f\circ g = f\circ h$, then:
    \[
      \begin{aligned}
        f\circ g &= f\circ h\\
        f'\circ f \circ g &= f'\circ f\circ h\\
        g &= h
      \end{aligned}
    \]
  \end{proof}
  \vspace{-\baselineskip}
\end{theorem}

\begin{definition}[Epimorphism]\index{Epimorphism}
  A morphism $f:C\to D$ is epic or an epimorphism when it is right
  cancellable~\parencite[p.~19]{lane:working_mathematician} i.e. when for any
  set-up of objects and morphisms:
  % https://q.uiver.app/#q=WzAsMyxbMCwwLCJDIl0sWzEsMCwiRCJdLFsyLDAsIlgiXSxbMSwyLCJoIiwyLHsib2Zmc2V0IjoxfV0sWzAsMSwiZiJdLFsxLDIsImciLDAseyJvZmZzZXQiOi0xfV1d
  \[\begin{tikzcd}[ampersand replacement=\&]
    C \& D \& X
    \arrow["h"', shift right=1, from=1-2, to=1-3]
    \arrow["f", from=1-1, to=1-2]
    \arrow["g", shift left=1, from=1-2, to=1-3]
  \end{tikzcd}\]

  The following holds:
  \[g \circ f = h \circ f \implies g = h\]
\end{definition}

\begin{remark}
  $f:a\epi b$ will be used to denote epimorphisms.
\end{remark}

\begin{theorem}[Right Inverses Imply Epimorphisms]\label{thm:right_inverse_implies_epi}
  If $f:C\to D$ has a right inverse $f':D\to C$, then it is a epimorphism.

  \begin{proof}
    By duality with Theorem \ref{thm:left_inverse_implies_mono}.
  \end{proof}
\end{theorem}

\subsection{Classification of Objects}

\begin{definition}[Terminal Object]\index{Terminal Object}
  For a category $\C$, an object $1$ is said to be terminal when there exists a
  unique morphism $f: C\to 1$ for every object $C\in
  \C_0$~\parencite[p.~48]{leinster:basic_category_theory}.
\end{definition}

\begin{theorem}[Uniqueness of Terminal Objects]\label{thm:terminal_object_iso}
  Initial objects in a category are unique up to unique isomorphism.

  \begin{proof}
    To proof uniqueness up to isomorphism, consider two terminal objects $C$ and
    $D$, the following diagram must commute as every arrow is unique:
    % https://q.uiver.app/#q=WzAsNCxbMSwwLCJEIl0sWzAsMCwiQyJdLFsyLDEsIkQiXSxbMSwxLCJDIl0sWzAsMiwiXFxpZF9EIl0sWzEsMywiXFxpZF9DIiwyXSxbMSwwLCJmIl0sWzAsMywiZyIsMl0sWzMsMiwiZiIsMl1d&macro_url=https%3A%2F%2Fraw.githubusercontent.com%2Faortega0703%2Fnotes-category-theory%2Fmain%2Fsrc%2Fmacros.tex
    \[\begin{tikzcd}[ampersand replacement=\&]
      C \& D \\
      \& C \& D
      \arrow["{\id_D}", from=1-2, to=2-3]
      \arrow["{\id_C}"', from=1-1, to=2-2]
      \arrow["f", from=1-1, to=1-2]
      \arrow["g"', from=1-2, to=2-2]
      \arrow["f"', from=2-2, to=2-3]
    \end{tikzcd}\]

    To proof this isomorphism is unique, consider two isomorphisms $f, g:
    C\overset{\cong}{\to}D$ between terminal objects. As there must be only one
    arrow $u$ is unique, then $f$ and $g$ must be equal.
  \end{proof}
\end{theorem}

\begin{theorem}\label{thm:iso_then_terminal}
  If an object is isomorphic to a terminal object, then it is terminal.

  \begin{proof}
    Consider a terminal object $1$ and an object $C$ together with an
    isomorphism $f:C\overset{\cong}{\to} 1$. It is necesary to proof the existence and uniqueness of an arrow from any object $D$ into $C$.

    \begin{description}
      \item[Existence:]
        Consider a morphism $g:C'\to 1$ given by the terminality of $1$, then:
        % https://q.uiver.app/#q=WzAsMyxbMCwxLCIxIl0sWzEsMSwiQyJdLFswLDAsIkQiXSxbMSwwLCJmIiwyLHsib2Zmc2V0IjoxfV0sWzAsMSwiZl57LTF9IiwyLHsib2Zmc2V0IjoxLCJzdHlsZSI6eyJib2R5Ijp7Im5hbWUiOiJkYXNoZWQifX19XSxbMiwwLCJnIiwyLHsib2Zmc2V0IjotMX1dLFsyLDEsImZeey0xfVxcY2lyYyBnIiwwLHsic3R5bGUiOnsiYm9keSI6eyJuYW1lIjoiZGFzaGVkIn19fV1d
        \[\begin{tikzcd}[ampersand replacement=\&]
          D \\
          1 \& C
          \arrow["f"', shift right=1, from=2-2, to=2-1]
          \arrow["{f^{-1}}"', shift right=1, dashed, from=2-1, to=2-2]
          \arrow["g"', shift left=1, from=1-1, to=2-1]
          \arrow["{f^{-1}\circ g}", dashed, from=1-1, to=2-2]
        \end{tikzcd}\]
        \vspace{-\baselineskip}
      \item[Uniqueness:]
        Consider a morphism $h:D \to C$, then by terminality of $1$:
        \[
          \begin{aligned}
            g &= f \circ h\\
            f^{-1} \circ g &= h
          \end{aligned}
        \]
    \end{description}
  \end{proof}
  \vspace{-1\baselineskip}
\end{theorem}

\begin{theorem}\label{thm:terminal_object_mono}
  Every arrow coming from a terminal object is a monomorphism.

  \begin{proof}
    Consider objects $X,C,D$ with $C$ terminal and morphisms $f:C\to D$,
    $g,h:X\to C$ with $f\circ g = f\circ h$. As $C$ is terminal there must be a
    unique morphism from any object into it, then $g=h$.
  \end{proof}
\end{theorem}

\begin{definition}[Initial Object]\index{Initial Object}
  For a category $\C$, an object $0$ is said to be initial when there exists a
  unique morphism $f: 0\to C$ for every $C\in
  \C_0$~\parencite[p.~48]{leinster:basic_category_theory}.
\end{definition}

\begin{theorem}[Uniqueness of Initial Objects\label{thm:initial_object_iso}]
  Initial objects in a category are unique up to unique isomorphism.

  \begin{proof}
    By duality of Theorem \ref{thm:terminal_object_iso}.
  \end{proof}
\end{theorem}

\begin{theorem}\label{thm:iso_initial_object}
  If an object is isomorphic to an initial object, it is also initial.

  \begin{proof}
    By duality of Theorem \ref{thm:iso_then_terminal}.
  \end{proof}
\end{theorem}

\begin{theorem}
  Every arrow coming into an initial object is an epimorphism.

  \begin{proof}
    By duality of Theorem \ref{thm:terminal_object_mono}.
  \end{proof}
\end{theorem}