\subsection{Classification of Morphisms}

\begin{definition}[Endomorphism]\index{Endomorphism}
  A morphism whose domain equals its codomain is an
  endomorphism~\parencite[p.~7]{riehl:category_theory_in_context}.
\end{definition}

\begin{definition}[Idempotent]\index{Idempotent}
  An endomorphism $f:C\to C$ is idempotent if $f\circ f =
  f$~\parencite[p.~20]{lane:working_mathematician}.
\end{definition}

\begin{definition}[Split]\index{Split}
  An endomorphism $f:C\to C$ is split if there exists morphisms $g:C\to D$ and
  $h:D\to C$ such that $f=h\circ g$ and $g\circ h=
  \id_{C}$~\parencite[p.~20]{lane:working_mathematician}.
\end{definition}

\begin{theorem}[Splits are Idempotent]
  Any split endomorphism is idempotent.

  \begin{proof}
    Consider an split endomorphism $f=h\circ g:C\to C$, then:
    \[
      \begin{aligned}
        f\circ f &= h\circ g \circ h\circ g\\
        &= h\circ g\\
        &= f
      \end{aligned}
    \]
  \end{proof}
  \vspace{-\baselineskip}
\end{theorem}

\begin{definition}[Left Inverse]\index{Left Inverse}\index{Retraction!see Left
  Inverse}
  A morphism $f: C\to D$ has a left inverse or a retraction $g: D\to C$ if
  $g\circ f = \id_C$~\parencite[p.~19]{lane:working_mathematician}.
\end{definition}

\begin{definition}[Right Inverse]\index{Right Inverse}\index{Section!see Right
  Inverse}
  A morphism $f: C\to D$ has a right inverse or a section $g: D\to C$ if
  $f\circ g = \id_{D}$~\parencite[p.~19]{lane:working_mathematician}.
\end{definition}

\begin{definition}[Isomorphism]\index{Isomorphism}
  A morphism is invertible or an isomorphism when it has a left and right
  inverse, and both are equal~\parencite[p.~19]{lane:working_mathematician}.
\end{definition}

\begin{remark}
  Two objects $A, B$ are said to be isomorphic (written $A\cong B$) if there
  exists is an isomorphism between them.
\end{remark}

\begin{theorem}[Uniqueness of Inverse]
  If $f: C\to D$ is an isomorphism, then it has a unique inverse $f^{-1}:D\to
  C$:
  \begin{proof}
    Consider two inverses of $f$, $f^{-1}_1$ and $f^{-1}_2$, then:
    \[
      \begin{aligned}
        \id_C &= \id_C\\
        f\circ f^{-1}_1 &= f\circ f^{-1}_2\\
        f^{-1}_1 \circ f \circ f^{-1}_1 &= f^{-1}_1 \circ f \circ f^{-1}_2\\
        f^{-1}_1 &= f^{-1}_2
      \end{aligned}
    \]
  \end{proof}
  \vspace{-\baselineskip}
\end{theorem}

\begin{theorem}
  Every identity is an isomorphism.

  \begin{proof}
    By unitality, every identity is its own inverse.
  \end{proof}
\end{theorem}

\begin{definition}[Automorphism]\index{Automorphism}
  An endomorphism which is also an isomorphism is an
  automorphism~\parencite[p.~7]{riehl:category_theory_in_context}.
\end{definition}

\begin{definition}[Monomorphisms]\index{Monomorphisms}
  A morphism $f:C\to D$ is monic or a monomorphism when it is left
  cancellable~\parencite[p.~19]{lane:working_mathematician}, i.e. when for any
  set-up of objects and morphisms:
  % https://q.uiver.app/#q=WzAsMyxbMSwwLCJDIl0sWzIsMCwiRCJdLFswLDAsIlgiXSxbMCwxLCJmIl0sWzIsMCwiZyIsMCx7Im9mZnNldCI6LTF9XSxbMiwwLCJoIiwyLHsib2Zmc2V0IjoxfV1d
  \[\begin{tikzcd}[ampersand replacement=\&]
    X \& C \& D
    \arrow["f", from=1-2, to=1-3]
    \arrow["g", shift left=1, from=1-1, to=1-2]
    \arrow["h"', shift right=1, from=1-1, to=1-2]
  \end{tikzcd}\]

  The following holds:
  \[f \circ g = f \circ h \implies g = h\]
\end{definition}

\begin{remark}
  $f:C\mono D$ will be used to denote monomorphisms.
\end{remark}

\begin{remark}
  $f:C\sub D$ will be used instead when $f$ represents inclusions.
\end{remark}

\begin{theorem}[Left Inverse Imply Monomorphisms]\label{thm:left_inverse_implies_mono}

  If $f:C\to D$ has a left inverse $f':D\to C$, then it is a monomorphism.

  \begin{proof}
    Consider the following set-up of objects and morphisms:
    % https://q.uiver.app/#q=WzAsMyxbMSwwLCJDIl0sWzIsMCwiRCJdLFswLDAsIlgiXSxbMCwxLCJmIiwwLHsib2Zmc2V0IjotMX1dLFsyLDAsImciLDAseyJvZmZzZXQiOi0xfV0sWzIsMCwiaCIsMix7Im9mZnNldCI6MX1dLFsxLDAsImYnIiwwLHsib2Zmc2V0IjotMX1dXQ==
    \[\begin{tikzcd}[ampersand replacement=\&]
      X \& C \& D
      \arrow["f", shift left=1, from=1-2, to=1-3]
      \arrow["g", shift left=1, from=1-1, to=1-2]
      \arrow["h"', shift right=1, from=1-1, to=1-2]
      \arrow["{f'}", shift left=1, from=1-3, to=1-2]
    \end{tikzcd}\]
    If $f'$ is left inverse of $f$ and $f\circ g = f\circ h$, then:
    \[
      \begin{aligned}
        f\circ g &= f\circ h\\
        f'\circ f \circ g &= f'\circ f\circ h\\
        g &= h
      \end{aligned}
    \]
  \end{proof}
  \vspace{-\baselineskip}
\end{theorem}

\begin{theorem}
  If $g\circ f: A\mono C$ is a monomorphism then so is $f$.

  \begin{proof}
    Consider an object $X$ with two morphisms $x_0, x_1:X\to A$ such that
    $f\circ x_0 = f\circ x_1$, then:
    \[
      \begin{aligned}
        f\circ x_0 &= f\circ x_1\\
        g\circ f \circ x_0 &= g \circ f \circ x_1\\
        x_0 &= x_1
      \end{aligned}
    \]
  \end{proof}
  \vspace{-\baselineskip}
\end{theorem}

\begin{definition}[Epimorphism]\index{Epimorphism}
  A morphism $f:C\to D$ is epic or an epimorphism when it is right
  cancellable~\parencite[p.~19]{lane:working_mathematician}, i.e. when for any
  set-up of objects and morphisms:
  % https://q.uiver.app/#q=WzAsMyxbMCwwLCJDIl0sWzEsMCwiRCJdLFsyLDAsIlgiXSxbMSwyLCJoIiwyLHsib2Zmc2V0IjoxfV0sWzAsMSwiZiJdLFsxLDIsImciLDAseyJvZmZzZXQiOi0xfV1d
  \[\begin{tikzcd}[ampersand replacement=\&]
    C \& D \& X
    \arrow["h"', shift right=1, from=1-2, to=1-3]
    \arrow["f", from=1-1, to=1-2]
    \arrow["g", shift left=1, from=1-2, to=1-3]
  \end{tikzcd}\]

  The following holds:
  \[g \circ f = h \circ f \implies g = h\]
\end{definition}

\begin{remark}
  $f:a\epi b$ will be used to denote epimorphisms.
\end{remark}

\begin{theorem}[Right Inverses Imply Epimorphisms]\label{thm:right_inverse_implies_epi}
  If $f:C\to D$ has a right inverse $f':D\to C$, then it is a epimorphism.

  \begin{proof}
    By duality with Theorem \ref{thm:left_inverse_implies_mono}.
  \end{proof}
\end{theorem}