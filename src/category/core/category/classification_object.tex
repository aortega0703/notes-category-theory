\subsection{Classification of Objects}

\begin{definition}[Terminal Object]\index{Terminal Object}
  For a category $\C$, an object $1$ is said to be terminal when there exists a
  unique morphism $f: C\to 1$ for every object $C\in
  \C_0$~\parencite[p.~48]{leinster:basic_category_theory}.
\end{definition}

\begin{theorem}[Uniqueness of Terminal Objects]\label{thm:terminal_object_iso}
  Initial objects in a category are unique up to unique isomorphism.

  \begin{proof}
    To proof uniqueness up to isomorphism, consider two terminal objects $C$ and
    $D$, the following diagram must commute as every arrow is unique:
    % https://q.uiver.app/#q=WzAsNCxbMSwwLCJEIl0sWzAsMCwiQyJdLFsyLDEsIkQiXSxbMSwxLCJDIl0sWzAsMiwiXFxpZF9EIl0sWzEsMywiXFxpZF9DIiwyXSxbMSwwLCJmIl0sWzAsMywiZyIsMl0sWzMsMiwiZiIsMl1d&macro_url=https%3A%2F%2Fraw.githubusercontent.com%2Faortega0703%2Fnotes-category-theory%2Fmain%2Fsrc%2Fmacros.tex
    \[\begin{tikzcd}[ampersand replacement=\&]
      C \& D \\
      \& C \& D
      \arrow["{\id_D}", from=1-2, to=2-3]
      \arrow["{\id_C}"', from=1-1, to=2-2]
      \arrow["f", from=1-1, to=1-2]
      \arrow["g"', from=1-2, to=2-2]
      \arrow["f"', from=2-2, to=2-3]
    \end{tikzcd}\]

    To proof this isomorphism is unique, consider two isomorphisms $f, g:
    C\overset{\cong}{\to}D$ between terminal objects. As there must be only one arrow $u$ is unique, then
    $f$ and $g$ must be equal.
  \end{proof}
\end{theorem}

\begin{theorem}\label{thm:iso_then_terminal}
  If an object is isomorphic to a terminal object, then it is terminal.

  \begin{proof}
    Consider a terminal object $1$ and an object $C$ together with an
    isomorphism $f:C\overset{\cong}{\to} 1$. It is necesary to proof the existence and uniqueness of an arrow from any object $D$ into $C$.

    \begin{description}
      \item[Existence:]
        Consider a morphism $g:C'\to 1$ given by the terminality of $1$, then:
        % https://q.uiver.app/#q=WzAsMyxbMCwxLCIxIl0sWzEsMSwiQyJdLFswLDAsIkQiXSxbMSwwLCJmIiwyLHsib2Zmc2V0IjoxfV0sWzAsMSwiZl57LTF9IiwyLHsib2Zmc2V0IjoxLCJzdHlsZSI6eyJib2R5Ijp7Im5hbWUiOiJkYXNoZWQifX19XSxbMiwwLCJnIiwyLHsib2Zmc2V0IjotMX1dLFsyLDEsImZeey0xfVxcY2lyYyBnIiwwLHsic3R5bGUiOnsiYm9keSI6eyJuYW1lIjoiZGFzaGVkIn19fV1d
        \[\begin{tikzcd}[ampersand replacement=\&]
          D \\
          1 \& C
          \arrow["f"', shift right=1, from=2-2, to=2-1]
          \arrow["{f^{-1}}"', shift right=1, dashed, from=2-1, to=2-2]
          \arrow["g"', shift left=1, from=1-1, to=2-1]
          \arrow["{f^{-1}\circ g}", dashed, from=1-1, to=2-2]
        \end{tikzcd}\]
      \item[Uniqueness:]
        Consider a morphism $h:D \to C$, then by terminality of $1$:
        \[
          \begin{aligned}
            g &= f \circ h\\
            f^{-1} \circ g &= h
          \end{aligned}
        \]
    \end{description}
  \end{proof}
\end{theorem}

\begin{theorem}\label{thm:terminal_object_mono}
  Every arrow coming from a terminal object is a monomorphism.

  \begin{proof}
    Consider objects $X,C,D$ with $C$ terminal and morphisms $f:C\to D$,
    $g,h:X\to C$ with $f\circ g = f\circ h$. As $C$ is terminal there must be a
    unique morphism from any object into it, then $g=h$.
  \end{proof}
\end{theorem}

\begin{definition}[Initial Object]\index{Initial Object}
  For a category $\C$, an object $0$ is said to be initial when there exists a
  unique morphism $f: 0\to C$ for every $C\in
  \C_0$~\parencite[p.~48]{leinster:basic_category_theory}.
\end{definition}

\begin{theorem}[Uniqueness of Initial Objects\label{thm:initial_object_iso}]
  Initial objects in a category are unique up to unique isomorphism.

  \begin{proof}
    By duality of Theorem \ref{thm:terminal_object_iso}.
  \end{proof}
\end{theorem}

\begin{theorem}\label{thm:iso_initial_object}
  If an object is isomorphic to an initial object, it is also initial.

  \begin{proof}
    By duality of Theorem \ref{thm:iso_then_terminal}.
  \end{proof}
\end{theorem}

\begin{theorem}
  Every arrow coming into an initial object is an epimorphism.

  \begin{proof}
    By duality of Theorem \ref{thm:terminal_object_mono}.
  \end{proof}
\end{theorem}