\subsection{Vertical Composition}

\begin{definition}[Vertical Composition\index{Vertical Composition}]
  For a set-up of categories, functors, and natural transformations:
  % https://q.uiver.app/?q=WzAsMixbMCwwLCJcXEMiXSxbMSwwLCJcXEQiXSxbMCwxLCJGIiwwLHsiY3VydmUiOi00fV0sWzAsMSwiSCIsMix7ImN1cnZlIjo0fV0sWzAsMSwiRyIsMV0sWzIsNCwiXFxhbHBoYSIsMCx7InNob3J0ZW4iOnsic291cmNlIjoyMCwidGFyZ2V0IjoyMH19XSxbNCwzLCJcXGJldGEiLDAseyJzaG9ydGVuIjp7InNvdXJjZSI6MjAsInRhcmdldCI6MjB9fV1d&macro_url=https%3A%2F%2Fraw.githubusercontent.com%2Faortega0703%2Fnotes-category-theory%2Fmain%2Fsrc%2Fmacros.tex
  \[\begin{tikzcd}[ampersand replacement=\&]
    \C \& \D
    \arrow[""{name=0, anchor=center, inner sep=0}, "F", curve={height=-24pt}, from=1-1, to=1-2]
    \arrow[""{name=1, anchor=center, inner sep=0}, "H"', curve={height=24pt}, from=1-1, to=1-2]
    \arrow[""{name=2, anchor=center, inner sep=0}, "G"{description}, from=1-1, to=1-2]
    \arrow["\alpha", shorten <=3pt, shorten >=3pt, Rightarrow, from=0, to=2]
    \arrow["\beta", shorten <=3pt, shorten >=3pt, Rightarrow, from=2, to=1]
  \end{tikzcd}\]

  The vertical composition $\beta\circ \alpha:F\to H$ consists
  of~\parencite[p.~30]{leinster:basic_category_theory}:

  \begin{itemize}
    \item $c$-Components:
      \[(\forall c\in \C_0)
        \big((\beta\circ\alpha)_c = \beta_c \circ \alpha_c\big)\]
    \item Naturality Condition:
      % https://q.uiver.app/?q=WzAsNCxbMCwwLCJGKGMpIl0sWzEsMCwiRihjJykiXSxbMCwxLCJIKGMpIl0sWzEsMSwiSChjJykiXSxbMCwxLCJGKGYpIl0sWzIsMywiSChmKSIsMl0sWzAsMiwiKFxcYmV0YVxcY2lyY1xcYWxwaGEpX2MiLDJdLFsxLDMsIihcXGJldGFcXGNpcmNcXGFscGhhKV97Yyd9Il1d
      \[\begin{tikzcd}[ampersand replacement=\&]
        {F(c)} \& {F(c')} \\
        {H(c)} \& {H(c')}
        \arrow["{F(f)}", from=1-1, to=1-2]
        \arrow["{H(f)}"', from=2-1, to=2-2]
        \arrow["{(\beta\circ\alpha)_c}"', from=1-1, to=2-1]
        \arrow["{(\beta\circ\alpha)_{c'}}", from=1-2, to=2-2]
      \end{tikzcd}\]
  \end{itemize}
\end{definition}

\begin{theorem}[Naturality of Vertical Composition]
  The naturality condition of vertical composition holds.
  \begin{proof}
    % https://q.uiver.app/?q=WzAsNixbMCwwLCJGKGMpIl0sWzEsMCwiRihjJykiXSxbMCwxLCJHKGMpIl0sWzAsMiwiSChjKSJdLFsxLDEsIkcoYycpIl0sWzEsMiwiSChjJykiXSxbMCwxLCJGKGYpIl0sWzIsNCwiRyhmKSJdLFszLDUsIkgoZikiLDJdLFswLDIsIlxcYWxwaGFfYyIsMl0sWzIsMywiXFxiZXRhX2MiLDJdLFsxLDQsIlxcYWxwaGFfe2MnfSJdLFs0LDUsIlxcYmV0YV97Yyd9Il1d
    \[\begin{tikzcd}
      {F(c)} & {F(c')} \\
      {G(c)} & {G(c')} \\
      {H(c)} & {H(c')}
      \arrow["{F(f)}", from=1-1, to=1-2]
      \arrow["{G(f)}", from=2-1, to=2-2]
      \arrow["{H(f)}"', from=3-1, to=3-2]
      \arrow["{\alpha_c}"', from=1-1, to=2-1]
      \arrow["{\beta_c}"', from=2-1, to=3-1]
      \arrow["{\alpha_{c'}}", from=1-2, to=2-2]
      \arrow["{\beta_{c'}}", from=2-2, to=3-2]
    \end{tikzcd}\]

    \[
      \begin{aligned}
        (\beta \circ \alpha)_{c'} \circ F(f)
        &= \beta_{c'} \circ \alpha_{c'} \circ F(f)\\
        &= \beta_{c'} \circ G(f) \circ \alpha_c\\
        &= H(f) \circ \beta_c \circ \alpha_c\\
        &= H(f) \circ (\alpha \circ \beta)_c
      \end{aligned}
    \]
  \end{proof}
\end{theorem}

\begin{theorem}[Unitality of Vertical Composition]
  For functors $F,G:\C\to \D$ and a natural transformation $\alpha:F\to G$, the
  following holds:
  \[\alpha \circ \id_F = \id_G \circ \alpha = \alpha\]

  \begin{proof}
    \[
      \begin{aligned}
        (\alpha \circ \id_F)_C
        &= \alpha_c \circ (\id_F)_c\\
        &= \alpha_c \circ \id_{F(c)}\\
        &= \alpha_c
      \end{aligned}
      \quad
      \begin{aligned}
        (\id_G \circ \alpha)_c
        &= (\id_G)_c \circ \alpha_c\\
        &= \id_{G(c)} \circ \alpha_c\\
        &= \alpha_c
      \end{aligned}
    \]
  \end{proof}
\end{theorem}

\begin{theorem}[Associativity of Vertical Composition]
  For a set-up of categories, functors, and natural transformations:
  % https://q.uiver.app/?q=WzAsMixbMCwwLCJcXEMiXSxbMiwwLCJcXEQiXSxbMCwxLCJGIiwwLHsiY3VydmUiOi01fV0sWzAsMSwiSSIsMix7ImN1cnZlIjo1fV0sWzAsMSwiRyIsMSx7ImN1cnZlIjotMn1dLFswLDEsIkgiLDEseyJjdXJ2ZSI6Mn1dLFsyLDQsIlxcYWxwaGEiLDAseyJzaG9ydGVuIjp7InNvdXJjZSI6MjAsInRhcmdldCI6MjB9fV0sWzQsNSwiXFxiZXRhIiwwLHsic2hvcnRlbiI6eyJzb3VyY2UiOjIwLCJ0YXJnZXQiOjIwfX1dLFs1LDMsIlxcZ2FtbWEiLDAseyJzaG9ydGVuIjp7InNvdXJjZSI6MjAsInRhcmdldCI6MjB9fV1d&macro_url=https%3A%2F%2Fraw.githubusercontent.com%2Faortega0703%2Fnotes-category-theory%2Fmain%2Fsrc%2Fmacros.tex
  \[\begin{tikzcd}[ampersand replacement=\&]
    \C \&\& \D
    \arrow[""{name=0, anchor=center, inner sep=0}, "F", curve={height=-30pt}, from=1-1, to=1-3]
    \arrow[""{name=1, anchor=center, inner sep=0}, "I"', curve={height=30pt}, from=1-1, to=1-3]
    \arrow[""{name=2, anchor=center, inner sep=0}, "G"{description}, curve={height=-12pt}, from=1-1, to=1-3]
    \arrow[""{name=3, anchor=center, inner sep=0}, "H"{description}, curve={height=12pt}, from=1-1, to=1-3]
    \arrow["\alpha", shorten <=2pt, shorten >=2pt, Rightarrow, from=0, to=2]
    \arrow["\beta", shorten <=3pt, shorten >=3pt, Rightarrow, from=2, to=3]
    \arrow["\gamma", shorten <=2pt, shorten >=2pt, Rightarrow, from=3, to=1]
  \end{tikzcd}\]

  The following holds:
  \[(\gamma \circ \beta) \circ \alpha = \gamma \circ (\beta \circ \alpha)\]

  \begin{proof}
    \[
      \begin{aligned}
        ((\gamma \circ \beta) \circ \alpha)_c
        &= (\gamma \circ \beta)_c \circ \alpha_c\\
        &= (\gamma_c \circ \beta_c) \circ \alpha_c\\
        &= \gamma_c \circ (\beta_c \circ \alpha_c)\\
        &= \gamma_c \circ (\beta \circ \alpha)_c\\
        &= (\gamma \circ (\beta \circ \alpha))_c\\
      \end{aligned}
    \]
  \end{proof}
\end{theorem}