\subsection{Vertical Composition}

\begin{definition}[Vertical Composition]\index{Vertical Composition}\label{def:nat_transformation_cmp}
  For a set-up of categories, functors, and natural transformations:
  % https://q.uiver.app/?q=WzAsMixbMCwwLCJcXEMiXSxbMSwwLCJcXEQiXSxbMCwxLCJGIiwwLHsiY3VydmUiOi00fV0sWzAsMSwiSCIsMix7ImN1cnZlIjo0fV0sWzAsMSwiRyIsMV0sWzIsNCwiXFxhbHBoYSIsMCx7InNob3J0ZW4iOnsic291cmNlIjoyMCwidGFyZ2V0IjoyMH19XSxbNCwzLCJcXGJldGEiLDAseyJzaG9ydGVuIjp7InNvdXJjZSI6MjAsInRhcmdldCI6MjB9fV1d&macro_url=https%3A%2F%2Fraw.githubusercontent.com%2Faortega0703%2Fnotes-category-theory%2Fmain%2Fsrc%2Fmacros.tex
  \[\begin{tikzcd}[ampersand replacement=\&]
    \C \& \D
    \arrow[""{name=0, anchor=center, inner sep=0}, "F", curve={height=-24pt}, from=1-1, to=1-2]
    \arrow[""{name=1, anchor=center, inner sep=0}, "H"', curve={height=24pt}, from=1-1, to=1-2]
    \arrow[""{name=2, anchor=center, inner sep=0}, "G"{description}, from=1-1, to=1-2]
    \arrow["\alpha", shorten <=3pt, shorten >=3pt, Rightarrow, from=0, to=2]
    \arrow["\beta", shorten <=3pt, shorten >=3pt, Rightarrow, from=2, to=1]
  \end{tikzcd}\]

  The vertical composition $\beta\circ \alpha:F\to H$ consists
  of~\parencite[p.~30]{leinster:basic_category_theory}:

  \begin{itemize}
    \item $C$-Components:
      \[(\forall C\in \C_0)
        \big((\beta\circ\alpha)_C = \beta_C \circ \alpha_C\big)\]
    \item Naturality Condition:
      % https://q.uiver.app/#q=WzAsNCxbMCwwLCJGKEMpIl0sWzEsMCwiRihEKSJdLFswLDEsIkgoQykiXSxbMSwxLCJIKEQpIl0sWzAsMSwiRihmKSJdLFsyLDMsIkgoZikiLDJdLFswLDIsIihcXGJldGFcXGNpcmNcXGFscGhhKV9DIiwyXSxbMSwzLCIoXFxiZXRhXFxjaXJjXFxhbHBoYSlfe0R9Il1d
      \[\begin{tikzcd}[ampersand replacement=\&]
        {F(C)} \& {F(D)} \\
        {H(C)} \& {H(D)}
        \arrow["{F(f)}", from=1-1, to=1-2]
        \arrow["{H(f)}"', from=2-1, to=2-2]
        \arrow["{(\beta\circ\alpha)_C}"', from=1-1, to=2-1]
        \arrow["{(\beta\circ\alpha)_{D}}", from=1-2, to=2-2]
      \end{tikzcd}\]
  \end{itemize}
\end{definition}

\begin{theorem}
  The naturality condition for vertical composition holds, therefore it is a
  natural transformation.
  \begin{proof}
    % https://q.uiver.app/#q=WzAsNixbMCwwLCJGKEMpIl0sWzEsMCwiRihEKSJdLFswLDEsIkcoQykiXSxbMCwyLCJIKEMpIl0sWzEsMSwiRyhEKSJdLFsxLDIsIkgoRCkiXSxbMCwxLCJGKGYpIl0sWzIsNCwiRyhmKSJdLFszLDUsIkgoZikiLDJdLFswLDIsIlxcYWxwaGFfQyIsMl0sWzIsMywiXFxiZXRhX0MiLDJdLFsxLDQsIlxcYWxwaGFfRCJdLFs0LDUsIlxcYmV0YV9EIl1d
    \[\begin{tikzcd}[ampersand replacement=\&]
      {F(C)} \& {F(D)} \\
      {G(C)} \& {G(D)} \\
      {H(C)} \& {H(D)}
      \arrow["{F(f)}", from=1-1, to=1-2]
      \arrow["{G(f)}", from=2-1, to=2-2]
      \arrow["{H(f)}"', from=3-1, to=3-2]
      \arrow["{\alpha_C}"', from=1-1, to=2-1]
      \arrow["{\beta_C}"', from=2-1, to=3-1]
      \arrow["{\alpha_D}", from=1-2, to=2-2]
      \arrow["{\beta_D}", from=2-2, to=3-2]
    \end{tikzcd}
    \qquad
    \begin{aligned}
      (\beta \circ \alpha)_{c'} \circ F(f)
      &= \beta_{c'} \circ \alpha_{c'} \circ F(f)\\
      &= \beta_{c'} \circ G(f) \circ \alpha_c\\
      &= H(f) \circ \beta_c \circ \alpha_c\\
      &= H(f) \circ (\alpha \circ \beta)_c
    \end{aligned}\]
  \end{proof}
  \vspace{-\baselineskip}
\end{theorem}

\begin{theorem}[Unitality of Vertical Composition]\label{thm:unitality_vertical}
  For functors $F,G:\C\to \D$ and a natural transformation $\alpha:F\to G$, the
  following holds:
  \[\alpha \circ \id_F = \id_G \circ \alpha = \alpha\]

  \begin{proof}
    \[
      \begin{aligned}
        (\alpha \circ \id_F)_C
        &= \alpha_C \circ (\id_F)_C\\
        &= \alpha_C \circ \id_{F(C)}\\
        &= \alpha_C
      \end{aligned}
      \qquad
      \begin{aligned}
        (\id_G \circ \alpha)_C
        &= (\id_G)_C \circ \alpha_C\\
        &= \id_{G(C)} \circ \alpha_C\\
        &= \alpha_C
      \end{aligned}
    \]
  \end{proof}
  \vspace{-\baselineskip}
\end{theorem}

\begin{theorem}[Associativity of Vertical Composition]\label{thm:assoc_vertical}
  For a set-up of categories, functors, and natural transformations:
  % https://q.uiver.app/?q=WzAsMixbMCwwLCJcXEMiXSxbMiwwLCJcXEQiXSxbMCwxLCJGIiwwLHsiY3VydmUiOi01fV0sWzAsMSwiSSIsMix7ImN1cnZlIjo1fV0sWzAsMSwiRyIsMSx7ImN1cnZlIjotMn1dLFswLDEsIkgiLDEseyJjdXJ2ZSI6Mn1dLFsyLDQsIlxcYWxwaGEiLDAseyJzaG9ydGVuIjp7InNvdXJjZSI6MjAsInRhcmdldCI6MjB9fV0sWzQsNSwiXFxiZXRhIiwwLHsic2hvcnRlbiI6eyJzb3VyY2UiOjIwLCJ0YXJnZXQiOjIwfX1dLFs1LDMsIlxcZ2FtbWEiLDAseyJzaG9ydGVuIjp7InNvdXJjZSI6MjAsInRhcmdldCI6MjB9fV1d&macro_url=https%3A%2F%2Fraw.githubusercontent.com%2Faortega0703%2Fnotes-category-theory%2Fmain%2Fsrc%2Fmacros.tex
  \[\begin{tikzcd}[ampersand replacement=\&]
    \C \&\& \D
    \arrow[""{name=0, anchor=center, inner sep=0}, "F", curve={height=-30pt}, from=1-1, to=1-3]
    \arrow[""{name=1, anchor=center, inner sep=0}, "I"', curve={height=30pt}, from=1-1, to=1-3]
    \arrow[""{name=2, anchor=center, inner sep=0}, "G"{description}, curve={height=-12pt}, from=1-1, to=1-3]
    \arrow[""{name=3, anchor=center, inner sep=0}, "H"{description}, curve={height=12pt}, from=1-1, to=1-3]
    \arrow["\alpha", shorten <=2pt, shorten >=2pt, Rightarrow, from=0, to=2]
    \arrow["\beta", shorten <=3pt, shorten >=3pt, Rightarrow, from=2, to=3]
    \arrow["\gamma", shorten <=2pt, shorten >=2pt, Rightarrow, from=3, to=1]
  \end{tikzcd}\]

  The following holds:
  \[(\gamma \circ \beta) \circ \alpha = \gamma \circ (\beta \circ \alpha)\]

  \begin{proof}
    \[
      \begin{aligned}
        ((\gamma \circ \beta) \circ \alpha)_C
        &= (\gamma \circ \beta)_C \circ \alpha_C\\
        &= (\gamma_C \circ \beta_C) \circ \alpha_C\\
        &= \gamma_C \circ (\beta_C \circ \alpha_C)\\
        &= \gamma_C \circ (\beta \circ \alpha)_C\\
        &= (\gamma \circ (\beta \circ \alpha))_C\\
      \end{aligned}
    \]
  \end{proof}
  \vspace{-\baselineskip}
\end{theorem}

\begin{theorem}
  For any categories $\C,\D$ there is a category with functors $\C\to\D$ as
  objects and natural transformations as morphisms. It is called the
  exponential category $D^C$.

  \begin{proof}
    By Definition \ref{def:id_nat_transformation},
    \ref{def:nat_transformation_cmp} and Theorems \ref{thm:unitality_vertical},
    \ref{thm:assoc_vertical}.
  \end{proof}
\end{theorem}