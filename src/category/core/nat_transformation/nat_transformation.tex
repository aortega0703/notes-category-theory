\section{Natural Transformations}

\begin{definition}[Natural Transformation\index{Natural Transformation}]
  For functors $F, G:\C \to \D$, a natural transformation $\alpha : F
  \Rightarrow G$ consists of~\parencite[p.~28]{leinster:basic_category_theory}:
  \begin{itemize}
    \item $c$-Components:
      \[(\forall c\in \C_0)(\alpha_c : F(c) \to G(c))\]
      If every $c$-component is an isomorphism the natural transformation is
      then called a natural isomorphism.
    \item Naturality Condition: The following diagram must commute for all
      $c,c'\in \C_0$ and $f \in \C(c, c')$:
      % https://q.uiver.app/?q=WzAsNCxbMCwwLCJGKGMpIl0sWzEsMCwiRihjJykiXSxbMCwxLCJHKGMpIl0sWzEsMSwiRyhjJykiXSxbMCwyLCJcXGFscGhhX2MiLDJdLFsxLDMsIlxcYWxwaGFfe2MnfSJdLFswLDEsIkYoZikiXSxbMiwzLCJHKGYpIiwyXV0=
      \[\begin{tikzcd}[ampersand replacement=\&]
        {F(c)} \& {F(c')} \\
        {G(c)} \& {G(c')}
        \arrow["{\alpha_c}"', from=1-1, to=2-1]
        \arrow["{\alpha_{c'}}", from=1-2, to=2-2]
        \arrow["{F(f)}", from=1-1, to=1-2]
        \arrow["{G(f)}"', from=2-1, to=2-2]
      \end{tikzcd}\]
  \end{itemize}
\end{definition}

\begin{example}
  Observe the following consiguration of categories, with functors $F,G:C\to D$
  and a natural tranformation $\alpha$ between them:
  % https://q.uiver.app/?q=WzAsOCxbMCwxLCJGKGEpIl0sWzEsMSwiRihiKSJdLFswLDIsIkcoYSkiXSxbMSwyLCJHKGIpIl0sWzAsMCwiYSJdLFsxLDAsImIiXSxbMiwxLCJGKGMpPUcoYykiXSxbMiwwLCJjIl0sWzAsMiwiXFxhbHBoYV9hIiwyXSxbMSwzLCJcXGFscGhhX2IiXSxbMCwxLCJGKGYpIl0sWzIsMywiRyhmKSIsMl0sWzQsNSwiZiJdLFs1LDcsImciXSxbMSw2LCJGKGcpIl0sWzMsNiwiRyhnKSIsMl1d
  \[\begin{tikzcd}[ampersand replacement=\&]
    a \& b \& c \\
    {F(a)} \& {F(b)} \& {F(c)=G(c)} \\
    {G(a)} \& {G(b)}
    \arrow["{\alpha_a}"', from=2-1, to=3-1]
    \arrow["{\alpha_b}", from=2-2, to=3-2]
    \arrow["{F(f)}", from=2-1, to=2-2]
    \arrow["{G(f)}"', from=3-1, to=3-2]
    \arrow["f", from=1-1, to=1-2]
    \arrow["g", from=1-2, to=1-3]
    \arrow["{F(g)}", from=2-2, to=2-3]
    \arrow["{G(g)}"', from=3-2, to=2-3]
  \end{tikzcd}\]
\end{example}

\subsection{Identity}

\begin{definition}[Identity Functor]\index{Identity
  Functor}\label{def:id_functor}
  For a category $\C$ there exists an identity functor $\id_\C: \C \to \C$ which
  consists of maps of~\parencite[p.~27]{adamek_herrlich_strecker:joy_cats}:
  \begin{itemize}
    \item Objects:
      \[(\forall c \in \C_0)
        (\id_\C(c) = c)\]
    \item Morphisms:
      \[\big(\forall (f: c\to c') \in \C_1\big)
        (\id_C(f) = f)\]
  \end{itemize}
\end{definition}

\begin{theorem}[$\id_\C$ Preserves Identity\label{thm:id_functor_identity}]
  For a category $\C$, the identity functor $\id_\C: \C \to \C$ preserves
  identity.

  \begin{proof}
    For all objects $c\in \C$, $\id_\C(\id_c) = \id_c$ by definition.
  \end{proof}
\end{theorem}

\begin{theorem}[$\id_\C$ Preserves Composition]\label{thm:id_functor_composition}

  For a category $\C$, the identity functor $\id_\C: \C \to \C$ preserves
  composition.

  \begin{proof}
    For all morphisms $f,g\in \C$ with $g$ composable after $f$:
    \[
      \begin{aligned}
        \id_\C(g\circ f)
        &= g\circ f\\
        &= \id_\C(g) \circ \id_\C(f)
      \end{aligned}
    \]
  \end{proof}
\end{theorem}

\begin{theorem}[$\id_\C$ is a Functor]
  For a category $\C$, the identity functor $\id_\C$ is a functor.

  \begin{proof}
    By Theorems \ref{thm:id_functor_identity} and
    \ref{thm:id_functor_composition} the identity functor is a functor.
  \end{proof}
\end{theorem}

\subsection{Vertical Composition}

\begin{definition}[Vertical Composition]\index{Vertical Composition}\label{def:nat_transformation_cmp}
  For a set-up of categories, functors, and natural transformations:
  % https://q.uiver.app/?q=WzAsMixbMCwwLCJcXEMiXSxbMSwwLCJcXEQiXSxbMCwxLCJGIiwwLHsiY3VydmUiOi00fV0sWzAsMSwiSCIsMix7ImN1cnZlIjo0fV0sWzAsMSwiRyIsMV0sWzIsNCwiXFxhbHBoYSIsMCx7InNob3J0ZW4iOnsic291cmNlIjoyMCwidGFyZ2V0IjoyMH19XSxbNCwzLCJcXGJldGEiLDAseyJzaG9ydGVuIjp7InNvdXJjZSI6MjAsInRhcmdldCI6MjB9fV1d&macro_url=https%3A%2F%2Fraw.githubusercontent.com%2Faortega0703%2Fnotes-category-theory%2Fmain%2Fsrc%2Fmacros.tex
  \[\begin{tikzcd}[ampersand replacement=\&]
    \C \& \D
    \arrow[""{name=0, anchor=center, inner sep=0}, "F", curve={height=-24pt}, from=1-1, to=1-2]
    \arrow[""{name=1, anchor=center, inner sep=0}, "H"', curve={height=24pt}, from=1-1, to=1-2]
    \arrow[""{name=2, anchor=center, inner sep=0}, "G"{description}, from=1-1, to=1-2]
    \arrow["\alpha", shorten <=3pt, shorten >=3pt, Rightarrow, from=0, to=2]
    \arrow["\beta", shorten <=3pt, shorten >=3pt, Rightarrow, from=2, to=1]
  \end{tikzcd}\]

  The vertical composition $\beta\circ \alpha:F\to H$ consists
  of~\parencite[p.~30]{leinster:basic_category_theory}:

  \begin{itemize}
    \item $C$-Components:
      \[(\forall C\in \C_0)
        \big((\beta\circ\alpha)_C = \beta_C \circ \alpha_C\big)\]
    \item Naturality Condition:
      % https://q.uiver.app/#q=WzAsNCxbMCwwLCJGKEMpIl0sWzEsMCwiRihEKSJdLFswLDEsIkgoQykiXSxbMSwxLCJIKEQpIl0sWzAsMSwiRihmKSJdLFsyLDMsIkgoZikiLDJdLFswLDIsIihcXGJldGFcXGNpcmNcXGFscGhhKV9DIiwyXSxbMSwzLCIoXFxiZXRhXFxjaXJjXFxhbHBoYSlfe0R9Il1d
      \[\begin{tikzcd}[ampersand replacement=\&]
        {F(C)} \& {F(D)} \\
        {H(C)} \& {H(D)}
        \arrow["{F(f)}", from=1-1, to=1-2]
        \arrow["{H(f)}"', from=2-1, to=2-2]
        \arrow["{(\beta\circ\alpha)_C}"', from=1-1, to=2-1]
        \arrow["{(\beta\circ\alpha)_{D}}", from=1-2, to=2-2]
      \end{tikzcd}\]
  \end{itemize}
\end{definition}

\begin{theorem}
  The naturality condition for vertical composition holds, therefore it is a
  natural transformation.
  \begin{proof}
    % https://q.uiver.app/#q=WzAsNixbMCwwLCJGKEMpIl0sWzEsMCwiRihEKSJdLFswLDEsIkcoQykiXSxbMCwyLCJIKEMpIl0sWzEsMSwiRyhEKSJdLFsxLDIsIkgoRCkiXSxbMCwxLCJGKGYpIl0sWzIsNCwiRyhmKSJdLFszLDUsIkgoZikiLDJdLFswLDIsIlxcYWxwaGFfQyIsMl0sWzIsMywiXFxiZXRhX0MiLDJdLFsxLDQsIlxcYWxwaGFfRCJdLFs0LDUsIlxcYmV0YV9EIl1d
    \[\begin{tikzcd}[ampersand replacement=\&]
      {F(C)} \& {F(D)} \\
      {G(C)} \& {G(D)} \\
      {H(C)} \& {H(D)}
      \arrow["{F(f)}", from=1-1, to=1-2]
      \arrow["{G(f)}", from=2-1, to=2-2]
      \arrow["{H(f)}"', from=3-1, to=3-2]
      \arrow["{\alpha_C}"', from=1-1, to=2-1]
      \arrow["{\beta_C}"', from=2-1, to=3-1]
      \arrow["{\alpha_D}", from=1-2, to=2-2]
      \arrow["{\beta_D}", from=2-2, to=3-2]
    \end{tikzcd}
    \qquad
    \begin{aligned}
      (\beta \circ \alpha)_{c'} \circ F(f)
      &= \beta_{c'} \circ \alpha_{c'} \circ F(f)\\
      &= \beta_{c'} \circ G(f) \circ \alpha_c\\
      &= H(f) \circ \beta_c \circ \alpha_c\\
      &= H(f) \circ (\alpha \circ \beta)_c
    \end{aligned}\]
  \end{proof}
  \vspace{-\baselineskip}
\end{theorem}

\begin{theorem}[Unitality of Vertical Composition]\label{thm:unitality_vertical}
  For functors $F,G:\C\to \D$ and a natural transformation $\alpha:F\to G$, the
  following holds:
  \[\alpha \circ \id_F = \id_G \circ \alpha = \alpha\]

  \begin{proof}
    \[
      \begin{aligned}
        (\alpha \circ \id_F)_C
        &= \alpha_C \circ (\id_F)_C\\
        &= \alpha_C \circ \id_{F(C)}\\
        &= \alpha_C
      \end{aligned}
      \qquad
      \begin{aligned}
        (\id_G \circ \alpha)_C
        &= (\id_G)_C \circ \alpha_C\\
        &= \id_{G(C)} \circ \alpha_C\\
        &= \alpha_C
      \end{aligned}
    \]
  \end{proof}
  \vspace{-\baselineskip}
\end{theorem}

\begin{theorem}[Associativity of Vertical Composition]\label{thm:assoc_vertical}
  For a set-up of categories, functors, and natural transformations:
  % https://q.uiver.app/?q=WzAsMixbMCwwLCJcXEMiXSxbMiwwLCJcXEQiXSxbMCwxLCJGIiwwLHsiY3VydmUiOi01fV0sWzAsMSwiSSIsMix7ImN1cnZlIjo1fV0sWzAsMSwiRyIsMSx7ImN1cnZlIjotMn1dLFswLDEsIkgiLDEseyJjdXJ2ZSI6Mn1dLFsyLDQsIlxcYWxwaGEiLDAseyJzaG9ydGVuIjp7InNvdXJjZSI6MjAsInRhcmdldCI6MjB9fV0sWzQsNSwiXFxiZXRhIiwwLHsic2hvcnRlbiI6eyJzb3VyY2UiOjIwLCJ0YXJnZXQiOjIwfX1dLFs1LDMsIlxcZ2FtbWEiLDAseyJzaG9ydGVuIjp7InNvdXJjZSI6MjAsInRhcmdldCI6MjB9fV1d&macro_url=https%3A%2F%2Fraw.githubusercontent.com%2Faortega0703%2Fnotes-category-theory%2Fmain%2Fsrc%2Fmacros.tex
  \[\begin{tikzcd}[ampersand replacement=\&]
    \C \&\& \D
    \arrow[""{name=0, anchor=center, inner sep=0}, "F", curve={height=-30pt}, from=1-1, to=1-3]
    \arrow[""{name=1, anchor=center, inner sep=0}, "I"', curve={height=30pt}, from=1-1, to=1-3]
    \arrow[""{name=2, anchor=center, inner sep=0}, "G"{description}, curve={height=-12pt}, from=1-1, to=1-3]
    \arrow[""{name=3, anchor=center, inner sep=0}, "H"{description}, curve={height=12pt}, from=1-1, to=1-3]
    \arrow["\alpha", shorten <=2pt, shorten >=2pt, Rightarrow, from=0, to=2]
    \arrow["\beta", shorten <=3pt, shorten >=3pt, Rightarrow, from=2, to=3]
    \arrow["\gamma", shorten <=2pt, shorten >=2pt, Rightarrow, from=3, to=1]
  \end{tikzcd}\]

  The following holds:
  \[(\gamma \circ \beta) \circ \alpha = \gamma \circ (\beta \circ \alpha)\]

  \begin{proof}
    \[
      \begin{aligned}
        ((\gamma \circ \beta) \circ \alpha)_C
        &= (\gamma \circ \beta)_C \circ \alpha_C\\
        &= (\gamma_C \circ \beta_C) \circ \alpha_C\\
        &= \gamma_C \circ (\beta_C \circ \alpha_C)\\
        &= \gamma_C \circ (\beta \circ \alpha)_C\\
        &= (\gamma \circ (\beta \circ \alpha))_C\\
      \end{aligned}
    \]
  \end{proof}
  \vspace{-\baselineskip}
\end{theorem}

\begin{theorem}
  For any categories $\C,\D$ there is a category with functors $\C\to\D$ as
  objects and natural transformations as morphisms. It is called the
  exponential category $D^C$.

  \begin{proof}
    By Definition \ref{def:id_nat_transformation},
    \ref{def:nat_transformation_cmp} and Theorems \ref{thm:unitality_vertical},
    \ref{thm:assoc_vertical}.
  \end{proof}
\end{theorem}

\subsection{Horizontal Composition}

\begin{definition}[Horizontal Composition]\index{Horizontal Composition}
  For a set-up of categories, functors, and natural transformations:
  % https://q.uiver.app/?q=WzAsMyxbMCwwLCJcXEEiXSxbMiwwLCJcXEIiXSxbNCwwLCJcXEMiXSxbMCwxLCJGIiwxLHsiY3VydmUiOi00fV0sWzAsMSwiRiciLDEseyJjdXJ2ZSI6NH1dLFsxLDIsIkciLDEseyJjdXJ2ZSI6LTR9XSxbMSwyLCJHJyIsMSx7ImN1cnZlIjo0fV0sWzMsNCwiXFxhbHBoYSIsMCx7InNob3J0ZW4iOnsic291cmNlIjoyMCwidGFyZ2V0IjoyMH19XSxbNSw2LCJcXGJldGEiLDAseyJzaG9ydGVuIjp7InNvdXJjZSI6MjAsInRhcmdldCI6MjB9fV1d&macro_url=https%3A%2F%2Fraw.githubusercontent.com%2Faortega0703%2Fnotes-category-theory%2Fmain%2Fsrc%2Fmacros.tex
  \[\begin{tikzcd}[ampersand replacement=\&]
    \A \&\& \B \&\& \C
    \arrow[""{name=0, anchor=center, inner sep=0}, "F"{description}, curve={height=-24pt}, from=1-1, to=1-3]
    \arrow[""{name=1, anchor=center, inner sep=0}, "{F'}"{description}, curve={height=24pt}, from=1-1, to=1-3]
    \arrow[""{name=2, anchor=center, inner sep=0}, "G"{description}, curve={height=-24pt}, from=1-3, to=1-5]
    \arrow[""{name=3, anchor=center, inner sep=0}, "{G'}"{description}, curve={height=24pt}, from=1-3, to=1-5]
    \arrow["\alpha", shorten <=6pt, shorten >=6pt, Rightarrow, from=0, to=1]
    \arrow["\beta", shorten <=6pt, shorten >=6pt, Rightarrow, from=2, to=3]
  \end{tikzcd}\]

  The horizontal composition $\beta * \alpha: (G\circ F) \to (G' \circ F')$
  consists of~\parencite[p.~37]{leinster:basic_category_theory}:
  \begin{itemize}
    \item $c$-Components:
      \[(\forall c\in\C_0)
        \big((\beta * \alpha)_c
        =\beta_{F'(c)} \circ G(\alpha_c)
        = G'(\alpha_c) \circ \beta_{F(c)}\big)
      \]
    \item Naturality Condition:
      The following diagram commutes:
      % https://q.uiver.app/?q=WzAsNCxbMCwwLCIoR1xcY2lyYyBGKShjKSJdLFswLDIsIihHJ1xcY2lyYyBGJykoYykiXSxbMiwwLCIoR1xcY2lyYyBGKShjJykiXSxbMiwyLCIoRydcXGNpcmMgRicpKGMnKSJdLFswLDIsIihHXFxjaXJjIEYpKGYpIl0sWzEsMywiKEcnXFxjaXJjIEYnKShmKSIsMl0sWzAsMSwiKFxcYmV0YSAqXFxhbHBoYSlfYyIsMV0sWzIsMywiKFxcYmV0YSAqXFxhbHBoYSlfe2MnfSIsMV1d
      \[\begin{tikzcd}
        {(G\circ F)(c)} && {(G\circ F)(c')} \\
        \\
        {(G'\circ F')(c)} && {(G'\circ F')(c')}
        \arrow["{(G\circ F)(f)}", from=1-1, to=1-3]
        \arrow["{(G'\circ F')(f)}"', from=3-1, to=3-3]
        \arrow["{(\beta *\alpha)_c}"{description}, from=1-1, to=3-1]
        \arrow["{(\beta *\alpha)_{c'}}"{description}, from=1-3, to=3-3]
      \end{tikzcd}\]
  \end{itemize}
\end{definition}

\begin{remark}
  The definition of the $c$-components for horizontal composition is motivated
  by the following diagram, where the equality of both definitions is given by
  the naturality condition of $\beta$.

  % https://q.uiver.app/?q=WzAsNyxbMCwwLCJjIl0sWzEsMCwiRihjKSJdLFsxLDIsIkYnKGMpIl0sWzIsMSwiKEdcXGNpcmMgRikoYykiXSxbMywwLCIoRydcXGNpcmMgRikoYykiXSxbMiwzLCIoR1xcY2lyYyBGJykoYykiXSxbMywyLCIoRydcXGNpcmMgRicpKGMpIl0sWzAsMSwiIiwwLHsiY29sb3VyIjpbMjQwLDYwLDYwXSwic3R5bGUiOnsiYm9keSI6eyJuYW1lIjoiZG90dGVkIn19fV0sWzAsMiwiIiwyLHsiY29sb3VyIjpbMCw2MCw2MF0sInN0eWxlIjp7ImJvZHkiOnsibmFtZSI6ImRvdHRlZCJ9fX1dLFsxLDIsIlxcYWxwaGFfYyIsMV0sWzEsMywiIiwwLHsiY29sb3VyIjpbMjQwLDYwLDYwXSwic3R5bGUiOnsiYm9keSI6eyJuYW1lIjoiZG90dGVkIn19fV0sWzEsNCwiIiwwLHsiY29sb3VyIjpbMCw2MCw2MF0sInN0eWxlIjp7ImJvZHkiOnsibmFtZSI6ImRvdHRlZCJ9fX1dLFsyLDUsIiIsMCx7ImNvbG91ciI6WzI0MCw2MCw2MF0sInN0eWxlIjp7ImJvZHkiOnsibmFtZSI6ImRvdHRlZCJ9fX1dLFsyLDYsIiIsMCx7ImNvbG91ciI6WzAsNjAsNjBdLCJzdHlsZSI6eyJib2R5Ijp7Im5hbWUiOiJkb3R0ZWQifX19XSxbMyw0LCJcXGJldGFfe0YoYyl9IiwxXSxbNSw2LCJcXGJldGFfe0YnKGMpfSIsMV0sWzQsNiwiRycoXFxhbHBoYV9jKSIsMV0sWzMsNSwiRyhcXGFscGhhX2MpIiwxXSxbMyw2LCJcXGJldGEqXFxhbHBoYSIsMV1d
  \[\begin{tikzcd}[ampersand replacement=\&]
    c \& {F(c)} \&\& {(G'\circ F)(c)} \\
    \&\& {(G\circ F)(c)} \\
    \& {F'(c)} \&\& {(G'\circ F')(c)} \\
    \&\& {(G\circ F')(c)}
    \arrow[draw={rgb,255:red,92;green,92;blue,214}, dotted, from=1-1, to=1-2]
    \arrow[draw={rgb,255:red,214;green,92;blue,92}, dotted, from=1-1, to=3-2]
    \arrow["{\alpha_c}"{description}, from=1-2, to=3-2]
    \arrow[draw={rgb,255:red,92;green,92;blue,214}, dotted, from=1-2, to=2-3]
    \arrow[draw={rgb,255:red,214;green,92;blue,92}, dotted, from=1-2, to=1-4]
    \arrow[draw={rgb,255:red,92;green,92;blue,214}, dotted, from=3-2, to=4-3]
    \arrow[draw={rgb,255:red,214;green,92;blue,92}, dotted, from=3-2, to=3-4]
    \arrow["{\beta_{F(c)}}"{description}, from=2-3, to=1-4]
    \arrow["{\beta_{F'(c)}}"{description}, from=4-3, to=3-4]
    \arrow["{G'(\alpha_c)}"{description}, from=1-4, to=3-4]
    \arrow["{G(\alpha_c)}"{description}, from=2-3, to=4-3]
    \arrow["{\beta*\alpha}"{description}, from=2-3, to=3-4]
  \end{tikzcd}\]
\end{remark}

\begin{theorem}[Naturality of Horizontal Composition]
  The naturality condition of horizontal composition (in \textcolor{blue}{blue})
  holds.

  \begin{proof}
    % https://q.uiver.app/?q=WzAsOCxbMSwwLCIoR1xcY2lyYyBGKShjKSJdLFszLDAsIihHJ1xcY2lyYyBGKShjKSJdLFsxLDIsIihHXFxjaXJjIEYnKShjKSJdLFszLDIsIihHJ1xcY2lyYyBGJykoYykiXSxbMiw0LCIoRydcXGNpcmMgRikoYycpIl0sWzAsNCwiKEdcXGNpcmMgRikoYycpIl0sWzAsNiwiKEdcXGNpcmMgRicpKGMnKSJdLFsyLDYsIihHJ1xcY2lyYyBGJykoYycpIl0sWzAsMiwiRyhcXGFscGhhX2MpIiwxXSxbMiwzLCJcXGJldGFfe0YnKGMpfSIsMV0sWzAsMSwiXFxiZXRhX3tGKGMpfSIsMV0sWzEsMywiRycoXFxhbHBoYV9jKSIsMV0sWzAsNSwiKEdcXGNpcmMgRikoZikiLDEseyJsYWJlbF9wb3NpdGlvbiI6NjAsImNvbG91ciI6WzI0MCw2MCw2MF19XSxbMiw2LCIoR1xcY2lyYyBGJykoZikiLDEseyJsYWJlbF9wb3NpdGlvbiI6MzB9XSxbMSw0LCIoRydcXGNpcmMgRikoZikiLDEseyJsYWJlbF9wb3NpdGlvbiI6NzB9XSxbMyw3LCIoRydcXGNpcmMgRicpKGYpIiwxLHsibGFiZWxfcG9zaXRpb24iOjQwLCJjb2xvdXIiOlsyNDAsNjAsNjBdfV0sWzUsNCwiXFxiZXRhX3tGKGMnKX0iLDFdLFs0LDcsIkcnKFxcYWxwaGFfe2MnfSkiLDFdLFs1LDYsIkcoXFxhbHBoYV97Yyd9KSIsMV0sWzYsNywiXFxiZXRhX3tGJyhjKX0iLDFdLFswLDMsIihcXGJldGEgKlxcYWxwaGEpX2MiLDEseyJjb2xvdXIiOlsyNDAsNjAsNjBdfV0sWzUsNywiKFxcYmV0YSAqXFxhbHBoYSlfe2MnfSIsMSx7ImNvbG91ciI6WzI0MCw2MCw2MF19XV0=
    \[\begin{tikzcd}[scale cd=0.9]
      & {(G\circ F)(c)} && {(G'\circ F)(c)} \\
      \\
      & {(G\circ F')(c)} && {(G'\circ F')(c)} \\
      \\
      {(G\circ F)(c')} && {(G'\circ F)(c')} \\
      \\
      {(G\circ F')(c')} && {(G'\circ F')(c')}
      \arrow["{G(\alpha_c)}"{description}, from=1-2, to=3-2]
      \arrow["{\beta_{F'(c)}}"{description}, from=3-2, to=3-4]
      \arrow["{\beta_{F(c)}}"{description}, from=1-2, to=1-4]
      \arrow["{G'(\alpha_c)}"{description}, from=1-4, to=3-4]
      \arrow["{(G\circ F)(f)}"{description, pos=0.6}, draw={rgb,255:red,92;green,92;blue,214}, from=1-2, to=5-1]
      \arrow["{(G\circ F')(f)}"{description, pos=0.3}, from=3-2, to=7-1]
      \arrow["{(G'\circ F)(f)}"{description, pos=0.7}, from=1-4, to=5-3]
      \arrow["{(G'\circ F')(f)}"{description, pos=0.4}, draw={rgb,255:red,92;green,92;blue,214}, from=3-4, to=7-3]
      \arrow["{\beta_{F(c')}}"{description}, from=5-1, to=5-3]
      \arrow["{G'(\alpha_{c'})}"{description}, from=5-3, to=7-3]
      \arrow["{G(\alpha_{c'})}"{description}, from=5-1, to=7-1]
      \arrow["{\beta_{F'(c)}}"{description}, from=7-1, to=7-3]
      \arrow["{(\beta *\alpha)_c}"{description}, draw={rgb,255:red,92;green,92;blue,214}, from=1-2, to=3-4]
      \arrow["{(\beta *\alpha)_{c'}}"{description}, draw={rgb,255:red,92;green,92;blue,214}, from=5-1, to=7-3]
    \end{tikzcd}\]
    \[
    \begin{aligned}
      &(G'\circ F')(f) \circ (\beta * \alpha)_c\\
      =& (G'\circ F')(f) \circ G'(\alpha_c)\circ\beta_{F(c)}\\
      =& G'(F'(f) \circ \alpha_c)\circ\beta_{F(c)}\\
      =& G'(\alpha_{c'} \circ F)(f) \circ \beta_{F(c)}\\
      =& G'(\alpha_{c'}) \circ (G'\circ F)(f) \circ \beta_{F(c)}\\
      =& G'(\alpha_{c'}) \circ \beta_{F(c')} \circ (G \circ F)(f)\\
      =& (\beta * \alpha)_{c'}\circ (G \circ F)(f)
    \end{aligned}
  \]
  \end{proof}
\end{theorem}

\begin{theorem}[Unitality of Horizontal Composition]
  For functors $F, G:\C\to \D$ and a natural transformations
  $\alpha:F\Rightarrow G$, the following holds:
  \[\alpha * \id_{\id_\C} = \id_{\id_\D} * \alpha = \alpha\]

  \begin{proof}
    \[
    \begin{aligned}[t]
      (\alpha * \id_{\id_\C})_c
      &= \alpha_{\id_\C(c)} \circ F\big((\id_{\id_\C})_c\big)\\
      &= \alpha_{c} \circ F(\id_{\id_\C(c)})\\
      &= \alpha_{c} \circ F(\id_{c})\\
      &= \alpha_{c} \circ \id_{F(c)}\\
      &= \alpha_{c}
    \end{aligned}
    \quad
    \begin{aligned}[t]
      (\id_{\id_\D} * \alpha)_c
      &= (\id_{\id_\D})_{G(c)} \circ \id_\D(\alpha_c)\\
      &= \id_{\id_\D(G(c))} \circ \alpha_c\\
      &= \id_{G(c)} \circ \alpha_c\\
      &= \alpha_c
    \end{aligned}
  \]
  \end{proof}
\end{theorem}

\begin{theorem}[Associativity of Horizontal Composition]
  For a set-up of categories, functors, and natural transformations:
  % https://q.uiver.app/?q=WzAsNCxbNCwwLCJcXEMiXSxbMiwwLCJcXEIiXSxbMCwwLCJcXEEiXSxbNiwwLCJcXEQiXSxbMSwwLCJHIiwwLHsiY3VydmUiOi00fV0sWzEsMCwiRyciLDIseyJjdXJ2ZSI6NH1dLFsyLDEsIkYiLDAseyJjdXJ2ZSI6LTR9XSxbMiwxLCJGJyIsMix7ImN1cnZlIjo0fV0sWzAsMywiSCIsMCx7ImN1cnZlIjotNH1dLFswLDMsIkgnIiwyLHsiY3VydmUiOjR9XSxbNiw3LCJcXGFscGhhIiwwLHsic2hvcnRlbiI6eyJzb3VyY2UiOjIwLCJ0YXJnZXQiOjIwfX1dLFs0LDUsIlxcYmV0YSIsMCx7InNob3J0ZW4iOnsic291cmNlIjoyMCwidGFyZ2V0IjoyMH19XSxbOCw5LCJcXGdhbW1hIiwwLHsic2hvcnRlbiI6eyJzb3VyY2UiOjIwLCJ0YXJnZXQiOjIwfX1dXQ==&macro_url=https%3A%2F%2Fraw.githubusercontent.com%2Faortega0703%2Fnotes-category-theory%2Fmain%2Fsrc%2Fmacros.tex
  \[\begin{tikzcd}[ampersand replacement=\&]
    \A \&\& \B \&\& \C \&\& \D
    \arrow[""{name=0, anchor=center, inner sep=0}, "G", curve={height=-24pt}, from=1-3, to=1-5]
    \arrow[""{name=1, anchor=center, inner sep=0}, "{G'}"', curve={height=24pt}, from=1-3, to=1-5]
    \arrow[""{name=2, anchor=center, inner sep=0}, "F", curve={height=-24pt}, from=1-1, to=1-3]
    \arrow[""{name=3, anchor=center, inner sep=0}, "{F'}"', curve={height=24pt}, from=1-1, to=1-3]
    \arrow[""{name=4, anchor=center, inner sep=0}, "H", curve={height=-24pt}, from=1-5, to=1-7]
    \arrow[""{name=5, anchor=center, inner sep=0}, "{H'}"', curve={height=24pt}, from=1-5, to=1-7]
    \arrow["\alpha", shorten <=6pt, shorten >=6pt, Rightarrow, from=2, to=3]
    \arrow["\beta", shorten <=6pt, shorten >=6pt, Rightarrow, from=0, to=1]
    \arrow["\gamma", shorten <=6pt, shorten >=6pt, Rightarrow, from=4, to=5]
  \end{tikzcd}\]

  The following holds:
  \[(\gamma * \beta) * \alpha = \gamma * (\beta * \alpha)\]

  \begin{proof}
    \[
      \begin{aligned}
        & ((\gamma * \beta) * \alpha)_a\\
        =& (\gamma * \beta)_{F'(a)} \circ (H\circ G)(\alpha_a)\\
        =& \left(\gamma_{(G'\circ F')(a)}\circ H(\beta_{F'(a)})\right)
        \circ (H\circ G)(\alpha_a)\\
        =& \gamma_{(G'\circ F')(a)} \circ
        \left(H(\beta_{F'(a)}) \circ (H\circ G)(\alpha_a)\right)\\
        =& \gamma_{(G'\circ F')(a)}\circ
        H\left((\beta_{F'(a)}) \circ G(\alpha_a)\right)\\
        =& \gamma_{(G'\circ F')(a)} \circ H( (\beta *\alpha)_a)\\
        =& (\gamma * (\beta * \alpha))_a
      \end{aligned}
    \]
  \end{proof}
\end{theorem}

\begin{definition}[Whiskering]\index{Whiskering}
  For a set-up of categories, functors, and natural transformations:
  % https://q.uiver.app/?q=WzAsMyxbMCwwLCJcXEEiXSxbMSwwLCJcXEIiXSxbMiwwLCJcXEMiXSxbMCwxLCJGIiwwLHsiY3VydmUiOi00fV0sWzAsMSwiRiciLDIseyJjdXJ2ZSI6NH1dLFsxLDIsIkciLDAseyJjdXJ2ZSI6LTR9XSxbMSwyLCJHJyIsMix7ImN1cnZlIjo0fV0sWzMsNCwiXFxhbHBoYSIsMCx7InNob3J0ZW4iOnsic291cmNlIjoyMCwidGFyZ2V0IjoyMH19XSxbNSw2LCJcXGJldGEiLDAseyJzaG9ydGVuIjp7InNvdXJjZSI6MjAsInRhcmdldCI6MjB9fV1d&macro_url=https%3A%2F%2Fraw.githubusercontent.com%2Faortega0703%2Fnotes-category-theory%2Fmain%2Fsrc%2Fmacros.tex
  \[\begin{tikzcd}[ampersand replacement=\&]
    \A \& \B \& \C
    \arrow[""{name=0, anchor=center, inner sep=0}, "F", curve={height=-24pt}, from=1-1, to=1-2]
    \arrow[""{name=1, anchor=center, inner sep=0}, "{F'}"', curve={height=24pt}, from=1-1, to=1-2]
    \arrow[""{name=2, anchor=center, inner sep=0}, "G", curve={height=-24pt}, from=1-2, to=1-3]
    \arrow[""{name=3, anchor=center, inner sep=0}, "{G'}"', curve={height=24pt}, from=1-2, to=1-3]
    \arrow["\alpha", shorten <=6pt, shorten >=6pt, Rightarrow, from=0, to=1]
    \arrow["\beta", shorten <=6pt, shorten >=6pt, Rightarrow, from=2, to=3]
  \end{tikzcd}\]

  Whiskering between functors and natural transformations is defined as:
  \[
    \begin{aligned}
      (\beta \circ F)_a &\coloneqq \beta_{F(a)}
    \end{aligned}
    \quad
    \begin{aligned}
      (G \circ \alpha)_a &\coloneqq G(\alpha_a)
    \end{aligned}
  \]
\end{definition}

\begin{remark}
  The definition of whiskering is motivated by considering the horizontal
  compositions with functor identities:
  \[
    % https://q.uiver.app/?q=WzAsNCxbMCwwLCIoR1xcY2lyYyBGKShhKSJdLFsxLDEsIihHJ1xcY2lyYyBGKShhKSJdLFsxLDAsIihHJ1xcY2lyYyBGKShhKSJdLFswLDEsIihHXFxjaXJjIEYpKGEpIl0sWzAsMiwiXFxiZXRhX3tGKGEpfSJdLFswLDMsIkcoXFxpZF97RihhKX0pIiwyXSxbMCwxLCJcXGJldGEqXFxpZF9GIiwxLHsic3R5bGUiOnsiYm9keSI6eyJuYW1lIjoiZG90dGVkIn19fV0sWzMsMSwiXFxiZXRhX3tGKGEpfSIsMl0sWzIsMSwiRycoXFxpZF97RihhKX0pIl1d&macro_url=https%3A%2F%2Fgist.githubusercontent.com%2Faortega0703%2Fa1fd97cb097b8142e63a6fbf0cdb0f76%2Fraw%2Fe6ec84fda0ef3d3ac45c61c03814b7a3a1c30dff%2Fmacros.tex
    \begin{tikzcd}
      {(G\circ F)(a)} & {(G'\circ F)(a)} \\
      {(G\circ F)(a)} & {(G'\circ F)(a)}
      \arrow["{\beta_{F(a)}}", from=1-1, to=1-2]
      \arrow["{G(\id_{F(a)})}"', from=1-1, to=2-1]
      \arrow["{\beta*\id_F}"{description}, dotted, from=1-1, to=2-2]
      \arrow["{\beta_{F(a)}}"', from=2-1, to=2-2]
      \arrow["{G'(\id_{F(a)})}", from=1-2, to=2-2]
    \end{tikzcd}
    \quad
    % https://q.uiver.app/?q=WzAsNCxbMCwwLCIoR1xcY2lyYyBGKShhKSJdLFsxLDEsIihHXFxjaXJjIEYnKShhKSJdLFsxLDAsIihHXFxjaXJjIEYpKGEpIl0sWzAsMSwiKEdcXGNpcmMgRicpKGEpIl0sWzAsMiwiKFxcaWRfRylfe0YoYSl9Il0sWzIsMSwiRyhcXGFscGhhX2EpIl0sWzAsMywiRyhcXGFscGhhX2EpIiwyXSxbMywxLCIoXFxpZF9HKV97RihhKX0iLDJdLFswLDEsIlxcaWRfRyAqIFxcYWxwaGEiLDEseyJzdHlsZSI6eyJib2R5Ijp7Im5hbWUiOiJkb3R0ZWQifX19XV0=&macro_url=https%3A%2F%2Fgist.githubusercontent.com%2Faortega0703%2Fa1fd97cb097b8142e63a6fbf0cdb0f76%2Fraw%2Fe6ec84fda0ef3d3ac45c61c03814b7a3a1c30dff%2Fmacros.tex
    \begin{tikzcd}
      {(G\circ F)(a)} & {(G\circ F)(a)} \\
      {(G\circ F')(a)} & {(G\circ F')(a)}
      \arrow["{(\id_G)_{F(a)}}", from=1-1, to=1-2]
      \arrow["{G(\alpha_a)}", from=1-2, to=2-2]
      \arrow["{G(\alpha_a)}"', from=1-1, to=2-1]
      \arrow["{(\id_G)_{F(a)}}"', from=2-1, to=2-2]
      \arrow["{\id_G * \alpha}"{description}, dotted, from=1-1, to=2-2]
    \end{tikzcd}
  \]
\end{remark}

\begin{theorem}[Interchange Law]
  For a set-up of categories, functors, and natural transformations:
  % https://q.uiver.app/?q=WzAsMyxbNCwwLCJcXEMiXSxbMiwwLCJcXEIiXSxbMCwwLCJcXEEiXSxbMSwwLCJHXzEiLDAseyJjdXJ2ZSI6LTR9XSxbMSwwLCJHXzMiLDIseyJjdXJ2ZSI6NH1dLFsyLDEsIkZfMSIsMCx7ImN1cnZlIjotNH1dLFsyLDEsIkZfMiIsMV0sWzIsMSwiRl8zIiwyLHsiY3VydmUiOjR9XSxbMSwwLCJHXzIiLDFdLFs1LDYsIlxcYWxwaGFfMSIsMCx7InNob3J0ZW4iOnsic291cmNlIjoyMCwidGFyZ2V0IjoyMH19XSxbNiw3LCJcXGFscGhhXzIiLDAseyJzaG9ydGVuIjp7InNvdXJjZSI6MjAsInRhcmdldCI6MjB9fV0sWzMsOCwiXFxiZXRhXzEiLDAseyJzaG9ydGVuIjp7InNvdXJjZSI6MjAsInRhcmdldCI6MjB9fV0sWzgsNCwiXFxiZXRhXzIiLDAseyJzaG9ydGVuIjp7InNvdXJjZSI6MjAsInRhcmdldCI6MjB9fV1d&macro_url=https%3A%2F%2Fraw.githubusercontent.com%2Faortega0703%2Fnotes-category-theory%2Fmain%2Fsrc%2Fmacros.tex
  \[\begin{tikzcd}[ampersand replacement=\&]
    \A \&\& \B \&\& \C
    \arrow[""{name=0, anchor=center, inner sep=0}, "{G_1}", curve={height=-24pt}, from=1-3, to=1-5]
    \arrow[""{name=1, anchor=center, inner sep=0}, "{G_3}"', curve={height=24pt}, from=1-3, to=1-5]
    \arrow[""{name=2, anchor=center, inner sep=0}, "{F_1}", curve={height=-24pt}, from=1-1, to=1-3]
    \arrow[""{name=3, anchor=center, inner sep=0}, "{F_2}"{description}, from=1-1, to=1-3]
    \arrow[""{name=4, anchor=center, inner sep=0}, "{F_3}"', curve={height=24pt}, from=1-1, to=1-3]
    \arrow[""{name=5, anchor=center, inner sep=0}, "{G_2}"{description}, from=1-3, to=1-5]
    \arrow["{\alpha_1}", shorten <=3pt, shorten >=3pt, Rightarrow, from=2, to=3]
    \arrow["{\alpha_2}", shorten <=3pt, shorten >=3pt, Rightarrow, from=3, to=4]
    \arrow["{\beta_1}", shorten <=3pt, shorten >=3pt, Rightarrow, from=0, to=5]
    \arrow["{\beta_2}", shorten <=3pt, shorten >=3pt, Rightarrow, from=5, to=1]
  \end{tikzcd}\]

  The following holds:
  \[(\beta_2\circ\beta_1)*(\alpha_2\circ\alpha_1) = (\beta_2*\alpha_2) \circ
  (\alpha_2*\alpha_2)\]

  \begin{proof}
    Consider the following diagram in the exponential category $\A^\C$:

    % https://q.uiver.app/?q=WzAsOSxbMCwwLCJHXzFcXGNpcmMgRl8xIl0sWzIsMiwiR18yXFxjaXJjIEZfMiJdLFsyLDAsIkdfMlxcY2lyYyBGXzEiXSxbMCwyLCJHXzEgXFxjaXJjIEZfMiJdLFs0LDQsIkdfMyBcXGNpcmMgRl8zIl0sWzAsNCwiR18xIFxcY2lyYyBGXzMiXSxbMiw0LCJHXzJcXGNpcmMgRl8zIl0sWzQsMiwiR18zIFxcY2lyYyBGXzIiXSxbNCwwLCJHXzNcXGNpcmMgRl8xIl0sWzAsMSwiXFxiZXRhXzEqXFxhbHBoYV8xIiwxXSxbMCwyLCJcXGJldGFfMVxcY2lyYyBGXzEiLDFdLFswLDMsIkdfMVxcY2lyY1xcYWxwaGFfMSIsMV0sWzMsMSwiXFxiZXRhXzFcXGNpcmMgRl8yIiwxXSxbMiwxLCJHXzJcXGNpcmMgXFxhbHBoYV8xIiwxXSxbMSw0LCJcXGJldGFfMiAqIFxcYWxwaGFfMiIsMV0sWzMsNSwiR18xXFxjaXJjXFxhbHBoYV8yIiwxXSxbNSw2LCJcXGJldGFfMVxcY2lyYyBGXzMiLDFdLFszLDYsIlxcYmV0YV8xKlxcYWxwaGFfMiIsMV0sWzEsNiwiR18yXFxjaXJjXFxhbHBoYV8yIiwxXSxbNiw0LCJcXGJldGFfMlxcY2lyYyBGXzMiLDFdLFsxLDcsIlxcYmV0YV8yXFxjaXJjIEZfMiIsMV0sWzcsNCwiR18zXFxjaXJjIFxcYWxwaGFfMiIsMV0sWzIsOCwiXFxiZXRhXzJcXGNpcmMgRjEiLDFdLFs4LDcsIkdfM1xcY2lyY1xcYWxwaGFfMSIsMV0sWzIsNywiXFxiZXRhXzIqXFxhbHBoYV8xIiwxXV0=&macro_url=https%3A%2F%2Fgist.githubusercontent.com%2Faortega0703%2Fa1fd97cb097b8142e63a6fbf0cdb0f76%2Fraw%2Fe6ec84fda0ef3d3ac45c61c03814b7a3a1c30dff%2Fmacros.tex
    \[\begin{tikzcd}[ampersand replacement=\&]
      {G_1\circ F_1} \&\& {G_2\circ F_1} \&\& {G_3\circ F_1} \\
      \\
      {G_1 \circ F_2} \&\& {G_2\circ F_2} \&\& {G_3 \circ F_2} \\
      \\
      {G_1 \circ F_3} \&\& {G_2\circ F_3} \&\& {G_3 \circ F_3}
      \arrow["{\beta_1*\alpha_1}"{description}, from=1-1, to=3-3]
      \arrow["{\beta_1\circ F_1}"{description}, from=1-1, to=1-3]
      \arrow["{G_1\circ\alpha_1}"{description}, from=1-1, to=3-1]
      \arrow["{\beta_1\circ F_2}"{description}, from=3-1, to=3-3]
      \arrow["{G_2\circ \alpha_1}"{description}, from=1-3, to=3-3]
      \arrow["{\beta_2 * \alpha_2}"{description}, from=3-3, to=5-5]
      \arrow["{G_1\circ\alpha_2}"{description}, from=3-1, to=5-1]
      \arrow["{\beta_1\circ F_3}"{description}, from=5-1, to=5-3]
      \arrow["{\beta_1*\alpha_2}"{description}, from=3-1, to=5-3]
      \arrow["{G_2\circ\alpha_2}"{description}, from=3-3, to=5-3]
      \arrow["{\beta_2\circ F_3}"{description}, from=5-3, to=5-5]
      \arrow["{\beta_2\circ F_2}"{description}, from=3-3, to=3-5]
      \arrow["{G_3\circ \alpha_2}"{description}, from=3-5, to=5-5]
      \arrow["{\beta_2\circ F1}"{description}, from=1-3, to=1-5]
      \arrow["{G_3\circ\alpha_1}"{description}, from=1-5, to=3-5]
      \arrow["{\beta_2*\alpha_1}"{description}, from=1-3, to=3-5]
    \end{tikzcd}\]

    \[
      \begin{aligned}
        (\beta_2 \circ \beta_1) * (\alpha_2 \circ \alpha_1)
        &= (G_3\circ \alpha_2) \circ (G_3\circ \alpha_1)
          \circ (beta_2 \circ F1) \circ (\beta_1 \circ F1)\\
        &= (G_3\circ \alpha_2) \circ (\beta_2\circ F_2)
          \circ (G_2\circ \alpha_1) \circ (\beta_1\circ F_1)\\
        &= (\beta_2 * \alpha_2) \circ (\beta_1 * \alpha_1)
      \end{aligned}
    \]
  \end{proof}
\end{theorem}