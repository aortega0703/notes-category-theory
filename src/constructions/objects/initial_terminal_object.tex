\subsection{Initial and Terminal Objects}

\begin{definition}[Initial Object\index{Initial Object}]
  For a category $\C$, an object $0$ is said to be initial when there exists a
  unique morphism $f: 0\to c$ for every $c\in \C_0$.
  \parencite{awodey:category_theory}
\end{definition}

\begin{remark}
  Initial objects are the dual of terminal objects.
\end{remark}

\begin{theorem}[Uniqueness of Initial Objects\label{thm:initial_object_iso}]
  If there are multiple terminal objects in a category, then they are all
  isomorphic.

  \begin{proof}
    Consider two initial objects $c$ and $c'$, the following diagrams must
    commute :
    % https://q.uiver.app/?q=WzAsMyxbMCwwLCJjIl0sWzEsMCwiYyciXSxbMSwxLCJjIl0sWzAsMiwiXFxpZF9jIiwyLHsic3R5bGUiOnsiYm9keSI6eyJuYW1lIjoiZGFzaGVkIn19fV0sWzAsMSwiZiIsMCx7InN0eWxlIjp7ImJvZHkiOnsibmFtZSI6ImRhc2hlZCJ9fX1dLFsxLDIsImciLDAseyJzdHlsZSI6eyJib2R5Ijp7Im5hbWUiOiJkYXNoZWQifX19XV0=&macro_url=https%3A%2F%2Fraw.githubusercontent.com%2Faortega0703%2Fnotes-category-theory%2Fmain%2Fsrc%2Fmacros.tex
    \[\begin{tikzcd}
      c & {c'} \\
      & c
      \arrow["{\id_c}"', dashed, from=1-1, to=2-2]
      \arrow["f", dashed, from=1-1, to=1-2]
      \arrow["g", dashed, from=1-2, to=2-2]
    \end{tikzcd}
    \quad
    % https://q.uiver.app/?q=WzAsMyxbMCwwLCJjJyJdLFsxLDAsImMiXSxbMSwxLCJjJyJdLFswLDIsIlxcaWRfYyIsMix7InN0eWxlIjp7ImJvZHkiOnsibmFtZSI6ImRhc2hlZCJ9fX1dLFswLDEsImciLDAseyJzdHlsZSI6eyJib2R5Ijp7Im5hbWUiOiJkYXNoZWQifX19XSxbMSwyLCJmIiwwLHsic3R5bGUiOnsiYm9keSI6eyJuYW1lIjoiZGFzaGVkIn19fV1d&macro_url=https%3A%2F%2Fraw.githubusercontent.com%2Faortega0703%2Fnotes-category-theory%2Fmain%2Fsrc%2Fmacros.tex
    \begin{tikzcd}
      {c'} & c \\
      & {c'}
      \arrow["{\id_c}"', dashed, from=1-1, to=2-2]
      \arrow["g", dashed, from=1-1, to=1-2]
      \arrow["f", dashed, from=1-2, to=2-2]
    \end{tikzcd}\]
  \end{proof}
\end{theorem}

\begin{theorem}[Isomorphism to an Initial Object\label{thm:iso_initial_object}]
  If an object is isomorphic to an initial object, it is also initial.

  \begin{proof}
    Consider an initial object $1$ isomorphic to an object $c$ with an arrow
    $f:1\to c$. The proof consists on proving the existence and uniqueness of an
    arrow from $c$ into an arbitrary object $c'$.

    \begin{description}
      \item[Existence:]
        % https://q.uiver.app/?q=WzAsMyxbMCwxLCIxIl0sWzEsMSwiYyJdLFswLDAsImMnIl0sWzAsMSwiZiIsMCx7Im9mZnNldCI6LTEsInN0eWxlIjp7ImJvZHkiOnsibmFtZSI6ImRhc2hlZCJ9fX1dLFsxLDAsImZeey0xfSIsMCx7Im9mZnNldCI6LTF9XSxbMCwyLCJnIiwwLHsib2Zmc2V0IjoxLCJzdHlsZSI6eyJib2R5Ijp7Im5hbWUiOiJkYXNoZWQifX19XSxbMSwyLCJnXFxjaXJjIGZeey0xfSIsMl1d
        \[\begin{tikzcd}[ampersand replacement=\&]
          {c'} \\
          1 \& c
          \arrow["f", shift left=1, dashed, from=2-1, to=2-2]
          \arrow["{f^{-1}}", shift left=1, from=2-2, to=2-1]
          \arrow["g", shift right=1, dashed, from=2-1, to=1-1]
          \arrow["{g\circ f^{-1}}"', from=2-2, to=1-1]
        \end{tikzcd}\]
      \item[Uniqueness:] Consider an arrow $h:c\to c'$, then by the initiality
        of $1$:
        \[
          \begin{aligned}
            g &= h\circ f\\
            g\circ f^{-1} &= h
          \end{aligned}
        \]
    \end{description}
  \end{proof}
\end{theorem}

\begin{definition}[Terminal Object\index{Terminal Object}]
  For a category $\C$, an object $1$ is said to be terminal when there exists a
  unique morphism $f: c\to 1$ for every object $c\in C$.
  \parencite{awodey:category_theory}
\end{definition}

\begin{remark}
  Terminal objects are the dual of initial objects.
\end{remark}

\begin{theorem}[Uniqueness of Terminal Objects\label{thm:terminal_object_iso}]
  If there are multiple terminal objects in a category, then they are all
  isomorphic.

  \begin{proof}
    By Theorem \ref{thm:initial_object_iso} initial objects are unique up to
    isomorphism. Then by duality of initial and terminal objects, the statement
    holds.
  \end{proof}
\end{theorem}

\begin{theorem}[Isomorphism to Terminal Objects]
  If an object is isomorphic to a terminal object, then it is terminal.

  \begin{proof}
    By Theorem \ref{thm:iso_initial_object} objects isomorphic to an initial
    object are initial. Then by duality of initial and terminal objects, the
    statement holds.
  \end{proof}
\end{theorem}