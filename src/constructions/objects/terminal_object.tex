\subsection{Terminal Object}

\begin{definition}[Terminal Object]
  For a category $C$, an object $1$ is said to be terminal when there exists a
  unique morphism $f: c\to 1$ for every object $c\in C$.
  \parencite{awodey:category_theory}
\end{definition}

\begin{theorem}[Isomorphism of Terminal Objects]\label{thm:term_obj_iso} If
  there are multiple terminal objects in a category, then they are all
  isomorphic.

  \begin{proof}
    Consider two terminal objects $c$ and $c'$ in the same category. By their
    terminality there exists a unique morphism into themselves from themselves,
    the identity morphism. Additionally the exists two unique morphisms $f:c\to
    c'$ and $g:c' \to c$ between them. Their compositions $g\circ f:c\to c$ and
    $f\circ g:c'\to c'$ must then be equal to their respective identities
    $\id_c$ and $\id_{c'}$ coinciding with the definition of isomorphism.
  \end{proof}
\end{theorem}

e.g. In Set, the singleton set is a terminal object. From every other set
there exists a unique function into the singleton set, that being the constant
function.