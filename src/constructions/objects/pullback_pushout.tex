\subsection{Pullbacks and Pushouts}

\subsubsection*{Pullbacks}

\begin{definition}[Pullback\index{Pullback}\label{def:pullback}]
  For a set-up in an arbitrary category:
  \parencite{leinster:basic_category_theory}
  % https://q.uiver.app/?q=WzAsMyxbMiwwLCJZIl0sWzAsMCwiWCJdLFsxLDAsIloiXSxbMCwyLCJ0Il0sWzEsMiwicyIsMl1d
  % https://q.uiver.app/?q=WzAsMyxbMiwwLCJ5Il0sWzAsMCwieCJdLFsxLDAsInoiXSxbMCwyLCJ0IiwyXSxbMSwyLCJzIl1d
  \[\begin{tikzcd}[ampersand replacement=\&]
    x \& z \& y
    \arrow["t"', from=1-3, to=1-2]
    \arrow["s", from=1-1, to=1-2]
  \end{tikzcd}\]

  A pullback candidate is an object $a$ with morphisms $j:a\to x$ and
  $j:a\to y$ such that the following commutes:
  % https://q.uiver.app/?q=WzAsNCxbMSwwLCJ5Il0sWzAsMSwieCJdLFsxLDEsInoiXSxbMCwwLCJhIl0sWzAsMiwidCJdLFsxLDIsInMiLDJdLFszLDEsImoiLDJdLFszLDAsImoiXV0=
  \[\begin{tikzcd}[ampersand replacement=\&]
    a \& y \\
    x \& z
    \arrow["t", from=1-2, to=2-2]
    \arrow["s"', from=2-1, to=2-2]
    \arrow["j"', from=1-1, to=2-1]
    \arrow["j", from=1-1, to=1-2]
  \end{tikzcd}\]

  A pullback is a candidate $\<p, i_1, i_2\>$ such that for any other candidate
  $\<a, j_1, j_2\>$ there exists a unique $h:a\to p$ that makes the following
  commute:
  % https://q.uiver.app/?q=WzAsNSxbMSwxLCJwIl0sWzIsMSwieSJdLFsxLDIsIngiXSxbMiwyLCJ6Il0sWzAsMCwiYSJdLFswLDIsImlfMiIsMl0sWzAsMSwiaV8xIl0sWzEsMywidCJdLFsyLDMsInMiLDJdLFs0LDIsImpfMiIsMl0sWzQsMSwial8xIl0sWzQsMCwiaCIsMSx7InN0eWxlIjp7ImJvZHkiOnsibmFtZSI6ImRhc2hlZCJ9fX1dXQ==
  \[\begin{tikzcd}[ampersand replacement=\&]
    a \\
    \& p \& y \\
    \& x \& z
    \arrow["{i_2}"', from=2-2, to=3-2]
    \arrow["{i_1}", from=2-2, to=2-3]
    \arrow["t", from=2-3, to=3-3]
    \arrow["s"', from=3-2, to=3-3]
    \arrow["{j_2}"', from=1-1, to=3-2]
    \arrow["{j_1}", from=1-1, to=2-3]
    \arrow["h"{description}, dashed, from=1-1, to=2-2]
  \end{tikzcd}\]
\end{definition}

\begin{definition}[Pullback as Limit\index{Pullback!as Limit}]
  Definition \ref{def:pullback} is equivalent to saying that a pullback is a
  limit from a diagram:
  % https://q.uiver.app/?q=WzAsMyxbMiwwLCJZIl0sWzAsMCwiWCJdLFsxLDAsIloiXSxbMCwyLCJ0Il0sWzEsMiwicyIsMl1d
  % https://q.uiver.app/?q=WzAsMyxbMiwwLCJ5Il0sWzAsMCwieCJdLFsxLDAsInoiXSxbMCwyLCJ0IiwyXSxbMSwyLCJzIl1d
  \[\begin{tikzcd}[ampersand replacement=\&]
    x \& z \& y
    \arrow["t"', from=1-3, to=1-2]
    \arrow["s", from=1-1, to=1-2]
  \end{tikzcd}\]
\end{definition}

\subsubsection*{Pushouts}

\begin{definition}[Pushout\index{Pushout}\label{def:pushout}]
  For a set-up in an arbitrary category:
  \parencite{leinster:basic_category_theory}
  % https://q.uiver.app/?q=WzAsMyxbMiwwLCJ6Il0sWzAsMCwieSJdLFsxLDAsIngiXSxbMiwwLCJ0Il0sWzIsMSwicyIsMl1d
  \[\begin{tikzcd}[ampersand replacement=\&]
    y \& x \& z
    \arrow["t", from=1-2, to=1-3]
    \arrow["s"', from=1-2, to=1-1]
  \end{tikzcd}\]

  A pushout candidate is an object $a$ with morphisms $j_1: y \to a$ and $j_2:
  z\to a$ such that the following commutes:
  % https://q.uiver.app/?q=WzAsNCxbMSwwLCJ6Il0sWzAsMSwieSJdLFsxLDEsIngiXSxbMCwwLCJhIl0sWzIsMCwidCIsMl0sWzIsMSwicyJdLFsxLDMsImpfMSJdLFswLDMsImpfMiIsMl1d
  \[\begin{tikzcd}[ampersand replacement=\&]
    a \& z \\
    y \& x
    \arrow["t"', from=2-2, to=1-2]
    \arrow["s", from=2-2, to=2-1]
    \arrow["{j_1}", from=2-1, to=1-1]
    \arrow["{j_2}"', from=1-2, to=1-1]
  \end{tikzcd}\]

  A pushout is a candidate $(p, i_1, i_2)$ such that for any candidate $(a,
  j_1, j_2)$ there exists a unique $h:p\to a$ that makes the following
  commute:
  % https://q.uiver.app/?q=WzAsNSxbMiwxLCJ6Il0sWzEsMiwieSJdLFsyLDIsIngiXSxbMSwxLCJwIl0sWzAsMCwiYSJdLFsyLDAsInQiLDJdLFsyLDEsInMiXSxbMSwzLCJpXzEiXSxbMCwzLCJpXzIiLDJdLFszLDQsImgiLDEseyJzdHlsZSI6eyJib2R5Ijp7Im5hbWUiOiJkYXNoZWQifX19XSxbMSw0LCJqXzEiXSxbMCw0LCJqXzIiLDJdXQ==
  \[\begin{tikzcd}[ampersand replacement=\&]
    a \\
    \& p \& z \\
    \& y \& x
    \arrow["t"', from=3-3, to=2-3]
    \arrow["s", from=3-3, to=3-2]
    \arrow["{i_1}", from=3-2, to=2-2]
    \arrow["{i_2}"', from=2-3, to=2-2]
    \arrow["h"{description}, dashed, from=2-2, to=1-1]
    \arrow["{j_1}", from=3-2, to=1-1]
    \arrow["{j_2}"', from=2-3, to=1-1]
  \end{tikzcd}\]
\end{definition}

\begin{definition}[Pushout as Colimit\index{Pushout!as Colimit}]
  Definition \ref{def:pullback} is equivalent to saying that a pushout is a
  colimit of a diagram:
  % https://q.uiver.app/?q=WzAsMyxbMiwwLCJ6Il0sWzAsMCwieSJdLFsxLDAsIngiXSxbMiwwLCJ0Il0sWzIsMSwicyIsMl1d
  \[\begin{tikzcd}[ampersand replacement=\&]
    y \& x \& z
    \arrow["t", from=1-2, to=1-3]
    \arrow["s"', from=1-2, to=1-1]
  \end{tikzcd}\]
\end{definition}