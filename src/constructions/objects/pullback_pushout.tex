\subsection{Pullbacks and Pushouts}

\begin{definition}[Pullback\index{Pullback}]\label{def:pullback}
  For a set-up in an arbitrary category:
  \parencite{leinster:basic_category_theory}
  % https://q.uiver.app/?q=WzAsMyxbMiwwLCJZIl0sWzAsMCwiWCJdLFsxLDAsIloiXSxbMCwyLCJ0Il0sWzEsMiwicyIsMl1d
  \[\begin{tikzcd}[ampersand replacement=\&]
    X \& Z \& Y
    \arrow["t", from=1-3, to=1-2]
    \arrow["s"', from=1-1, to=1-2]
  \end{tikzcd}\]

  A pullback candidate is an object $P$ with morphisms $p_1:P\to X$ and
  $p_2:P\to Y$ such that the following commutes:
  % https://q.uiver.app/?q=WzAsNCxbMSwwLCJZIl0sWzAsMSwiWCJdLFsxLDEsIloiXSxbMCwwLCJQIl0sWzAsMiwidCJdLFsxLDIsInMiLDJdLFszLDAsInBfMiJdLFszLDEsInBfMSIsMl1d
  \[\begin{tikzcd}[ampersand replacement=\&]
    P \& Y \\
    X \& Z
    \arrow["t", from=1-2, to=2-2]
    \arrow["s"', from=2-1, to=2-2]
    \arrow["{p_2}", from=1-1, to=1-2]
    \arrow["{p_1}"', from=1-1, to=2-1]
  \end{tikzcd}\]

  A pullback is a candidate terminal among all other candidates.
\end{definition}