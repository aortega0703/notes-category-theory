\subsection{Pullbacks and Pushouts}

\subsubsection*{Pullbacks}

\begin{definition}[Pullback\index{Pullback}\label{def:pullback}]
  For a set-up in an arbitrary category:
  \parencite{leinster:basic_category_theory}
  % https://q.uiver.app/?q=WzAsMyxbMiwwLCJZIl0sWzAsMCwiWCJdLFsxLDAsIloiXSxbMCwyLCJ0Il0sWzEsMiwicyIsMl1d
  \[\begin{tikzcd}[ampersand replacement=\&]
    X \& Z \& Y
    \arrow["t", from=1-3, to=1-2]
    \arrow["s"', from=1-1, to=1-2]
  \end{tikzcd}\]

  A pullback candidate is an object $P$ with morphisms $p_1:P\to X$ and
  $p_2:P\to Y$ such that the following commutes:
  % https://q.uiver.app/?q=WzAsNCxbMSwwLCJZIl0sWzAsMSwiWCJdLFsxLDEsIloiXSxbMCwwLCJQIl0sWzAsMiwidCJdLFsxLDIsInMiLDJdLFszLDAsInBfMiJdLFszLDEsInBfMSIsMl1d
  \[\begin{tikzcd}[ampersand replacement=\&]
    P \& Y \\
    X \& Z
    \arrow["t", from=1-2, to=2-2]
    \arrow["s"', from=2-1, to=2-2]
    \arrow["{p_2}", from=1-1, to=1-2]
    \arrow["{p_1}"', from=1-1, to=2-1]
  \end{tikzcd}\]

  A pullback is a candidate $(P, p_1, p_2)$ such that for any other candidate
  $(A, p'_1, p'_2)$ there exists a unique $h:A\to P$ that makes the following
  commute:
  % https://q.uiver.app/?q=WzAsNSxbMSwxLCJQIl0sWzIsMSwiWSJdLFsxLDIsIlgiXSxbMiwyLCJaIl0sWzAsMCwiQSJdLFswLDIsInBfMiIsMl0sWzAsMSwicF8xIl0sWzEsMywidCJdLFsyLDMsInMiLDJdLFs0LDIsInAnXzIiLDJdLFs0LDEsInAnXzEiXSxbNCwwLCJoIiwxLHsic3R5bGUiOnsiYm9keSI6eyJuYW1lIjoiZGFzaGVkIn19fV1d
  \[\begin{tikzcd}[ampersand replacement=\&]
    A \\
    \& P \& Y \\
    \& X \& Z
    \arrow["{p_2}"', from=2-2, to=3-2]
    \arrow["{p_1}", from=2-2, to=2-3]
    \arrow["t", from=2-3, to=3-3]
    \arrow["s"', from=3-2, to=3-3]
    \arrow["{p'_2}"', from=1-1, to=3-2]
    \arrow["{p'_1}", from=1-1, to=2-3]
    \arrow["h"{description}, dashed, from=1-1, to=2-2]
  \end{tikzcd}\]
\end{definition}

\begin{definition}[Pullback as Limit\index{Pullback!as Limit}]
  Definition \ref{def:pullback} is equivalent to saying that a pullback is a
  limit from a diagram:
  % https://q.uiver.app/?q=WzAsMyxbMiwwLCJZIl0sWzAsMCwiWCJdLFsxLDAsIloiXSxbMCwyLCJ0Il0sWzEsMiwicyIsMl1d
  \[\begin{tikzcd}[ampersand replacement=\&]
    X \& Z \& Y
    \arrow["t", from=1-3, to=1-2]
    \arrow["s"', from=1-1, to=1-2]
  \end{tikzcd}\]
\end{definition}

\subsubsection*{Pushouts}

\begin{definition}[Pushout\index{Pushout}\label{def:pushout}]
  For a set-up in an arbitrary category:
  \parencite{leinster:basic_category_theory}
  % https://q.uiver.app/?q=WzAsMyxbMiwwLCJaIl0sWzAsMCwiWSJdLFsxLDAsIlgiXSxbMiwwLCJ0Il0sWzIsMSwicyIsMl1d
  \[\begin{tikzcd}[ampersand replacement=\&]
    Y \& X \& Z
    \arrow["t", from=1-2, to=1-3]
    \arrow["s"', from=1-2, to=1-1]
  \end{tikzcd}\]

  A pushout candidate is an object $P$ with morphisms $p_1: Y \to P$ and $p_2:
  Z\to P$ such that the following commutes:
  % https://q.uiver.app/?q=WzAsNCxbMSwwLCJaIl0sWzAsMSwiWSJdLFsxLDEsIlgiXSxbMCwwLCJQIl0sWzIsMCwidCIsMl0sWzIsMSwicyJdLFsxLDMsInBfMSJdLFswLDMsInBfMiIsMl1d
  \[\begin{tikzcd}[ampersand replacement=\&]
    P \& Z \\
    Y \& X
    \arrow["t"', from=2-2, to=1-2]
    \arrow["s", from=2-2, to=2-1]
    \arrow["{p_1}", from=2-1, to=1-1]
    \arrow["{p_2}"', from=1-2, to=1-1]
  \end{tikzcd}\]

  A pushout is a candidate $(P, p_1, p_2)$ such that for any candidate $(A,
  p'_1, p'_2)$ there exists a unique $h:P\to A$ that makes the following
  commute:
  % https://q.uiver.app/?q=WzAsNSxbMiwxLCJaIl0sWzEsMiwiWSJdLFsyLDIsIlgiXSxbMSwxLCJQIl0sWzAsMCwiQSJdLFsyLDAsInQiLDJdLFsyLDEsInMiXSxbMSwzLCJwXzEiXSxbMCwzLCJwXzIiLDJdLFszLDQsImgiLDEseyJzdHlsZSI6eyJib2R5Ijp7Im5hbWUiOiJkYXNoZWQifX19XSxbMSw0LCJwJ18xIl0sWzAsNCwicCdfMiIsMl1d
  \[\begin{tikzcd}[ampersand replacement=\&]
    A \\
    \& P \& Z \\
    \& Y \& X
    \arrow["t"', from=3-3, to=2-3]
    \arrow["s", from=3-3, to=3-2]
    \arrow["{p_1}", from=3-2, to=2-2]
    \arrow["{p_2}"', from=2-3, to=2-2]
    \arrow["h"{description}, dashed, from=2-2, to=1-1]
    \arrow["{p'_1}", from=3-2, to=1-1]
    \arrow["{p'_2}"', from=2-3, to=1-1]
  \end{tikzcd}\]
\end{definition}

\begin{definition}[Pushout as Colimit\index{Pushout!as Colimit}]
  Definition \ref{def:pullback} is equivalent to saying that a pushout is a colimit
  from a diagram:
  % https://q.uiver.app/?q=WzAsMyxbMiwwLCJaIl0sWzAsMCwiWSJdLFsxLDAsIlgiXSxbMiwwLCJ0Il0sWzIsMSwicyIsMl1d
  \[\begin{tikzcd}[ampersand replacement=\&]
    Y \& X \& Z
    \arrow["t", from=1-2, to=1-3]
    \arrow["s"', from=1-2, to=1-1]
  \end{tikzcd}\]
\end{definition}