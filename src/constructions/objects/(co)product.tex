\subsection{Products and Coproducts}
The product of a pair of objects, if it exists, is another object such that it
represents the cartesian product of both of them.

\begin{definition}[Binary Product\index{Product!Binary}]\label{def:product}
	For a pair of objects $(A, B)$, a product candidate consists of an object $P$
	together with morphisms $\mathrm{fst}: A\times B\to A$ and
	$\mathrm{scn}:A\times B\to B$:
	% https://q.uiver.app/?q=WzAsMyxbMCwxLCJBIl0sWzIsMSwiQiJdLFsxLDAsIlAiXSxbMiwwXSxbMiwxXV0=
	\[\begin{tikzcd}[ampersand replacement=\&]
		\& P \\
		A \&\& B
		\arrow[from=1-2, to=2-1]
		\arrow[from=1-2, to=2-3]
	\end{tikzcd}\]

	A product $A\times B$ is a candidate such that for any other candidate $P$ the
	following commutes: \parencite{leinster:basic_category_theory}
	% https://q.uiver.app/?q=WzAsNCxbMiwxLCJBIl0sWzEsMSwiQiJdLFswLDIsIkFcXHRpbWVzIEIiXSxbMCwwLCJQIl0sWzIsMF0sWzIsMV0sWzMsMF0sWzMsMV0sWzMsMiwiIiwxLHsic3R5bGUiOnsiYm9keSI6eyJuYW1lIjoiZGFzaGVkIn19fV1d
	% https://q.uiver.app/?q=WzAsNCxbMSwxLCJBIl0sWzIsMSwiQiJdLFswLDIsIkFcXHRpbWVzIEIiXSxbMCwwLCJQIl0sWzIsMCwiXFxtYXRocm17ZnN0fSJdLFsyLDEsIlxcbWF0aHJte3NuZH0iLDJdLFszLDAsInBfMSIsMl0sWzMsMSwicF8yIl0sWzMsMiwiaCIsMix7InN0eWxlIjp7ImJvZHkiOnsibmFtZSI6ImRhc2hlZCJ9fX1dXQ==
	\[\begin{tikzcd}[ampersand replacement=\&]
		P \\
		\& A \& B \\
		{A\times B}
		\arrow["{\mathrm{fst}}", from=3-1, to=2-2]
		\arrow["{\mathrm{snd}}"', from=3-1, to=2-3]
		\arrow["{p_1}"', from=1-1, to=2-2]
		\arrow["{p_2}", from=1-1, to=2-3]
		\arrow["h"', dashed, from=1-1, to=3-1]
	\end{tikzcd}\]
\end{definition}

\begin{definition}[Product as Limit\index{Product!as Limit}]\label{def:product:limit}
	More generally, for a discrete category $I$, the functor $F:I\to C$ is an
	$I$-indexed family of objects $(c_i)_{i\in I}$ in $C$. The product
	$\prod_{i\in I} c_i$ is the limit of such a functor $F$.
	\parencite{leinster:basic_category_theory}
\end{definition}

\begin{remark}
	In the case where $I=\underline{0}$ the product is denoted as $1$ and
	corresponds to a terminal object.
\end{remark}

\begin{remark}
	In the case where $I=\underline{2}$, Definitions \ref{def:product} and
	\ref{def:product:def:limit} are equivalent.
\end{remark}

\begin{theorem}[Uniqueness of Product]
	The product of a collection of objects is unique up to isomorphism.

	\begin{proof}
		As the product of objects is a limit, and by Theorem \ref{thm:limit_iso}
		limits are unique up to isomorphism, then products are also unique up to
		isomorphism.
	\end{proof}
\end{theorem}

The coproduct of a pair of objects, if it exists, is another object such that it
represents the disjunt union of both of them.
\begin{definition}[Coproduct\index{Product!Coproduct}\index{Sum}\label{def:coproduct}]
	For a discrete category $I$, the functor $F:I\to C$ is an $I$-indexed family
	of objects $(c_i)_{i\in I}$ in $C$. The coproduct or sum $\Sigma_{i\in I} c_i$
	is the colimit of such a functor $F$.
	\parencite{leinster:basic_category_theory}
\end{definition}

\begin{remark}
	Products and coproducts are dual concepts.
\end{remark}

\begin{remark}
	In the case where $I=\underline{0}$ the product is denoted as $0$ and
	corresponds to an initial object.
\end{remark}

\begin{remark}
	In the case where $I=\underline{2}$ the coproduct of objects $a$ and $b$ is
	denoted $a + b$ and the resulting diagram (which must commute for all $c$)
	is as follows: \parencite{leinster:basic_category_theory}
	% https://q.uiver.app/?q=WzAsNixbMywxLCJhKyBiIl0sWzIsMSwiYSJdLFs0LDEsImIiXSxbMywwLCJjIl0sWzAsMSwiMCJdLFsxLDEsIjEiXSxbMSwwLCJcXG1hdGhybXtsZWZ0fSIsMl0sWzIsMCwiXFxtYXRocm17cmlnaHR9Il0sWzAsMywiaCIsMCx7InN0eWxlIjp7ImJvZHkiOnsibmFtZSI6ImRhc2hlZCJ9fX1dLFsxLDMsImYiXSxbMiwzLCJnIiwyXSxbNCwxLCIiLDAseyJjdXJ2ZSI6NCwiY29sb3VyIjpbMjQwLDYwLDYwXSwic3R5bGUiOnsiYm9keSI6eyJuYW1lIjoiZG90dGVkIn19fV0sWzUsMiwiIiwwLHsiY3VydmUiOjQsImNvbG91ciI6WzI0MCw2MCw2MF0sInN0eWxlIjp7ImJvZHkiOnsibmFtZSI6ImRvdHRlZCJ9fX1dXQ==
	\[\begin{tikzcd}[ampersand replacement=\&]
		\&\&\& c \\
		0 \& 1 \& a \& {a+ b} \& b
		\arrow["{\mathrm{left}}"', from=2-3, to=2-4]
		\arrow["{\mathrm{right}}", from=2-5, to=2-4]
		\arrow["h", dashed, from=2-4, to=1-4]
		\arrow["f", from=2-3, to=1-4]
		\arrow["g"', from=2-5, to=1-4]
		\arrow[draw={rgb,255:red,92;green,92;blue,214}, curve={height=24pt}, dotted, from=2-1, to=2-3]
		\arrow[draw={rgb,255:red,92;green,92;blue,214}, curve={height=24pt}, dotted, from=2-2, to=2-5]
	\end{tikzcd}\]
\end{remark}

\begin{remark}
	For the binary case, Definition \ref{def:coproduct} can be expressed as an
	initial morphism from the (binary) diagonal functor to $(a, b)$:
	% https://q.uiver.app/?q=WzAsNSxbMiwwLCIoYSxiKSJdLFsxLDAsIlxcRGVsdGEoYSsgYikiXSxbMSwxLCJcXERlbHRhKGMpIl0sWzAsMSwiYyJdLFswLDAsImErIGIiXSxbMCwxLCIoXFxtYXRocm17bGVmdH0sXFxtYXRocm17cmlnaHR9KSIsMl0sWzAsMiwiZiciXSxbMSwyLCJcXERlbHRhKGgpIiwyLHsic3R5bGUiOnsiYm9keSI6eyJuYW1lIjoiZGFzaGVkIn19fV0sWzQsMywiaCIsMix7InN0eWxlIjp7ImJvZHkiOnsibmFtZSI6ImRhc2hlZCJ9fX1dLFs0LDEsIiIsMix7ImNvbG91ciI6WzI0MCw2MCw2MF0sInN0eWxlIjp7ImJvZHkiOnsibmFtZSI6ImRvdHRlZCJ9fX1dLFszLDIsIiIsMix7ImNvbG91ciI6WzI0MCw2MCw2MF0sInN0eWxlIjp7ImJvZHkiOnsibmFtZSI6ImRvdHRlZCJ9fX1dLFs4LDcsIiIsMSx7ImxhYmVsX3Bvc2l0aW9uIjozMCwic2hvcnRlbiI6eyJzb3VyY2UiOjIwLCJ0YXJnZXQiOjUwfSwibGV2ZWwiOjEsImNvbG91ciI6WzI0MCw2MCw2MF0sInN0eWxlIjp7ImJvZHkiOnsibmFtZSI6ImRvdHRlZCJ9fX1dXQ==
	\[\begin{tikzcd}[ampersand replacement=\&]
		{a+ b} \& {\Delta(a+ b)} \& {(a,b)} \\
		c \& {\Delta(c)}
		\arrow["{(\mathrm{left},\mathrm{right})}"', from=1-3, to=1-2]
		\arrow["{f'}", from=1-3, to=2-2]
		\arrow[""{name=0, anchor=center, inner sep=0}, "{\Delta(h)}"', dashed, from=1-2, to=2-2]
		\arrow[""{name=1, anchor=center, inner sep=0}, "h"', dashed, from=1-1, to=2-1]
		\arrow[draw={rgb,255:red,92;green,92;blue,214}, dotted, from=1-1, to=1-2]
		\arrow[draw={rgb,255:red,92;green,92;blue,214}, dotted, from=2-1, to=2-2]
		\arrow[draw={rgb,255:red,92;green,92;blue,214}, shorten <=8pt, shorten >=20pt, dotted, from=1, to=0]
	\end{tikzcd}\]
\end{remark}

\begin{theorem}[Uniqueness of Coproduct]
	The coproduct of a collection of objects is unique up to isomorphism.

	\begin{proof}
		As the product of objects is a colimit, and by Theorem \ref{thm:colimit_iso}
		colimits are unique up to isomorphism, then coproducts are also unique up to
		isomorphism.
	\end{proof}
\end{theorem}