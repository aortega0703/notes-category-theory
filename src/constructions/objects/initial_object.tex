\subsection{Initial Object}

\begin{definition}
  For a category $C$, an object $0$ is said to be initial when there exists a
  unique morphism $f: 0\to c$ for every object $c\in C$.
  \parencite{awodey:category_theory}
\end{definition}

\begin{theorem}[Isomorphism of Initial Objects]\label{thm:init_obj_iso} If
  there are multiple terminal objects in a category, then they are all
  isomorphic.

  \begin{proof}
    Consider two initial objects $c$ and $c'$ in the same category. By their
    initiality there exists a unique morphism from themselves into themselves,
    the identity morphism. Additionally the exists two unique morphisms $f:c\to
    c'$ and $g:c' \to c$ between them. Their compositions $g\circ f:c\to c$ and
    $f\circ g:c'\to c'$ must then be equal to their respective identities
    $\id_c$ and $\id_{c'}$ coinciding with the definition of isomorphism.
  \end{proof}
\end{theorem}
e.g. In Set, the empty set is an initial object. There exists only one function
from it to every other set, that being the empty function.

\subsubsection*{Uniqueness up to isomorphism}

Consider a category with a pair of terminal objects $c$ and $c'$. By being
terminal there exists a unique morphism from any object into $c$ and $c'$,
particularly, there exists unique morphisms $f: c' \to c$, $g: c \to c'$ between
themselves and $\id_c: c\to c$, $\id_c: c\to c$ to themselves.
Then the compositions $f\circ g:c\to c$ and $g\circ f:c' \to c'$ must coincide
with the identity morphisms, making $f,g$ an isomorphism between $c$ and $c'$.