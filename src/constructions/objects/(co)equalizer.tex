\subsection{Equalizers and Coequalizers}

\subsubsection*{Equalizer}

\begin{definition}[Equalizer\index{Equalizer}\label{def:equalizer}]
  For a set-up in an arbitrary category:
  \parencite{leinster:basic_category_theory}

  % https://q.uiver.app/?q=WzAsMixbMCwwLCJYIl0sWzEsMCwiWSJdLFswLDEsImYiLDAseyJvZmZzZXQiOi0yfV0sWzAsMSwiZyIsMix7Im9mZnNldCI6Mn1dXQ==
  \[\begin{tikzcd}[ampersand replacement=\&]
    X \& Y
    \arrow["f", shift left=2, from=1-1, to=1-2]
    \arrow["g"', shift right=2, from=1-1, to=1-2]
  \end{tikzcd}\]

  An equalizer candidate of $f$ and $g$ consists of an object $E$ together with
  a morphism $i:E\to X$ such that $fi = gi$, i.e. the following commutes:
  % https://q.uiver.app/?q=WzAsNCxbMCwwLCJFIl0sWzEsMCwiWCJdLFswLDEsIlgiXSxbMSwxLCJZIl0sWzEsMywiZiJdLFsyLDMsImciLDJdLFswLDIsImkiLDJdLFswLDEsImkiXV0=
  \[\begin{tikzcd}[ampersand replacement=\&]
    E \& X \\
    X \& Y
    \arrow["f", from=1-2, to=2-2]
    \arrow["g"', from=2-1, to=2-2]
    \arrow["i"', from=1-1, to=2-1]
    \arrow["i", from=1-1, to=1-2]
  \end{tikzcd}\]

  An equalizer is a candidate $(E, i)$ such that for any candidate $(A,
  i')$ there exists a unique $h:A\to E$ that makes the following commute:
  % https://q.uiver.app/?q=WzAsNSxbMSwxLCJFIl0sWzIsMSwiWCJdLFsxLDIsIlgiXSxbMiwyLCJZIl0sWzAsMCwiQSJdLFswLDIsImkiLDJdLFswLDEsImkiXSxbMSwzLCJmIl0sWzIsMywiZyIsMl0sWzQsMiwiaSciLDJdLFs0LDEsImknIl0sWzQsMCwiaCIsMSx7InN0eWxlIjp7ImJvZHkiOnsibmFtZSI6ImRhc2hlZCJ9fX1dXQ==
  \[\begin{tikzcd}[ampersand replacement=\&]
    A \\
    \& E \& X \\
    \& X \& Y
    \arrow["i"', from=2-2, to=3-2]
    \arrow["i", from=2-2, to=2-3]
    \arrow["f", from=2-3, to=3-3]
    \arrow["g"', from=3-2, to=3-3]
    \arrow["{i'}"', from=1-1, to=3-2]
    \arrow["{i'}", from=1-1, to=2-3]
    \arrow["h"{description}, dashed, from=1-1, to=2-2]
  \end{tikzcd}\]
\end{definition}

\begin{definition}[Equalizer as Limit\index{Equalizer!as Limit}]
  Definition \ref{def:equalizer} is equivalent to saying that an equalizer is
  the limit from a diagram:
  % https://q.uiver.app/?q=WzAsMixbMCwwLCJYIl0sWzEsMCwiWSJdLFswLDEsImYiLDAseyJvZmZzZXQiOi0yfV0sWzAsMSwiZyIsMix7Im9mZnNldCI6Mn1dXQ==
  \[\begin{tikzcd}[ampersand replacement=\&]
    X \& Y
    \arrow["f", shift left=2, from=1-1, to=1-2]
    \arrow["g"', shift right=2, from=1-1, to=1-2]
  \end{tikzcd}\]
\end{definition}

\subsubsection*{Coequalizer}

\begin{definition}[Coequalizer\index{Coequalizer}]
  For a set-up in an arbitrary category:
  \parencite{leinster:basic_category_theory}
  % https://q.uiver.app/?q=WzAsMixbMCwwLCJYIl0sWzEsMCwiWSJdLFswLDEsImYiLDAseyJvZmZzZXQiOi0yfV0sWzAsMSwiZyIsMix7Im9mZnNldCI6Mn1dXQ==
  \[\begin{tikzcd}[ampersand replacement=\&]
    X \& Y
    \arrow["f", shift left=2, from=1-1, to=1-2]
    \arrow["g"', shift right=2, from=1-1, to=1-2]
  \end{tikzcd}\]

  A co-equalizer candidate of $f$ and $g$ consists of an object $E$ together
  with a morphism $i:Y\to E$ such that $if = ig$, i.e. the following commutes:
  % https://q.uiver.app/?q=WzAsNCxbMCwwLCJYIl0sWzEsMCwiWSJdLFswLDEsIlkiXSxbMSwxLCJFIl0sWzEsMywiaSJdLFsyLDMsImkiLDJdLFswLDIsImciLDJdLFswLDEsImYiXV0=
  \[\begin{tikzcd}[ampersand replacement=\&]
    X \& Y \\
    Y \& E
    \arrow["i", from=1-2, to=2-2]
    \arrow["i"', from=2-1, to=2-2]
    \arrow["g"', from=1-1, to=2-1]
    \arrow["f", from=1-1, to=1-2]
  \end{tikzcd}\]

  A co-equalizer is a candidate $(E, i)$ such that for any candidate $(A, i')$
  there exists a unique $h: E\to A$ that makes the following commute:
  % https://q.uiver.app/?q=WzAsNSxbMCwwLCJYIl0sWzEsMCwiWSJdLFswLDEsIlkiXSxbMSwxLCJFIl0sWzIsMiwiQSJdLFsxLDMsImkiXSxbMiwzLCJpIiwyXSxbMCwyLCJnIiwyXSxbMCwxLCJmIl0sWzIsNCwiaSciLDJdLFsxLDQsImknIl0sWzMsNCwiaCIsMSx7InN0eWxlIjp7ImJvZHkiOnsibmFtZSI6ImRhc2hlZCJ9fX1dXQ==
  \[\begin{tikzcd}[ampersand replacement=\&]
    X \& Y \\
    Y \& E \\
    \&\& A
    \arrow["i", from=1-2, to=2-2]
    \arrow["i"', from=2-1, to=2-2]
    \arrow["g"', from=1-1, to=2-1]
    \arrow["f", from=1-1, to=1-2]
    \arrow["{i'}"', from=2-1, to=3-3]
    \arrow["{i'}", from=1-2, to=3-3]
    \arrow["h"{description}, dashed, from=2-2, to=3-3]
  \end{tikzcd}\]
\end{definition}

\begin{definition}[Co-equalizer as Colimit\index{Coequalizer!as Colimit}]

\end{definition}