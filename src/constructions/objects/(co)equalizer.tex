\subsection{Equalizers and Coequalizers}

\subsubsection*{Equalizer}

\begin{definition}[Equalizer\index{Equalizer}\label{def:equalizer}]
  For a set-up in an arbitrary
  category~\parencite{leinster:basic_category_theory}:
  % https://q.uiver.app/?q=WzAsMixbMCwwLCJ4Il0sWzEsMCwieSJdLFswLDEsImYiLDAseyJvZmZzZXQiOi0xfV0sWzAsMSwiZyIsMix7Im9mZnNldCI6MX1dXQ==
  \[\begin{tikzcd}[ampersand replacement=\&]
    x \& y
    \arrow["f", shift left=1, from=1-1, to=1-2]
    \arrow["g"', shift right=1, from=1-1, to=1-2]
  \end{tikzcd}\]

  An equalizer candidate of $f$ and $g$ consists of an object $a$ together with
  a morphism $j:a\to x$ such that $f\circ j = g\circ j$, i.e. the following commutes:
  % https://q.uiver.app/?q=WzAsNCxbMCwwLCJhIl0sWzEsMCwieCJdLFswLDEsIngiXSxbMSwxLCJ5Il0sWzEsMywiZiJdLFsyLDMsImciLDJdLFswLDIsImoiLDJdLFswLDEsImoiXV0=
  \[\begin{tikzcd}[ampersand replacement=\&]
    a \& x \\
    x \& y
    \arrow["f", from=1-2, to=2-2]
    \arrow["g"', from=2-1, to=2-2]
    \arrow["j"', from=1-1, to=2-1]
    \arrow["j", from=1-1, to=1-2]
  \end{tikzcd}\]

  An equalizer is a candidate $\<e, i\>$ such that for any candidate $\<a, j\>$
  there exists a unique $h:a\to e$ that makes the following commute:
  % https://q.uiver.app/?q=WzAsNSxbMSwxLCJlIl0sWzIsMSwieCJdLFsxLDIsIngiXSxbMiwyLCJ5Il0sWzAsMCwiYSJdLFswLDIsImkiLDJdLFswLDEsImkiXSxbMSwzLCJmIl0sWzIsMywiZyIsMl0sWzQsMiwiaiIsMl0sWzQsMSwiaiJdLFs0LDAsImgiLDEseyJzdHlsZSI6eyJib2R5Ijp7Im5hbWUiOiJkYXNoZWQifX19XV0=
  \[\begin{tikzcd}[ampersand replacement=\&]
    a \\
    \& e \& x \\
    \& x \& y
    \arrow["i"', from=2-2, to=3-2]
    \arrow["i", from=2-2, to=2-3]
    \arrow["f", from=2-3, to=3-3]
    \arrow["g"', from=3-2, to=3-3]
    \arrow["j"', from=1-1, to=3-2]
    \arrow["j", from=1-1, to=2-3]
    \arrow["h"{description}, dashed, from=1-1, to=2-2]
  \end{tikzcd}\]
\end{definition}

\begin{definition}[Equalizer as Limit\index{Equalizer!as Limit}]
  Definition \ref{def:equalizer} is equivalent to saying that an equalizer is
  the limit of a diagram:
  % https://q.uiver.app/?q=WzAsMixbMCwwLCJ4Il0sWzEsMCwieSJdLFswLDEsImYiLDAseyJvZmZzZXQiOi0xfV0sWzAsMSwiZyIsMix7Im9mZnNldCI6MX1dXQ==
  \[\begin{tikzcd}[ampersand replacement=\&]
    x \& y
    \arrow["f", shift left=1, from=1-1, to=1-2]
    \arrow["g"', shift right=1, from=1-1, to=1-2]
  \end{tikzcd}\]
\end{definition}

\subsubsection*{Coequalizer}

\begin{definition}[Coequalizer\index{Coequalizer}\label{def:coequalizer}]
  For a set-up in an arbitrary
  category~\parencite{leinster:basic_category_theory}:
  % https://q.uiver.app/?q=WzAsMixbMCwwLCJ4Il0sWzEsMCwieSJdLFswLDEsImYiLDAseyJvZmZzZXQiOi0xfV0sWzAsMSwiZyIsMix7Im9mZnNldCI6MX1dXQ==
  \[\begin{tikzcd}[ampersand replacement=\&]
    x \& y
    \arrow["f", shift left=1, from=1-1, to=1-2]
    \arrow["g"', shift right=1, from=1-1, to=1-2]
  \end{tikzcd}\]

  A co-equalizer candidate of $f$ and $g$ consists of an object $a$ together
  with a morphism $j:y\to e$ such that $j\circ f = j\circ g$, i.e. the following
  commutes:
  % https://q.uiver.app/?q=WzAsNCxbMCwwLCJ4Il0sWzEsMCwieSJdLFswLDEsInkiXSxbMSwxLCJhIl0sWzIsMywiaiIsMl0sWzAsMiwiZyIsMl0sWzAsMSwiZiJdLFsxLDMsImoiXV0=
  \[\begin{tikzcd}[ampersand replacement=\&]
    x \& y \\
    y \& a
    \arrow["j"', from=2-1, to=2-2]
    \arrow["g"', from=1-1, to=2-1]
    \arrow["f", from=1-1, to=1-2]
    \arrow["j", from=1-2, to=2-2]
  \end{tikzcd}\]

  A co-equalizer is a candidate $\<e, i\>$ such that for any candidate $\<a,
  j\>$ there exists a unique $h: e\to a$ that makes the following commute:
  % https://q.uiver.app/?q=WzAsNSxbMCwwLCJ4Il0sWzEsMCwieSJdLFswLDEsInkiXSxbMSwxLCJlIl0sWzIsMiwiYSJdLFsxLDMsImkiXSxbMiwzLCJpIiwyXSxbMCwyLCJnIiwyXSxbMCwxLCJmIl0sWzIsNCwiaiIsMl0sWzEsNCwiaiJdLFszLDQsImgiLDEseyJzdHlsZSI6eyJib2R5Ijp7Im5hbWUiOiJkYXNoZWQifX19XV0=
  \[\begin{tikzcd}[ampersand replacement=\&]
    x \& y \\
    y \& e \\
    \&\& a
    \arrow["i", from=1-2, to=2-2]
    \arrow["i"', from=2-1, to=2-2]
    \arrow["g"', from=1-1, to=2-1]
    \arrow["f", from=1-1, to=1-2]
    \arrow["j"', from=2-1, to=3-3]
    \arrow["j", from=1-2, to=3-3]
    \arrow["h"{description}, dashed, from=2-2, to=3-3]
  \end{tikzcd}\]
\end{definition}

\begin{definition}[Coequalizer as Colimit\index{Coequalizer!as Colimit}]
  Definition \ref{def:coequalizer} is equivalent to saying that a coequalizer is
  the colimit of a diagram:
  % https://q.uiver.app/?q=WzAsMixbMCwwLCJ4Il0sWzEsMCwieSJdLFswLDEsImYiLDAseyJvZmZzZXQiOi0xfV0sWzAsMSwiZyIsMix7Im9mZnNldCI6MX1dXQ==
  \[\begin{tikzcd}[ampersand replacement=\&]
    x \& y
    \arrow["f", shift left=1, from=1-1, to=1-2]
    \arrow["g"', shift right=1, from=1-1, to=1-2]
  \end{tikzcd}\]
\end{definition}